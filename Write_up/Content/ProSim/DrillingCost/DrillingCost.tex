The following section aims to investigate the impact of drilling costs on the design of single flash \ac{DSC} and binary \ac{ORC} geothermal power plants.

\subsection{Boundary Conditions}
    The geofluid is assumed to be pure water, arriving at the wellhead at a temperature of \qty{473}{\K} (\qty{200}{\degreeCelsius}) and a steam quality \qty{20}{\percent}. The geofluid inlet pressure is back-calculated from the inlet conditions. The mass rate of geofluid is fixed at \qty{50}{\kg\per\s}. The working fluid is assumed to be n-Butane.

    \begin{table}[H]
        \centering
        \caption{The boundary conditions used for the single flash \ac{DSC} and the binary \ac{ORC} geothermal power plants.}
        \label{table:prosim_pure_water_drilling_boundary}
        \begin{tabular}{|c | c |}
    \hline
    \rowcolor{bluepoli!40} % comment this line to remove the color
    \textbf{Condition} & \textbf{Values} \T\B \\
    \hline \hline
    Inlet Temperature & \qty{473}{\K} \T\B \\
    Inlet Steam Quality & \qty{20}{\percent} \T\B \\
    Inlet Pressure & calculated \T\B \\
    Outlet Pressure &  \(P_{geo,\;out} = P_{geo,\;in}\) \T\B \\
    Working Fluid &  n-Butane \T\B \\
    \hline
\end{tabular}        
    \end{table}

\subsection{Optimisation Configuration}
    Three optimisations were considered: a thermodynamic optimisation (i.e. maximising the net electrical power) and two techno-economic optimisations (i.e. maximising the specific plant cost), and excluding and the other including the drilling cost.
    
    For all optimisations the evaporation pressure, the degree of super-heating and condensation temperature could be adjusted by the optimiser. For the thermodynamic optimisation, the minimum approach temperature differences in the pre-heater, evaporator, super-heater and condenser are treated as constraints, whereas for the techno-economic optimisations they are adjusted by the optimiser, Table~\ref{table:prosim_pure_water_drilling_opt_params}

    \begin{table}[H]
        \centering
        \caption{The optimisation parameters used for the single flash \ac{DSC} and the binary \ac{ORC} geothermal power plants.}
        \label{table:prosim_pure_water_drilling_opt_params}
        \resizebox{\linewidth}{!}{\begin{tabular}{|c | c | c | c |}
    \hline
    \rowcolor{bluepoli!40} % comment this line to remove the color
       &  & \multicolumn{2}{c|}{\textbf{Techno-economic}} \T\B \\
    \rowcolor{bluepoli!40} % comment this line to remove the color
       & \textbf{Thermodynamic} & \textbf{incl. drilling cost} & \textbf{excl. drilling costs} \T\B \\
    \hline \hline
    Objective Function & \(W_{elec}^{net}\) & \(\frac{C_{plant} + C_{drilling}}{W_{elec}^{net}}\) & \(\frac{C_{plant}}{W_{elec}^{net}}\) \T\B \\
    \hline
    \multicolumn{1}{|l|}{\multirow{7}{*}{Controls}} & \qty{303}{\K}\(\leq T_{cond}\leq\)\qty{400}{\K} & \qty{303}{\K}\(\leq T_{cond}\leq\)\qty{400}{\K} & \qty{303}{\K}\(\leq T_{cond}\leq\)\qty{400}{\K} \T\B \\
    \multicolumn{1}{|l|}{} & \num{0.2}\(\leq \frac{P_{evap}}{P_{crit}}\leq\)\num{0.8} & \num{0.2}\(\leq \frac{P_{evap}}{P_{crit}}\leq\)\num{0.8} & \num{0.2}\(\leq \frac{P_{evap}}{P_{crit}}\leq\)\num{0.8} \T\B \\
    \multicolumn{1}{|l|}{} & \qty{3}{\K}\(\leq \Delta T_{sh}\leq\)\qty{15}{\K} & \qty{3}{\K}\(\leq \Delta T_{sh}\leq\)\qty{15}{\K} & \qty{3}{\K}\(\leq \Delta T_{sh}\leq\)\qty{15}{\K} \T\B \\
    \multicolumn{1}{|l|}{}  &   & \qty{5}{\K}\(\leq \Delta T_{preh}^{min}\leq\)\qty{30}{\K} & \qty{5}{\K}\(\leq \Delta T_{preh}^{min}\leq\)\qty{30}{\K} \T\B \\
    \multicolumn{1}{|l|}{}  &   & \qty{10}{\K}\(\leq \Delta T_{evap}^{min}\leq\)\qty{30}{\K} & \qty{10}{\K}\(\leq \Delta T_{evap}^{min}\leq\)\qty{30}{\K} \T\B \\
    \multicolumn{1}{|l|}{} &   & \qty{10}{\K}\(\leq \Delta T_{suph}^{min}\leq\)\qty{30}{\K} & \qty{10}{\K}\(\leq \Delta T_{suph}^{min}\leq\)\qty{30}{\K} \T\B \\
    \multicolumn{1}{|l|}{} &   & \qty{5}{\K}\(\leq \Delta T_{cond}^{min}\leq\)\qty{30}{\K} & \qty{5}{\K}\(\leq \Delta T_{cond}^{min}\leq\)\qty{30}{\K} \T\B \\
    \hline
    \multicolumn{1}{|l|}{\multirow{4}{*}{Constraints}} & \qty{5}{\K}\(\leq \Delta T_{preh}^{min}\leq\)\qty{30}{\K} &   &   \T\B \\
    \multicolumn{1}{|l|}{}  & \qty{10}{\K}\(\leq \Delta T_{evap}^{min}\leq\)\qty{30}{\K} &  &  \T\B \\
    \multicolumn{1}{|l|}{} & \qty{10}{\K}\(\leq \Delta T_{suph}^{min}\leq\)\qty{30}{\K} &  &  \T\B \\
    \multicolumn{1}{|l|}{} & \qty{5}{\K}\(\leq \Delta T_{cond}^{min}\leq\)\qty{30}{\K} &  &  \T\B \\
    \hline
\end{tabular}}        
    \end{table}

\subsection{Results}
    The power plant was optimised for drilling costs ranging between zero to \qty{60}{\mega\USD\of{2023}}, Figure~\ref{fig:prosim_purewater_drilling}. The specific plant cost obtained from the thermodynamic and simple techno-economic optimisation, are by definition, independent of the drilling cost and form an upper and lower bound for the specific plant cost.

    When the drilling cost is included in the objective function, the specific plant cost tends towards the upper bound determined from the thermodynamic optimisation. This is a result of the contribution of the plant costs towards the total cost diminishing with increasing drilling costs. As such, greater reductions in specific total plant cost can be achieved by maximising the net electrical power, instead of reducing the cost the power plant, which also tends to reduce net electrical power. The significance of this effect also depends on the working fluid, see Figure~\ref{fig:prosim_purewater_drilling_cyclopentane} for CycloPentane for comparison. 
    
    \begin{figure}[H]
        \centering
        \resizebox{\linewidth}{!}{% This file was created with tikzplotlib v0.10.1.
\begin{tikzpicture}

\definecolor{darkgray176}{RGB}{176,176,176}
\definecolor{darkorange25512714}{RGB}{255,127,14}
\definecolor{lightgray204}{RGB}{204,204,204}
\definecolor{steelblue31119180}{RGB}{31,119,180}

\begin{groupplot}[group style={group size=1 by 2}]
\nextgroupplot[
legend cell align={left},
legend style={
  fill opacity=0.8,
  draw opacity=1,
  text opacity=1,
  at={(0.97,0.03)},
  anchor=south east,
  draw=lightgray204
},
tick align=outside,
tick pos=left,
x grid style={darkgray176},
xlabel={Inlet Steam Quality/\unit{\percent}},
xmin=-5, xmax=105,
xtick style={color=black},
y grid style={darkgray176},
ylabel={Net electrical power/\unit{\mega\watt}},
ymin=0, ymax=25,
ytick style={color=black}
]
\addplot [semithick, steelblue31119180]
table {%
0 3.75045845803564
10 5.3233703834005
20 6.52143371948183
30 7.72325173820182
40 8.90734085878374
50 10.0900882744848
60 11.2859816404375
70 12.4799494051215
80 13.6582927603204
90 14.8580333710726
100 16.0578783341511
};
\addlegendentry{Binary ORC}
\addplot [semithick, darkorange25512714]
table {%
0 2.00704976317278
10 3.48622124228406
20 5.29070301214816
30 7.40602998595349
40 9.81065940251747
50 12.2687276928164
60 14.6430124095696
70 17.1478727162591
80 19.6069623546236
90 22.0296823002723
100 24.5007022429487
};
\addlegendentry{Single Flash DSC}
\addplot [semithick, black, dotted, forget plot]
table {%
25 0
25 30
};

\nextgroupplot[
legend cell align={left},
legend style={fill opacity=0.8, draw opacity=1, text opacity=1, draw=lightgray204},
tick align=outside,
tick pos=left,
x grid style={darkgray176},
xlabel={Inlet Steam Quality/\unit{\percent}},
xmin=-5, xmax=105,
xtick style={color=black},
y grid style={darkgray176},
ylabel={Specific Cost/\unit{\USD\of{2023}\per\kilo\watt}},
ymin=0, ymax=4000,
ytick style={color=black}
]
\addplot [semithick, steelblue31119180]
table {%
0 3585.04906524689
10 3062.95078885213
20 2601.17267795214
30 2369.45547809689
40 2183.42127474037
50 2088.17516605231
60 1989.79208588077
70 1937.15559430886
80 1880.15962224779
90 1827.92110615512
100 1784.4193788523
};
\addlegendentry{Binary ORC}
\addplot [semithick, darkorange25512714]
table {%
0 3997.591337054
10 3127.91569148296
20 2572.52535475938
30 2188.20053767263
40 1953.30434225776
50 1797.74666476365
60 1672.2143192466
70 1578.73530661691
80 1498.53083730244
90 1430.09446218898
100 1371.73706471868
};
\addlegendentry{Single Flash DSC}
\addplot [semithick, black, dotted, forget plot]
table {%
25 0
25 4000
};
\end{groupplot}

\end{tikzpicture}
}
        \caption[The specific plant cost (excl. drilling costs) and net electrical power (right) obtained from thermodynamic and techno-economic optimisations using n-Butane as the working fluid.]{The specific plant cost (excl. drilling costs) (left) and the net electrical power (right) as obtained from the thermodynamic and techno-economic optimisations as a function of the drilling cost. The working fluid is n-Butane}
        \label{fig:prosim_purewater_drilling}
    \end{figure}

    \begin{figure}[H]
        \centering
        \resizebox{\linewidth}{!}{% This file was created with tikzplotlib v0.10.1.
\begin{tikzpicture}

\definecolor{darkgray176}{RGB}{176,176,176}
\definecolor{darkorange25512714}{RGB}{255,127,14}
\definecolor{forestgreen4416044}{RGB}{44,160,44}
\definecolor{lightgray204}{RGB}{204,204,204}
\definecolor{steelblue31119180}{RGB}{31,119,180}

\begin{groupplot}[
    group style={
        group size=2 by 1, 
        vertical sep=2.5cm, 
        horizontal sep=2.5cm},
    height=6cm, 
    width=7cm, 
]
\nextgroupplot[
legend cell align={left},
legend style={
  fill opacity=0.8,
  draw opacity=1,
  text opacity=1,
  at={(1.15, -0.35)},
  anchor=north,
  draw=lightgray204
},
tick align=outside,
tick pos=left,
x grid style={darkgray176},
xlabel={Drilling Cost/\unit{\mega\USD\of{2023}}},
xmin=0, xmax=60,
xtick style={color=black},
y grid style={darkgray176},
ylabel={Specific plant cost/\unit{\USD\of{2023}\kilo\watt}},
ymin=0, ymax=4000,
ytick style={color=black},
ytick distance=1000
]
\addplot [semithick, steelblue31119180]
table {%
0 3324.1004335473
1 3324.1004335473
2 3324.1004335473
4 3324.1004335473
8 3324.1004335473
12 3324.1004335473
16 3324.1004335473
20 3324.1004335473
25 3324.1004335473
30 3324.1004335473
35 3324.1004335473
40 3324.1004335473
50 3324.1004335473
60 3324.1004335473
};
\addlegendentry{Thermodynamic Opt.}
\addplot [semithick, darkorange25512714]
table {%
0 2125.94793147044
1 2125.94793147044
2 2125.94793147044
4 2125.94793147044
8 2125.94793147044
12 2125.94793147044
16 2125.94793147044
20 2125.94793147044
25 2125.94793147044
30 2125.94793147044
35 2125.94793147044
40 2125.94793147044
50 2125.94793147044
60 2125.94793147044
};
\addlegendentry{Techno-economic Opt. (excluding drilling costs)}
\addplot [semithick, forestgreen4416044]
table {%
0 2134.86158164739
1 2240.07273722083
2 2235.67594525793
4 2269.77662935057
8 2287.91086994343
12 2349.86685781443
16 2378.85680973178
20 2375.26499444228
25 2429.34208179927
30 2496.20905156471
35 2465.9562363496
40 2530.17562567409
50 2673.29019016132
60 3153.76749697075
};
\addlegendentry{Techno-economic Opt. (including drilling costs)}

\nextgroupplot[
tick align=outside,
tick pos=left,
x grid style={darkgray176},
xlabel={Drilling Cost/\unit{\mega\USD\of{2023}}},
xmin=0, xmax=60,
xtick style={color=black},
y grid style={darkgray176},
ylabel={Net electrical power/\unit{\mega\watt}},
ymin=0, ymax=10,
ytick style={color=black}
]
\addplot [semithick, steelblue31119180]
table {%
0 8.67312917722771
1 8.67312917722771
2 8.67312917722771
4 8.67312917722771
8 8.67312917722771
12 8.67312917722771
16 8.67312917722771
20 8.67312917722771
25 8.67312917722771
30 8.67312917722771
35 8.67312917722771
40 8.67312917722771
50 8.67312917722771
60 8.67312917722771
};
\addplot [semithick, darkorange25512714]
table {%
0 3.24768016868321
1 3.24768016868321
2 3.24768016868321
4 3.24768016868321
8 3.24768016868321
12 3.24768016868321
16 3.24768016868321
20 3.24768016868321
25 3.24768016868321
30 3.24768016868321
35 3.24768016868321
40 3.24768016868321
50 3.24768016868321
60 3.24768016868321
};
\addplot [semithick, forestgreen4416044]
table {%
0 3.14095734400058
1 6.21228907960342
2 6.29436173932615
4 6.55325520823824
8 6.84360915751591
12 7.168577206856
16 7.28400752660467
20 7.26659096414527
25 7.38675295828168
30 7.51110970957674
35 7.43616663970955
40 7.57638400329378
50 7.75201144638119
60 8.36888127426749
};
\end{groupplot}

\end{tikzpicture}
}
        \caption[The specific plant cost (excl. drilling costs) and net electrical power (right) obtained from thermodynamic and techno-economic optimisations using CycloPentane as the working fluid.]{The specific plant cost (excl. drilling costs) (left) and the net electrical power (right) as obtained from the thermodynamic and techno-economic optimisations as a function of the drilling cost. The working fluid is CycloPentane}
        \label{fig:prosim_purewater_drilling_cyclopentane}
    \end{figure}

\clearpage