The presence of impurities like \ac{NCG} in the geofluid represents a challenge for the operation of geothermal power plants, as they can not only be harmful and toxic, but may also pose a global warming potential. Harmful and toxic substances maybe removed by means of scrubbing the \ac{NCG} stream (e.g. AMIS or CarbFix process), however species like carbon dioxide and methane are mostly released into the environment.

The following chapter focuses on the latter constituents of \ac{NCG} (i.e. \ce{CO2} and \ce{CH4}), and aims to investigate the impact of the presence of \ac{NCG} in the geofluid on the performance of the power plant, as well as various options for their disposal.

\subsection{NCG Venting}
    To prevent the build of \ac{NCG} with the power plant, \ac{NCG} must be released from the power plant. Venting to the atmosphere (following the removal of harmful constituents) represents a low-effort solution, but raises the carbon-foot print of the power plant.

    Nevertheless, venting still represents a challenge to \ac{DSC} geothermal power plants, as the vapour is typically expanded to sub-atmospheric conditions, see Section~\ref{sec:thermo_opt_DSCresults}, and as such the \ac{NCG} must be re-pressurised to atmospheric pressure to be vented. Binary \ac{ORC}s do not face this challenge as the pressure losses within the \ac{PHE} are comparatively low, and so the \ac{NCG} remains about atmospheric pressure and can be vented without issue.

    \subsubsection{Plant Configurations}
        The \ac{DSC} plant used for the thermodynamic, see Section~\ref{sec:thermo_opt_DSC_config} is modified to include an \ac{NCG} separator following the condenser, as well as compressor and a pump to repressurise the \ac{NCG} and condensate streams to atmospheric pressure (if required, Figure~\ref{fig:prosim_NCG_ventingDSC}.

        \begin{figure}[H]
            \centering
            \resizebox{\linewidth}{!}{\begin{tikzpicture}
    % draw equipment
    \pic (producer) at (0,0) {producer};
    \pic (valve) at ($(producer-anchor) + (1, 2.5)$) {lamination valve};
    \pic[scale=0.4] (separator) at ($(valve-anchor) + (1.5, 0)$) {gas-liquid separator};
    \pic (turbine) at ($(separator-anchor) + (2, 1.5)$) {turbine_gen};

    \pic[rotate=90] (condenser) at ($(turbine-outlet bottom) + (3, -1)$) {heat exchanger biphase};

    \pic[scale=0.4] (NCGsep) at ($(condenser-shell bottom) + (2,0)$) {gas-liquid separator};
    \pic (NCGcomp) at ($(NCGsep-gas outlet) + (1, 1)$) {compressor};

    \pic (condpump) at ($(condenser-anchor) + (0, -2.2)$) {centrifugal pump};
    \pic (brinepump) at ($(separator-liquid outlet) + (+1, -1.7)$) {centrifugal pump};
    \pic (injector) at ($0.5*(brinepump-anchor) + 0.5*(condpump-anchor)$) {injector};
    \pic[rotate=180] (joint) at ($(injector-anchor) + (0, 1)$) {valve triple=main};
    
    % % draw connectors
    \draw[main stream] (producer-top) |- (valve-inlet);
    \draw[main stream] (valve-outlet) -- (separator-inlet left);
    \draw[main stream] (separator-gas outlet) |- (turbine-inlet top);

    \draw[main stream] (turbine-outlet bottom) |- (condenser-shell top);
    \draw[main stream] (condenser-shell bottom) -- (NCGsep-inlet left);
    \draw[main stream] (NCGsep-liquid outlet) |- (condpump-anchor);
    \draw[main stream] (NCGsep-gas outlet) |- (NCGcomp-inlet bottom);
    \draw[main stream] (NCGcomp-outlet top) -- ($(NCGcomp-outlet top) + (1.5, 0)$);

    \draw[main stream] (condenser-pipes top) -- ++ (0, 1);
    \draw[main stream] ($(condenser-pipes bottom) + (0, 0.5)$) -- (condenser-pipes bottom);

    \draw[main stream] (condpump-top) |- (joint-left);
    \draw[main stream] (separator-liquid outlet) |- (brinepump-anchor);
    \draw[main stream] (brinepump-top) |- (joint-right);
    \draw[main stream] (joint-top) -- (injector-top);

    % % draw labels
    \node[below] at (producer-bottom) {Producer};
    \node[above, align=center] at (valve-top) {Expansion\\ Valve};
    \node[right, align=left] at (separator-inlet right) {Steam \\ Separator};
    \node[above] at (turbine-top) {Turbine \& Generator};
    \node[right] at (NCGsep-inlet right) {NCG Separator.};
    \node[above] at ($(NCGcomp-anchor) + (0, 0.5)$) {NCG Compressor};

    \node[below] at (condenser-shell left) {Condenser};
    \node[above, align=left] at ($(condenser-pipes top) + (0, 1)$) {Coolant out};
    \node[above right, align=left] at ($(condenser-pipes bottom) + (-0.3, 0.5)$) {Coolant in};

    \node[below, align=center] at (condpump-bottom) {Condensate\\Pump};
    \node[below, align=center] at (brinepump-bottom) {Brine\\Pump};
    \node[below] at (injector-bottom) {Injector};
    \node[below] at ($(NCGcomp-outlet top) + (1.5, 0)$) {To atmosphere};
    
\end{tikzpicture}}
            \caption{Single flash \ac{DSC} geothermal power plant with \ac{NCG} re-pressurisation to atmospheric pressure to allow venting.}
            \label{fig:prosim_NCG_ventingDSC}
        \end{figure}

        Besides a \ac{NCG} separator, the binary \ac{ORC} does not required any modifications to facilitate the venting of \ac{NCG} to the atmosphere, Figure~\ref{fig:prosim_NCG_ventingORC}.

        \begin{figure}[H]
            \centering
            \begin{tikzpicture}
    % draw equipment
    \pic (producer) at (0,0) {producer};
    \pic (injector) at (2.0,0) {injector};
    
    \pic[rotate=180] (preheater) at ($(injector-anchor) + (2.225, 3)$) {heat exchanger biphase};    
    \pic[rotate=180] (evaporator) at ($(preheater-anchor) + (0, 1.5)$) {heat exchanger biphase};
    \pic[rotate=180] (superheater) at ($(evaporator-anchor) + (0, 1.5)$) {heat exchanger biphase};
    
    \pic (turbine) at ($(superheater-anchor) + (5, 0.5)$) {turbine_gen};

    % \pic[yscale=-1, xscale=-1] (condenser) at ($(turbine-anchor) + (-1, -4.5)$) {condenser};
    \pic (condenser) at ($(turbine-outlet bottom) + (0, -2.2)$) {heat exchanger biphase};
    
    \pic (pump) at ($(condenser-shell bottom) + (-3, -1.8)$) {centrifugal pump};

    \pic[scale=0.4] (NCGsep) at ($(preheater-pipes top) + (-1.8, 0)$) {gas-liquid separator};
    
    % % draw connectors
    \draw[main stream] (producer-top) |- (superheater-pipes bottom);
    \draw[main stream] (superheater-pipes top) 
        -- ($(superheater-pipes top) + (-0.5, 0)$) 
        |- (evaporator-pipes bottom);
    \draw[main stream] (evaporator-pipes top) 
        -- ($(evaporator-pipes top) + (-0.5, 0)$) 
        |- (preheater-pipes bottom);
    \draw[main stream] (preheater-pipes top) 
        -| (NCGsep-inlet right);
    \draw[main stream] (NCGsep-liquid outlet)
        -- (injector-top);
    \draw[main stream] (NCGsep-gas outlet)
        -- ($(NCGsep-gas outlet) + (0,1)$);

    \draw[main stream] (preheater-shell bottom) -- (evaporator-shell top);
    \draw[main stream] (evaporator-shell bottom) -- (superheater-shell top);
    \draw[main stream] (superheater-shell bottom) |- (turbine-inlet top);

    \draw[main stream] (turbine-outlet bottom) -- (condenser-shell top);
    \draw[main stream] (condenser-pipes top) -- ++ (1, 0);
    \draw[main stream] ($(condenser-pipes bottom) + (1, 0)$) -- (condenser-pipes bottom);
    
    \draw[main stream] (condenser-shell bottom) |- (pump-anchor);
    \draw[main stream] (pump-top) -| (preheater-shell top);
    
    % % draw labels
    \node[below] at (producer-bottom) {Producer};
    \node[right] at (preheater-shell left) {Pre-Heater};
    \node[right] at (evaporator-shell left) {Evaporator};
    \node[right] at (superheater-shell left) {Super-Heater};
    \node[above] at (turbine-top) {Turbine \& Generator};
    \node[left, align=left] at (condenser-shell left) {Condenser};
    \node[above right, align=left] at ($(condenser-pipes top) + (1, 0)$) {Coolant out};
    \node[below right, align=left] at ($(condenser-pipes bottom) + (1, 0)$) {Coolant in};
    
    \node[below, align=center] at (pump-bottom) {Circulation \\ Pump};
    \node[below] at (injector-bottom) {Injector};
    \node[left, align=left] at (NCGsep-inlet left) {NCG\\Sep.};
    \node[above, align=center] at ($(NCGsep-gas outlet) + (0, 1)$) {To\\atmosphere};
\end{tikzpicture}
            \caption{Binary \ac{ORC} geothermal power plant with \ac{NCG} venting to atmosphere.}
            \label{fig:prosim_NCG_ventingORC}
        \end{figure}

    \subsubsection{Boundary Conditions and Assumptions}
        Unlike the previous investigations, the geofluid is assumed to be a mixture of water and carbon dioxide, ranging between pure water and up to \qty{12}{\mol\percent\of{\ce{CO2}}} (corresponding to \qty{30}{mass\percent\of{\ce{CO2}}}). The working fluid used in the binary \ac{ORC} is n-Butane.

        The geofluid inlet temperature and vapour quality were chosen to be \qty{473}{\K} (\qty{200}{\degreeCelsius}) and \qty{20}{\percent} respectively, such that for the pure water case the binary \ac{ORC} and single flash \ac{DSC} have a similar net electrical power and specific cost, Figure~\ref{fig:prosim_NCG_nButane_Wnet_specCost}. 

        \begin{figure}[H]
            \centering
            \input{Content/ProSim/NCGHandling/Plots/nButane_Wnet_specCost}
            \caption[The net electrical power and specific plant cost of a thermodynamically optimised binary \ac{ORC} using n-Butane as the working fluid and a single flash \ac{DSC} geothermal power plant.]{The net electrical power (left) and specific plant cost (right) of a thermodynamically optimised binary \ac{ORC} using n-Butane as the working fluid and a thermodynamically optimised single flash \ac{DSC} geothermal power plant, Section~\ref{sec:thermo_comp}.}
            \label{fig:prosim_NCG_nButane_Wnet_specCost}
        \end{figure}

        The geofluid mass flow rate is scaled to ensure a constant heat flow rate into the power plant (equal to that of pure water at the aforementioned conditions), Equation~\ref{eq:NCG_mrate_scaling}.

        \begin{align}
            \Dot{m}_{geo} = \Dot{m}_{\ce{H2O}} * \frac{h_{\ce{H2O}}^{in}-h_{\ce{H2O}}^{ref}}{h_{geo}^{in}-h_{geo}^{ref}} \label{eq:NCG_mrate_scaling}
        \end{align}

        With this in mind, to minimise the scaling of the geofluid mass rate and to ensure that the geofluid compositions remain \emph{similar}, for this study, the vapour quality of the geofluid is defined solely based on the fraction of the total amount of water that is in the vapour phase, Equation~\ref{eq:NCG_vapqual_def}.

        \begin{align}
            x_{geo} = \frac{\Dot{n}_{\ce{H2O}}^{vap}}{\Dot{n}_{\ce{H2O}}^{vap}+\Dot{n}_{\ce{H2O}}^{liq}} \label{eq:NCG_vapqual_def}
        \end{align}

        \begin{table}[H]
            \centering
            \caption{The boundary conditions used for the single flash \ac{DSC} and the binary \ac{ORC} geothermal power plants.}
            \label{table:NCG_Venting_BCs}
            \begin{tabular}{| c | c |}
    \hline
    \rowcolor{bluepoli!40} % comment this line to remove the color
    \textbf{Condition} & \textbf{Values} \T\B \\
    \hline \hline
    \ce{CO2} Content & \qty{0}{\mol\percent}\(\leq z_{NCG,\;in}\leq\)\qty{15}{\mol\percent} \T\B \\
    Inlet Temperature & \qty{473}{\K} \T\B \\
    Inlet Steam Quality & \qty{25}{\percent} \T\B \\
    Inlet Pressure & calculated \T\B \\
    NCG Outlet Pressure &  \qty{1.1}{\bar} \T\B \\
    Brine/Condensate Outlet Pressure &  \(P_{brine/cond,\;out} = P_{brine/condo,\;in}\) \T\B \\
    Working Fluid &  n-Butane \T\B \\
    \hline
\end{tabular}        
        \end{table}
    
    \subsubsection{Optimisation Configuration}
        \label{sec:NCG_Venting_opt_config}
        The two power plants are thermodynamically optimised (i.e. to maximise their net electrical power) using the same optimisation configuration as previously used in Section~\ref{sec:thermo_comp}, see Table~\ref{table:NCG_Venting_opt_config}.

        \begin{table}[H]
            \centering
            \caption{The optimisation parameters used for the single flash \ac{DSC} and the binary \ac{ORC} geothermal power plants.}
            \label{table:NCG_Venting_opt_config}
            \input{Content/ProSim/NCGHandling/DataTables/OptConfig_Venting}        
        \end{table}

    \subsubsection{Results}
        The performance of the binary \ac{ORC} is unaffected by the presence of \ac{NCG} within the geofluid, Figure~\ref{fig:prosim_NCG_Ventilation_Wnet}. This is a result of the heat flow being the same as in the pure water case, and with the pinch-point being located at the pre-heater inlet (on the working fluid side) the operating conditions of the \ac{ORC} are unchanged. Moreover, as there are no pressure losses in the \ac{PHE}, the geofluid exits the \ac{PHE} at a pressure equal to the inlet pressure, which is well above atmospheric pressure, Figure~\ref{fig:prosim_NCG_Ventilation_Power_Breakdown}.

        \begin{figure}[H]
            \centering
            % This file was created with tikzplotlib v0.10.1.
\begin{tikzpicture}

\definecolor{burlywood253194140}{RGB}{253,194,140}
\definecolor{chocolate2369815}{RGB}{236,98,15}
\definecolor{darkgray176}{RGB}{176,176,176}
\definecolor{lightblue182212233}{RGB}{182,212,233}
\definecolor{lightgray204}{RGB}{204,204,204}
\definecolor{midnightblue848107}{RGB}{8,48,107}
\definecolor{saddlebrown127394}{RGB}{127,39,4}
\definecolor{steelblue59139194}{RGB}{59,139,194}

\begin{groupplot}[group style={group size=1 by 2}]
\nextgroupplot[
legend cell align={left},
legend style={
  fill opacity=0.8,
  draw opacity=1,
  text opacity=1,
  at={(0.97,0.03)},
  anchor=south east,
  draw=lightgray204
},
tick align=outside,
tick pos=left,
title={n-Butane},
x grid style={darkgray176},
xlabel={Drilling Cost/\unit{\mega\USD\of{2023}}},
xmin=0, xmax=63,
xtick style={color=black},
y grid style={darkgray176},
ylabel={Net electrical power/\unit{\mega\watt}},
ymin=0, ymax=9,
ytick style={color=black}
]
\addplot [semithick, burlywood253194140]
table {%
0 6.52143371948183
1 6.52143371948183
2 6.52143371948183
4 6.52143371948183
8 6.52143371948183
12 6.52143371948183
16 6.52143371948183
20 6.52143371948183
25 6.52143371948183
30 6.52143371948183
35 6.52143371948183
40 6.52143371948183
50 6.52143371948183
60 6.52143371948183
};
\addlegendentry{Thermodynamic Opt.}
\addplot [semithick, saddlebrown127394]
table {%
0 3.35312380103824
1 3.35312380103824
2 3.35312380103824
4 3.35312380103824
8 3.35312380103824
12 3.35312380103824
16 3.35312380103824
20 3.35312380103824
25 3.35312380103824
30 3.35312380103824
35 3.35312380103824
40 3.35312380103824
50 3.35312380103824
60 3.35312380103824
};
\addlegendentry{Techno-economic Opt. (excluding drilling costs)}
\addplot [semithick, chocolate2369815]
table {%
0 3.41586816872176
1 3.45484599362514
2 3.59827794491314
4 5.44747672618843
8 5.71741855543081
12 6.13905459187174
16 6.22879829708411
20 6.33475894364317
25 6.35406610908275
30 6.39668208743175
35 6.4680377937982
40 6.40986537612564
50 6.50452302914855
60 6.47623187635999
};
\addlegendentry{Techno-economic Opt. (including drilling costs)}

\nextgroupplot[
legend cell align={left},
legend style={fill opacity=0.8, draw opacity=1, text opacity=1, draw=lightgray204},
tick align=outside,
tick pos=left,
title={Cyclopentane},
x grid style={darkgray176},
xlabel={Drilling Cost/\unit{\mega\USD\of{2023}}},
xmin=0, xmax=63,
xtick style={color=black},
y grid style={darkgray176},
ylabel={Net electrical power/\unit{\mega\watt}},
ymin=0, ymax=9,
ytick style={color=black}
]
\addplot [semithick, lightblue182212233]
table {%
0 8.67312917722771
1 8.67312917722771
2 8.67312917722771
4 8.67312917722771
8 8.67312917722771
12 8.67312917722771
16 8.67312917722771
20 8.67312917722771
25 8.67312917722771
30 8.67312917722771
35 8.67312917722771
40 8.67312917722771
50 8.67312917722771
60 8.67312917722771
};
\addlegendentry{Thermodynamic Opt.}
\addplot [semithick, midnightblue848107]
table {%
0 3.24768016868321
1 3.24768016868321
2 3.24768016868321
4 3.24768016868321
8 3.24768016868321
12 3.24768016868321
16 3.24768016868321
20 3.24768016868321
25 3.24768016868321
30 3.24768016868321
35 3.24768016868321
40 3.24768016868321
50 3.24768016868321
60 3.24768016868321
};
\addlegendentry{Techno-economic Opt. (excluding drilling costs)}
\addplot [semithick, steelblue59139194]
table {%
0 3.14095734400058
1 6.21228907960342
2 6.29436173932615
4 6.55325520823824
8 6.84360915751591
12 7.168577206856
16 7.28400752660467
20 7.26659096414527
25 7.38675295828168
30 7.51110970957674
35 7.43616663970955
40 7.57638400329378
50 7.75201144638119
60 8.36888127426749
};
\addlegendentry{Techno-economic Opt. (including drilling costs)}
\end{groupplot}

\end{tikzpicture}

            \caption[The net electrical power of a thermodynamically optimised binary \ac{ORC} and single flash \ac{DSC} geothermal power plants venting \ac{NCG} to atmosphere.]{The net electrical power of thermodynamically optimised binary \ac{ORC} (using n-Butane as the working fluid) and single flash \ac{DSC} geothermal power plants with \ac{NCG} venting to atmosphere.}
            \label{fig:prosim_NCG_Ventilation_Wnet}
        \end{figure}

        On the other hand, the net electrical power of the single flash \ac{DSC} initiall decreases for geofluid \ce{CO2} contents up to about \qty{6}{\percent}, but then exceeds the net power observed with pure water as the geofluid, Figure~\ref{fig:prosim_NCG_Ventilation_Wnet}. The increase in net power can be attributed to the increase in geofluid inlet pressure, which allows for larger expansion ratios and higher turbine power, Figures~\ref{fig:prosim_NCG_Ventilation_Power_Breakdown} and ~\ref{fig:prosim_NCG_Ventilation_OPs}. Although \ac{NCG} handling power could be expected to increase inline with \ce{CO2} content, reducing the degree of flashing, thus lowering the vapour mass rate, and raising the condensation pressure, thus lowering re-pressurisation power requirements, help to minimise this effect.

        \begin{figure}[H]
            \centering
            % This file was created with tikzplotlib v0.10.1.
\begin{tikzpicture}

\definecolor{crimson2143940}{RGB}{214,39,40}
\definecolor{darkgray176}{RGB}{176,176,176}
\definecolor{darkorange25512714}{RGB}{255,127,14}
\definecolor{forestgreen4416044}{RGB}{44,160,44}
\definecolor{lightgray204}{RGB}{204,204,204}
\definecolor{steelblue31119180}{RGB}{31,119,180}

\begin{groupplot}[
    group style={
        group size=2 by 1, 
        vertical sep=2.5cm, 
        horizontal sep=2cm},
    height=6cm, 
    width=7cm, 
]
\nextgroupplot[
legend cell align={left},
legend style={
  fill opacity=0.8,
  draw opacity=1,
  text opacity=1,
  at={(1.15, -0.35)},
  anchor=north,
  draw=lightgray204
},
legend columns=-1,
tick align=outside,
tick pos=left,
title={Binary ORC},
x grid style={darkgray176},
xlabel={Geofluid \ce{CO2} content/\unit{\mol\percent}},
xmin=0, xmax=16,
xtick style={color=black},
xtick distance=4,
y grid style={darkgray176},
ylabel={Net electrical power /\unit{\mega\watt}},
ymin=-2, ymax=10,
ytick style={color=black},
ytick distance=2,
]
\addplot [semithick, black, dotted, forget plot]
table {%
0 0
15 0
};
\addplot [semithick, steelblue31119180]
table {%
1 7.01448533671156
2 7.09927339965173
3 7.1464152434167
4 7.07749065024379
5 7.01679759619276
7 7.11050265098926
9 7.11962449744598
11 7.1600886565322
13 7.09670547840913
15 7.09712060873134
};
\addlegendentry{Net}
\addplot [semithick, darkorange25512714]
table {%
1 7.76806090357288
2 7.87592091391509
3 8.0364303101519
4 7.83800597707966
5 7.80280936226836
7 7.87193917422286
9 7.9096332957137
11 8.00038818344087
13 7.9509565715266
15 8.17994924596017
};
\addlegendentry{Cycle}
\addplot [semithick, forestgreen4416044]
table {%
1 -0.753575566861316
2 -0.776647514263361
3 -0.890015066735204
4 -0.760515326835868
5 -0.786011766075599
7 -0.761436523233596
9 -0.790008798267723
11 -0.840299526908673
13 -0.854251093117472
15 -1.08282863722884
};
\addlegendentry{Parasitic}
\addplot [semithick, crimson2143940]
table {%
1 0
2 0
3 0
4 0
5 0
7 0
9 0
11 0
13 0
15 0
};
\addlegendentry{NCG Handling}

\nextgroupplot[
tick align=outside,
tick pos=left,
title={Single flash DSC},
x grid style={darkgray176},
xlabel={Geofluid \ce{CO2} content/\unit{\mol\percent}},
xmin=0, xmax=16,
xtick style={color=black},
xtick distance=4,
y grid style={darkgray176},
ylabel={Net electrical power /\unit{\mega\watt}},
ymin=-2, ymax=10,
ytick style={color=black},
ytick distance=2
]
\addplot [semithick, black, dotted]
table {%
0 0
15 0
};
\addplot [semithick, steelblue31119180]
table {%
1 6.05622907082415
2 5.82162034028068
3 5.7796192166751
4 5.81472582415529
5 5.91325095696919
7 6.19689216122147
9 6.56577446488
11 7.0369710587965
13 7.57368260572439
15 8.0978373660655
};
\addplot [semithick, darkorange25512714]
table {%
1 6.90032913124049
2 6.51557306121636
3 6.74939354239069
4 6.70268719099458
5 6.83401811219064
7 7.2990108939922
9 7.79568408471777
11 7.70341961475493
13 7.78644734762378
15 8.97712779560728
};
\addplot [semithick, forestgreen4416044]
table {%
1 -0.844100060416344
2 -0.693952720935684
3 -0.969774325715595
4 -0.887961366839294
5 -0.920767155221454
7 -1.10211873277074
9 -1.22990961983777
11 -0.666448555958437
13 -0.212764741899389
15 -0.879290429541779
};
\addplot [semithick, crimson2143940]
table {%
1 -0.650281713579303
2 -0.553332372935146
3 -0.834544053797203
4 -0.769052461656707
5 -0.813423035442634
7 -0.998876384547606
9 -1.12901589347039
11 -0.581156214843795
13 -0.134697073760764
15 -0.791714449304675
};
\end{groupplot}

\end{tikzpicture}

            \caption{Breakdown of the net electrical power for thermodynamically optimised binary \ac{ORC} (using n-Butane as the working fluid) (left) and single flash \ac{DSC} (right) geothermal power plants with \ac{NCG} venting to atmosphere.}
            \label{fig:prosim_NCG_Ventilation_Power_Breakdown}
        \end{figure}

        \begin{figure}[H]
            \centering
            % This file was created with tikzplotlib v0.10.1.
\begin{tikzpicture}

\definecolor{crimson2143940}{RGB}{214,39,40}
\definecolor{darkgray176}{RGB}{176,176,176}
\definecolor{darkorange25512714}{RGB}{255,127,14}
\definecolor{forestgreen4416044}{RGB}{44,160,44}
\definecolor{lightgray204}{RGB}{204,204,204}
\definecolor{steelblue31119180}{RGB}{31,119,180}

\begin{axis}[
legend cell align={left},
legend style={
  fill opacity=0.8,
  draw opacity=1,
  text opacity=1,
  at={(0.97,0.03)},
  anchor=south east,
  draw=lightgray204
},
log basis y={10},
tick align=outside,
tick pos=left,
x grid style={darkgray176},
xlabel={Geofluid \ce{CO2} content/\unit{\mol\percent}},
xmin=0.3, xmax=15.7,
xtick style={color=black},
y grid style={darkgray176},
ylabel={Pressure/\unit{\bar}},
ymin=0.142832233002548, ymax=34.1714024287807,
ymode=log,
ytick style={color=black},
ytick={0.01,0.1,1,10,100,1000},
yticklabels={
  \(\displaystyle {10^{-2}}\),
  \(\displaystyle {10^{-1}}\),
  \(\displaystyle {10^{0}}\),
  \(\displaystyle {10^{1}}\),
  \(\displaystyle {10^{2}}\),
  \(\displaystyle {10^{3}}\)
}
]
\addplot [semithick, steelblue31119180]
table {%
1 15.6630536688748
2 16.3198841873615
3 16.9938113190951
4 17.6848295204466
5 18.3935694555122
7 19.8673757145131
9 21.4215479157084
11 23.0630210435187
13 24.799575491809
15 26.6399783604382
};
\addlegendentry{Inlet}
\addplot [semithick, darkorange25512714]
table {%
1 8.91280006879514
2 11.619514683976
3 12.742153323404
4 15.2413970634192
5 17.0756406138661
7 18.4471428119401
9 20.4355582975179
11 22.6041335886886
13 23.6051569686054
15 26.2202186408983
};
\addlegendentry{Flash Sep.}
\addplot [semithick, forestgreen4416044]
table {%
1 0.183212525464347
2 0.298665812602474
3 0.3069901933328
4 0.385668843640606
5 0.42807608145105
7 0.465726351278403
9 0.503653005999805
11 0.769740739905919
13 1.02082952300736
15 0.777502984609375
};
\addlegendentry{Condenser}
\addplot [semithick, crimson2143940]
table {%
1 1.1
2 1.1
3 1.1
4 1.1
5 1.1
7 1.1
9 1.1
11 1.1
13 1.1
15 1.1
};
\addlegendentry{Outlet}
\end{axis}

\end{tikzpicture}

            \caption{The pressures of the geofluid at different points in the power plant.}
            \label{fig:prosim_NCG_Ventilation_OPs}
        \end{figure}

        At higher geofluid \ce{CO2} contents (above \qty{9}{\percent}, the optimisation problem becomes degenerate as many combinations of process variables yield similar values for the objective function. For instance, increasing the degree of flashing and raising the condensation pressure has the same effect as reducing the degree of flashing and lowering the condensation pressure. While this does not significantly alter the maximum net electrical power found by the optimisation, it has a considerable impact on the cost of the power plant, Figure~\ref{fig:prosim_NCG_Ventilation_Cost}. This is because, in one case virtually no \ac{NCG} re-pressurisation is required, while in the other it is required and represents a considerable cost item, Figure~\ref{fig:prosim_NCG_Ventilation_CostBreakdown}.

        \begin{figure}[H]
            \centering
            % This file was created with tikzplotlib v0.10.1.
\begin{tikzpicture}

\definecolor{darkgray176}{RGB}{176,176,176}
\definecolor{darkorange25512714}{RGB}{255,127,14}
\definecolor{lightgray204}{RGB}{204,204,204}
\definecolor{steelblue31119180}{RGB}{31,119,180}

\begin{axis}[
legend cell align={left},
legend style={fill opacity=0.8, draw opacity=1, text opacity=1, draw=lightgray204},
tick align=outside,
tick pos=left,
x grid style={darkgray176},
xlabel={Geofluid \ce{CO2} content/\unit{\mol\percent}},
xmin=0, xmax=16,
xtick style={color=black},
y grid style={darkgray176},
ylabel={Plant Cost/\unit{\mega\USD\of{2023}}},
ymin=0, ymax=100,
ytick style={color=black}
]
\addplot [semithick, steelblue31119180]
table {%
1 19.1926875017161
2 21.6943139984392
3 24.1558293022766
4 26.1475096140439
5 27.9974960372848
7 31.0536811733364
9 33.3720702989873
11 36.1093680763004
13 38.0355481783677
15 39.7521298098009
};
\addlegendentry{Binary ORC}
\addplot [semithick, darkorange25512714]
table {%
1 29.6515541903645
2 33.671382180979
3 38.8981961750247
4 44.5477762308278
5 52.9838408653663
7 54.8933994932988
9 59.1315906110587
11 67.6388936909268
13 72.5615993918084
15 76.8767388176412
};
\addlegendentry{Single Flash DSC}
\end{axis}

\end{tikzpicture}

            \caption{The total plant cost of thermodynamically optimised binary \ac{ORC} (using n-Butane as the working fluid) and single flash \ac{DSC} geothermal power plants with \ac{NCG} venting to atmosphere.}
            \label{fig:prosim_NCG_Ventilation_Cost}
        \end{figure}

        \begin{figure}[H]
            \centering
            \input{Content/ProSim/NCGHandling/Plots/Ventilation/Cost_Breakdown}
            \caption{The cost breakdown of thermodynamically optimised binary \ac{ORC} (using n-Butane as the working fluid) and single flash \ac{DSC} geothermal power plants with \ac{NCG} venting to atmosphere.}
            \label{fig:prosim_NCG_Ventilation_CostBreakdown}
        \end{figure}

\subsection{NCG Re-injection}
    In previous studies the geofluid was discharged at a pressure equal to the pressure at the power plant inlet, under the assumption that this would be sufficient to re-inject the geofluid into the reservoir. Even if higher pressures were required, as the geofluid was assumed to be as pure water and in its liquid state at the outlet, the re-pressurisation to higher pressure would have been possible with only small reductions in net electrical power and small capital cost increases.

    For geofluids comprised of water and \ac{NCG}, the existence of an \ac{NCG}-rich vapour phase is all but guaranteed, which poses a significant challenge for re-injection. Due to the low density of \ac{NCG}, the pressures required for re-injection are about one to two orders of magnitude higher, and owing to the high compressibility of gases, the power required for compression is significant.

    \begin{notes}{Note}
        Owing to the large differences in density between the liquid and vapour phases, injection of a two-phase fluid is likely to be technically challenging, if not impossible.

        As the vapour has a lower density than the liquid, it has a natural tendency to rise in the wellbore (i.e. buoyancy). Hence, the liquid must exert a significant drag on the gas phase to prevent the vapour phase from moving upwards, towards the surface. This may be possible when the fluid is predominantly liquid (e.g. bubble flow) or predominantly vapour (i.e. mist flow), where the effect of buoyancy is reduced due to the significant drag between the bulk fluid and the bubbles/droplets. 
        
        For this reason, injection of single phase fluids is preferred. Where two-phase fluids are to be injected (e.g. brine and \ac{NCG}), this can be achieved by means of a secondary tubing within the wellbore, with the one fluid flowing through the secondary tubing, and the other in the annular space between the primary and secondary tubing.  
    \end{notes}

    \begin{notes}{Note}
        The injection of liquids containing dissolved gases (e.g. brine containing dissolved \ac{NCG} is also not without risks. For instance, loss in wellhead pressure or increase in fluid temperature (i.e. during shut-in conditions), may result in previously dissolved gases coming out of solution. The presence of gas as well as its upward movement due to buoyancy, lightens the fluid in the wellbore resulting in a reducing in pressure all along the wellbore, causing more gas to come out of solution.

        This run cascading effect could simply result in the accumulation of gas near the wellhead (if in shut-in conditions) or in the worst-case lead to a blow-out event.
    \end{notes}

    \subsubsection{Injection Pressure and Depth of Injection}
        \label{sec:prosim_NCGhandling_Pinj}
        To determine the maximum depth of injection, given the wellhead conditions of the geothermal brine and the \ac{NCG} (i.e. temperature and pressure) it is useful to consider the static pressure profiles within the wellbore.

        The static pressure of the geothermal brine at a depth \(z\) can be estimated using Equation~\ref{eq:brine_static_grad}, assuming that the brine is pure water and in-compressible. Where, \(\rho_{brine}\) is the density of the brine, taken to be \qty{1000}{\kg\per\cubic\m} (i.e. independent of temperature and pressure), \(g\) the acceleration due to gravity and \(z\) the \ac{TVD} along the wellbore. 

        \begin{align}
            P_{brine}^{static}(z) = \rho_{brine} \cdot g \cdot z  \label{eq:brine_static_grad}
        \end{align}

        For the \ac{NCG}, the static pressure was determined by assuming negligible heat transfer between the \ac{NCG}, the brine and the surrounding formation along the wellbore, i.e. constant enthalpy. The enthalpy of the \ac{NCG} at the wellhead is defined by the wellhead temperature and pressure. The static pressure along the wellbore is then determined by discretising the wellbore into \(N\) nodes and iteratively solving Equations~\ref{eq:ncg_static_grad} and \ref{eq:ncg_average_density}. Here, \(P_i\) and \(P_{i+1}\) are the static pressure at node \(i\) and \(i+1\) respectively, \(g\) is the acceleration due to gravity, \(\Delta z\) is the difference in vertical depth between nodes \(i+1\) and \(i\), and \(\rho_{av}\) is the average density between nodes \(i\) and \(i+1\). The inlet enthalpy and density are calculated using \emph{CoolProp} treating the \ac{NCG} as pure \ce{CO2}.

        \begin{align}
            P_{i+1} = P_i + \rho_{av} \cdot g \cdot \Delta z \label{eq:ncg_static_grad}
        \end{align}

        \begin{align}
            \rho_{av} = \frac{\rho (h, P_i) + \rho (h, P_{i+1})}{2} \label{eq:ncg_average_density}
        \end{align}

        The depth at which the pressure of the brine and the pressure of the \ac{NCG} are equal, is the maximum depth of injection. In reality, frictional losses, as well as heat transfer effects between the surrounding rock, the geothermal brine and the \ac{NCG} result in injection at shallower depths.

        The static pressure profiles for brine and \ac{NCG} along a wellbore were calculated for a range of \ac{NCG} injection pressures, Figure~\ref{fig:prosim_NCG_Reinjection_PvD_by_Pinj}, assuming the brine to be pure water, a brine injection temperature and pressure (at the wellhead) of \qty{298}{\K} and \qty{1}{\bar} respectively, the \ac{NCG} is pure \ce{CO2}, and a \ac{NCG} injection temperature (at the wellhead) of \qty{298}{\K}. 
        
        \begin{figure}[H]
            \centering
            % This file was created with tikzplotlib v0.10.1.
\begin{tikzpicture}

\definecolor{crimson2143940}{RGB}{214,39,40}
\definecolor{darkgray176}{RGB}{176,176,176}
\definecolor{darkorange25512714}{RGB}{255,127,14}
\definecolor{darkturquoise23190207}{RGB}{23,190,207}
\definecolor{forestgreen4416044}{RGB}{44,160,44}
\definecolor{goldenrod18818934}{RGB}{188,189,34}
\definecolor{gray127}{RGB}{127,127,127}
\definecolor{lightgray204}{RGB}{204,204,204}
\definecolor{mediumpurple148103189}{RGB}{148,103,189}
\definecolor{orchid227119194}{RGB}{227,119,194}
\definecolor{sienna1408675}{RGB}{140,86,75}
\definecolor{steelblue31119180}{RGB}{31,119,180}

\begin{axis}[y dir = {reverse},
                 xlabel = {Pressure/\unit{\bar}},
                 ylabel = {Depth/\unit{\m}},
                 xmin = 0,
                 xmax = 450,
                 ymin = 0,
                 ymax = 4000,
                 axis x line=top,
                 legend cell align={left},
legend style={fill opacity=0.8, draw opacity=1, text opacity=1, draw=lightgray204, at={(1.03, 0.5)}, anchor=west}, width=10cm, height=6.5cm,]
\addplot [semithick, steelblue31119180]
table {%
10 0
10.4905 5
10.981 10
11.4715 15
11.962 20
12.4525 25
12.943 30
13.4335 35
13.924 40
14.4145 45
14.905 50
15.3955 55
15.886 60
16.3765 65
16.867 70
17.3575 75
17.848 80
18.3385 85
18.829 90
19.3195 95
19.81 100
20.3005 105
20.791 110
21.2815 115
21.772 120
22.2625 125
22.753 130
23.2435 135
23.734 140
24.2245 145
24.715 150
25.2055 155
25.696 160
26.1865 165
26.677 170
27.1675 175
27.658 180
28.1485 185
28.639 190
29.1295 195
29.62 200
30.1105 205
30.601 210
31.0915 215
31.582 220
32.0725 225
32.563 230
33.0535 235
33.544 240
34.0345 245
34.525 250
35.0155 255
35.506 260
35.9965 265
36.487 270
36.9775 275
37.468 280
37.9585 285
38.449 290
38.9395 295
39.43 300
39.9205 305
40.411 310
40.9015 315
41.392 320
41.8825 325
42.373 330
42.8635 335
43.354 340
43.8445 345
44.335 350
44.8255 355
45.316 360
45.8065 365
46.297 370
46.7875 375
47.278 380
47.7685 385
48.259 390
48.7495 395
49.24 400
49.7305 405
50.221 410
50.7115 415
51.202 420
51.6925 425
52.183 430
52.6735 435
53.164 440
53.6545 445
54.145 450
54.6355 455
55.126 460
55.6165 465
56.107 470
56.5975 475
57.088 480
57.5785 485
58.069 490
58.5595 495
59.05 500
59.5405 505
60.031 510
60.5215 515
61.012 520
61.5025 525
61.993 530
62.4835 535
62.974 540
63.4645 545
63.955 550
64.4455 555
64.936 560
65.4265 565
65.917 570
66.4075 575
66.898 580
67.3885 585
67.879 590
68.3695 595
68.86 600
69.3505 605
69.841 610
70.3315 615
70.822 620
71.3125 625
71.803 630
72.2935 635
72.784 640
73.2745 645
73.765 650
74.2555 655
74.746 660
75.2365 665
75.727 670
76.2175 675
76.708 680
77.1985 685
77.689 690
78.1795 695
78.67 700
79.1605 705
79.651 710
80.1415 715
80.632 720
81.1225 725
81.613 730
82.1035 735
82.594 740
83.0845 745
83.575 750
84.0655 755
84.556 760
85.0465 765
85.537 770
86.0275 775
86.518 780
87.0085 785
87.499 790
87.9895 795
88.48 800
88.9705 805
89.461 810
89.9515 815
90.442 820
90.9325 825
91.423 830
91.9135 835
92.404 840
92.8945 845
93.385 850
93.8755 855
94.366 860
94.8565 865
95.347 870
95.8375 875
96.328 880
96.8185 885
97.309 890
97.7995 895
98.29 900
98.7805 905
99.271 910
99.7615 915
100.252 920
100.7425 925
101.233 930
101.7235 935
102.214 940
102.7045 945
103.195 950
103.6855 955
104.176 960
104.6665 965
105.157 970
105.6475 975
106.138 980
106.6285 985
107.119 990
107.6095 995
108.1 1000
108.5905 1005
109.081 1010
109.5715 1015
110.062 1020
110.5525 1025
111.043 1030
111.5335 1035
112.024 1040
112.5145 1045
113.005 1050
113.4955 1055
113.986 1060
114.4765 1065
114.967 1070
115.4575 1075
115.948 1080
116.4385 1085
116.929 1090
117.4195 1095
117.91 1100
118.4005 1105
118.891 1110
119.3815 1115
119.872 1120
120.3625 1125
120.853 1130
121.3435 1135
121.834 1140
122.3245 1145
122.815 1150
123.3055 1155
123.796 1160
124.2865 1165
124.777 1170
125.2675 1175
125.758 1180
126.2485 1185
126.739 1190
127.2295 1195
127.72 1200
128.2105 1205
128.701 1210
129.1915 1215
129.682 1220
130.1725 1225
130.663 1230
131.1535 1235
131.644 1240
132.1345 1245
132.625 1250
133.1155 1255
133.606 1260
134.0965 1265
134.587 1270
135.0775 1275
135.568 1280
136.0585 1285
136.549 1290
137.0395 1295
137.53 1300
138.0205 1305
138.511 1310
139.0015 1315
139.492 1320
139.9825 1325
140.473 1330
140.9635 1335
141.454 1340
141.9445 1345
142.435 1350
142.9255 1355
143.416 1360
143.9065 1365
144.397 1370
144.8875 1375
145.378 1380
145.8685 1385
146.359 1390
146.8495 1395
147.34 1400
147.8305 1405
148.321 1410
148.8115 1415
149.302 1420
149.7925 1425
150.283 1430
150.7735 1435
151.264 1440
151.7545 1445
152.245 1450
152.7355 1455
153.226 1460
153.7165 1465
154.207 1470
154.6975 1475
155.188 1480
155.6785 1485
156.169 1490
156.6595 1495
157.15 1500
157.6405 1505
158.131 1510
158.6215 1515
159.112 1520
159.6025 1525
160.093 1530
160.5835 1535
161.074 1540
161.5645 1545
162.055 1550
162.5455 1555
163.036 1560
163.5265 1565
164.017 1570
164.5075 1575
164.998 1580
165.4885 1585
165.979 1590
166.4695 1595
166.96 1600
167.4505 1605
167.941 1610
168.4315 1615
168.922 1620
169.4125 1625
169.903 1630
170.3935 1635
170.884 1640
171.3745 1645
171.865 1650
172.3555 1655
172.846 1660
173.3365 1665
173.827 1670
174.3175 1675
174.808 1680
175.2985 1685
175.789 1690
176.2795 1695
176.77 1700
177.2605 1705
177.751 1710
178.2415 1715
178.732 1720
179.2225 1725
179.713 1730
180.2035 1735
180.694 1740
181.1845 1745
181.675 1750
182.1655 1755
182.656 1760
183.1465 1765
183.637 1770
184.1275 1775
184.618 1780
185.1085 1785
185.599 1790
186.0895 1795
186.58 1800
187.0705 1805
187.561 1810
188.0515 1815
188.542 1820
189.0325 1825
189.523 1830
190.0135 1835
190.504 1840
190.9945 1845
191.485 1850
191.9755 1855
192.466 1860
192.9565 1865
193.447 1870
193.9375 1875
194.428 1880
194.9185 1885
195.409 1890
195.8995 1895
196.39 1900
196.8805 1905
197.371 1910
197.8615 1915
198.352 1920
198.8425 1925
199.333 1930
199.8235 1935
200.314 1940
200.8045 1945
201.295 1950
201.7855 1955
202.276 1960
202.7665 1965
203.257 1970
203.7475 1975
204.238 1980
204.7285 1985
205.219 1990
205.7095 1995
206.2 2000
206.6905 2005
207.181 2010
207.6715 2015
208.162 2020
208.6525 2025
209.143 2030
209.6335 2035
210.124 2040
210.6145 2045
211.105 2050
211.5955 2055
212.086 2060
212.5765 2065
213.067 2070
213.5575 2075
214.048 2080
214.5385 2085
215.029 2090
215.5195 2095
216.01 2100
216.5005 2105
216.991 2110
217.4815 2115
217.972 2120
218.4625 2125
218.953 2130
219.4435 2135
219.934 2140
220.4245 2145
220.915 2150
221.4055 2155
221.896 2160
222.3865 2165
222.877 2170
223.3675 2175
223.858 2180
224.3485 2185
224.839 2190
225.3295 2195
225.82 2200
226.3105 2205
226.801 2210
227.2915 2215
227.782 2220
228.2725 2225
228.763 2230
229.2535 2235
229.744 2240
230.2345 2245
230.725 2250
231.2155 2255
231.706 2260
232.1965 2265
232.687 2270
233.1775 2275
233.668 2280
234.1585 2285
234.649 2290
235.1395 2295
235.63 2300
236.1205 2305
236.611 2310
237.1015 2315
237.592 2320
238.0825 2325
238.573 2330
239.0635 2335
239.554 2340
240.0445 2345
240.535 2350
241.0255 2355
241.516 2360
242.0065 2365
242.497 2370
242.9875 2375
243.478 2380
243.9685 2385
244.459 2390
244.9495 2395
245.44 2400
245.9305 2405
246.421 2410
246.9115 2415
247.402 2420
247.8925 2425
248.383 2430
248.8735 2435
249.364 2440
249.8545 2445
250.345 2450
250.8355 2455
251.326 2460
251.8165 2465
252.307 2470
252.7975 2475
253.288 2480
253.7785 2485
254.269 2490
254.7595 2495
255.25 2500
255.7405 2505
256.231 2510
256.7215 2515
257.212 2520
257.7025 2525
258.193 2530
258.6835 2535
259.174 2540
259.6645 2545
260.155 2550
260.6455 2555
261.136 2560
261.6265 2565
262.117 2570
262.6075 2575
263.098 2580
263.5885 2585
264.079 2590
264.5695 2595
265.06 2600
265.5505 2605
266.041 2610
266.5315 2615
267.022 2620
267.5125 2625
268.003 2630
268.4935 2635
268.984 2640
269.4745 2645
269.965 2650
270.4555 2655
270.946 2660
271.4365 2665
271.927 2670
272.4175 2675
272.908 2680
273.3985 2685
273.889 2690
274.3795 2695
274.87 2700
275.3605 2705
275.851 2710
276.3415 2715
276.832 2720
277.3225 2725
277.813 2730
278.3035 2735
278.794 2740
279.2845 2745
279.775 2750
280.2655 2755
280.756 2760
281.2465 2765
281.737 2770
282.2275 2775
282.718 2780
283.2085 2785
283.699 2790
284.1895 2795
284.68 2800
285.1705 2805
285.661 2810
286.1515 2815
286.642 2820
287.1325 2825
287.623 2830
288.1135 2835
288.604 2840
289.0945 2845
289.585 2850
290.0755 2855
290.566 2860
291.0565 2865
291.547 2870
292.0375 2875
292.528 2880
293.0185 2885
293.509 2890
293.9995 2895
294.49 2900
294.9805 2905
295.471 2910
295.9615 2915
296.452 2920
296.9425 2925
297.433 2930
297.9235 2935
298.414 2940
298.9045 2945
299.395 2950
299.8855 2955
300.376 2960
300.8665 2965
301.357 2970
301.8475 2975
302.338 2980
302.8285 2985
303.319 2990
303.8095 2995
304.3 3000
304.7905 3005
305.281 3010
305.7715 3015
306.262 3020
306.7525 3025
307.243 3030
307.7335 3035
308.224 3040
308.7145 3045
309.205 3050
309.6955 3055
310.186 3060
310.6765 3065
311.167 3070
311.6575 3075
312.148 3080
312.6385 3085
313.129 3090
313.6195 3095
314.11 3100
314.6005 3105
315.091 3110
315.5815 3115
316.072 3120
316.5625 3125
317.053 3130
317.5435 3135
318.034 3140
318.5245 3145
319.015 3150
319.5055 3155
319.996 3160
320.4865 3165
320.977 3170
321.4675 3175
321.958 3180
322.4485 3185
322.939 3190
323.4295 3195
323.92 3200
324.4105 3205
324.901 3210
325.3915 3215
325.882 3220
326.3725 3225
326.863 3230
327.3535 3235
327.844 3240
328.3345 3245
328.825 3250
329.3155 3255
329.806 3260
330.2965 3265
330.787 3270
331.2775 3275
331.768 3280
332.2585 3285
332.749 3290
333.2395 3295
333.73 3300
334.2205 3305
334.711 3310
335.2015 3315
335.692 3320
336.1825 3325
336.673 3330
337.1635 3335
337.654 3340
338.1445 3345
338.635 3350
339.1255 3355
339.616 3360
340.1065 3365
340.597 3370
341.0875 3375
341.578 3380
342.0685 3385
342.559 3390
343.0495 3395
343.54 3400
344.0305 3405
344.521 3410
345.0115 3415
345.502 3420
345.9925 3425
346.483 3430
346.9735 3435
347.464 3440
347.9545 3445
348.445 3450
348.9355 3455
349.426 3460
349.9165 3465
350.407 3470
350.8975 3475
351.388 3480
351.8785 3485
352.369 3490
352.8595 3495
353.35 3500
353.8405 3505
354.331 3510
354.8215 3515
355.312 3520
355.8025 3525
356.293 3530
356.7835 3535
357.274 3540
357.7645 3545
358.255 3550
358.7455 3555
359.236 3560
359.7265 3565
360.217 3570
360.7075 3575
361.198 3580
361.6885 3585
362.179 3590
362.6695 3595
363.16 3600
363.6505 3605
364.141 3610
364.6315 3615
365.122 3620
365.6125 3625
366.103 3630
366.5935 3635
367.084 3640
367.5745 3645
368.065 3650
368.5555 3655
369.046 3660
369.5365 3665
370.027 3670
370.5175 3675
371.008 3680
371.4985 3685
371.989 3690
372.4795 3695
372.97 3700
373.4605 3705
373.951 3710
374.4415 3715
374.932 3720
375.4225 3725
375.913 3730
376.4035 3735
376.894 3740
377.3845 3745
377.875 3750
378.3655 3755
378.856 3760
379.3465 3765
379.837 3770
380.3275 3775
380.818 3780
381.3085 3785
381.799 3790
382.2895 3795
382.78 3800
383.2705 3805
383.761 3810
384.2515 3815
384.742 3820
385.2325 3825
385.723 3830
386.2135 3835
386.704 3840
387.1945 3845
387.685 3850
388.1755 3855
388.666 3860
389.1565 3865
389.647 3870
390.1375 3875
390.628 3880
391.1185 3885
391.609 3890
392.0995 3895
392.59 3900
393.0805 3905
393.571 3910
394.0615 3915
394.552 3920
395.0425 3925
395.533 3930
396.0235 3935
396.514 3940
397.0045 3945
397.495 3950
397.9855 3955
398.476 3960
398.9665 3965
399.457 3970
399.9475 3975
400.438 3980
400.9285 3985
401.419 3990
401.9095 3995
};
\addlegendentry{Brine}
\addplot [semithick, darkorange25512714]
table {%
15 0
15.0141945926766 5
15.0284028660842 10
15.0426248338267 15
15.056860509522 20
15.0711099068023 25
15.085373039314 30
15.0996499207177 35
15.1139405646881 40
15.1282449849144 45
15.1425631950997 50
15.1568952089618 55
};
\addlegendentry{\(P_{inj}\)=\qty{15}{\bar}}
\addplot [semithick, forestgreen4416044]
table {%
20 0
20.0195328695005 5
20.039085272273 10
20.0586572285953 15
20.0782487587673 20
20.0978598831108 25
20.1174906219699 30
20.1371409957109 35
20.1568110247219 40
20.1765007294136 45
20.1962101302187 50
20.215939247592 55
20.2356881020109 60
20.2554567139746 65
20.2752451040051 70
20.2950532926463 75
20.3148813004648 80
20.3347291480494 85
20.3545968560113 90
20.3744844449842 95
20.3943919356241 100
20.4143193486096 105
20.4342667046419 110
};
\addlegendentry{\(P_{inj}\)=\qty{20}{\bar}}
\addplot [semithick, mediumpurple148103189]
table {%
30 0
30.0314826764753 5
30.0629994771659 10
30.0945504405837 15
30.1261356052851 20
30.1577550098704 25
30.1894086929846 30
30.2210966933166 35
30.2528190495999 40
30.2845758006126 45
30.3163669851769 50
30.3481926421598 55
30.3800528104729 60
30.411947529072 65
30.4438768369581 70
30.4758407731765 75
30.5078393768174 80
30.5398726870157 85
30.5719407429512 90
30.6040435838486 95
30.6361812489774 100
30.6683537776522 105
30.7005612092324 110
30.7328035831226 115
30.7650809387726 120
30.797393315677 125
30.829740753376 130
30.8621232914547 135
30.8945409695435 140
30.9269938273184 145
30.9594819045003 150
30.992005240856 155
31.0245638761974 160
31.057157850382 165
31.0897872033128 170
31.1224519749384 175
31.155152205253 180
31.1878879342967 185
31.2206592021549 190
31.253466048959 195
31.2863085148863 200
31.3191866401596 205
31.352100465048 210
31.3850500298662 215
31.4180353749752 220
};
\addlegendentry{\(P_{inj}\)=\qty{30}{\bar}}
\addplot [semithick, orchid227119194]
table {%
40 0
40.0458717428958 5
40.0917980440623 10
40.1377789699778 15
40.1838145871948 20
40.2299049623405 25
40.2760501621169 30
40.3222502533003 35
40.3685053027418 40
40.4148153773672 45
40.461180544177 50
40.5076008702466 55
40.5540764227259 60
40.6006072688398 65
40.6471934758881 70
40.6938351112452 75
40.7405322423609 80
40.7872849367593 85
40.83409326204 90
40.8809572858772 95
40.9278770760205 100
40.9748527002942 105
41.0218842265978 110
41.0689717229061 115
41.1161152572686 120
41.1633148978103 125
41.2105707127312 130
41.2578827703068 135
41.3052511388874 140
41.3526758868988 145
41.4001570828422 150
41.4476947952938 155
41.4952890929054 160
41.542940044404 165
41.5906477185921 170
41.6384121843475 175
41.6862335106235 180
41.7341117664488 185
41.7820470209276 190
41.8300393432395 195
41.8780888026398 200
41.9261954684593 205
41.9743594101042 210
42.0225806970564 215
42.0708593988734 220
42.1191955851884 225
42.16758932571 230
42.2160406902228 235
42.2645497485867 240
42.3131165707376 245
42.3617412266871 250
42.4104237865223 255
42.4591643204063 260
42.5079628985779 265
42.5568195913516 270
42.6057344691178 275
42.6547076023428 280
42.7037390615686 285
42.752828917413 290
42.8019772405698 295
42.8511841018088 300
42.9004495719754 305
42.9497737219911 310
42.9991566228534 315
43.0485983456356 320
43.0980989614869 325
43.1476585416328 330
43.1972771573743 335
43.2469548800888 340
};
\addlegendentry{\(P_{inj}\)=\qty{40}{\bar}}
\addplot [semithick, goldenrod18818934]
table {%
50 0
50.0645533035053 5
50.1291925783721 10
50.1939179356399 15
50.2587294864301 20
50.3236273419598 25
50.3886116135415 30
50.4536824125831 35
50.5188398505871 40
50.5840840391512 45
50.6494150899674 50
50.7148331148223 55
50.7803382255967 60
50.8459305342651 65
50.9116101528959 70
50.9773771936512 75
51.0432317687862 80
51.1091739906494 85
51.1752039716819 90
51.2413218244179 95
51.3075276614836 100
51.3738215955977 105
51.4402037395708 110
51.5066742063053 115
51.5732331087952 120
51.6398805601257 125
51.7066166734731 130
51.7734415621047 135
51.8403553393781 140
51.9073581187417 145
51.9744500137335 150
52.041631137982 155
52.1089016052048 160
52.1762615292093 165
52.2437110238918 170
52.3112502032377 175
52.378879181321 180
52.4465980723042 185
52.5144069904378 190
52.5823060500602 195
52.6502953655978 200
52.7183750515639 205
52.7865452225594 210
52.8548059932718 215
52.9231574784753 220
52.9915997930305 225
53.060133051884 230
53.1287573700683 235
53.1974728627013 240
53.2662796449866 245
53.3351778322122 250
53.4041675397514 255
53.4732488830617 260
53.5424219776848 265
53.6116869392462 270
53.6810438834553 275
53.7504929261047 280
53.8200341830701 285
53.8896677703098 290
53.9593938038649 295
54.0292123998585 300
54.0991236744956 305
54.169127744063 310
54.2392247249285 315
54.3094147335414 320
54.3796978864313 325
54.4500743002084 330
54.5205440915632 335
54.5911073772657 340
54.6617642741658 345
54.7325148991924 350
54.8033593693533 355
54.8742978017351 360
54.9453303135024 365
55.0164570218981 370
55.0876780442427 375
55.158993497934 380
55.2304035004467 385
55.3019081693326 390
55.3735076222196 395
55.4452019768118 400
55.516991350889 405
55.5888758623066 410
55.6608556289949 415
55.7329307689592 420
55.8051014002791 425
55.8773676411083 430
55.9497296096744 435
56.0221874242785 440
56.0947412032946 445
56.1673910651697 450
56.2401371284232 455
56.3129795116464 460
56.3859183335028 465
56.4589537127268 470
56.5320857681242 475
};
\addlegendentry{\(P_{inj}\)=\qty{50}{\bar}}
\addplot [semithick, steelblue31119180]
table {%
60 0
60.0939411024971 5
60.1880291675174 10
60.2822643822429 15
60.3766469338008 20
60.4711770092613 25
60.5658547956363 30
60.660680479877 35
60.7556542488721 40
60.8507762894464 45
60.9460467883579 50
61.0414659322969 55
61.1370339078834 60
61.2327509016653 65
61.3286171001167 70
61.4246326896356 75
61.5207978565422 80
61.6171127870765 85
61.7135776673969 90
61.8101926835777 95
61.9069580216075 100
62.0038738673868 105
62.1009404067262 110
62.1981578253442 115
62.2955263088655 120
62.3930460428185 125
62.4907172126336 130
62.5885400036408 135
62.686514601068 140
62.7846411900388 145
62.88291995557 150
62.9813510825701 155
63.0799347558369 160
63.1786711600554 165
63.2775604797955 170
63.3766028995104 175
63.4757986035338 180
63.5751477760788 185
63.6746506012346 190
63.7743072629643 195
63.8741179451036 200
63.9740828313584 205
64.0742021053022 210
64.1744759503745 215
64.274904549878 220
64.3754880869769 225
64.4762267446944 230
64.5771207059109 235
64.6781701533611 240
64.7793752696324 245
64.8807362371625 250
64.982253238237 255
65.0839264549874 260
65.1857560693889 265
65.2877422632578 270
65.3898852182497 275
65.4921851158569 280
65.5946421374064 285
65.6972564640576 290
65.8000282767998 295
65.9029577564505 300
66.0060450836525 305
66.1092904388718 310
66.2126940023958 315
66.3162559543306 320
66.4199764745985 325
66.5238557429365 330
66.6278939388931 335
66.7320912418269 340
66.8364478309036 345
66.9409638850943 350
67.0456395831729 355
67.1504751037136 360
67.2554706250894 365
67.3606263254689 370
67.4659423828147 375
67.5714189748808 380
67.6770562792104 385
67.7828544731337 390
67.8888137337654 395
67.9949342380027 400
68.1012161625231 405
68.2076596837816 410
68.3142649780089 415
68.4210322212093 420
68.5279615891579 425
68.6350532573986 430
68.742307401242 435
68.849724195763 440
68.9573038157987 445
69.0650464359457 450
69.1729522305585 455
69.281021373747 460
69.3892540393741 465
69.4976504010537 470
69.6062106321486 475
69.7149349057678 480
69.8238233947651 485
69.9328762717359 490
70.042093709016 495
70.1514758786789 500
70.2610229525335 505
70.3707351021224 510
70.4806124987193 515
70.5906553133274 520
70.7008637166765 525
70.8112378792217 530
70.9217779711405 535
71.0324841623315 540
71.1433566224116 545
71.2543955207142 550
71.3656010262873 555
71.4769733078911 560
71.5885125339963 565
71.7002188727817 570
71.8120924921324 575
71.9241335596376 580
72.036342242589 585
72.1487187079783 590
72.2612631224955 595
72.3739756525269 600
72.486856464153 605
72.5999057231469 610
72.713123594972 615
72.8265102447803 620
72.9400658374104 625
73.0537905373858 630
73.1676845089126 635
73.2817479158782 640
73.3959809218492 645
73.5103836900694 650
};
\addlegendentry{\(P_{inj}\)=\qty{60}{\bar}}
\addplot [semithick, darkorange25512714]
table {%
65 0
65.3528994986556 5
65.7061064390694 10
66.059618579142 15
66.4134337490375 20
66.7675498457739 25
67.1219648283842 30
67.4766767135806 35
67.8316835718612 40
68.1869835240076 45
68.5425747379277 50
68.8984554258037 55
69.2546238415126 60
69.6110782782858 65
69.9678170665846 70
70.3248385721653 75
70.6821411943153 80
71.0397233642403 85
71.3975835435883 90
71.7557202230946 95
72.1141319213361 100
72.472817183583 105
72.8317745807388 110
73.1910027083595 115
73.550500185743 120
73.910265655084 125
74.2702977806858 130
74.6305952482249 135
74.991156764063 140
75.351981054602 145
75.7130668656777 150
76.0744129619894 155
76.4360181265615 160
76.7978811602345 165
77.1600008811832 170
77.5223761244593 175
77.885005741556 180
78.2478885999939 185
78.6110235829264 190
78.9744095887616 195
79.3380455308007 200
79.7019303368917 205
80.0660629490968 210
80.4304423233725 215
80.7950674292622 220
81.1599372495988 225
81.5250507802192 230
81.8904070296886 235
82.2560050190322 240
82.6218437814773 245
82.9879223622034 250
83.3542398180971 255
83.7207952175187 260
84.0875876400753 265
84.4546161763957 270
84.821879927916 275
85.1893780066694 280
85.5571095350823 285
85.9250736457752 290
86.2932694813691 295
86.6616961942971 300
87.0303529466201 305
87.399238909848 310
87.768353264764 315
88.1376952012552 320
88.5072639181443 325
88.8770586230276 330
89.2470785321177 335
89.6173228700867 340
89.9877908699154 345
90.358481772745 350
90.7293948277326 355
91.1005292919099 360
91.4718844300451 365
91.8434595145072 370
92.215253825135 375
92.5872666491068 380
92.9594972808147 385
93.331945021741 390
93.7046091803371 395
94.0774890719055 400
94.4505840184838 405
94.8238933487319 410
95.1974163978211 415
95.5711525073255 420
95.9451010251162 425
96.3192613052574 430
96.6936327079048 435
97.0682145992062 440
97.4430063512042 445
97.8180073417405 450
98.1932169543628 455
98.5686345782339 460
98.9442596080418 465
99.3200914439122 470
99.6961294913229 475
100.07237316102 480
100.448821868934 485
100.825475036102 490
101.202332088587 495
101.579392457402 500
101.956655578432 505
102.334120892363 510
102.711787844605 515
103.089655885226 520
103.46772446888 525
103.845993054736 530
104.224461106417 535
104.60312809193 540
104.981993483602 545
105.361056758018 550
105.740317395961 555
106.119774882348 560
106.499428706175 565
106.879278360453 570
107.259323342158 575
107.63956315217 580
108.01999729522 585
108.400625279838 590
108.781446618297 595
109.162460826564 600
109.54366742425 605
109.925065934557 610
110.306655884233 615
110.688436803523 620
111.070408226121 625
111.452569689125 630
111.834920732994 635
112.217460901501 640
112.60018974169 645
112.983106803835 650
113.366211641397 655
113.749503810985 660
114.132982872313 665
114.516648388163 670
114.900499924344 675
115.284537049656 680
115.668759335853 685
116.053166357605 690
116.437757692462 695
116.822532920819 700
117.207491625884 705
117.592633393638 710
117.97795781281 715
118.363464474835 720
118.749152973828 725
119.135022906552 730
119.521073872384 735
119.907305473285 740
120.293717313774 745
120.680309000893 750
121.067080144181 755
121.454030355645 760
121.841159249733 765
122.228466443305 770
122.615951555607 775
123.003614208242 780
123.391454025148 785
123.779470632569 790
124.167663659031 795
124.556032735318 800
124.944577494445 805
125.333297571637 810
125.722192604302 815
126.111262232012 820
126.500506096476 825
126.889923841521 830
127.279515113065 835
127.669279559103 840
128.059216829677 845
128.44932657686 850
128.839608454736 855
129.230062119374 860
129.620687228815 865
130.011483443046 870
130.402450423985 875
130.79358783546 880
131.184895343189 885
131.576372614763 890
131.968019319628 895
132.359835129063 900
132.75181971617 905
133.143972755848 910
133.536293924781 915
133.92878290142 920
134.321439365963 925
134.714263000346 930
135.107253488217 935
135.500410514928 940
135.893733767515 945
136.287222934683 950
136.680877706793 955
137.074697775843 960
137.468682835456 965
137.862832580864 970
138.257146708894 975
138.651624917953 980
139.046266908016 985
139.441072380609 990
139.836041038796 995
140.231172587169 1000
140.626466731828 1005
141.021923180375 1010
141.417541641895 1015
141.813321826948 1020
142.209263447552 1025
142.605366217175 1030
143.001629850717 1035
143.398054064505 1040
143.794638576273 1045
144.191383105156 1050
144.588287371679 1055
144.985351097738 1060
145.382574006596 1065
145.77995582287 1070
146.177496272518 1075
146.575195082829 1080
146.973051982412 1085
147.371066701186 1090
147.769238970368 1095
148.167568522465 1100
148.566055091261 1105
148.964698411807 1110
149.363498220414 1115
149.762454254638 1120
150.161566253275 1125
150.560833956349 1130
150.960257105102 1135
151.359835441984 1140
151.759568710647 1145
152.159456655932 1150
152.559499023861 1155
152.959695561629 1160
153.360046017593 1165
153.760550141265 1170
154.161207683303 1175
154.5620183955 1180
154.962982030779 1185
155.364098343183 1190
155.765367087865 1195
156.166788021081 1200
156.568360900182 1205
156.970085483609 1210
157.371961530876 1215
157.773988802573 1220
158.176167060349 1225
158.578496066912 1230
158.980975586014 1235
159.383605382449 1240
159.786385222042 1245
160.189314871644 1250
160.592394099124 1255
160.995622673361 1260
161.399000364234 1265
161.802526942624 1270
162.206202180394 1275
162.610025850395 1280
163.013997726449 1285
163.418117583348 1290
163.822385196844 1295
164.226800343644 1300
164.631362801404 1305
165.03607234872 1310
165.440928765124 1315
165.845931831076 1320
166.251081327958 1325
166.656377038067 1330
167.061818744611 1335
167.467406231699 1340
167.87313928434 1345
168.279017688431 1350
168.685041230757 1355
169.091209698979 1360
169.497522881634 1365
169.903980568125 1370
170.310582548715 1375
170.717328614526 1380
171.124218557527 1385
171.531252170533 1390
171.938429247198 1395
172.345749582008 1400
172.753212970278 1405
173.160819208146 1410
173.568568092566 1415
173.976459421302 1420
174.384492992928 1425
174.792668606817 1430
175.200986063139 1435
175.609445162853 1440
176.018045707706 1445
176.426787500225 1450
176.835670343713 1455
177.244694042242 1460
177.653858400652 1465
178.063163224543 1470
178.472608320271 1475
178.882193494945 1480
179.291918556419 1485
179.701783313288 1490
180.111787574888 1495
180.521931151283 1500
180.932213853269 1505
181.342635492363 1510
181.753195880802 1515
182.163894831539 1520
182.574732158234 1525
182.985707675255 1530
183.396821197671 1535
183.808072541248 1540
184.219461522446 1545
184.630987958411 1550
185.042651666976 1555
185.454452466652 1560
185.866390176627 1565
186.278464616763 1570
186.690675607587 1575
187.10302297029 1580
187.515506526726 1585
187.928126099402 1590
188.340881511477 1595
188.753772586759 1600
189.166799149701 1605
189.579961025396 1610
189.993258039571 1615
190.40669001859 1620
190.820256789442 1625
191.233958179745 1630
191.647794017736 1635
192.061764132271 1640
192.475868352821 1645
192.890106509466 1650
193.304478432895 1655
193.718983954399 1660
194.133622905871 1665
194.5483951198 1670
194.963300429266 1675
195.378338667942 1680
195.793509670087 1685
196.20881327054 1690
196.624249304722 1695
197.039817608632 1700
197.455518018837 1705
197.871350372479 1710
198.287314507262 1715
198.703410261457 1720
199.119637473894 1725
199.535995983957 1730
199.952485631589 1735
200.369106257278 1740
200.785857702064 1745
201.202739807528 1750
201.619752415796 1755
202.036895369529 1760
202.454168511924 1765
202.871571686712 1770
203.289104738153 1775
203.706767511032 1780
204.124559850657 1785
204.54248160286 1790
204.960532613987 1795
205.378712730902 1800
205.797021800979 1805
206.215459672101 1810
206.634026192658 1815
207.052721211546 1820
207.471544578157 1825
207.890496142386 1830
208.309575754621 1835
208.728783265743 1840
209.148118527123 1845
209.56758139062 1850
209.987171708578 1855
210.406889333822 1860
210.826734119658 1865
211.246705919869 1870
211.666804588711 1875
212.087029980913 1880
212.507381951673 1885
212.927860356657 1890
213.348465051993 1895
213.769195894275 1900
214.190052740553 1905
214.611035448334 1910
215.032143875583 1915
215.453377880713 1920
215.87473732259 1925
216.296222060527 1930
216.71783195428 1935
217.139566864052 1940
217.561426650482 1945
217.98341117465 1950
218.405520298071 1955
218.827753882695 1960
219.250111790901 1965
219.672593885499 1970
220.095200029726 1975
220.517930087243 1980
220.940783922134 1985
221.363761398903 1990
221.786862382473 1995
222.210086738183 2000
222.633434331785 2005
223.056905029444 2010
223.480498697734 2015
223.904215203638 2020
224.328054414542 2025
224.752016198239 2030
225.176100422922 2035
225.600306957181 2040
226.024635670008 2045
226.449086430788 2050
226.873659109299 2055
227.298353575711 2060
227.723169700585 2065
228.148107354868 2070
228.573166409893 2075
228.998346737377 2080
229.423648209418 2085
229.849070698497 2090
230.274614077469 2095
230.700278219567 2100
231.126062998399 2105
231.551968287945 2110
231.977993962556 2115
232.40413989695 2120
232.830405966215 2125
233.256792045802 2130
233.683298011526 2135
234.109923739565 2140
234.536669106454 2145
234.963533989091 2150
235.390518264725 2155
235.817621810962 2160
236.244844505764 2165
236.672186227439 2170
237.099646854648 2175
237.527226266399 2180
237.954924342047 2185
238.382740961291 2190
238.810676004172 2195
239.238729351074 2200
239.666900882719 2205
240.095190480169 2210
240.523598024822 2215
240.952123398409 2220
241.380766482997 2225
241.809527160983 2230
242.238405315094 2235
242.667400828386 2240
243.096513584244 2245
243.525743466374 2250
243.955090358811 2255
244.384554145909 2260
244.814134712345 2265
245.243831943113 2270
245.673645723529 2275
246.103575939221 2280
246.533622476135 2285
246.96378522053 2290
247.394064058976 2295
247.824458878356 2300
248.25496956586 2305
248.685596008987 2310
249.116338095541 2315
249.547195713634 2320
249.978168751678 2325
250.40925709839 2330
250.840460642787 2335
251.271779274185 2340
251.703212882198 2345
252.134761356739 2350
252.566424588012 2355
252.998202466521 2360
253.430094883057 2365
253.862101728706 2370
254.294222894844 2375
254.726458273134 2380
255.158807755528 2385
255.591271234263 2390
256.023848601864 2395
256.456539751136 2400
256.889344575168 2405
257.32226296733 2410
257.755294821273 2415
258.188440030926 2420
258.621698490495 2425
259.055070094462 2430
259.488554737585 2435
259.922152314896 2440
260.355862721698 2445
260.789685853568 2450
261.22362160635 2455
261.65766987616 2460
262.091830559381 2465
262.526103552661 2470
262.960488752915 2475
263.394986057324 2480
263.829595363329 2485
264.264316568636 2490
264.69914957121 2495
265.134094269276 2500
265.56915056132 2505
266.004318346082 2510
266.439597522561 2515
266.874987990012 2520
267.310489647942 2525
267.746102396113 2530
268.181826134539 2535
268.617660763484 2540
269.053606183464 2545
269.489662295241 2550
269.925828999828 2555
270.362106198483 2560
270.79849379271 2565
271.234991684259 2570
271.671599775122 2575
272.108317967536 2580
272.545146163977 2585
272.982084267164 2590
273.419132180055 2595
273.856289805845 2600
274.29355704797 2605
274.730933810099 2610
275.168419996141 2615
275.606015510235 2620
276.043720256758 2625
276.481534140318 2630
276.919457065754 2635
277.357488938138 2640
277.795629662769 2645
278.233879145179 2650
278.672237291124 2655
279.110704006591 2660
279.549279197789 2665
279.987962771157 2670
280.426754633354 2675
280.865654691266 2680
281.304662851999 2685
281.743779022882 2690
282.183003111465 2695
282.622335025516 2700
283.061774673025 2705
283.501321962198 2710
283.940976801458 2715
284.380739099445 2720
284.820608765015 2725
285.260585707239 2730
285.7006698354 2735
286.140861058995 2740
286.581159287733 2745
287.021564431535 2750
287.462076400531 2755
287.902695105061 2760
288.343420455675 2765
288.784252363129 2770
289.225190738387 2775
289.66623549262 2780
290.107386537204 2785
290.548643783718 2790
290.990007143948 2795
291.431476529881 2800
291.873051853706 2805
292.314733027814 2810
292.756519964799 2815
293.19841257745 2820
293.640410778761 2825
294.082514481919 2830
294.524723600312 2835
294.967038047525 2840
295.409457737336 2845
295.851982583722 2850
296.294612500852 2855
296.73734740309 2860
297.180187204994 2865
297.623131821312 2870
298.066181166985 2875
298.509335157146 2880
298.952593707117 2885
299.395956732408 2890
299.839424148722 2895
300.282995871945 2900
300.726671818153 2905
301.170451903609 2910
301.614336044762 2915
302.058324158243 2920
302.502416160873 2925
302.946611969651 2930
303.390911501764 2935
303.835314674579 2940
304.279821405644 2945
304.724431612691 2950
305.16914521363 2955
305.613962126551 2960
306.058882269724 2965
306.503905561597 2970
306.949031920796 2975
307.394261266124 2980
307.83959351656 2985
308.28502859126 2990
308.730566409554 2995
309.176206890946 3000
309.621949955116 3005
310.067795521916 3010
310.51374351137 3015
310.959793843675 3020
311.4059464392 3025
311.852201218482 3030
312.298558102232 3035
312.745017011326 3040
313.191577866814 3045
313.638240589911 3050
314.085005101999 3055
314.531871324629 3060
314.978839179519 3065
315.425908588551 3070
315.873079473773 3075
316.320351757398 3080
316.767725361803 3085
317.215200209529 3090
317.662776223278 3095
318.110453325917 3100
318.558231440473 3105
319.006110490135 3110
319.454090398253 3115
319.902171088335 3120
320.350352484052 3125
320.798634509231 3130
321.247017087858 3135
321.695500144078 3140
322.144083602192 3145
322.592767386659 3150
323.041551422092 3155
323.490435633263 3160
323.939419945097 3165
324.388504282672 3170
324.837688571224 3175
325.286972736139 3180
325.736356702958 3185
326.185840397374 3190
326.63542374523 3195
327.085106672524 3200
327.534889105401 3205
327.98477097016 3210
328.434752193247 3215
328.884832701259 3220
329.335012420942 3225
329.785291279187 3230
330.235669203038 3235
330.686146119681 3240
331.136721956454 3245
331.587396640837 3250
332.038170100459 3255
332.489042263091 3260
332.940013056651 3265
333.391082409202 3270
333.842250248949 3275
334.293516504241 3280
334.74488110357 3285
335.196343975571 3290
335.64790504902 3295
336.099564252835 3300
336.551321516074 3305
337.003176767936 3310
337.455129937762 3315
337.907180955029 3320
338.359329749355 3325
338.811576250498 3330
339.263920388352 3335
339.71636209295 3340
340.16890129446 3345
340.621537923191 3350
341.074271909584 3355
341.527103184218 3360
341.980031677809 3365
342.433057321204 3370
342.886180045389 3375
343.339399781482 3380
343.792716460734 3385
344.24613001453 3390
344.699640374388 3395
345.153247471959 3400
345.606951239025 3405
346.0607516075 3410
346.514648509429 3415
346.968641876987 3420
347.422731642481 3425
347.876917738347 3430
348.33120009715 3435
348.785578651585 3440
349.240053334473 3445
349.694624078768 3450
350.149290817547 3455
350.604053484017 3460
351.058912011511 3465
351.51386633349 3470
351.968916383539 3475
352.424062095371 3480
352.879303402822 3485
353.334640239856 3490
353.790072540558 3495
354.245600239141 3500
354.701223269938 3505
355.156941567409 3510
355.612755066134 3515
356.068663700817 3520
356.524667406285 3525
356.980766117486 3530
357.436959769489 3535
357.893248297484 3540
358.349631636785 3545
358.806109722821 3550
359.262682491145 3555
359.719349877428 3560
360.176111817462 3565
360.632968247154 3570
361.089919102534 3575
361.546964319747 3580
362.004103835058 3585
362.461337584847 3590
362.918665505613 3595
363.376087533972 3600
363.833603606653 3605
364.291213660506 3610
364.748917632493 3615
365.206715459691 3620
365.664607079295 3625
366.122592428613 3630
366.580671445065 3635
};
\addlegendentry{\(P_{inj}\)=\qty{65}{\bar}}
\addplot [semithick, forestgreen4416044]
table {%
70 0
70.3658184465872 5
70.7318964508825 10
71.0982327475693 15
71.4648260918431 20
71.8316752587184 25
72.1987790423804 30
72.5661362555746 35
72.9337457290343 40
73.30160631094 45
73.6697168664097 50
74.0380762770174 55
74.4066834403345 60
74.7755372694976 65
75.1446366927985 70
75.5139806532897 75
75.8835681084111 80
76.2533980296357 85
76.6234694021271 90
76.9937812244106 95
77.3643325080639 100
77.735122277415 105
78.106149569254 110
78.4774134325554 115
78.8489129282106 120
79.2206471287706 125
79.5926151181964 130
79.964815991619 135
80.3372488551074 140
80.7099128254434 145
81.0828070299044 150
81.455930606052 155
81.8292827015275 160
82.2028624738538 165
82.5766690902424 170
82.9507017274065 175
83.3249595713787 180
83.6994418173341 185
84.0741476694184 190
84.44907634058 195
84.8242270524071 200
85.1995990349685 205
85.5751915266586 210
85.951003774046 215
86.3270350317277 220
86.7032845621842 225
87.0797516356392 230
87.4564355299233 235
87.8333355303405 240
88.2104509295373 245
88.5877810273758 250
88.965325130809 255
89.3430825537592 260
89.7210526169995 265
90.0992346480374 270
90.4776279810015 275
90.8562319565303 280
91.2350459216639 285
91.6140692297384 290
91.9933012402805 295
92.3727413189071 300
92.7523888372261 305
93.1322431727388 310
93.5123037087456 315
93.8925698342516 320
94.2730409438766 325
94.6537164377654 330
95.0345957215005 335
95.4156782060168 340
95.7969633075181 345
96.178450447395 350
96.5601390521447 355
96.9420285532924 360
97.3241183873142 365
97.7064079955622 370
98.0888968241903 375
98.471584324082 380
98.8544699507796 385
99.2375531644146 390
99.6208334296398 395
100.004310215563 400
100.387982995681 405
100.771851247816 410
101.155914454053 415
101.540172100678 420
101.924623678117 425
102.30926868088 430
102.6941066075 435
103.079136960474 440
103.464359246216 445
103.849772974991 450
104.235377660872 455
104.62117282168 460
105.007157978936 465
105.39333265781 470
105.77969638707 475
106.166248699037 480
106.552989129532 485
106.939917217833 490
107.327032506631 495
107.714334541979 500
108.101822873251 505
108.489497053101 510
108.877356637417 515
109.265401185279 520
109.653630258921 525
110.04204342369 530
110.430640248002 535
110.819420303311 540
111.208383164063 545
111.597528407666 550
111.986855614446 555
112.376364367617 560
112.766054253242 565
113.1559248602 570
113.545975780152 575
113.936206607503 580
114.326616939377 585
114.717206375579 590
115.107974518561 595
115.4989209734 600
115.890045347757 605
116.281347251852 610
116.672826298436 615
117.064482102755 620
117.45631428253 625
117.848322457921 630
118.240506251506 635
118.632865288247 640
119.02539919547 645
119.418107602834 650
119.810990142305 655
120.204046448133 660
120.597276156828 665
120.990678907129 670
121.384254339988 675
121.77800209854 680
122.171921828081 685
122.566013176047 690
122.96027579199 695
123.354709327553 700
123.749313436453 705
124.144087774456 710
124.539031999356 715
124.934145770955 720
125.32942875104 725
125.724880603367 730
126.120500993636 735
126.516289589476 740
126.912246060421 745
127.308370077893 750
127.704661315186 755
128.101119447442 760
128.497744151636 765
128.894535106557 770
129.291491992792 775
129.688614492706 780
130.085902290426 785
130.483355071825 790
130.880972524502 795
131.278754337769 800
131.676700202635 805
132.074809811785 810
132.473082859572 815
132.871519041992 820
133.270118056679 825
133.66887960288 830
134.067803381448 835
134.466889094823 840
134.866136447018 845
135.265545143605 850
135.665114891702 855
136.064845399957 860
136.464736378537 865
136.864787539112 870
137.264998594842 875
137.665369260366 880
138.065899251784 885
138.466588286652 890
138.86743608396 895
139.268442364129 900
139.669606848991 905
140.07092926178 910
140.472409327121 915
140.874046771015 920
141.275841320832 925
141.677792705294 930
142.079900654467 935
142.482164899749 940
142.88458517386 945
143.287161210828 950
143.68989274598 955
144.092779515932 960
144.495821258578 965
144.899017713078 970
145.30236861985 975
145.705873720557 980
146.109532758099 985
146.513345476604 990
146.917311621416 995
147.321430939084 1000
147.725703177356 1005
148.130128085169 1010
148.534705412637 1015
148.939434911042 1020
149.344316332828 1025
149.74934943159 1030
150.154533962063 1035
150.559869680117 1040
150.965356342745 1045
151.370993708057 1050
151.776781535268 1055
152.182719584694 1060
152.588807617741 1065
152.995045396894 1070
153.401432685716 1075
153.807969248833 1080
154.214654851929 1085
154.621489261739 1090
155.028472246039 1095
155.435603573639 1100
155.842883014376 1105
156.250310339106 1110
156.657885319696 1115
157.065607729018 1120
157.473477340939 1125
157.881493930317 1130
158.289657272992 1135
158.697967145778 1140
159.106423326459 1145
159.515025593777 1150
159.923773727432 1155
160.33266750807 1160
160.741706717275 1165
161.150891137569 1170
161.5602205524 1175
161.969694746135 1180
162.379313504058 1185
162.789076612359 1190
163.19898385813 1195
163.609035029359 1200
164.019229914923 1205
164.429568304581 1210
164.840049988969 1215
165.250674759594 1220
165.661442408828 1225
166.072352729902 1230
166.4834055169 1235
166.894600564751 1240
167.305937669228 1245
167.71741662694 1250
168.129037235323 1255
168.540799292641 1260
168.952702597974 1265
169.364746951217 1270
169.776932153074 1275
170.189258005049 1280
170.601724309446 1285
171.014330869359 1290
171.427077488669 1295
171.83996397204 1300
172.252990124911 1305
172.666155753494 1310
173.079460664765 1315
173.492904666464 1320
173.906487567086 1325
174.320209175876 1330
174.734069302829 1335
175.148067758679 1340
175.562204354898 1345
175.976478903692 1350
176.390891217991 1355
176.80544111145 1360
177.220128398444 1365
177.634952894059 1370
178.049914414092 1375
178.465012775043 1380
178.880247794115 1385
179.295619289205 1390
179.711127078902 1395
180.126770982482 1400
180.542550819906 1405
180.95846641181 1410
181.374517579507 1415
181.790704144981 1420
182.20702593088 1425
182.623482760515 1430
183.040074457856 1435
183.456800847525 1440
183.873661754795 1445
184.290657005585 1450
184.707786426455 1455
185.125049844606 1460
185.542447087869 1465
185.959977984709 1470
186.377642364216 1475
186.795440056103 1480
187.213370890701 1485
187.631434698959 1490
188.049631312435 1495
188.467960563296 1500
188.886422284314 1505
189.305016308862 1510
189.723742470908 1515
190.142600605015 1520
190.561590546338 1525
190.980712130616 1530
191.399965194173 1535
191.819349573911 1540
192.23886510731 1545
192.658511632424 1550
193.078288987873 1555
193.498197012848 1560
193.918235547099 1565
194.338404430939 1570
194.758703505236 1575
195.179132611412 1580
195.599691591438 1585
196.020380287834 1590
196.441198543663 1595
196.862146202529 1600
197.283223108573 1605
197.704429106471 1610
198.125764041432 1615
198.547227759191 1620
198.96882010601 1625
199.390540928675 1630
199.81239007449 1635
200.234367391276 1640
200.656472727367 1645
201.078705931611 1650
201.501066853361 1655
201.923555342477 1660
202.34617124932 1665
202.768914424753 1670
203.191784720133 1675
203.614781987315 1680
204.037906078641 1685
204.461156846946 1690
204.884534145547 1695
205.308037828249 1700
205.731667749332 1705
206.15542376356 1710
206.579305726168 1715
207.003313492866 1720
207.427446919834 1725
207.85170586372 1730
208.276090181637 1735
208.700599731161 1740
209.125234370327 1745
209.549993957629 1750
209.974878352017 1755
210.399887412892 1760
210.825021000106 1765
211.250278973961 1770
211.6756611952 1775
212.101167525015 1780
212.526797825036 1785
212.95255195733 1790
213.378429784403 1795
213.804431169194 1800
214.230555975073 1805
214.656804065841 1810
215.083175305725 1815
215.509669559377 1820
215.936286691871 1825
216.363026568702 1830
216.789889055785 1835
217.216874019449 1840
217.643981326437 1845
218.071210843906 1850
218.498562439419 1855
218.926035980949 1860
219.353631336875 1865
219.781348375979 1870
220.209186967442 1875
220.637146980847 1880
221.065228286173 1885
221.493430753795 1890
221.92175425448 1895
222.350198659387 1900
222.778763840064 1905
223.207449668446 1910
223.636256016853 1915
224.065182757991 1920
224.494229764942 1925
224.923396911173 1930
225.352684070525 1935
225.782091117217 1940
226.211617925839 1945
226.641264371356 1950
227.071030329101 1955
227.500915674776 1960
227.930920284449 1965
228.361044034554 1970
228.791286801885 1975
229.2216484636 1980
229.652128897215 1985
230.082727980603 1990
230.513445591993 1995
230.944281609969 2000
231.375235913464 2005
231.806308381766 2010
232.237498894508 2015
232.668807331673 2020
233.100233573588 2025
233.531777500923 2030
233.963438994691 2035
234.395217936246 2040
234.827114207279 2045
235.259127689821 2050
235.691258266235 2055
236.123505819221 2060
236.55587023181 2065
236.988351387364 2070
237.420949169573 2075
237.853663462456 2080
238.286494150358 2085
238.719441117948 2090
239.152504250217 2095
239.585683432481 2100
240.018978550371 2105
240.452389489841 2110
240.885916137158 2115
241.319558378907 2120
241.753316101985 2125
242.187189193605 2130
242.621177541286 2135
243.055281032861 2140
243.489499556469 2145
243.923833000555 2150
244.358281253872 2155
244.792844205474 2160
245.227521744719 2165
245.662313761266 2170
246.097220145073 2175
246.532240786397 2180
246.967375575791 2185
247.402624404106 2190
247.837987162483 2195
248.273463742361 2200
248.709054035465 2205
249.144757933816 2210
249.58057532972 2215
250.016506115771 2220
250.45255018485 2225
250.888707430124 2230
251.324977745042 2235
251.761361023336 2240
252.19785715902 2245
252.634466046387 2250
253.071187580009 2255
253.508021654735 2260
253.944968165691 2265
254.382027008277 2270
254.819198078167 2275
255.256481271309 2280
255.69387648392 2285
256.131383612489 2290
256.569002553773 2295
257.006733204795 2300
257.444575462849 2305
257.882529225491 2310
258.320594390542 2315
258.758770856085 2320
259.197058520468 2325
259.635457282296 2330
260.073967040437 2335
260.512587694014 2340
260.951319142412 2345
261.390161285268 2350
261.829114022476 2355
262.268177254184 2360
262.707350880793 2365
263.146634802957 2370
263.586028921577 2375
264.025533137808 2380
264.465147353053 2385
264.904871468959 2390
265.344705387424 2395
265.78464901059 2400
266.224702240842 2405
266.66486498081 2410
267.105137133366 2415
267.545518601623 2420
267.986009288935 2425
268.426609098895 2430
268.867317935334 2435
269.308135702321 2440
269.749062304162 2445
270.190097645395 2450
270.631241630796 2455
271.072494165374 2460
271.513855154368 2465
271.95532450325 2470
272.396902117723 2475
272.838587903719 2480
273.280381767399 2485
273.72228361515 2490
274.164293353588 2495
274.606410889553 2500
275.04863613011 2505
275.49096898255 2510
275.933409354384 2515
276.375957153347 2520
276.818612287395 2525
277.261374664703 2530
277.704244193668 2535
278.147220782902 2540
278.590304341238 2545
279.033494777724 2550
279.476792001623 2555
279.920195922415 2560
280.363706449793 2565
280.807323493664 2570
281.251046964146 2575
281.694876771569 2580
282.138812826476 2585
282.582855039616 2590
283.02700332195 2595
283.471257584646 2600
283.91561773908 2605
284.360083696833 2610
284.804655369694 2615
285.249332669656 2620
285.694115508915 2625
286.139003799872 2630
286.583997455129 2635
287.02909638749 2640
287.474300509962 2645
287.919609735749 2650
288.365023978257 2655
288.810543151089 2660
289.256167168046 2665
289.701895943126 2670
290.147729390524 2675
290.59366742463 2680
291.039709960029 2685
291.485856911499 2690
291.932108194012 2695
292.378463722734 2700
292.82492341302 2705
293.271487180418 2710
293.718154940665 2715
294.164926609689 2720
294.611802103606 2725
295.058781338719 2730
295.505864231521 2735
295.953050698689 2740
296.400340657089 2745
296.847734023769 2750
297.295230715964 2755
297.742830651092 2760
298.190533746754 2765
298.638339920734 2770
299.086249090997 2775
299.534261175691 2780
299.982376093142 2785
300.430593761857 2790
300.878914100522 2795
301.327337028003 2800
301.77586246334 2805
302.224490325755 2810
302.673220534641 2815
303.122053009572 2820
303.570987670294 2825
304.020024436729 2830
304.469163228971 2835
304.91840396729 2840
305.367746572126 2845
305.817190964092 2850
306.266737063972 2855
306.716384792723 2860
307.166134071468 2865
307.615984821503 2870
308.065936964291 2875
308.515990421464 2880
308.966145114821 2885
309.416400966328 2890
309.866757898118 2895
310.31721583249 2900
310.767774691907 2905
311.218434398998 2910
311.669194876555 2915
312.120056047534 2920
312.571017835054 2925
313.022080162395 2930
313.473242953001 2935
313.924506130475 2940
314.375869618581 2945
314.827333341244 2950
315.278897222547 2955
315.730561186733 2960
316.182325158203 2965
316.634189061515 2970
317.086152821384 2975
317.538216362684 2980
317.990379610441 2985
318.442642489841 2990
318.895004926222 2995
319.347466845077 3000
319.800028172052 3005
320.25268883295 3010
320.705448753722 3015
321.158307860474 3020
321.611266079464 3025
322.0643233371 3030
322.517479559941 3035
322.970734674696 3040
323.424088608225 3045
323.877541287535 3050
324.331092639784 3055
324.784742592276 3060
325.238491072463 3065
325.692338007946 3070
326.14628332647 3075
326.600326955928 3080
327.054468824357 3085
327.508708859942 3090
327.963046991008 3095
328.417483146028 3100
328.872017253618 3105
329.326649242535 3110
329.781379041682 3115
330.2362065801 3120
330.691131786976 3125
331.146154591636 3130
331.601274923546 3135
332.056492712315 3140
332.511807887688 3145
332.967220379553 3150
333.422730117935 3155
333.878337032997 3160
334.334041055042 3165
334.789842114508 3170
335.245740141971 3175
335.701735068145 3180
336.157826823879 3185
336.614015340156 3190
337.070300548098 3195
337.526682378958 3200
337.983160764125 3205
338.439735635122 3210
338.896406923606 3215
339.353174561366 3220
339.810038480324 3225
340.266998612535 3230
340.724054890182 3235
341.181207245585 3240
341.638455611191 3245
342.095799919578 3250
342.553240103455 3255
343.01077609566 3260
343.46840782916 3265
343.926135237052 3270
344.383958252558 3275
344.841876809033 3280
345.299890839955 3285
345.758000278931 3290
346.216205059696 3295
346.674505116109 3300
347.132900382156 3305
347.591390791947 3310
348.049976279721 3315
348.508656779837 3320
348.967432226782 3325
349.426302555164 3330
349.885267699716 3335
350.344327595295 3340
350.803482176878 3345
351.262731379568 3350
351.722075138586 3355
352.181513389279 3360
352.641046067111 3365
353.100673107669 3370
353.560394446659 3375
354.020210019911 3380
354.480119763368 3385
354.9401236131 3390
355.400221505289 3395
355.860413376239 3400
356.320699162373 3405
356.781078800231 3410
357.241552226468 3415
357.70211937786 3420
358.162780191297 3425
358.623534603787 3430
359.084382552453 3435
359.545323974533 3440
360.006358807384 3445
360.467486988473 3450
360.928708455384 3455
361.390023145817 3460
361.851430997582 3465
362.312931948606 3470
362.774525936927 3475
363.236212900697 3480
363.69799277818 3485
364.159865507753 3490
364.621831027904 3495
365.083889277233 3500
365.54604019445 3505
366.008283718377 3510
366.470619787948 3515
366.933048342203 3520
367.395569320296 3525
367.858182661488 3530
368.32088830515 3535
368.783686190761 3540
369.246576257911 3545
369.709558446296 3550
370.17263269572 3555
370.635798946095 3560
371.09905713744 3565
371.562407209882 3570
372.025849103652 3575
372.489382759091 3580
372.953008116643 3585
373.416725116859 3590
373.880533700394 3595
374.34443380801 3600
374.808425380572 3605
375.272508359051 3610
375.736682684521 3615
376.200948298159 3620
376.665305141248 3625
377.129753155173 3630
377.59429228142 3635
378.058922461581 3640
378.523643637347 3645
378.988455750513 3650
379.453358742977 3655
379.918352556734 3660
380.383437133885 3665
380.848612416629 3670
381.313878347266 3675
381.779234868196 3680
382.24468192192 3685
382.710219451037 3690
383.175847398248 3695
383.641565706351 3700
384.107374318243 3705
384.57327317692 3710
385.039262225477 3715
385.505341407104 3720
385.971510665093 3725
386.437769942829 3730
386.904119183798 3735
387.370558331581 3740
387.837087329856 3745
388.303706122395 3750
388.770414653071 3755
389.237212865848 3760
389.704100704787 3765
390.171078114046 3770
390.638145037876 3775
391.105301420624 3780
391.572547206729 3785
392.039882340727 3790
392.507306767246 3795
392.974820431008 3800
393.442423276831 3805
393.910115249621 3810
394.377896294381 3815
394.845766356206 3820
395.313725380281 3825
395.781773311886 3830
396.249910096391 3835
396.718135679257 3840
397.186450006039 3845
397.654853022381 3850
398.123344674018 3855
398.591924906776 3860
399.060593666571 3865
399.52935089941 3870
399.998196551387 3875
400.46713056869 3880
400.936152897592 3885
401.405263484458 3890
401.87446227574 3895
402.343749217981 3900
402.813124257809 3905
403.282587341943 3910
403.752138417188 3915
404.221777430438 3920
404.691504328674 3925
405.161319058963 3930
405.63122156846 3935
406.101211804408 3940
406.571289714133 3945
407.041455245051 3950
407.511708344661 3955
407.982048960549 3960
408.452477040387 3965
408.922992531932 3970
409.393595383026 3975
409.864285541594 3980
410.33506295565 3985
410.805927573287 3990
411.276879342687 3995
};
\addlegendentry{\(P_{inj}\)=\qty{70}{\bar}}
\end{axis}

\end{tikzpicture}

            \caption{The static pressure with depth of \ac{NCG} for a range of injection wellhead pressures.}
            \label{fig:prosim_NCG_Reinjection_PvD_by_Pinj}
        \end{figure}

        From these static pressure profiles, Figure\ref{fig:prosim_NCG_Reinjection_PvD_by_Pinj}, the depth of injection be seen to be a few hundreds of meters for \ac{NCG} injection pressures up to \qty{60}{\bar}, which is insufficient for geothermal reservoirs. Injection depths of a few thousands of meters can be achieved with injection pressures in excess of \qty{65}{\bar}, which can be attributed to the \ce{CO2} being a liquid at the wellhead. Liquefaction is possible in this case, as the \ac{NCG} injection temperature at the wellhead (\qty{298}{\K}) is below the critical temperature of \ce{CO2} (\qty{304.12}{\K}) and the injection pressure (\qty{65}{\bar}) is above the saturation pressure of \ce{CO2} (\qty{64.12}{\bar} at \qty{298}{\K}).   

        \begin{notes}{Note}

            For low \ac{NCG} content geofluids, it can be thermodynamically possible to fully dissolve the \ac{NCG} the geofluids at these shallow depths of injection, allowing for the \ac{NCG} to be carried to the reservoir. However, it is also important to consider the rate of mass transfer, as the dissolution must take place before the \ac{NCG} can large enough bubbles for buoyancy effects to manifest, which could lead to well unloading and a potentially catastrophic blow-out event.

            To prevent this, large mass transfer coefficients and mass transfer area are required. 
        \end{notes}

        Besides the injection pressure, the depth of injection also depends on the temperature of the \ac{NCG} at the wellhead, as this affects the density in the wellbore, Figure~\ref{fig:prosim_NCG_Reinjection_Pinj_v_Depth_v_Tinj}. For injection temperatures below \qty{313}{\K}, \ac{NCG} can be injected to depths of \qty{4000}{m} with injection pressures below \qty{100}{\bar}. For injection temperatures below \qty{303}{\K} similar depths can be obtained with just \qty{80}{\bar} of pressure at the wellhead.

        \begin{figure}[H]
            \centering
            \input{Content/ProSim/NCGHandling/Plots/Reinjection/Pinj_vs_Depth_vs_Tinj}
            \caption[The minimum \ac{NCG} injection pressure required to inject at a given depth for different \ac{NCG} injection temperatures.]{The minimum \ac{NCG} injection pressure required to inject at a given depth for different \ac{NCG} injection temperatures. The brine injection pressure is assumed to be \qty{15.5}{\bar} corresponding to the saturation pressure of pure water at \qty{473}{\K}.}
            \label{fig:prosim_NCG_Reinjection_Pinj_v_Depth_v_Tinj}
        \end{figure}

    % \subsubsection{Determination of Re-injection Pressure}
        
    %     The maximum depth of \ac{NCG} re-injection was estimated for a range of wellhead pressures. Here, the \ac{NCG} was assumed to be pure \ce{CO2}, at a temperature of \qty{298}{\K} (\qty{25}{\degreeCelsius}) at the wellhead. The static pressure gradient was then evaluated assuming negligible heat transfer with the surrounding formation along the wellbore (i.e. constant enthalpy). For reference, the static pressure gradient of pure water (also at \qty{298}{\K} (\qty{25}{\degreeCelsius}) and \qty{1}{\bar} of pressure at the wellhead) was calculated in a similar fashion. The depth of injection, is thus given by the intersection of the static pressure profiles, Figure~\ref{fig:prosim_NCG_Reinjection_PvD_by_Pinj}. The corresponding injection pressures required to reach a given depth are shown in Figure~\ref{fig:prosim_NCG_Reinjection_Pinj_v_D}.

    %     \begin{figure}[H]
    %         \centering
    %         % This file was created with tikzplotlib v0.10.1.
\begin{tikzpicture}

\definecolor{crimson2143940}{RGB}{214,39,40}
\definecolor{darkgray176}{RGB}{176,176,176}
\definecolor{darkorange25512714}{RGB}{255,127,14}
\definecolor{darkturquoise23190207}{RGB}{23,190,207}
\definecolor{forestgreen4416044}{RGB}{44,160,44}
\definecolor{goldenrod18818934}{RGB}{188,189,34}
\definecolor{gray127}{RGB}{127,127,127}
\definecolor{lightgray204}{RGB}{204,204,204}
\definecolor{mediumpurple148103189}{RGB}{148,103,189}
\definecolor{orchid227119194}{RGB}{227,119,194}
\definecolor{sienna1408675}{RGB}{140,86,75}
\definecolor{steelblue31119180}{RGB}{31,119,180}

\begin{axis}[y dir = {reverse},
                 xlabel = {Pressure/\unit{\bar}},
                 ylabel = {Depth/\unit{\m}},
                 xmin = 0,
                 xmax = 450,
                 ymin = 0,
                 ymax = 4000,
                 axis x line=top,
                 legend cell align={left},
legend style={fill opacity=0.8, draw opacity=1, text opacity=1, draw=lightgray204, at={(1.03, 0.5)}, anchor=west}, width=10cm, height=6.5cm,]
\addplot [semithick, steelblue31119180]
table {%
10 0
10.4905 5
10.981 10
11.4715 15
11.962 20
12.4525 25
12.943 30
13.4335 35
13.924 40
14.4145 45
14.905 50
15.3955 55
15.886 60
16.3765 65
16.867 70
17.3575 75
17.848 80
18.3385 85
18.829 90
19.3195 95
19.81 100
20.3005 105
20.791 110
21.2815 115
21.772 120
22.2625 125
22.753 130
23.2435 135
23.734 140
24.2245 145
24.715 150
25.2055 155
25.696 160
26.1865 165
26.677 170
27.1675 175
27.658 180
28.1485 185
28.639 190
29.1295 195
29.62 200
30.1105 205
30.601 210
31.0915 215
31.582 220
32.0725 225
32.563 230
33.0535 235
33.544 240
34.0345 245
34.525 250
35.0155 255
35.506 260
35.9965 265
36.487 270
36.9775 275
37.468 280
37.9585 285
38.449 290
38.9395 295
39.43 300
39.9205 305
40.411 310
40.9015 315
41.392 320
41.8825 325
42.373 330
42.8635 335
43.354 340
43.8445 345
44.335 350
44.8255 355
45.316 360
45.8065 365
46.297 370
46.7875 375
47.278 380
47.7685 385
48.259 390
48.7495 395
49.24 400
49.7305 405
50.221 410
50.7115 415
51.202 420
51.6925 425
52.183 430
52.6735 435
53.164 440
53.6545 445
54.145 450
54.6355 455
55.126 460
55.6165 465
56.107 470
56.5975 475
57.088 480
57.5785 485
58.069 490
58.5595 495
59.05 500
59.5405 505
60.031 510
60.5215 515
61.012 520
61.5025 525
61.993 530
62.4835 535
62.974 540
63.4645 545
63.955 550
64.4455 555
64.936 560
65.4265 565
65.917 570
66.4075 575
66.898 580
67.3885 585
67.879 590
68.3695 595
68.86 600
69.3505 605
69.841 610
70.3315 615
70.822 620
71.3125 625
71.803 630
72.2935 635
72.784 640
73.2745 645
73.765 650
74.2555 655
74.746 660
75.2365 665
75.727 670
76.2175 675
76.708 680
77.1985 685
77.689 690
78.1795 695
78.67 700
79.1605 705
79.651 710
80.1415 715
80.632 720
81.1225 725
81.613 730
82.1035 735
82.594 740
83.0845 745
83.575 750
84.0655 755
84.556 760
85.0465 765
85.537 770
86.0275 775
86.518 780
87.0085 785
87.499 790
87.9895 795
88.48 800
88.9705 805
89.461 810
89.9515 815
90.442 820
90.9325 825
91.423 830
91.9135 835
92.404 840
92.8945 845
93.385 850
93.8755 855
94.366 860
94.8565 865
95.347 870
95.8375 875
96.328 880
96.8185 885
97.309 890
97.7995 895
98.29 900
98.7805 905
99.271 910
99.7615 915
100.252 920
100.7425 925
101.233 930
101.7235 935
102.214 940
102.7045 945
103.195 950
103.6855 955
104.176 960
104.6665 965
105.157 970
105.6475 975
106.138 980
106.6285 985
107.119 990
107.6095 995
108.1 1000
108.5905 1005
109.081 1010
109.5715 1015
110.062 1020
110.5525 1025
111.043 1030
111.5335 1035
112.024 1040
112.5145 1045
113.005 1050
113.4955 1055
113.986 1060
114.4765 1065
114.967 1070
115.4575 1075
115.948 1080
116.4385 1085
116.929 1090
117.4195 1095
117.91 1100
118.4005 1105
118.891 1110
119.3815 1115
119.872 1120
120.3625 1125
120.853 1130
121.3435 1135
121.834 1140
122.3245 1145
122.815 1150
123.3055 1155
123.796 1160
124.2865 1165
124.777 1170
125.2675 1175
125.758 1180
126.2485 1185
126.739 1190
127.2295 1195
127.72 1200
128.2105 1205
128.701 1210
129.1915 1215
129.682 1220
130.1725 1225
130.663 1230
131.1535 1235
131.644 1240
132.1345 1245
132.625 1250
133.1155 1255
133.606 1260
134.0965 1265
134.587 1270
135.0775 1275
135.568 1280
136.0585 1285
136.549 1290
137.0395 1295
137.53 1300
138.0205 1305
138.511 1310
139.0015 1315
139.492 1320
139.9825 1325
140.473 1330
140.9635 1335
141.454 1340
141.9445 1345
142.435 1350
142.9255 1355
143.416 1360
143.9065 1365
144.397 1370
144.8875 1375
145.378 1380
145.8685 1385
146.359 1390
146.8495 1395
147.34 1400
147.8305 1405
148.321 1410
148.8115 1415
149.302 1420
149.7925 1425
150.283 1430
150.7735 1435
151.264 1440
151.7545 1445
152.245 1450
152.7355 1455
153.226 1460
153.7165 1465
154.207 1470
154.6975 1475
155.188 1480
155.6785 1485
156.169 1490
156.6595 1495
157.15 1500
157.6405 1505
158.131 1510
158.6215 1515
159.112 1520
159.6025 1525
160.093 1530
160.5835 1535
161.074 1540
161.5645 1545
162.055 1550
162.5455 1555
163.036 1560
163.5265 1565
164.017 1570
164.5075 1575
164.998 1580
165.4885 1585
165.979 1590
166.4695 1595
166.96 1600
167.4505 1605
167.941 1610
168.4315 1615
168.922 1620
169.4125 1625
169.903 1630
170.3935 1635
170.884 1640
171.3745 1645
171.865 1650
172.3555 1655
172.846 1660
173.3365 1665
173.827 1670
174.3175 1675
174.808 1680
175.2985 1685
175.789 1690
176.2795 1695
176.77 1700
177.2605 1705
177.751 1710
178.2415 1715
178.732 1720
179.2225 1725
179.713 1730
180.2035 1735
180.694 1740
181.1845 1745
181.675 1750
182.1655 1755
182.656 1760
183.1465 1765
183.637 1770
184.1275 1775
184.618 1780
185.1085 1785
185.599 1790
186.0895 1795
186.58 1800
187.0705 1805
187.561 1810
188.0515 1815
188.542 1820
189.0325 1825
189.523 1830
190.0135 1835
190.504 1840
190.9945 1845
191.485 1850
191.9755 1855
192.466 1860
192.9565 1865
193.447 1870
193.9375 1875
194.428 1880
194.9185 1885
195.409 1890
195.8995 1895
196.39 1900
196.8805 1905
197.371 1910
197.8615 1915
198.352 1920
198.8425 1925
199.333 1930
199.8235 1935
200.314 1940
200.8045 1945
201.295 1950
201.7855 1955
202.276 1960
202.7665 1965
203.257 1970
203.7475 1975
204.238 1980
204.7285 1985
205.219 1990
205.7095 1995
206.2 2000
206.6905 2005
207.181 2010
207.6715 2015
208.162 2020
208.6525 2025
209.143 2030
209.6335 2035
210.124 2040
210.6145 2045
211.105 2050
211.5955 2055
212.086 2060
212.5765 2065
213.067 2070
213.5575 2075
214.048 2080
214.5385 2085
215.029 2090
215.5195 2095
216.01 2100
216.5005 2105
216.991 2110
217.4815 2115
217.972 2120
218.4625 2125
218.953 2130
219.4435 2135
219.934 2140
220.4245 2145
220.915 2150
221.4055 2155
221.896 2160
222.3865 2165
222.877 2170
223.3675 2175
223.858 2180
224.3485 2185
224.839 2190
225.3295 2195
225.82 2200
226.3105 2205
226.801 2210
227.2915 2215
227.782 2220
228.2725 2225
228.763 2230
229.2535 2235
229.744 2240
230.2345 2245
230.725 2250
231.2155 2255
231.706 2260
232.1965 2265
232.687 2270
233.1775 2275
233.668 2280
234.1585 2285
234.649 2290
235.1395 2295
235.63 2300
236.1205 2305
236.611 2310
237.1015 2315
237.592 2320
238.0825 2325
238.573 2330
239.0635 2335
239.554 2340
240.0445 2345
240.535 2350
241.0255 2355
241.516 2360
242.0065 2365
242.497 2370
242.9875 2375
243.478 2380
243.9685 2385
244.459 2390
244.9495 2395
245.44 2400
245.9305 2405
246.421 2410
246.9115 2415
247.402 2420
247.8925 2425
248.383 2430
248.8735 2435
249.364 2440
249.8545 2445
250.345 2450
250.8355 2455
251.326 2460
251.8165 2465
252.307 2470
252.7975 2475
253.288 2480
253.7785 2485
254.269 2490
254.7595 2495
255.25 2500
255.7405 2505
256.231 2510
256.7215 2515
257.212 2520
257.7025 2525
258.193 2530
258.6835 2535
259.174 2540
259.6645 2545
260.155 2550
260.6455 2555
261.136 2560
261.6265 2565
262.117 2570
262.6075 2575
263.098 2580
263.5885 2585
264.079 2590
264.5695 2595
265.06 2600
265.5505 2605
266.041 2610
266.5315 2615
267.022 2620
267.5125 2625
268.003 2630
268.4935 2635
268.984 2640
269.4745 2645
269.965 2650
270.4555 2655
270.946 2660
271.4365 2665
271.927 2670
272.4175 2675
272.908 2680
273.3985 2685
273.889 2690
274.3795 2695
274.87 2700
275.3605 2705
275.851 2710
276.3415 2715
276.832 2720
277.3225 2725
277.813 2730
278.3035 2735
278.794 2740
279.2845 2745
279.775 2750
280.2655 2755
280.756 2760
281.2465 2765
281.737 2770
282.2275 2775
282.718 2780
283.2085 2785
283.699 2790
284.1895 2795
284.68 2800
285.1705 2805
285.661 2810
286.1515 2815
286.642 2820
287.1325 2825
287.623 2830
288.1135 2835
288.604 2840
289.0945 2845
289.585 2850
290.0755 2855
290.566 2860
291.0565 2865
291.547 2870
292.0375 2875
292.528 2880
293.0185 2885
293.509 2890
293.9995 2895
294.49 2900
294.9805 2905
295.471 2910
295.9615 2915
296.452 2920
296.9425 2925
297.433 2930
297.9235 2935
298.414 2940
298.9045 2945
299.395 2950
299.8855 2955
300.376 2960
300.8665 2965
301.357 2970
301.8475 2975
302.338 2980
302.8285 2985
303.319 2990
303.8095 2995
304.3 3000
304.7905 3005
305.281 3010
305.7715 3015
306.262 3020
306.7525 3025
307.243 3030
307.7335 3035
308.224 3040
308.7145 3045
309.205 3050
309.6955 3055
310.186 3060
310.6765 3065
311.167 3070
311.6575 3075
312.148 3080
312.6385 3085
313.129 3090
313.6195 3095
314.11 3100
314.6005 3105
315.091 3110
315.5815 3115
316.072 3120
316.5625 3125
317.053 3130
317.5435 3135
318.034 3140
318.5245 3145
319.015 3150
319.5055 3155
319.996 3160
320.4865 3165
320.977 3170
321.4675 3175
321.958 3180
322.4485 3185
322.939 3190
323.4295 3195
323.92 3200
324.4105 3205
324.901 3210
325.3915 3215
325.882 3220
326.3725 3225
326.863 3230
327.3535 3235
327.844 3240
328.3345 3245
328.825 3250
329.3155 3255
329.806 3260
330.2965 3265
330.787 3270
331.2775 3275
331.768 3280
332.2585 3285
332.749 3290
333.2395 3295
333.73 3300
334.2205 3305
334.711 3310
335.2015 3315
335.692 3320
336.1825 3325
336.673 3330
337.1635 3335
337.654 3340
338.1445 3345
338.635 3350
339.1255 3355
339.616 3360
340.1065 3365
340.597 3370
341.0875 3375
341.578 3380
342.0685 3385
342.559 3390
343.0495 3395
343.54 3400
344.0305 3405
344.521 3410
345.0115 3415
345.502 3420
345.9925 3425
346.483 3430
346.9735 3435
347.464 3440
347.9545 3445
348.445 3450
348.9355 3455
349.426 3460
349.9165 3465
350.407 3470
350.8975 3475
351.388 3480
351.8785 3485
352.369 3490
352.8595 3495
353.35 3500
353.8405 3505
354.331 3510
354.8215 3515
355.312 3520
355.8025 3525
356.293 3530
356.7835 3535
357.274 3540
357.7645 3545
358.255 3550
358.7455 3555
359.236 3560
359.7265 3565
360.217 3570
360.7075 3575
361.198 3580
361.6885 3585
362.179 3590
362.6695 3595
363.16 3600
363.6505 3605
364.141 3610
364.6315 3615
365.122 3620
365.6125 3625
366.103 3630
366.5935 3635
367.084 3640
367.5745 3645
368.065 3650
368.5555 3655
369.046 3660
369.5365 3665
370.027 3670
370.5175 3675
371.008 3680
371.4985 3685
371.989 3690
372.4795 3695
372.97 3700
373.4605 3705
373.951 3710
374.4415 3715
374.932 3720
375.4225 3725
375.913 3730
376.4035 3735
376.894 3740
377.3845 3745
377.875 3750
378.3655 3755
378.856 3760
379.3465 3765
379.837 3770
380.3275 3775
380.818 3780
381.3085 3785
381.799 3790
382.2895 3795
382.78 3800
383.2705 3805
383.761 3810
384.2515 3815
384.742 3820
385.2325 3825
385.723 3830
386.2135 3835
386.704 3840
387.1945 3845
387.685 3850
388.1755 3855
388.666 3860
389.1565 3865
389.647 3870
390.1375 3875
390.628 3880
391.1185 3885
391.609 3890
392.0995 3895
392.59 3900
393.0805 3905
393.571 3910
394.0615 3915
394.552 3920
395.0425 3925
395.533 3930
396.0235 3935
396.514 3940
397.0045 3945
397.495 3950
397.9855 3955
398.476 3960
398.9665 3965
399.457 3970
399.9475 3975
400.438 3980
400.9285 3985
401.419 3990
401.9095 3995
};
\addlegendentry{Brine}
\addplot [semithick, darkorange25512714]
table {%
15 0
15.0141945926766 5
15.0284028660842 10
15.0426248338267 15
15.056860509522 20
15.0711099068023 25
15.085373039314 30
15.0996499207177 35
15.1139405646881 40
15.1282449849144 45
15.1425631950997 50
15.1568952089618 55
};
\addlegendentry{\(P_{inj}\)=\qty{15}{\bar}}
\addplot [semithick, forestgreen4416044]
table {%
20 0
20.0195328695005 5
20.039085272273 10
20.0586572285953 15
20.0782487587673 20
20.0978598831108 25
20.1174906219699 30
20.1371409957109 35
20.1568110247219 40
20.1765007294136 45
20.1962101302187 50
20.215939247592 55
20.2356881020109 60
20.2554567139746 65
20.2752451040051 70
20.2950532926463 75
20.3148813004648 80
20.3347291480494 85
20.3545968560113 90
20.3744844449842 95
20.3943919356241 100
20.4143193486096 105
20.4342667046419 110
};
\addlegendentry{\(P_{inj}\)=\qty{20}{\bar}}
\addplot [semithick, mediumpurple148103189]
table {%
30 0
30.0314826764753 5
30.0629994771659 10
30.0945504405837 15
30.1261356052851 20
30.1577550098704 25
30.1894086929846 30
30.2210966933166 35
30.2528190495999 40
30.2845758006126 45
30.3163669851769 50
30.3481926421598 55
30.3800528104729 60
30.411947529072 65
30.4438768369581 70
30.4758407731765 75
30.5078393768174 80
30.5398726870157 85
30.5719407429512 90
30.6040435838486 95
30.6361812489774 100
30.6683537776522 105
30.7005612092324 110
30.7328035831226 115
30.7650809387726 120
30.797393315677 125
30.829740753376 130
30.8621232914547 135
30.8945409695435 140
30.9269938273184 145
30.9594819045003 150
30.992005240856 155
31.0245638761974 160
31.057157850382 165
31.0897872033128 170
31.1224519749384 175
31.155152205253 180
31.1878879342967 185
31.2206592021549 190
31.253466048959 195
31.2863085148863 200
31.3191866401596 205
31.352100465048 210
31.3850500298662 215
31.4180353749752 220
};
\addlegendentry{\(P_{inj}\)=\qty{30}{\bar}}
\addplot [semithick, orchid227119194]
table {%
40 0
40.0458717428958 5
40.0917980440623 10
40.1377789699778 15
40.1838145871948 20
40.2299049623405 25
40.2760501621169 30
40.3222502533003 35
40.3685053027418 40
40.4148153773672 45
40.461180544177 50
40.5076008702466 55
40.5540764227259 60
40.6006072688398 65
40.6471934758881 70
40.6938351112452 75
40.7405322423609 80
40.7872849367593 85
40.83409326204 90
40.8809572858772 95
40.9278770760205 100
40.9748527002942 105
41.0218842265978 110
41.0689717229061 115
41.1161152572686 120
41.1633148978103 125
41.2105707127312 130
41.2578827703068 135
41.3052511388874 140
41.3526758868988 145
41.4001570828422 150
41.4476947952938 155
41.4952890929054 160
41.542940044404 165
41.5906477185921 170
41.6384121843475 175
41.6862335106235 180
41.7341117664488 185
41.7820470209276 190
41.8300393432395 195
41.8780888026398 200
41.9261954684593 205
41.9743594101042 210
42.0225806970564 215
42.0708593988734 220
42.1191955851884 225
42.16758932571 230
42.2160406902228 235
42.2645497485867 240
42.3131165707376 245
42.3617412266871 250
42.4104237865223 255
42.4591643204063 260
42.5079628985779 265
42.5568195913516 270
42.6057344691178 275
42.6547076023428 280
42.7037390615686 285
42.752828917413 290
42.8019772405698 295
42.8511841018088 300
42.9004495719754 305
42.9497737219911 310
42.9991566228534 315
43.0485983456356 320
43.0980989614869 325
43.1476585416328 330
43.1972771573743 335
43.2469548800888 340
};
\addlegendentry{\(P_{inj}\)=\qty{40}{\bar}}
\addplot [semithick, goldenrod18818934]
table {%
50 0
50.0645533035053 5
50.1291925783721 10
50.1939179356399 15
50.2587294864301 20
50.3236273419598 25
50.3886116135415 30
50.4536824125831 35
50.5188398505871 40
50.5840840391512 45
50.6494150899674 50
50.7148331148223 55
50.7803382255967 60
50.8459305342651 65
50.9116101528959 70
50.9773771936512 75
51.0432317687862 80
51.1091739906494 85
51.1752039716819 90
51.2413218244179 95
51.3075276614836 100
51.3738215955977 105
51.4402037395708 110
51.5066742063053 115
51.5732331087952 120
51.6398805601257 125
51.7066166734731 130
51.7734415621047 135
51.8403553393781 140
51.9073581187417 145
51.9744500137335 150
52.041631137982 155
52.1089016052048 160
52.1762615292093 165
52.2437110238918 170
52.3112502032377 175
52.378879181321 180
52.4465980723042 185
52.5144069904378 190
52.5823060500602 195
52.6502953655978 200
52.7183750515639 205
52.7865452225594 210
52.8548059932718 215
52.9231574784753 220
52.9915997930305 225
53.060133051884 230
53.1287573700683 235
53.1974728627013 240
53.2662796449866 245
53.3351778322122 250
53.4041675397514 255
53.4732488830617 260
53.5424219776848 265
53.6116869392462 270
53.6810438834553 275
53.7504929261047 280
53.8200341830701 285
53.8896677703098 290
53.9593938038649 295
54.0292123998585 300
54.0991236744956 305
54.169127744063 310
54.2392247249285 315
54.3094147335414 320
54.3796978864313 325
54.4500743002084 330
54.5205440915632 335
54.5911073772657 340
54.6617642741658 345
54.7325148991924 350
54.8033593693533 355
54.8742978017351 360
54.9453303135024 365
55.0164570218981 370
55.0876780442427 375
55.158993497934 380
55.2304035004467 385
55.3019081693326 390
55.3735076222196 395
55.4452019768118 400
55.516991350889 405
55.5888758623066 410
55.6608556289949 415
55.7329307689592 420
55.8051014002791 425
55.8773676411083 430
55.9497296096744 435
56.0221874242785 440
56.0947412032946 445
56.1673910651697 450
56.2401371284232 455
56.3129795116464 460
56.3859183335028 465
56.4589537127268 470
56.5320857681242 475
};
\addlegendentry{\(P_{inj}\)=\qty{50}{\bar}}
\addplot [semithick, steelblue31119180]
table {%
60 0
60.0939411024971 5
60.1880291675174 10
60.2822643822429 15
60.3766469338008 20
60.4711770092613 25
60.5658547956363 30
60.660680479877 35
60.7556542488721 40
60.8507762894464 45
60.9460467883579 50
61.0414659322969 55
61.1370339078834 60
61.2327509016653 65
61.3286171001167 70
61.4246326896356 75
61.5207978565422 80
61.6171127870765 85
61.7135776673969 90
61.8101926835777 95
61.9069580216075 100
62.0038738673868 105
62.1009404067262 110
62.1981578253442 115
62.2955263088655 120
62.3930460428185 125
62.4907172126336 130
62.5885400036408 135
62.686514601068 140
62.7846411900388 145
62.88291995557 150
62.9813510825701 155
63.0799347558369 160
63.1786711600554 165
63.2775604797955 170
63.3766028995104 175
63.4757986035338 180
63.5751477760788 185
63.6746506012346 190
63.7743072629643 195
63.8741179451036 200
63.9740828313584 205
64.0742021053022 210
64.1744759503745 215
64.274904549878 220
64.3754880869769 225
64.4762267446944 230
64.5771207059109 235
64.6781701533611 240
64.7793752696324 245
64.8807362371625 250
64.982253238237 255
65.0839264549874 260
65.1857560693889 265
65.2877422632578 270
65.3898852182497 275
65.4921851158569 280
65.5946421374064 285
65.6972564640576 290
65.8000282767998 295
65.9029577564505 300
66.0060450836525 305
66.1092904388718 310
66.2126940023958 315
66.3162559543306 320
66.4199764745985 325
66.5238557429365 330
66.6278939388931 335
66.7320912418269 340
66.8364478309036 345
66.9409638850943 350
67.0456395831729 355
67.1504751037136 360
67.2554706250894 365
67.3606263254689 370
67.4659423828147 375
67.5714189748808 380
67.6770562792104 385
67.7828544731337 390
67.8888137337654 395
67.9949342380027 400
68.1012161625231 405
68.2076596837816 410
68.3142649780089 415
68.4210322212093 420
68.5279615891579 425
68.6350532573986 430
68.742307401242 435
68.849724195763 440
68.9573038157987 445
69.0650464359457 450
69.1729522305585 455
69.281021373747 460
69.3892540393741 465
69.4976504010537 470
69.6062106321486 475
69.7149349057678 480
69.8238233947651 485
69.9328762717359 490
70.042093709016 495
70.1514758786789 500
70.2610229525335 505
70.3707351021224 510
70.4806124987193 515
70.5906553133274 520
70.7008637166765 525
70.8112378792217 530
70.9217779711405 535
71.0324841623315 540
71.1433566224116 545
71.2543955207142 550
71.3656010262873 555
71.4769733078911 560
71.5885125339963 565
71.7002188727817 570
71.8120924921324 575
71.9241335596376 580
72.036342242589 585
72.1487187079783 590
72.2612631224955 595
72.3739756525269 600
72.486856464153 605
72.5999057231469 610
72.713123594972 615
72.8265102447803 620
72.9400658374104 625
73.0537905373858 630
73.1676845089126 635
73.2817479158782 640
73.3959809218492 645
73.5103836900694 650
};
\addlegendentry{\(P_{inj}\)=\qty{60}{\bar}}
\addplot [semithick, darkorange25512714]
table {%
65 0
65.3528994986556 5
65.7061064390694 10
66.059618579142 15
66.4134337490375 20
66.7675498457739 25
67.1219648283842 30
67.4766767135806 35
67.8316835718612 40
68.1869835240076 45
68.5425747379277 50
68.8984554258037 55
69.2546238415126 60
69.6110782782858 65
69.9678170665846 70
70.3248385721653 75
70.6821411943153 80
71.0397233642403 85
71.3975835435883 90
71.7557202230946 95
72.1141319213361 100
72.472817183583 105
72.8317745807388 110
73.1910027083595 115
73.550500185743 120
73.910265655084 125
74.2702977806858 130
74.6305952482249 135
74.991156764063 140
75.351981054602 145
75.7130668656777 150
76.0744129619894 155
76.4360181265615 160
76.7978811602345 165
77.1600008811832 170
77.5223761244593 175
77.885005741556 180
78.2478885999939 185
78.6110235829264 190
78.9744095887616 195
79.3380455308007 200
79.7019303368917 205
80.0660629490968 210
80.4304423233725 215
80.7950674292622 220
81.1599372495988 225
81.5250507802192 230
81.8904070296886 235
82.2560050190322 240
82.6218437814773 245
82.9879223622034 250
83.3542398180971 255
83.7207952175187 260
84.0875876400753 265
84.4546161763957 270
84.821879927916 275
85.1893780066694 280
85.5571095350823 285
85.9250736457752 290
86.2932694813691 295
86.6616961942971 300
87.0303529466201 305
87.399238909848 310
87.768353264764 315
88.1376952012552 320
88.5072639181443 325
88.8770586230276 330
89.2470785321177 335
89.6173228700867 340
89.9877908699154 345
90.358481772745 350
90.7293948277326 355
91.1005292919099 360
91.4718844300451 365
91.8434595145072 370
92.215253825135 375
92.5872666491068 380
92.9594972808147 385
93.331945021741 390
93.7046091803371 395
94.0774890719055 400
94.4505840184838 405
94.8238933487319 410
95.1974163978211 415
95.5711525073255 420
95.9451010251162 425
96.3192613052574 430
96.6936327079048 435
97.0682145992062 440
97.4430063512042 445
97.8180073417405 450
98.1932169543628 455
98.5686345782339 460
98.9442596080418 465
99.3200914439122 470
99.6961294913229 475
100.07237316102 480
100.448821868934 485
100.825475036102 490
101.202332088587 495
101.579392457402 500
101.956655578432 505
102.334120892363 510
102.711787844605 515
103.089655885226 520
103.46772446888 525
103.845993054736 530
104.224461106417 535
104.60312809193 540
104.981993483602 545
105.361056758018 550
105.740317395961 555
106.119774882348 560
106.499428706175 565
106.879278360453 570
107.259323342158 575
107.63956315217 580
108.01999729522 585
108.400625279838 590
108.781446618297 595
109.162460826564 600
109.54366742425 605
109.925065934557 610
110.306655884233 615
110.688436803523 620
111.070408226121 625
111.452569689125 630
111.834920732994 635
112.217460901501 640
112.60018974169 645
112.983106803835 650
113.366211641397 655
113.749503810985 660
114.132982872313 665
114.516648388163 670
114.900499924344 675
115.284537049656 680
115.668759335853 685
116.053166357605 690
116.437757692462 695
116.822532920819 700
117.207491625884 705
117.592633393638 710
117.97795781281 715
118.363464474835 720
118.749152973828 725
119.135022906552 730
119.521073872384 735
119.907305473285 740
120.293717313774 745
120.680309000893 750
121.067080144181 755
121.454030355645 760
121.841159249733 765
122.228466443305 770
122.615951555607 775
123.003614208242 780
123.391454025148 785
123.779470632569 790
124.167663659031 795
124.556032735318 800
124.944577494445 805
125.333297571637 810
125.722192604302 815
126.111262232012 820
126.500506096476 825
126.889923841521 830
127.279515113065 835
127.669279559103 840
128.059216829677 845
128.44932657686 850
128.839608454736 855
129.230062119374 860
129.620687228815 865
130.011483443046 870
130.402450423985 875
130.79358783546 880
131.184895343189 885
131.576372614763 890
131.968019319628 895
132.359835129063 900
132.75181971617 905
133.143972755848 910
133.536293924781 915
133.92878290142 920
134.321439365963 925
134.714263000346 930
135.107253488217 935
135.500410514928 940
135.893733767515 945
136.287222934683 950
136.680877706793 955
137.074697775843 960
137.468682835456 965
137.862832580864 970
138.257146708894 975
138.651624917953 980
139.046266908016 985
139.441072380609 990
139.836041038796 995
140.231172587169 1000
140.626466731828 1005
141.021923180375 1010
141.417541641895 1015
141.813321826948 1020
142.209263447552 1025
142.605366217175 1030
143.001629850717 1035
143.398054064505 1040
143.794638576273 1045
144.191383105156 1050
144.588287371679 1055
144.985351097738 1060
145.382574006596 1065
145.77995582287 1070
146.177496272518 1075
146.575195082829 1080
146.973051982412 1085
147.371066701186 1090
147.769238970368 1095
148.167568522465 1100
148.566055091261 1105
148.964698411807 1110
149.363498220414 1115
149.762454254638 1120
150.161566253275 1125
150.560833956349 1130
150.960257105102 1135
151.359835441984 1140
151.759568710647 1145
152.159456655932 1150
152.559499023861 1155
152.959695561629 1160
153.360046017593 1165
153.760550141265 1170
154.161207683303 1175
154.5620183955 1180
154.962982030779 1185
155.364098343183 1190
155.765367087865 1195
156.166788021081 1200
156.568360900182 1205
156.970085483609 1210
157.371961530876 1215
157.773988802573 1220
158.176167060349 1225
158.578496066912 1230
158.980975586014 1235
159.383605382449 1240
159.786385222042 1245
160.189314871644 1250
160.592394099124 1255
160.995622673361 1260
161.399000364234 1265
161.802526942624 1270
162.206202180394 1275
162.610025850395 1280
163.013997726449 1285
163.418117583348 1290
163.822385196844 1295
164.226800343644 1300
164.631362801404 1305
165.03607234872 1310
165.440928765124 1315
165.845931831076 1320
166.251081327958 1325
166.656377038067 1330
167.061818744611 1335
167.467406231699 1340
167.87313928434 1345
168.279017688431 1350
168.685041230757 1355
169.091209698979 1360
169.497522881634 1365
169.903980568125 1370
170.310582548715 1375
170.717328614526 1380
171.124218557527 1385
171.531252170533 1390
171.938429247198 1395
172.345749582008 1400
172.753212970278 1405
173.160819208146 1410
173.568568092566 1415
173.976459421302 1420
174.384492992928 1425
174.792668606817 1430
175.200986063139 1435
175.609445162853 1440
176.018045707706 1445
176.426787500225 1450
176.835670343713 1455
177.244694042242 1460
177.653858400652 1465
178.063163224543 1470
178.472608320271 1475
178.882193494945 1480
179.291918556419 1485
179.701783313288 1490
180.111787574888 1495
180.521931151283 1500
180.932213853269 1505
181.342635492363 1510
181.753195880802 1515
182.163894831539 1520
182.574732158234 1525
182.985707675255 1530
183.396821197671 1535
183.808072541248 1540
184.219461522446 1545
184.630987958411 1550
185.042651666976 1555
185.454452466652 1560
185.866390176627 1565
186.278464616763 1570
186.690675607587 1575
187.10302297029 1580
187.515506526726 1585
187.928126099402 1590
188.340881511477 1595
188.753772586759 1600
189.166799149701 1605
189.579961025396 1610
189.993258039571 1615
190.40669001859 1620
190.820256789442 1625
191.233958179745 1630
191.647794017736 1635
192.061764132271 1640
192.475868352821 1645
192.890106509466 1650
193.304478432895 1655
193.718983954399 1660
194.133622905871 1665
194.5483951198 1670
194.963300429266 1675
195.378338667942 1680
195.793509670087 1685
196.20881327054 1690
196.624249304722 1695
197.039817608632 1700
197.455518018837 1705
197.871350372479 1710
198.287314507262 1715
198.703410261457 1720
199.119637473894 1725
199.535995983957 1730
199.952485631589 1735
200.369106257278 1740
200.785857702064 1745
201.202739807528 1750
201.619752415796 1755
202.036895369529 1760
202.454168511924 1765
202.871571686712 1770
203.289104738153 1775
203.706767511032 1780
204.124559850657 1785
204.54248160286 1790
204.960532613987 1795
205.378712730902 1800
205.797021800979 1805
206.215459672101 1810
206.634026192658 1815
207.052721211546 1820
207.471544578157 1825
207.890496142386 1830
208.309575754621 1835
208.728783265743 1840
209.148118527123 1845
209.56758139062 1850
209.987171708578 1855
210.406889333822 1860
210.826734119658 1865
211.246705919869 1870
211.666804588711 1875
212.087029980913 1880
212.507381951673 1885
212.927860356657 1890
213.348465051993 1895
213.769195894275 1900
214.190052740553 1905
214.611035448334 1910
215.032143875583 1915
215.453377880713 1920
215.87473732259 1925
216.296222060527 1930
216.71783195428 1935
217.139566864052 1940
217.561426650482 1945
217.98341117465 1950
218.405520298071 1955
218.827753882695 1960
219.250111790901 1965
219.672593885499 1970
220.095200029726 1975
220.517930087243 1980
220.940783922134 1985
221.363761398903 1990
221.786862382473 1995
222.210086738183 2000
222.633434331785 2005
223.056905029444 2010
223.480498697734 2015
223.904215203638 2020
224.328054414542 2025
224.752016198239 2030
225.176100422922 2035
225.600306957181 2040
226.024635670008 2045
226.449086430788 2050
226.873659109299 2055
227.298353575711 2060
227.723169700585 2065
228.148107354868 2070
228.573166409893 2075
228.998346737377 2080
229.423648209418 2085
229.849070698497 2090
230.274614077469 2095
230.700278219567 2100
231.126062998399 2105
231.551968287945 2110
231.977993962556 2115
232.40413989695 2120
232.830405966215 2125
233.256792045802 2130
233.683298011526 2135
234.109923739565 2140
234.536669106454 2145
234.963533989091 2150
235.390518264725 2155
235.817621810962 2160
236.244844505764 2165
236.672186227439 2170
237.099646854648 2175
237.527226266399 2180
237.954924342047 2185
238.382740961291 2190
238.810676004172 2195
239.238729351074 2200
239.666900882719 2205
240.095190480169 2210
240.523598024822 2215
240.952123398409 2220
241.380766482997 2225
241.809527160983 2230
242.238405315094 2235
242.667400828386 2240
243.096513584244 2245
243.525743466374 2250
243.955090358811 2255
244.384554145909 2260
244.814134712345 2265
245.243831943113 2270
245.673645723529 2275
246.103575939221 2280
246.533622476135 2285
246.96378522053 2290
247.394064058976 2295
247.824458878356 2300
248.25496956586 2305
248.685596008987 2310
249.116338095541 2315
249.547195713634 2320
249.978168751678 2325
250.40925709839 2330
250.840460642787 2335
251.271779274185 2340
251.703212882198 2345
252.134761356739 2350
252.566424588012 2355
252.998202466521 2360
253.430094883057 2365
253.862101728706 2370
254.294222894844 2375
254.726458273134 2380
255.158807755528 2385
255.591271234263 2390
256.023848601864 2395
256.456539751136 2400
256.889344575168 2405
257.32226296733 2410
257.755294821273 2415
258.188440030926 2420
258.621698490495 2425
259.055070094462 2430
259.488554737585 2435
259.922152314896 2440
260.355862721698 2445
260.789685853568 2450
261.22362160635 2455
261.65766987616 2460
262.091830559381 2465
262.526103552661 2470
262.960488752915 2475
263.394986057324 2480
263.829595363329 2485
264.264316568636 2490
264.69914957121 2495
265.134094269276 2500
265.56915056132 2505
266.004318346082 2510
266.439597522561 2515
266.874987990012 2520
267.310489647942 2525
267.746102396113 2530
268.181826134539 2535
268.617660763484 2540
269.053606183464 2545
269.489662295241 2550
269.925828999828 2555
270.362106198483 2560
270.79849379271 2565
271.234991684259 2570
271.671599775122 2575
272.108317967536 2580
272.545146163977 2585
272.982084267164 2590
273.419132180055 2595
273.856289805845 2600
274.29355704797 2605
274.730933810099 2610
275.168419996141 2615
275.606015510235 2620
276.043720256758 2625
276.481534140318 2630
276.919457065754 2635
277.357488938138 2640
277.795629662769 2645
278.233879145179 2650
278.672237291124 2655
279.110704006591 2660
279.549279197789 2665
279.987962771157 2670
280.426754633354 2675
280.865654691266 2680
281.304662851999 2685
281.743779022882 2690
282.183003111465 2695
282.622335025516 2700
283.061774673025 2705
283.501321962198 2710
283.940976801458 2715
284.380739099445 2720
284.820608765015 2725
285.260585707239 2730
285.7006698354 2735
286.140861058995 2740
286.581159287733 2745
287.021564431535 2750
287.462076400531 2755
287.902695105061 2760
288.343420455675 2765
288.784252363129 2770
289.225190738387 2775
289.66623549262 2780
290.107386537204 2785
290.548643783718 2790
290.990007143948 2795
291.431476529881 2800
291.873051853706 2805
292.314733027814 2810
292.756519964799 2815
293.19841257745 2820
293.640410778761 2825
294.082514481919 2830
294.524723600312 2835
294.967038047525 2840
295.409457737336 2845
295.851982583722 2850
296.294612500852 2855
296.73734740309 2860
297.180187204994 2865
297.623131821312 2870
298.066181166985 2875
298.509335157146 2880
298.952593707117 2885
299.395956732408 2890
299.839424148722 2895
300.282995871945 2900
300.726671818153 2905
301.170451903609 2910
301.614336044762 2915
302.058324158243 2920
302.502416160873 2925
302.946611969651 2930
303.390911501764 2935
303.835314674579 2940
304.279821405644 2945
304.724431612691 2950
305.16914521363 2955
305.613962126551 2960
306.058882269724 2965
306.503905561597 2970
306.949031920796 2975
307.394261266124 2980
307.83959351656 2985
308.28502859126 2990
308.730566409554 2995
309.176206890946 3000
309.621949955116 3005
310.067795521916 3010
310.51374351137 3015
310.959793843675 3020
311.4059464392 3025
311.852201218482 3030
312.298558102232 3035
312.745017011326 3040
313.191577866814 3045
313.638240589911 3050
314.085005101999 3055
314.531871324629 3060
314.978839179519 3065
315.425908588551 3070
315.873079473773 3075
316.320351757398 3080
316.767725361803 3085
317.215200209529 3090
317.662776223278 3095
318.110453325917 3100
318.558231440473 3105
319.006110490135 3110
319.454090398253 3115
319.902171088335 3120
320.350352484052 3125
320.798634509231 3130
321.247017087858 3135
321.695500144078 3140
322.144083602192 3145
322.592767386659 3150
323.041551422092 3155
323.490435633263 3160
323.939419945097 3165
324.388504282672 3170
324.837688571224 3175
325.286972736139 3180
325.736356702958 3185
326.185840397374 3190
326.63542374523 3195
327.085106672524 3200
327.534889105401 3205
327.98477097016 3210
328.434752193247 3215
328.884832701259 3220
329.335012420942 3225
329.785291279187 3230
330.235669203038 3235
330.686146119681 3240
331.136721956454 3245
331.587396640837 3250
332.038170100459 3255
332.489042263091 3260
332.940013056651 3265
333.391082409202 3270
333.842250248949 3275
334.293516504241 3280
334.74488110357 3285
335.196343975571 3290
335.64790504902 3295
336.099564252835 3300
336.551321516074 3305
337.003176767936 3310
337.455129937762 3315
337.907180955029 3320
338.359329749355 3325
338.811576250498 3330
339.263920388352 3335
339.71636209295 3340
340.16890129446 3345
340.621537923191 3350
341.074271909584 3355
341.527103184218 3360
341.980031677809 3365
342.433057321204 3370
342.886180045389 3375
343.339399781482 3380
343.792716460734 3385
344.24613001453 3390
344.699640374388 3395
345.153247471959 3400
345.606951239025 3405
346.0607516075 3410
346.514648509429 3415
346.968641876987 3420
347.422731642481 3425
347.876917738347 3430
348.33120009715 3435
348.785578651585 3440
349.240053334473 3445
349.694624078768 3450
350.149290817547 3455
350.604053484017 3460
351.058912011511 3465
351.51386633349 3470
351.968916383539 3475
352.424062095371 3480
352.879303402822 3485
353.334640239856 3490
353.790072540558 3495
354.245600239141 3500
354.701223269938 3505
355.156941567409 3510
355.612755066134 3515
356.068663700817 3520
356.524667406285 3525
356.980766117486 3530
357.436959769489 3535
357.893248297484 3540
358.349631636785 3545
358.806109722821 3550
359.262682491145 3555
359.719349877428 3560
360.176111817462 3565
360.632968247154 3570
361.089919102534 3575
361.546964319747 3580
362.004103835058 3585
362.461337584847 3590
362.918665505613 3595
363.376087533972 3600
363.833603606653 3605
364.291213660506 3610
364.748917632493 3615
365.206715459691 3620
365.664607079295 3625
366.122592428613 3630
366.580671445065 3635
};
\addlegendentry{\(P_{inj}\)=\qty{65}{\bar}}
\addplot [semithick, forestgreen4416044]
table {%
70 0
70.3658184465872 5
70.7318964508825 10
71.0982327475693 15
71.4648260918431 20
71.8316752587184 25
72.1987790423804 30
72.5661362555746 35
72.9337457290343 40
73.30160631094 45
73.6697168664097 50
74.0380762770174 55
74.4066834403345 60
74.7755372694976 65
75.1446366927985 70
75.5139806532897 75
75.8835681084111 80
76.2533980296357 85
76.6234694021271 90
76.9937812244106 95
77.3643325080639 100
77.735122277415 105
78.106149569254 110
78.4774134325554 115
78.8489129282106 120
79.2206471287706 125
79.5926151181964 130
79.964815991619 135
80.3372488551074 140
80.7099128254434 145
81.0828070299044 150
81.455930606052 155
81.8292827015275 160
82.2028624738538 165
82.5766690902424 170
82.9507017274065 175
83.3249595713787 180
83.6994418173341 185
84.0741476694184 190
84.44907634058 195
84.8242270524071 200
85.1995990349685 205
85.5751915266586 210
85.951003774046 215
86.3270350317277 220
86.7032845621842 225
87.0797516356392 230
87.4564355299233 235
87.8333355303405 240
88.2104509295373 245
88.5877810273758 250
88.965325130809 255
89.3430825537592 260
89.7210526169995 265
90.0992346480374 270
90.4776279810015 275
90.8562319565303 280
91.2350459216639 285
91.6140692297384 290
91.9933012402805 295
92.3727413189071 300
92.7523888372261 305
93.1322431727388 310
93.5123037087456 315
93.8925698342516 320
94.2730409438766 325
94.6537164377654 330
95.0345957215005 335
95.4156782060168 340
95.7969633075181 345
96.178450447395 350
96.5601390521447 355
96.9420285532924 360
97.3241183873142 365
97.7064079955622 370
98.0888968241903 375
98.471584324082 380
98.8544699507796 385
99.2375531644146 390
99.6208334296398 395
100.004310215563 400
100.387982995681 405
100.771851247816 410
101.155914454053 415
101.540172100678 420
101.924623678117 425
102.30926868088 430
102.6941066075 435
103.079136960474 440
103.464359246216 445
103.849772974991 450
104.235377660872 455
104.62117282168 460
105.007157978936 465
105.39333265781 470
105.77969638707 475
106.166248699037 480
106.552989129532 485
106.939917217833 490
107.327032506631 495
107.714334541979 500
108.101822873251 505
108.489497053101 510
108.877356637417 515
109.265401185279 520
109.653630258921 525
110.04204342369 530
110.430640248002 535
110.819420303311 540
111.208383164063 545
111.597528407666 550
111.986855614446 555
112.376364367617 560
112.766054253242 565
113.1559248602 570
113.545975780152 575
113.936206607503 580
114.326616939377 585
114.717206375579 590
115.107974518561 595
115.4989209734 600
115.890045347757 605
116.281347251852 610
116.672826298436 615
117.064482102755 620
117.45631428253 625
117.848322457921 630
118.240506251506 635
118.632865288247 640
119.02539919547 645
119.418107602834 650
119.810990142305 655
120.204046448133 660
120.597276156828 665
120.990678907129 670
121.384254339988 675
121.77800209854 680
122.171921828081 685
122.566013176047 690
122.96027579199 695
123.354709327553 700
123.749313436453 705
124.144087774456 710
124.539031999356 715
124.934145770955 720
125.32942875104 725
125.724880603367 730
126.120500993636 735
126.516289589476 740
126.912246060421 745
127.308370077893 750
127.704661315186 755
128.101119447442 760
128.497744151636 765
128.894535106557 770
129.291491992792 775
129.688614492706 780
130.085902290426 785
130.483355071825 790
130.880972524502 795
131.278754337769 800
131.676700202635 805
132.074809811785 810
132.473082859572 815
132.871519041992 820
133.270118056679 825
133.66887960288 830
134.067803381448 835
134.466889094823 840
134.866136447018 845
135.265545143605 850
135.665114891702 855
136.064845399957 860
136.464736378537 865
136.864787539112 870
137.264998594842 875
137.665369260366 880
138.065899251784 885
138.466588286652 890
138.86743608396 895
139.268442364129 900
139.669606848991 905
140.07092926178 910
140.472409327121 915
140.874046771015 920
141.275841320832 925
141.677792705294 930
142.079900654467 935
142.482164899749 940
142.88458517386 945
143.287161210828 950
143.68989274598 955
144.092779515932 960
144.495821258578 965
144.899017713078 970
145.30236861985 975
145.705873720557 980
146.109532758099 985
146.513345476604 990
146.917311621416 995
147.321430939084 1000
147.725703177356 1005
148.130128085169 1010
148.534705412637 1015
148.939434911042 1020
149.344316332828 1025
149.74934943159 1030
150.154533962063 1035
150.559869680117 1040
150.965356342745 1045
151.370993708057 1050
151.776781535268 1055
152.182719584694 1060
152.588807617741 1065
152.995045396894 1070
153.401432685716 1075
153.807969248833 1080
154.214654851929 1085
154.621489261739 1090
155.028472246039 1095
155.435603573639 1100
155.842883014376 1105
156.250310339106 1110
156.657885319696 1115
157.065607729018 1120
157.473477340939 1125
157.881493930317 1130
158.289657272992 1135
158.697967145778 1140
159.106423326459 1145
159.515025593777 1150
159.923773727432 1155
160.33266750807 1160
160.741706717275 1165
161.150891137569 1170
161.5602205524 1175
161.969694746135 1180
162.379313504058 1185
162.789076612359 1190
163.19898385813 1195
163.609035029359 1200
164.019229914923 1205
164.429568304581 1210
164.840049988969 1215
165.250674759594 1220
165.661442408828 1225
166.072352729902 1230
166.4834055169 1235
166.894600564751 1240
167.305937669228 1245
167.71741662694 1250
168.129037235323 1255
168.540799292641 1260
168.952702597974 1265
169.364746951217 1270
169.776932153074 1275
170.189258005049 1280
170.601724309446 1285
171.014330869359 1290
171.427077488669 1295
171.83996397204 1300
172.252990124911 1305
172.666155753494 1310
173.079460664765 1315
173.492904666464 1320
173.906487567086 1325
174.320209175876 1330
174.734069302829 1335
175.148067758679 1340
175.562204354898 1345
175.976478903692 1350
176.390891217991 1355
176.80544111145 1360
177.220128398444 1365
177.634952894059 1370
178.049914414092 1375
178.465012775043 1380
178.880247794115 1385
179.295619289205 1390
179.711127078902 1395
180.126770982482 1400
180.542550819906 1405
180.95846641181 1410
181.374517579507 1415
181.790704144981 1420
182.20702593088 1425
182.623482760515 1430
183.040074457856 1435
183.456800847525 1440
183.873661754795 1445
184.290657005585 1450
184.707786426455 1455
185.125049844606 1460
185.542447087869 1465
185.959977984709 1470
186.377642364216 1475
186.795440056103 1480
187.213370890701 1485
187.631434698959 1490
188.049631312435 1495
188.467960563296 1500
188.886422284314 1505
189.305016308862 1510
189.723742470908 1515
190.142600605015 1520
190.561590546338 1525
190.980712130616 1530
191.399965194173 1535
191.819349573911 1540
192.23886510731 1545
192.658511632424 1550
193.078288987873 1555
193.498197012848 1560
193.918235547099 1565
194.338404430939 1570
194.758703505236 1575
195.179132611412 1580
195.599691591438 1585
196.020380287834 1590
196.441198543663 1595
196.862146202529 1600
197.283223108573 1605
197.704429106471 1610
198.125764041432 1615
198.547227759191 1620
198.96882010601 1625
199.390540928675 1630
199.81239007449 1635
200.234367391276 1640
200.656472727367 1645
201.078705931611 1650
201.501066853361 1655
201.923555342477 1660
202.34617124932 1665
202.768914424753 1670
203.191784720133 1675
203.614781987315 1680
204.037906078641 1685
204.461156846946 1690
204.884534145547 1695
205.308037828249 1700
205.731667749332 1705
206.15542376356 1710
206.579305726168 1715
207.003313492866 1720
207.427446919834 1725
207.85170586372 1730
208.276090181637 1735
208.700599731161 1740
209.125234370327 1745
209.549993957629 1750
209.974878352017 1755
210.399887412892 1760
210.825021000106 1765
211.250278973961 1770
211.6756611952 1775
212.101167525015 1780
212.526797825036 1785
212.95255195733 1790
213.378429784403 1795
213.804431169194 1800
214.230555975073 1805
214.656804065841 1810
215.083175305725 1815
215.509669559377 1820
215.936286691871 1825
216.363026568702 1830
216.789889055785 1835
217.216874019449 1840
217.643981326437 1845
218.071210843906 1850
218.498562439419 1855
218.926035980949 1860
219.353631336875 1865
219.781348375979 1870
220.209186967442 1875
220.637146980847 1880
221.065228286173 1885
221.493430753795 1890
221.92175425448 1895
222.350198659387 1900
222.778763840064 1905
223.207449668446 1910
223.636256016853 1915
224.065182757991 1920
224.494229764942 1925
224.923396911173 1930
225.352684070525 1935
225.782091117217 1940
226.211617925839 1945
226.641264371356 1950
227.071030329101 1955
227.500915674776 1960
227.930920284449 1965
228.361044034554 1970
228.791286801885 1975
229.2216484636 1980
229.652128897215 1985
230.082727980603 1990
230.513445591993 1995
230.944281609969 2000
231.375235913464 2005
231.806308381766 2010
232.237498894508 2015
232.668807331673 2020
233.100233573588 2025
233.531777500923 2030
233.963438994691 2035
234.395217936246 2040
234.827114207279 2045
235.259127689821 2050
235.691258266235 2055
236.123505819221 2060
236.55587023181 2065
236.988351387364 2070
237.420949169573 2075
237.853663462456 2080
238.286494150358 2085
238.719441117948 2090
239.152504250217 2095
239.585683432481 2100
240.018978550371 2105
240.452389489841 2110
240.885916137158 2115
241.319558378907 2120
241.753316101985 2125
242.187189193605 2130
242.621177541286 2135
243.055281032861 2140
243.489499556469 2145
243.923833000555 2150
244.358281253872 2155
244.792844205474 2160
245.227521744719 2165
245.662313761266 2170
246.097220145073 2175
246.532240786397 2180
246.967375575791 2185
247.402624404106 2190
247.837987162483 2195
248.273463742361 2200
248.709054035465 2205
249.144757933816 2210
249.58057532972 2215
250.016506115771 2220
250.45255018485 2225
250.888707430124 2230
251.324977745042 2235
251.761361023336 2240
252.19785715902 2245
252.634466046387 2250
253.071187580009 2255
253.508021654735 2260
253.944968165691 2265
254.382027008277 2270
254.819198078167 2275
255.256481271309 2280
255.69387648392 2285
256.131383612489 2290
256.569002553773 2295
257.006733204795 2300
257.444575462849 2305
257.882529225491 2310
258.320594390542 2315
258.758770856085 2320
259.197058520468 2325
259.635457282296 2330
260.073967040437 2335
260.512587694014 2340
260.951319142412 2345
261.390161285268 2350
261.829114022476 2355
262.268177254184 2360
262.707350880793 2365
263.146634802957 2370
263.586028921577 2375
264.025533137808 2380
264.465147353053 2385
264.904871468959 2390
265.344705387424 2395
265.78464901059 2400
266.224702240842 2405
266.66486498081 2410
267.105137133366 2415
267.545518601623 2420
267.986009288935 2425
268.426609098895 2430
268.867317935334 2435
269.308135702321 2440
269.749062304162 2445
270.190097645395 2450
270.631241630796 2455
271.072494165374 2460
271.513855154368 2465
271.95532450325 2470
272.396902117723 2475
272.838587903719 2480
273.280381767399 2485
273.72228361515 2490
274.164293353588 2495
274.606410889553 2500
275.04863613011 2505
275.49096898255 2510
275.933409354384 2515
276.375957153347 2520
276.818612287395 2525
277.261374664703 2530
277.704244193668 2535
278.147220782902 2540
278.590304341238 2545
279.033494777724 2550
279.476792001623 2555
279.920195922415 2560
280.363706449793 2565
280.807323493664 2570
281.251046964146 2575
281.694876771569 2580
282.138812826476 2585
282.582855039616 2590
283.02700332195 2595
283.471257584646 2600
283.91561773908 2605
284.360083696833 2610
284.804655369694 2615
285.249332669656 2620
285.694115508915 2625
286.139003799872 2630
286.583997455129 2635
287.02909638749 2640
287.474300509962 2645
287.919609735749 2650
288.365023978257 2655
288.810543151089 2660
289.256167168046 2665
289.701895943126 2670
290.147729390524 2675
290.59366742463 2680
291.039709960029 2685
291.485856911499 2690
291.932108194012 2695
292.378463722734 2700
292.82492341302 2705
293.271487180418 2710
293.718154940665 2715
294.164926609689 2720
294.611802103606 2725
295.058781338719 2730
295.505864231521 2735
295.953050698689 2740
296.400340657089 2745
296.847734023769 2750
297.295230715964 2755
297.742830651092 2760
298.190533746754 2765
298.638339920734 2770
299.086249090997 2775
299.534261175691 2780
299.982376093142 2785
300.430593761857 2790
300.878914100522 2795
301.327337028003 2800
301.77586246334 2805
302.224490325755 2810
302.673220534641 2815
303.122053009572 2820
303.570987670294 2825
304.020024436729 2830
304.469163228971 2835
304.91840396729 2840
305.367746572126 2845
305.817190964092 2850
306.266737063972 2855
306.716384792723 2860
307.166134071468 2865
307.615984821503 2870
308.065936964291 2875
308.515990421464 2880
308.966145114821 2885
309.416400966328 2890
309.866757898118 2895
310.31721583249 2900
310.767774691907 2905
311.218434398998 2910
311.669194876555 2915
312.120056047534 2920
312.571017835054 2925
313.022080162395 2930
313.473242953001 2935
313.924506130475 2940
314.375869618581 2945
314.827333341244 2950
315.278897222547 2955
315.730561186733 2960
316.182325158203 2965
316.634189061515 2970
317.086152821384 2975
317.538216362684 2980
317.990379610441 2985
318.442642489841 2990
318.895004926222 2995
319.347466845077 3000
319.800028172052 3005
320.25268883295 3010
320.705448753722 3015
321.158307860474 3020
321.611266079464 3025
322.0643233371 3030
322.517479559941 3035
322.970734674696 3040
323.424088608225 3045
323.877541287535 3050
324.331092639784 3055
324.784742592276 3060
325.238491072463 3065
325.692338007946 3070
326.14628332647 3075
326.600326955928 3080
327.054468824357 3085
327.508708859942 3090
327.963046991008 3095
328.417483146028 3100
328.872017253618 3105
329.326649242535 3110
329.781379041682 3115
330.2362065801 3120
330.691131786976 3125
331.146154591636 3130
331.601274923546 3135
332.056492712315 3140
332.511807887688 3145
332.967220379553 3150
333.422730117935 3155
333.878337032997 3160
334.334041055042 3165
334.789842114508 3170
335.245740141971 3175
335.701735068145 3180
336.157826823879 3185
336.614015340156 3190
337.070300548098 3195
337.526682378958 3200
337.983160764125 3205
338.439735635122 3210
338.896406923606 3215
339.353174561366 3220
339.810038480324 3225
340.266998612535 3230
340.724054890182 3235
341.181207245585 3240
341.638455611191 3245
342.095799919578 3250
342.553240103455 3255
343.01077609566 3260
343.46840782916 3265
343.926135237052 3270
344.383958252558 3275
344.841876809033 3280
345.299890839955 3285
345.758000278931 3290
346.216205059696 3295
346.674505116109 3300
347.132900382156 3305
347.591390791947 3310
348.049976279721 3315
348.508656779837 3320
348.967432226782 3325
349.426302555164 3330
349.885267699716 3335
350.344327595295 3340
350.803482176878 3345
351.262731379568 3350
351.722075138586 3355
352.181513389279 3360
352.641046067111 3365
353.100673107669 3370
353.560394446659 3375
354.020210019911 3380
354.480119763368 3385
354.9401236131 3390
355.400221505289 3395
355.860413376239 3400
356.320699162373 3405
356.781078800231 3410
357.241552226468 3415
357.70211937786 3420
358.162780191297 3425
358.623534603787 3430
359.084382552453 3435
359.545323974533 3440
360.006358807384 3445
360.467486988473 3450
360.928708455384 3455
361.390023145817 3460
361.851430997582 3465
362.312931948606 3470
362.774525936927 3475
363.236212900697 3480
363.69799277818 3485
364.159865507753 3490
364.621831027904 3495
365.083889277233 3500
365.54604019445 3505
366.008283718377 3510
366.470619787948 3515
366.933048342203 3520
367.395569320296 3525
367.858182661488 3530
368.32088830515 3535
368.783686190761 3540
369.246576257911 3545
369.709558446296 3550
370.17263269572 3555
370.635798946095 3560
371.09905713744 3565
371.562407209882 3570
372.025849103652 3575
372.489382759091 3580
372.953008116643 3585
373.416725116859 3590
373.880533700394 3595
374.34443380801 3600
374.808425380572 3605
375.272508359051 3610
375.736682684521 3615
376.200948298159 3620
376.665305141248 3625
377.129753155173 3630
377.59429228142 3635
378.058922461581 3640
378.523643637347 3645
378.988455750513 3650
379.453358742977 3655
379.918352556734 3660
380.383437133885 3665
380.848612416629 3670
381.313878347266 3675
381.779234868196 3680
382.24468192192 3685
382.710219451037 3690
383.175847398248 3695
383.641565706351 3700
384.107374318243 3705
384.57327317692 3710
385.039262225477 3715
385.505341407104 3720
385.971510665093 3725
386.437769942829 3730
386.904119183798 3735
387.370558331581 3740
387.837087329856 3745
388.303706122395 3750
388.770414653071 3755
389.237212865848 3760
389.704100704787 3765
390.171078114046 3770
390.638145037876 3775
391.105301420624 3780
391.572547206729 3785
392.039882340727 3790
392.507306767246 3795
392.974820431008 3800
393.442423276831 3805
393.910115249621 3810
394.377896294381 3815
394.845766356206 3820
395.313725380281 3825
395.781773311886 3830
396.249910096391 3835
396.718135679257 3840
397.186450006039 3845
397.654853022381 3850
398.123344674018 3855
398.591924906776 3860
399.060593666571 3865
399.52935089941 3870
399.998196551387 3875
400.46713056869 3880
400.936152897592 3885
401.405263484458 3890
401.87446227574 3895
402.343749217981 3900
402.813124257809 3905
403.282587341943 3910
403.752138417188 3915
404.221777430438 3920
404.691504328674 3925
405.161319058963 3930
405.63122156846 3935
406.101211804408 3940
406.571289714133 3945
407.041455245051 3950
407.511708344661 3955
407.982048960549 3960
408.452477040387 3965
408.922992531932 3970
409.393595383026 3975
409.864285541594 3980
410.33506295565 3985
410.805927573287 3990
411.276879342687 3995
};
\addlegendentry{\(P_{inj}\)=\qty{70}{\bar}}
\end{axis}

\end{tikzpicture}

    %         \caption{The static pressure with depth of \ac{NCG} for a range of injection wellhead pressures.}
    %         \label{fig:prosim_NCG_Reinjection_PvD_by_Pinj}
    %     \end{figure}

    %     Under the previously used boundary conditions (i.e. \(P_{geo}^{out} = P_{geo}^{in}\)) power plant discharge pressures of around \qtyrange{10}{20}{\bar} could be expected - however this would only permit injection of \ac{NCG} to depths of a few hundred meters. While the \ac{NCG} remains in its vapour state, even at elevated pressures (e.g. \qty{60}{\bar} only relatively shallow depths of around \qty{650}{\m} can be reached.

    %     \begin{notes}{Note:}
    %         For low geofluid \ac{NCG} contents, it can be thermodynamically possible to inject the \ac{NCG} at these shallow depths, and for the \ac{NCG} to fully dissolve in the geofluid. However, here it is also important to consider the rate of mass transfer, as the dissolution of the \ac{NCG} with in the geofluid, must take place before the \ac{NCG} coalesce into bubbles large enough for buoyancy effects to manifest, which could lead to well unloading and a potential catastrophic blow-out event.

    %         To prevent this, large mass transfer coefficients and mass transfer area are required. 
    %     \end{notes}

    %     However, as the injection temperature (\qty{298}{\K} or \qty{25}{\degreeCelsius}) is below the critical temperature of \ce{CO2} (\qty{304.1}{\K} or \qty{31}{\degreeCelsius}), and as such it is possible to liquefy to \ce{CO2} at sufficiently high pressures, in this case, at pressures in excess of \qty{64.1}{\bar}. Liquefaction has several advantages over gaseous \ce{CO2}, since liquid \ce{CO2} has a similar density to water thus allowing deeper injection depths to be obtained, and owing to its low compressibility pressurisation to even higher pressures is energetically cheap. With pressures of \qty{70}{\bar} injection depths in excess of \qty{4000}{\m} can be obtained.
        
    %     \begin{figure}[H]
    %         \centering
    %         % This file was created with tikzplotlib v0.10.1.
\begin{tikzpicture}

\definecolor{darkgray176}{RGB}{176,176,176}
\definecolor{steelblue31119180}{RGB}{31,119,180}

\begin{axis}[
tick align=outside,
tick pos=left,
unbounded coords=jump,
x grid style={darkgray176},
xlabel={Reservoir Depth/\unit{\m}},
xmin=-187.5, xmax=4000,
xtick style={color=black},
xtick distance=500,
y grid style={darkgray176},
ylabel={Minimum Wellhead Pressure/\unit{\bar}},
ymin=10, ymax=70,
ytick style={color=black},
ytick distance=10,
width=12cm, height=6.5cm
]
\addplot [semithick, steelblue31119180]
table {%
0 5
0 8.10636014739756
20 11.2127202947951
50 14.3190804421927
80 17.4254405895902
120 20.5318007369878
150 23.6381608843854
190 26.7445210317829
220 29.8508811791805
260 32.957241326578
300 36.0636014739756
330 39.1699616213732
370 42.2763217687707
410 45.3826819161683
460 48.4890420635659
500 51.5954022109634
550 54.701762358361
610 57.8081225057585
670 60.9144826531561
760 64.0208428005537
nan nan
3360 64.2208428005536
3750 65.2682112216063
nan 66.3155796426589
nan 67.3629480637115
nan 68.4103164847642
nan 69.4576849058168
nan 70.5050533268694
nan 71.5524217479221
nan 72.5997901689747
nan 73.6471585900273
nan 74.69452701108
nan 75.7418954321326
nan 76.7892638531852
nan 77.8366322742379
nan 78.8840006952905
nan 79.9313691163431
nan 80.9787375373957
nan 82.0261059584484
nan 83.073474379501
nan 84.1208428005536
};
\end{axis}

\end{tikzpicture}

    %         \caption{The minimum wellhead pressure required to inject \ac{NCG} at a given depth against a static column of water with a header pressure of \qty{10}{\bar}.}
    %         \label{fig:prosim_NCG_Reinjection_Pinj_v_D}
    %     \end{figure}

    \subsubsection{Boundary Conditions}
        \label{sec:prosim_NCG_reinj}
        The boundary conditions used for this study resemble those used in the previous study, the only difference being the geofluid outlet pressure, which is assumed to be \qty{75}{\bar} for all outlet geofluid streams (i.e. brine, condensate and \ac{NCG}).

        \begin{table}[H]
            \centering
            \caption{The boundary conditions used for the single flash \ac{DSC} and the binary \ac{ORC} geothermal power plants.}
            \label{table:NCG_Reinj_BCs}
            \begin{tabular}{| c | c |}
    \hline
    \rowcolor{bluepoli!40} % comment this line to remove the color
    \textbf{Condition} & \textbf{Values} \T\B \\
    \hline \hline
    \ce{CO2} Content & \qty{0}{\mol\percent}\(\leq z_{NCG,\;in}\leq\)\qty{15}{\mol\percent} \T\B \\
    Inlet Temperature & \qty{473}{\K} \T\B \\
    Inlet Steam Quality & \qty{25}{\percent} \T\B \\
    Inlet Pressure & calculated \T\B \\
    Brine/Condensate/NCG Outlet Pressure &  \qty{75}{\bar} \T\B \\
    NCG Liquefaction Temperature &  \qty{300}{\K} \T\B \\
    Working Fluid &  n-Butane \T\B \\
    \hline
\end{tabular}        
        \end{table}

    \subsubsection{Plant Configurations}
    \label{sec:NCG_reinjection_plant_config}
        The power plants are extended with \ac{NCG} handling facilities: the vapour and liquid streams are separated and then separately pressurised to the target outlet pressure of \qty{75}{\bar}, Figure~\ref{fig:prosim_NCG_reinjection}. The brine and condensate streams (\ac{DSC} only) are pressurised with pumps, and the \ac{NCG} stream is pressurised to its saturation pressure at \qty{298}{\K}, around \qty{64.1}{\bar}, before liquefied and then pressurised to the target pressure.

        \begin{figure}[H]
            \centering
            \begin{tikzpicture}
    % draw equipment

    \pic (producer) at (0,0) {producer};
    \pic (injector) at (9.5,0) {injector};
    \pic[scale=0.6, yscale=2] (pplant) at ($(producer-top) + (1.5, 3)$) {block};

    \pic (Bpump) at ($(pplant-bottom right) + (2, -2)$) {centrifugal pump};

    \pic[scale=0.4] (NCGsep) at ($(pplant-top right) + (2.5, 0)$) {gas-liquid separator};
    
    \pic (Cpump) at ($(NCGsep-liquid outlet) + (1.5, -1)$) {centrifugal pump};
    
    \pic (NCGcomp) at ($(NCGsep-gas outlet) + (1, 1)$) {compressor};
    \pic[rotate=90] (NCGcond) at ($(NCGcomp-outlet bottom) + (1.5, 0)$) {heat exchanger biphase};
    \pic (NCGpump) at ($(NCGcond-shell bottom) + (1.2, 0)$) {centrifugal pump};

    \pic[rotate=90] (Ljoint) at ($(injector-top) + (-0.2, 0.7)$) {valve triple=main};

    % draw connectors
    \draw[main stream] (producer-top) |- (pplant-left);
    \draw[main stream] (pplant-top right) -- (NCGsep-inlet left);

    \draw[main stream] (pplant-bottom right) -| ($0.75*(pplant-bottom right) + 0.25*(Bpump-anchor)$) |- (Bpump-anchor);
    
    \draw[main stream] (Bpump-top) -- (Ljoint-top);
    \draw[main stream] (Cpump-top) -| (Ljoint-right);
    
    \draw[main stream] (Ljoint-left) -- ($(injector-top) + (-0.2, 0)$);
    
    \draw[main stream] (NCGsep-liquid outlet) |- (Cpump-anchor);
    \draw[main stream] (NCGsep-gas outlet) |- (NCGcomp-inlet top);
    
    \draw[main stream] (NCGcomp-outlet bottom) -- (NCGcond-shell top);
    \draw[main stream] (NCGcond-shell bottom) -- (NCGpump-anchor);
    \draw[main stream] (NCGpump-top) -| ++(1.5, -3) -| ($(injector-top) + (0.2, 0)$);

    \draw[main stream] ($(NCGcond-pipes bottom) + (0,1)$) -- (NCGcond-pipes bottom);
    \draw[main stream] (NCGcond-pipes top) -- ($(NCGcond-pipes top) + (0,1.75)$);


    % draw labels
    \node[below] at (producer-bottom) {Producer};
    \node[below] at (injector-bottom) {Injector};
    \node[align=center] at (pplant-anchor) {Power\\Plant};

    \node[below] at (Bpump-bottom) {Brine Pump};
    \node[below] at (Cpump-bottom) {Condensate Pump};

    \node[below right] at (NCGsep-inlet right) {NCG Sep.};
    \node[above] at (NCGcomp-inlet top) {NCG Compr.};
    \node[below] at (NCGcond-shell left) {NCG Liquefier};
    \node[above] at (NCGpump-top) {NCG pump};

    \node[below right, align=left, font=\footnotesize] at ($(pplant-top right) - (0, 0.1)$) {From DSC\textsuperscript{2}};
    \node[above right, align=left, font=\footnotesize] at ($(pplant-top right) + (0, 0.1)$) {From ORC\textsuperscript{1}};
    \node[below left, align=left, font=\footnotesize] at ($(pplant-bottom right) - (-0.4, 0.75)$) {DSC only\textsuperscript{3}};

    \node[above right] at ($(NCGcond-pipes bottom) + (0,1)$) {Coolant In};
    \node[above] at ($(NCGcond-pipes top) + (0,1.75)$) {Coolant Out};
\end{tikzpicture}
            \caption[Binary \ac{ORC} or single flash \ac{DSC} geothermal power plants with \ac{NCG} reinjection into the reservoir.]{Binary \ac{ORC} or single flash \ac{DSC} geothermal power plants with \ac{NCG} reinjection into the reservoir. \textsuperscript{1}Geofluid from \ac{PHE}. \textsuperscript{2}Condensate and \ac{NCG} from Condenser. \textsuperscript{3}Brine from flash separator.}
            \label{fig:prosim_NCG_reinjection}
        \end{figure}

    \subsubsection{Optimisation Configuration}
        The power plants were optimised for the net electrical power based on the optimisation configuration used in the previous study, Section~\ref{sec:NCG_Venting_opt_config}.

    \subsubsection{Results}
        Re-pressurisation of \ac{NCG} for re-injection into the reservoir, significantly reduces the net electrical power of the \ac{DSC} plant, with it just about breaking even at a geofluid \ce{CO2} content of \qty{15}{\percent} (i.e. a \qty{100}{\percent} reduction compared to the pure water case), Figure~\ref{fig:prosim_NCG_Reinjection_Wnet}. The net electrical power of the \ac{ORC}, decreases by about \qty{20}{\percent} over the full range of \ce{CO2} contents studied. In both cases, the reduction can be seen to be driven entirely by the power requirements of the \ac{NCG} re-pressurisation, Figure~\ref{fig:prosim_NCG_Reinjection_Power_Breakdown}
    
        \begin{figure}[H]
            \centering
            % This file was created with tikzplotlib v0.10.1.
\begin{tikzpicture}

\definecolor{burlywood253194140}{RGB}{253,194,140}
\definecolor{chocolate2369815}{RGB}{236,98,15}
\definecolor{darkgray176}{RGB}{176,176,176}
\definecolor{lightblue182212233}{RGB}{182,212,233}
\definecolor{lightgray204}{RGB}{204,204,204}
\definecolor{midnightblue848107}{RGB}{8,48,107}
\definecolor{saddlebrown127394}{RGB}{127,39,4}
\definecolor{steelblue59139194}{RGB}{59,139,194}

\begin{groupplot}[group style={group size=1 by 2}]
\nextgroupplot[
legend cell align={left},
legend style={
  fill opacity=0.8,
  draw opacity=1,
  text opacity=1,
  at={(0.97,0.03)},
  anchor=south east,
  draw=lightgray204
},
tick align=outside,
tick pos=left,
title={n-Butane},
x grid style={darkgray176},
xlabel={Drilling Cost/\unit{\mega\USD\of{2023}}},
xmin=0, xmax=63,
xtick style={color=black},
y grid style={darkgray176},
ylabel={Net electrical power/\unit{\mega\watt}},
ymin=0, ymax=9,
ytick style={color=black}
]
\addplot [semithick, burlywood253194140]
table {%
0 6.52143371948183
1 6.52143371948183
2 6.52143371948183
4 6.52143371948183
8 6.52143371948183
12 6.52143371948183
16 6.52143371948183
20 6.52143371948183
25 6.52143371948183
30 6.52143371948183
35 6.52143371948183
40 6.52143371948183
50 6.52143371948183
60 6.52143371948183
};
\addlegendentry{Thermodynamic Opt.}
\addplot [semithick, saddlebrown127394]
table {%
0 3.35312380103824
1 3.35312380103824
2 3.35312380103824
4 3.35312380103824
8 3.35312380103824
12 3.35312380103824
16 3.35312380103824
20 3.35312380103824
25 3.35312380103824
30 3.35312380103824
35 3.35312380103824
40 3.35312380103824
50 3.35312380103824
60 3.35312380103824
};
\addlegendentry{Techno-economic Opt. (excluding drilling costs)}
\addplot [semithick, chocolate2369815]
table {%
0 3.41586816872176
1 3.45484599362514
2 3.59827794491314
4 5.44747672618843
8 5.71741855543081
12 6.13905459187174
16 6.22879829708411
20 6.33475894364317
25 6.35406610908275
30 6.39668208743175
35 6.4680377937982
40 6.40986537612564
50 6.50452302914855
60 6.47623187635999
};
\addlegendentry{Techno-economic Opt. (including drilling costs)}

\nextgroupplot[
legend cell align={left},
legend style={fill opacity=0.8, draw opacity=1, text opacity=1, draw=lightgray204},
tick align=outside,
tick pos=left,
title={Cyclopentane},
x grid style={darkgray176},
xlabel={Drilling Cost/\unit{\mega\USD\of{2023}}},
xmin=0, xmax=63,
xtick style={color=black},
y grid style={darkgray176},
ylabel={Net electrical power/\unit{\mega\watt}},
ymin=0, ymax=9,
ytick style={color=black}
]
\addplot [semithick, lightblue182212233]
table {%
0 8.67312917722771
1 8.67312917722771
2 8.67312917722771
4 8.67312917722771
8 8.67312917722771
12 8.67312917722771
16 8.67312917722771
20 8.67312917722771
25 8.67312917722771
30 8.67312917722771
35 8.67312917722771
40 8.67312917722771
50 8.67312917722771
60 8.67312917722771
};
\addlegendentry{Thermodynamic Opt.}
\addplot [semithick, midnightblue848107]
table {%
0 3.24768016868321
1 3.24768016868321
2 3.24768016868321
4 3.24768016868321
8 3.24768016868321
12 3.24768016868321
16 3.24768016868321
20 3.24768016868321
25 3.24768016868321
30 3.24768016868321
35 3.24768016868321
40 3.24768016868321
50 3.24768016868321
60 3.24768016868321
};
\addlegendentry{Techno-economic Opt. (excluding drilling costs)}
\addplot [semithick, steelblue59139194]
table {%
0 3.14095734400058
1 6.21228907960342
2 6.29436173932615
4 6.55325520823824
8 6.84360915751591
12 7.168577206856
16 7.28400752660467
20 7.26659096414527
25 7.38675295828168
30 7.51110970957674
35 7.43616663970955
40 7.57638400329378
50 7.75201144638119
60 8.36888127426749
};
\addlegendentry{Techno-economic Opt. (including drilling costs)}
\end{groupplot}

\end{tikzpicture}

            \caption[The net electrical power of thermodynamically optimised binary \ac{ORC} and single flash \ac{DSC} geothermal power plants re-injecting \ac{NCG} into the reservoir.]{The net electrical power of thermodynamically optimised binary \ac{ORC} (using n-Butane as the working fluid) and single flash \ac{DSC} geothermal power plants with \ac{NCG} re-pressurisation to \qty{75}{\bar} for re-injection into the reservoir.}
            \label{fig:prosim_NCG_Reinjection_Wnet}
        \end{figure}

        \begin{figure}[H]
            \centering
            % This file was created with tikzplotlib v0.10.1.
\begin{tikzpicture}

\definecolor{crimson2143940}{RGB}{214,39,40}
\definecolor{darkgray176}{RGB}{176,176,176}
\definecolor{darkorange25512714}{RGB}{255,127,14}
\definecolor{forestgreen4416044}{RGB}{44,160,44}
\definecolor{lightgray204}{RGB}{204,204,204}
\definecolor{steelblue31119180}{RGB}{31,119,180}

\begin{groupplot}[group style={group size=1 by 2}]
\nextgroupplot[
legend cell align={left},
legend style={
  fill opacity=0.8,
  draw opacity=1,
  text opacity=1,
  at={(0.5,0.09)},
  anchor=south,
  draw=lightgray204
},
tick align=outside,
tick pos=left,
title={Binary ORC},
x grid style={darkgray176},
xlabel={Geofluid \ce{CO2} content/\unit{\mol\percent}},
xmin=0, xmax=16,
xtick style={color=black},
y grid style={darkgray176},
ylabel={Net electrical power /\unit{\mega\watt}},
ymin=-10, ymax=10,
ytick style={color=black}
]
\addplot [semithick, black, dotted, forget plot]
table {%
0 0
15 0
};
\addplot [semithick, steelblue31119180]
table {%
1 6.55053136664628
2 6.45897092818831
3 6.39492422245896
4 6.17205961707096
5 5.99848102437757
7 5.82878375693394
9 5.62850587440294
11 5.50896456987317
13 5.32202151383855
15 5.26612762767
};
\addlegendentry{Net}
\addplot [semithick, darkorange25512714]
table {%
1 7.85336079595166
2 7.84244748331583
3 7.96249454533374
4 7.93702933335966
5 7.94737112855562
7 7.98529882390817
9 7.89450599346438
11 8.07288573249151
13 8.04087158282228
15 8.06032755071626
};
\addlegendentry{Cycle}
\addplot [semithick, forestgreen4416044]
table {%
1 -1.30282942930539
2 -1.38347655512752
3 -1.56757032287478
4 -1.7649697162887
5 -1.94889010417805
7 -2.15651506697422
9 -2.26600011906144
11 -2.56392116261835
13 -2.71885006898373
15 -2.79419992304626
};
\addlegendentry{Parasitic}
\addplot [semithick, crimson2143940]
table {%
1 -0.471010741119679
2 -0.6343006750446
3 -0.78363034850153
4 -0.917783133932867
5 -1.04459330405587
7 -1.28569212412358
9 -1.48348362777772
11 -1.65255708233234
13 -1.76603687017963
15 -1.89346942668607
};
\addlegendentry{NCG Handling}

\nextgroupplot[
tick align=outside,
tick pos=left,
title={Single flash DSC},
x grid style={darkgray176},
xlabel={Geofluid \ce{CO2} content/\unit{\mol\percent}},
xmin=0, xmax=16,
xtick style={color=black},
y grid style={darkgray176},
ylabel={Net electrical power /\unit{\mega\watt}},
ymin=-10, ymax=10,
ytick style={color=black}
]
\addplot [semithick, black, dotted]
table {%
0 0
15 0
};
\addplot [semithick, steelblue31119180]
table {%
1 4.93051246835845
2 4.19029885324923
3 3.66627047825475
4 3.13203730200944
5 2.64291217913743
7 1.85880097825933
9 1.32052238269359
11 0.776437586549237
13 0.299772606188324
15 -0.139823843070852
};
\addplot [semithick, darkorange25512714]
table {%
1 6.77342896109221
2 6.44579489179965
3 6.46216096146654
4 6.51105135863938
5 6.93950521444493
7 6.35909370310959
9 6.26481184649974
11 6.64634365721825
13 6.68394426019312
15 6.68144873794428
};
\addplot [semithick, forestgreen4416044]
table {%
1 -1.84291649273376
2 -2.25549603855041
3 -2.79589048321179
4 -3.37901405662994
5 -4.2965930353075
7 -4.50029272485026
9 -4.94428946380614
11 -5.86990607066902
13 -6.3841716540048
15 -6.82127258101513
};
\addplot [semithick, crimson2143940]
table {%
1 -1.66767841375239
2 -2.12548986408418
3 -2.6758716273896
4 -3.27296395886854
5 -4.184428663601
7 -4.41774048970405
9 -4.87527410008465
11 -5.80200767519317
13 -6.32111465623136
15 -6.76145255194789
};
\end{groupplot}

\end{tikzpicture}

            \caption[Breakdown of the net electrical power for thermodynamically optimised binary \ac{ORC} and single flash \ac{DSC} geothermal power plants re-injecting \ac{NCG} into the reservoir.]{Breakdown of the net electrical power for thermodynamically optimised binary \ac{ORC} (using n-Butane as the working fluid) (left) and single flash \ac{DSC} (right) geothermal power plants with \ac{NCG} re-pressurisation to \qty{75}{\bar} for re-injection into the reservoir.}
            \label{fig:prosim_NCG_Reinjection_Power_Breakdown}
        \end{figure}

        However, unlike in the venting case, where the cycle (or turbine power) could be seen to marginally increase with geofluid \ce{CO2} content for the single flash \ac{DSC}, the cycle power is not strongly affected by geofluid \ce{CO2} content. Given the re-pressurisation power is on a comparable magnitude to the turbine power, it is favourable to raise the condensation pressure to reduce the required compression ratio and hence power requirement, Figure~\ref{fig:prosim_NCG_Reinjection_OPs}.  
        \begin{figure}[H]
            \centering
            % This file was created with tikzplotlib v0.10.1.
\begin{tikzpicture}

\definecolor{crimson2143940}{RGB}{214,39,40}
\definecolor{darkgray176}{RGB}{176,176,176}
\definecolor{darkorange25512714}{RGB}{255,127,14}
\definecolor{forestgreen4416044}{RGB}{44,160,44}
\definecolor{lightgray204}{RGB}{204,204,204}
\definecolor{steelblue31119180}{RGB}{31,119,180}

\begin{axis}[
legend cell align={left},
legend style={
  fill opacity=0.8,
  draw opacity=1,
  text opacity=1,
  at={(0.97,0.03)},
  anchor=south east,
  draw=lightgray204
},
log basis y={10},
tick align=outside,
tick pos=left,
x grid style={darkgray176},
xlabel={Geofluid \ce{CO2} content/\unit{\mol\percent}},
xmin=0.3, xmax=15.7,
xtick style={color=black},
y grid style={darkgray176},
ylabel={Pressure/\unit{\bar}},
ymin=0.142832233002548, ymax=34.1714024287807,
ymode=log,
ytick style={color=black},
ytick={0.01,0.1,1,10,100,1000},
yticklabels={
  \(\displaystyle {10^{-2}}\),
  \(\displaystyle {10^{-1}}\),
  \(\displaystyle {10^{0}}\),
  \(\displaystyle {10^{1}}\),
  \(\displaystyle {10^{2}}\),
  \(\displaystyle {10^{3}}\)
}
]
\addplot [semithick, steelblue31119180]
table {%
1 15.6630536688748
2 16.3198841873615
3 16.9938113190951
4 17.6848295204466
5 18.3935694555122
7 19.8673757145131
9 21.4215479157084
11 23.0630210435187
13 24.799575491809
15 26.6399783604382
};
\addlegendentry{Inlet}
\addplot [semithick, darkorange25512714]
table {%
1 8.91280006879514
2 11.619514683976
3 12.742153323404
4 15.2413970634192
5 17.0756406138661
7 18.4471428119401
9 20.4355582975179
11 22.6041335886886
13 23.6051569686054
15 26.2202186408983
};
\addlegendentry{Flash Sep.}
\addplot [semithick, forestgreen4416044]
table {%
1 0.183212525464347
2 0.298665812602474
3 0.3069901933328
4 0.385668843640606
5 0.42807608145105
7 0.465726351278403
9 0.503653005999805
11 0.769740739905919
13 1.02082952300736
15 0.777502984609375
};
\addlegendentry{Condenser}
\addplot [semithick, crimson2143940]
table {%
1 1.1
2 1.1
3 1.1
4 1.1
5 1.1
7 1.1
9 1.1
11 1.1
13 1.1
15 1.1
};
\addlegendentry{Outlet}
\end{axis}

\end{tikzpicture}

            \caption{The pressures of the geofluid at different points in the single flash geothermal power plant.}
            \label{fig:prosim_NCG_Reinjection_OPs}
        \end{figure}

        The cost of the binary \ac{ORC} roughly doubles compared to the pure water, while the single flash \ac{DSC} increases by a factor of \num{2.6}, Figure~\ref{fig:prosim_NCG_Reinjection_Cost}. Compared to the venting case, where \ac{NCG} handling already presented a significant cost item for the \ac{DSC}, the \ac{NCG} handling is now the primary cost item for both the binary \ac{ORC} and the \ac{DSC} plant, Figure~\ref{fig:prosim_NCG_Reinjection_CostBreakdown}.
        
        \begin{figure}[H]
            \centering
            % This file was created with tikzplotlib v0.10.1.
\begin{tikzpicture}

\definecolor{darkgray176}{RGB}{176,176,176}
\definecolor{darkorange25512714}{RGB}{255,127,14}
\definecolor{lightgray204}{RGB}{204,204,204}
\definecolor{steelblue31119180}{RGB}{31,119,180}

\begin{axis}[
legend cell align={left},
legend style={fill opacity=0.8, draw opacity=1, text opacity=1, draw=lightgray204},
tick align=outside,
tick pos=left,
x grid style={darkgray176},
xlabel={Geofluid \ce{CO2} content/\unit{\mol\percent}},
xmin=0, xmax=16,
xtick style={color=black},
y grid style={darkgray176},
ylabel={Plant Cost/\unit{\mega\USD\of{2023}}},
ymin=0, ymax=100,
ytick style={color=black}
]
\addplot [semithick, steelblue31119180]
table {%
1 19.1926875017161
2 21.6943139984392
3 24.1558293022766
4 26.1475096140439
5 27.9974960372848
7 31.0536811733364
9 33.3720702989873
11 36.1093680763004
13 38.0355481783677
15 39.7521298098009
};
\addlegendentry{Binary ORC}
\addplot [semithick, darkorange25512714]
table {%
1 29.6515541903645
2 33.671382180979
3 38.8981961750247
4 44.5477762308278
5 52.9838408653663
7 54.8933994932988
9 59.1315906110587
11 67.6388936909268
13 72.5615993918084
15 76.8767388176412
};
\addlegendentry{Single Flash DSC}
\end{axis}

\end{tikzpicture}

            \caption[The total plant cost of thermodynamically optimised binary \ac{ORC} and single flash \ac{DSC} geothermal power plants re-injecting \ac{NCG} into the reservoir.]{The total plant cost of thermodynamically optimised binary \ac{ORC} (using n-Butane as the working fluid) and single flash \ac{DSC} geothermal power plants with \ac{NCG} re-pressurisation to \qty{75}{\bar} for re-injection into the reservoir.}
            \label{fig:prosim_NCG_Reinjection_Cost}
        \end{figure}

        \begin{figure}[H]
            \centering
            % This file was created with tikzplotlib v0.10.1.
\begin{tikzpicture}

\definecolor{crimson2143940}{RGB}{214,39,40}
\definecolor{darkgray176}{RGB}{176,176,176}
\definecolor{darkorange25512714}{RGB}{255,127,14}
\definecolor{forestgreen4416044}{RGB}{44,160,44}
\definecolor{lightgray204}{RGB}{204,204,204}
\definecolor{mediumpurple148103189}{RGB}{148,103,189}
\definecolor{sienna1408675}{RGB}{140,86,75}
\definecolor{steelblue31119180}{RGB}{31,119,180}

\begin{groupplot}[
    group style={
        group size=2 by 1, 
        vertical sep=2.5cm, 
        horizontal sep=2cm},
    height=6cm, 
    width=7cm, 
]
\nextgroupplot[
legend cell align={left},
legend style={fill opacity=0.8, draw opacity=1, text opacity=1, draw=lightgray204, at={(1.15, -0.35)}, anchor=north},
legend columns=3,
tick align=outside,
tick pos=left,
title={Binary ORC},
x grid style={darkgray176},
xlabel={Geofluid \ce{CO2} content/\unit{\mol\percent}},
xmin=0, xmax=16,
xtick style={color=black}, xtick distance=4,
y grid style={darkgray176},
ylabel={Cost/\unit{\mega\USD\of{2023}}},
ymin=0, ymax=80,
ytick style={color=black}
]
\draw[draw=none,fill=steelblue31119180] (axis cs:0.55,0) rectangle (axis cs:1.45,3.32727904879437);
\addlegendimage{ybar,area legend,draw=none,fill=steelblue31119180}
\addlegendentry{Turbine}

\draw[draw=none,fill=steelblue31119180] (axis cs:1.55,0) rectangle (axis cs:2.45,3.31506746344461);
\draw[draw=none,fill=steelblue31119180] (axis cs:2.55,0) rectangle (axis cs:3.45,3.36226651630017);
\draw[draw=none,fill=steelblue31119180] (axis cs:3.55,0) rectangle (axis cs:4.45,3.35478449081312);
\draw[draw=none,fill=steelblue31119180] (axis cs:4.55,0) rectangle (axis cs:5.45,3.35260405374644);
\draw[draw=none,fill=steelblue31119180] (axis cs:6.55,0) rectangle (axis cs:7.45,3.36705687357246);
\draw[draw=none,fill=steelblue31119180] (axis cs:8.55,0) rectangle (axis cs:9.45,3.33149983973469);
\draw[draw=none,fill=steelblue31119180] (axis cs:10.55,0) rectangle (axis cs:11.45,3.38894267582062);
\draw[draw=none,fill=steelblue31119180] (axis cs:12.55,0) rectangle (axis cs:13.45,3.38984333013471);
\draw[draw=none,fill=steelblue31119180] (axis cs:14.55,0) rectangle (axis cs:15.45,3.37829092884536);
\draw[draw=none,fill=darkorange25512714] (axis cs:0.55,3.32727904879437) rectangle (axis cs:1.45,6.01588520601527);
\addlegendimage{ybar,area legend,draw=none,fill=darkorange25512714}
\addlegendentry{Condenser}

\draw[draw=none,fill=darkorange25512714] (axis cs:1.55,3.31506746344461) rectangle (axis cs:2.45,5.94113258125354);
\draw[draw=none,fill=darkorange25512714] (axis cs:2.55,3.36226651630017) rectangle (axis cs:3.45,6.06994026468901);
\draw[draw=none,fill=darkorange25512714] (axis cs:3.55,3.35478449081312) rectangle (axis cs:4.45,6.08565724906099);
\draw[draw=none,fill=darkorange25512714] (axis cs:4.55,3.35260405374644) rectangle (axis cs:5.45,6.10543087227862);
\draw[draw=none,fill=darkorange25512714] (axis cs:6.55,3.36705687357246) rectangle (axis cs:7.45,6.12154494156857);
\draw[draw=none,fill=darkorange25512714] (axis cs:8.55,3.33149983973469) rectangle (axis cs:9.45,5.99203064415512);
\draw[draw=none,fill=darkorange25512714] (axis cs:10.55,3.38894267582062) rectangle (axis cs:11.45,6.18827656491483);
\draw[draw=none,fill=darkorange25512714] (axis cs:12.55,3.38984333013471) rectangle (axis cs:13.45,6.21578655527019);
\draw[draw=none,fill=darkorange25512714] (axis cs:14.55,3.37829092884536) rectangle (axis cs:15.45,6.15505506967462);
\draw[draw=none,fill=forestgreen4416044] (axis cs:0.55,6.01588520601527) rectangle (axis cs:1.45,9.24154697100833);
\addlegendimage{ybar,area legend,draw=none,fill=forestgreen4416044}
\addlegendentry{Other Equipment}

\draw[draw=none,fill=forestgreen4416044] (axis cs:1.55,5.94113258125354) rectangle (axis cs:2.45,9.58723577428724);
\draw[draw=none,fill=forestgreen4416044] (axis cs:2.55,6.06994026468901) rectangle (axis cs:3.45,10.1297435087752);
\draw[draw=none,fill=forestgreen4416044] (axis cs:3.55,6.08565724906099) rectangle (axis cs:4.45,10.4801966800108);
\draw[draw=none,fill=forestgreen4416044] (axis cs:4.55,6.10543087227862) rectangle (axis cs:5.45,10.8108923911147);
\draw[draw=none,fill=forestgreen4416044] (axis cs:6.55,6.12154494156857) rectangle (axis cs:7.45,11.3406510210149);
\draw[draw=none,fill=forestgreen4416044] (axis cs:8.55,5.99203064415512) rectangle (axis cs:9.45,11.6007819549027);
\draw[draw=none,fill=forestgreen4416044] (axis cs:10.55,6.18827656491483) rectangle (axis cs:11.45,12.2570779222926);
\draw[draw=none,fill=forestgreen4416044] (axis cs:12.55,6.21578655527019) rectangle (axis cs:13.45,12.6083156609361);
\draw[draw=none,fill=forestgreen4416044] (axis cs:14.55,6.15505506967462) rectangle (axis cs:15.45,12.8360852897125);
\draw[draw=none,fill=crimson2143940] (axis cs:0.55,9.24154697100833) rectangle (axis cs:1.45,9.93149525416407);
\addlegendimage{ybar,area legend,draw=none,fill=crimson2143940}
\addlegendentry{NCG Handling}

\draw[draw=none,fill=crimson2143940] (axis cs:1.55,9.58723577428724) rectangle (axis cs:2.45,11.3951118080087);
\draw[draw=none,fill=crimson2143940] (axis cs:2.55,10.1297435087752) rectangle (axis cs:3.45,12.8498061458072);
\draw[draw=none,fill=crimson2143940] (axis cs:3.55,10.4801966800108) rectangle (axis cs:4.45,13.9945653937189);
\draw[draw=none,fill=crimson2143940] (axis cs:4.55,10.8108923911147) rectangle (axis cs:5.45,15.0110638594064);
\draw[draw=none,fill=crimson2143940] (axis cs:6.55,11.3406510210149) rectangle (axis cs:7.45,16.7939432695488);
\draw[draw=none,fill=crimson2143940] (axis cs:8.55,11.6007819549027) rectangle (axis cs:9.45,18.1137420468739);
\draw[draw=none,fill=crimson2143940] (axis cs:10.55,12.2570779222926) rectangle (axis cs:11.45,19.6927010901965);
\draw[draw=none,fill=crimson2143940] (axis cs:12.55,12.6083156609361) rectangle (axis cs:13.45,20.7787517869734);
\draw[draw=none,fill=crimson2143940] (axis cs:14.55,12.8360852897125) rectangle (axis cs:15.45,21.7463683022343);
\draw[draw=none,fill=mediumpurple148103189] (axis cs:0.55,9.93149525416407) rectangle (axis cs:1.45,11.2898161774757);
\addlegendimage{ybar,area legend,draw=none,fill=mediumpurple148103189}
\addlegendentry{PHE}

\draw[draw=none,fill=mediumpurple148103189] (axis cs:1.55,11.3951118080087) rectangle (axis cs:2.45,12.7613611756179);
\draw[draw=none,fill=mediumpurple148103189] (axis cs:2.55,12.8498061458072) rectangle (axis cs:3.45,14.2093113543018);
\draw[draw=none,fill=mediumpurple148103189] (axis cs:3.55,13.9945653937189) rectangle (axis cs:4.45,15.3808880083243);
\draw[draw=none,fill=mediumpurple148103189] (axis cs:4.55,15.0110638594064) rectangle (axis cs:5.45,16.4691153159264);
\draw[draw=none,fill=mediumpurple148103189] (axis cs:6.55,16.7939432695488) rectangle (axis cs:7.45,18.2668712780623);
\draw[draw=none,fill=mediumpurple148103189] (axis cs:8.55,18.1137420468739) rectangle (axis cs:9.45,19.6306295876164);
\draw[draw=none,fill=mediumpurple148103189] (axis cs:10.55,19.6927010901965) rectangle (axis cs:11.45,21.2408047508221);
\draw[draw=none,fill=mediumpurple148103189] (axis cs:12.55,20.7787517869734) rectangle (axis cs:13.45,22.3738518698306);
\draw[draw=none,fill=mediumpurple148103189] (axis cs:14.55,21.7463683022343) rectangle (axis cs:15.45,23.3836057701325);
\draw[draw=none,fill=sienna1408675] (axis cs:0.55,11.2898161774757) rectangle (axis cs:1.45,19.1926875017087);
\addlegendimage{ybar,area legend,draw=none,fill=sienna1408675}
\addlegendentry{Construction}

\draw[draw=none,fill=sienna1408675] (axis cs:1.55,12.7613611756179) rectangle (axis cs:2.45,21.6943139985505);
\draw[draw=none,fill=sienna1408675] (axis cs:2.55,14.2093113543018) rectangle (axis cs:3.45,24.155829302313);
\draw[draw=none,fill=sienna1408675] (axis cs:3.55,15.3808880083243) rectangle (axis cs:4.45,26.1475096141514);
\draw[draw=none,fill=sienna1408675] (axis cs:4.55,16.4691153159264) rectangle (axis cs:5.45,27.9974960370749);
\draw[draw=none,fill=sienna1408675] (axis cs:6.55,18.2668712780623) rectangle (axis cs:7.45,31.0536811727059);
\draw[draw=none,fill=sienna1408675] (axis cs:8.55,19.6306295876164) rectangle (axis cs:9.45,33.3720702989479);
\draw[draw=none,fill=sienna1408675] (axis cs:10.55,21.2408047508221) rectangle (axis cs:11.45,36.1093680763976);
\draw[draw=none,fill=sienna1408675] (axis cs:12.55,22.3738518698306) rectangle (axis cs:13.45,38.035548178712);
\draw[draw=none,fill=sienna1408675] (axis cs:14.55,23.3836057701325) rectangle (axis cs:15.45,39.7521298092252);

\nextgroupplot[
tick align=outside,
tick pos=left,
title={Single flash DSC},
x grid style={darkgray176},
xlabel={Geofluid \ce{CO2} content/\unit{\mol\percent}},
xmin=0, xmax=16,
xtick style={color=black},
y grid style={darkgray176},
ylabel={Cost/\unit{\mega\USD\of{2023}}},
ymin=0, ymax=80,
ytick style={color=black}
]
\draw[draw=none,fill=steelblue31119180] (axis cs:0.55,0) rectangle (axis cs:1.45,4.8299271786291);
\draw[draw=none,fill=steelblue31119180] (axis cs:1.55,0) rectangle (axis cs:2.45,4.78966508875775);
\draw[draw=none,fill=steelblue31119180] (axis cs:2.55,0) rectangle (axis cs:3.45,4.89756629898145);
\draw[draw=none,fill=steelblue31119180] (axis cs:3.55,0) rectangle (axis cs:4.45,5.03557954169426);
\draw[draw=none,fill=steelblue31119180] (axis cs:4.55,0) rectangle (axis cs:5.45,5.33716548108707);
\draw[draw=none,fill=steelblue31119180] (axis cs:6.55,0) rectangle (axis cs:7.45,5.27049748244794);
\draw[draw=none,fill=steelblue31119180] (axis cs:8.55,0) rectangle (axis cs:9.45,5.42262752684113);
\draw[draw=none,fill=steelblue31119180] (axis cs:10.55,0) rectangle (axis cs:11.45,5.83975084530036);
\draw[draw=none,fill=steelblue31119180] (axis cs:12.55,0) rectangle (axis cs:13.45,6.07773382730978);
\draw[draw=none,fill=steelblue31119180] (axis cs:14.55,0) rectangle (axis cs:15.45,6.29970797022702);
\draw[draw=none,fill=darkorange25512714] (axis cs:0.55,4.8299271786291) rectangle (axis cs:1.45,6.38328438990831);
\draw[draw=none,fill=darkorange25512714] (axis cs:1.55,4.78966508875775) rectangle (axis cs:2.45,6.10704167963618);
\draw[draw=none,fill=darkorange25512714] (axis cs:2.55,4.89756629898145) rectangle (axis cs:3.45,6.09259749718013);
\draw[draw=none,fill=darkorange25512714] (axis cs:3.55,5.03557954169426) rectangle (axis cs:4.45,6.19594232041816);
\draw[draw=none,fill=darkorange25512714] (axis cs:4.55,5.33716548108707) rectangle (axis cs:5.45,6.50721632960603);
\draw[draw=none,fill=darkorange25512714] (axis cs:6.55,5.27049748244794) rectangle (axis cs:7.45,6.30221685156145);
\draw[draw=none,fill=darkorange25512714] (axis cs:8.55,5.42262752684113) rectangle (axis cs:9.45,6.43268223251233);
\draw[draw=none,fill=darkorange25512714] (axis cs:10.55,5.83975084530036) rectangle (axis cs:11.45,6.84654139814259);
\draw[draw=none,fill=darkorange25512714] (axis cs:12.55,6.07773382730978) rectangle (axis cs:13.45,7.07109713789424);
\draw[draw=none,fill=darkorange25512714] (axis cs:14.55,6.29970797022702) rectangle (axis cs:15.45,7.2523966792832);
\draw[draw=none,fill=forestgreen4416044] (axis cs:0.55,6.38328438990831) rectangle (axis cs:1.45,11.3667388756839);
\draw[draw=none,fill=forestgreen4416044] (axis cs:1.55,6.10704167963618) rectangle (axis cs:2.45,11.7660975083721);
\draw[draw=none,fill=forestgreen4416044] (axis cs:2.55,6.09259749718013) rectangle (axis cs:3.45,12.6301094593691);
\draw[draw=none,fill=forestgreen4416044] (axis cs:3.55,6.19594232041816) rectangle (axis cs:4.45,13.6829635356833);
\draw[draw=none,fill=forestgreen4416044] (axis cs:4.55,6.50721632960603) rectangle (axis cs:5.45,15.4120635338693);
\draw[draw=none,fill=forestgreen4416044] (axis cs:6.55,6.30221685156145) rectangle (axis cs:7.45,15.5279982790066);
\draw[draw=none,fill=forestgreen4416044] (axis cs:8.55,6.43268223251233) rectangle (axis cs:9.45,16.3707646881525);
\draw[draw=none,fill=forestgreen4416044] (axis cs:10.55,6.84654139814259) rectangle (axis cs:11.45,18.2144226907353);
\draw[draw=none,fill=forestgreen4416044] (axis cs:12.55,7.07109713789424) rectangle (axis cs:13.45,19.2663239264335);
\draw[draw=none,fill=forestgreen4416044] (axis cs:14.55,7.2523966792832) rectangle (axis cs:15.45,20.1728569847691);
\draw[draw=none,fill=crimson2143940] (axis cs:0.55,11.3667388756839) rectangle (axis cs:1.45,17.4420907002144);
\draw[draw=none,fill=crimson2143940] (axis cs:1.55,11.7660975083721) rectangle (axis cs:2.45,19.8066954005759);
\draw[draw=none,fill=crimson2143940] (axis cs:2.55,12.6301094593691) rectangle (axis cs:3.45,22.8812918676616);
\draw[draw=none,fill=crimson2143940] (axis cs:3.55,13.6829635356833) rectangle (axis cs:4.45,26.2045742534281);
\draw[draw=none,fill=crimson2143940] (axis cs:4.55,15.4120635338693) rectangle (axis cs:5.45,31.1669652149213);
\draw[draw=none,fill=crimson2143940] (axis cs:6.55,15.5279982790066) rectangle (axis cs:7.45,32.2902349960581);
\draw[draw=none,fill=crimson2143940] (axis cs:8.55,16.3707646881525) rectangle (axis cs:9.45,34.7832885947404);
\draw[draw=none,fill=crimson2143940] (axis cs:10.55,18.2144226907353) rectangle (axis cs:11.45,39.7875845240746);
\draw[draw=none,fill=crimson2143940] (axis cs:12.55,19.2663239264335) rectangle (axis cs:13.45,42.6832937598873);
\draw[draw=none,fill=crimson2143940] (axis cs:14.55,20.1728569847691) rectangle (axis cs:15.45,45.2216110692007);
\draw[draw=none,fill=mediumpurple148103189] (axis cs:0.55,17.4420907002144) rectangle (axis cs:1.45,17.4420907002144);
\draw[draw=none,fill=mediumpurple148103189] (axis cs:1.55,19.8066954005759) rectangle (axis cs:2.45,19.8066954005759);
\draw[draw=none,fill=mediumpurple148103189] (axis cs:2.55,22.8812918676616) rectangle (axis cs:3.45,22.8812918676616);
\draw[draw=none,fill=mediumpurple148103189] (axis cs:3.55,26.2045742534281) rectangle (axis cs:4.45,26.2045742534281);
\draw[draw=none,fill=mediumpurple148103189] (axis cs:4.55,31.1669652149213) rectangle (axis cs:5.45,31.1669652149213);
\draw[draw=none,fill=mediumpurple148103189] (axis cs:6.55,32.2902349960581) rectangle (axis cs:7.45,32.2902349960581);
\draw[draw=none,fill=mediumpurple148103189] (axis cs:8.55,34.7832885947404) rectangle (axis cs:9.45,34.7832885947404);
\draw[draw=none,fill=mediumpurple148103189] (axis cs:10.55,39.7875845240746) rectangle (axis cs:11.45,39.7875845240746);
\draw[draw=none,fill=mediumpurple148103189] (axis cs:12.55,42.6832937598873) rectangle (axis cs:13.45,42.6832937598873);
\draw[draw=none,fill=mediumpurple148103189] (axis cs:14.55,45.2216110692007) rectangle (axis cs:15.45,45.2216110692007);
\draw[draw=none,fill=sienna1408675] (axis cs:0.55,17.4420907002144) rectangle (axis cs:1.45,29.6515541903645);
\draw[draw=none,fill=sienna1408675] (axis cs:1.55,19.8066954005759) rectangle (axis cs:2.45,33.671382180979);
\draw[draw=none,fill=sienna1408675] (axis cs:2.55,22.8812918676616) rectangle (axis cs:3.45,38.8981961750247);
\draw[draw=none,fill=sienna1408675] (axis cs:3.55,26.2045742534281) rectangle (axis cs:4.45,44.5477762308278);
\draw[draw=none,fill=sienna1408675] (axis cs:4.55,31.1669652149213) rectangle (axis cs:5.45,52.9838408653662);
\draw[draw=none,fill=sienna1408675] (axis cs:6.55,32.2902349960581) rectangle (axis cs:7.45,54.8933994932988);
\draw[draw=none,fill=sienna1408675] (axis cs:8.55,34.7832885947404) rectangle (axis cs:9.45,59.1315906110587);
\draw[draw=none,fill=sienna1408675] (axis cs:10.55,39.7875845240746) rectangle (axis cs:11.45,67.6388936909268);
\draw[draw=none,fill=sienna1408675] (axis cs:12.55,42.6832937598873) rectangle (axis cs:13.45,72.5615993918084);
\draw[draw=none,fill=sienna1408675] (axis cs:14.55,45.2216110692007) rectangle (axis cs:15.45,76.8767388176412);
\end{groupplot}

\end{tikzpicture}

            \caption[The cost breakdown of thermodynamically optimised binary \ac{ORC} and single flash \ac{DSC} geothermal power plants re-injecting \ac{NCG} into the reservoir.]{The cost breakdown of thermodynamically optimised binary \ac{ORC} (using n-Butane as the working fluid) and single flash \ac{DSC} geothermal power plants with \ac{NCG} re-pressurisation to \qty{75}{\bar} for re-injection into the reservoir.}
            \label{fig:prosim_NCG_Reinjection_CostBreakdown}
        \end{figure}

\subsection{Re-injection with partial dissolution}
    To reduce the power requirements of pressurising the \ac{NCG} to the re-injection pressure of \qty{75}{\bar}, it may be possible to re-dissolve some \ac{NCG} in the brine. This could reduce the re-pressurisation power requirements by reducing the mass rate of \ac{NCG} requiring further pressurisation to the target pressure. 

    \subsubsection{Plant Configurations}
        The plant configuration shown for the previous study, Section~\ref{sec:NCG_reinjection_plant_config}, were adapted to include an absorption unit to redissolve the \ac{NCG} in the brine. 
        
        The streams exiting the power plant, split into their constituent phases, which are then re-pressurised to a common absorption pressure \(P_{absorb}\) and then fed to an absorption column. The absorption column is modelled using a Mixer and a Separator element, to combine and equilibrate the streams and to then separate them into a vapour and liquid stream. 

        \begin{figure}[H]
            \centering
            \resizebox{\linewidth}{!}{\begin{tikzpicture}
    % draw equipment
    \pic (producer) at (0,0) {producer};
    \pic (injector) at (14,0) {injector};
    \pic[scale=0.6, yscale=2] (pplant) at ($(producer-top) + (1.5, 3)$) {block};

    \pic (Bpump) at ($(pplant-bottom right) + (1, -2.2)$) {centrifugal pump};

    \pic[scale=0.4] (NCGsep) at ($(pplant-top right) + (2.5, 0)$) {gas-liquid separator};
    
    \pic (Cpump) at ($(NCGsep-liquid outlet) + (1.2, -0.5)$) {centrifugal pump};
    \pic (NCGcomp_lp) at ($(NCGsep-gas outlet) + (1, 1.5)$) {compressor};

    \pic[yscale=0.5] (absorb) at ($(NCGsep-inlet right) + (3.75, 0)$) {column=packed};
    \pic (NCGcomp_hp) at ($(NCGcomp_lp-anchor) + (4, 0)$) {compressor};
    \pic[rotate=90] (Ljoint) at ($(Cpump-top) + (1.55, 0)$) {valve triple=main};

    
    \pic[rotate=90] (NCGcond) at ($(NCGcomp_hp-outlet bottom) + (1.75, 0)$) {heat exchanger biphase};
    \pic (NCGpump) at ($(NCGcond-shell bottom) + (1.5, 0)$) {centrifugal pump};

    \pic (Bpump_hp) at ($(absorb-bottom) + (2, -1)$) {centrifugal pump};

    % \pic[rotate=90] (Ljoint) at ($(injector-top) + (-0.2, 0.7)$) {valve triple=main};

    % draw connectors
    \draw[main stream] (producer-top) |- (pplant-left);
    \draw[main stream] (pplant-top right) -- (NCGsep-inlet left);

    \draw[main stream] (pplant-bottom right) -| ($0.75*(pplant-bottom right) + 0.25*(Bpump-anchor)$) |- (Bpump-anchor);

    \draw[main stream] (Cpump-top) -- (Ljoint-top);
    \draw[main stream] (Bpump-top) -| (Ljoint-left);
    \draw[main stream] (Ljoint-right) |- (absorb-top left);
    
    
    \draw[main stream] (NCGsep-liquid outlet) |- (Cpump-anchor);
    \draw[main stream] (NCGsep-gas outlet) |- (NCGcomp_lp-inlet top);
    
    \draw[main stream] (NCGcomp_lp-outlet bottom) -- ++(1.75,0) |- (absorb-bottom left);

    \draw[main stream] (absorb-top) |- (NCGcomp_hp-inlet top);
    \draw[main stream] (absorb-bottom) |- (Bpump_hp-anchor);
    \draw[main stream] (NCGcomp_hp-outlet bottom) -- (NCGcond-shell top);

    \draw[main stream] (NCGcond-shell bottom) -- (NCGpump-anchor);
    \draw[main stream] (NCGpump-top) -| ++(1.5, -3) -| ($(injector-top) + (0.2, 0)$);
    \draw[main stream] (Bpump_hp-top) -| ($(injector-top) + (-0.2, 0)$);

    \draw[main stream] ($(NCGcond-pipes bottom) + (0,1)$) -- (NCGcond-pipes bottom);
    \draw[main stream] (NCGcond-pipes top) -- ($(NCGcond-pipes top) + (0,1.75)$);


    % draw labels
    \node[below] at (producer-bottom) {Producer};
    \node[below] at (injector-bottom) {Injector};
    \node[align=center] at (pplant-anchor) {Power\\Plant};

    \node[below] at (Bpump-bottom) {Brine Pump};
    \node[below, align=center] at (Cpump-bottom) {Condensate or\\Brine Pump};

    \node[right] at (NCGsep-inlet right) {NCG Sep.};
    \node[above] at (NCGcomp_lp-inlet top) {LP NCG Compr.};
    \node[below] at (NCGcond-shell left) {NCG Liquefier};
    \node[above] at (NCGpump-top) {NCG pump};

    \node[right] at (absorb-right) {\ce{CO2} Absorber};
    \node[below] at (Bpump_hp-bottom) {HP Brine Pump};
    \node[above] at (NCGcomp_hp-inlet top) {HP NCG Compr.};

    \node[below right, align=left, font=\footnotesize] at ($(pplant-top right) - (0, 0.1)$) {From DSC\textsuperscript{2}};
    \node[above right, align=left, font=\footnotesize] at ($(pplant-top right) + (0, 0.1)$) {From ORC\textsuperscript{1}};
    \node[below left, align=left, font=\footnotesize] at ($(pplant-bottom right) - (-0.2, 0.75)$) {DSC only\textsuperscript{3}};

    \node[above right] at ($(NCGcond-pipes bottom) + (0,1)$) {Coolant In};
    \node[above] at ($(NCGcond-pipes top) + (0,1.75)$) {Coolant Out};
\end{tikzpicture}}
            \caption[Binary \ac{ORC} or single flash \ac{DSC} geothermal power plants with partial dissolution of \ac{NCG} subsequent re-injection into the reservoir.]{Binary \ac{ORC} or single flash \ac{DSC} geothermal power plants with partial dissolution of \ac{NCG} subsequent re-injection into the reservoir. \textsuperscript{1}Geofluid from \ac{PHE}. \textsuperscript{2}Condensate and \ac{NCG} from Condenser. \textsuperscript{3}Brine from flash separator.}
            \label{fig:prosim_NCG_reinjection_CarbFix}
        \end{figure}

    \subsubsection{Boundary Conditions}
        The same boundary conditions as in the previous study are used, see Section~\ref{sec:prosim_NCG_reinj}.

    \subsubsection{Optimisation Configuration}
        The binary \ac{ORC} and single flash \ac{DSC} power plants were thermodynamically optimised using the same optimisation configuration as used for previous studies, but with an additional optimisation variable, the absorption pressure \(P_{absorb}\), which is allowed to range between the power plant outlet pressure and saturation pressure of \ac{NCG}, Table~\ref{table:NCG_CarbFix_opt_config}.

        \begin{table}[H]
            \centering
            \caption{The optimisation parameters used for the single flash \ac{DSC} and the binary \ac{ORC} geothermal power plants.}
            \label{table:NCG_CarbFix_opt_config}
            \begin{tabular}{|c | c c |}
    \hline
    \rowcolor{bluepoli!40} % comment this line to remove the color
      & \textbf{Single Flash DSC} & \textbf{Binary ORC} \T\B \\
    \hline \hline
    Objective Function & \(W_{net,\;elec}\) & \(W_{net,\;elec}\) \T\B \\
    \hline
    \multicolumn{1}{|l|}{\multirow{4}{*}{Constraints}} & \multirow{4}{*}{\(\Delta T_{cond}^{min}\geq5\)\unit{\K}} & \(\Delta T_{cond}^{min}\geq5\)\unit{\K} \T\B \\
    \multicolumn{1}{|l|}{} &  & \(\Delta T_{preh}^{min}\geq5\)\unit{\K} \T\B \\
    \multicolumn{1}{|l|}{} &  & \(\Delta T_{evap}^{min}\geq10\)\unit{\K} \T\B \\
    \multicolumn{1}{|l|}{} &  & \(\Delta T_{sh}^{min}\geq10\)\unit{\K} \T\B \\
    \hline
    \multicolumn{1}{|l|}{\multirow{4}{*}{Controls}} & \num{0}\(\leq \frac{P_{absorb} - P_{geo}^{out}}{P_{sat} - P_{geo}^{out}}\leq\)\num{1} & \num{0}\(\leq \frac{P_{absorb} - P_{geo}^{out}}{P_{sat} - P_{geo}^{out}}\leq\)\num{1} \T\B \\
    \multicolumn{1}{|l|}{} & \qty{0.1}{\bar}\(\leq P_{cond}\leq\)\qty{5.0}{\bar} & \qty{303}{\K}\(\leq T_{cond}\leq\)\qty{400}{\K} \T\B \\
 
    \multicolumn{1}{|l|}{} & \num{0.3}\(\leq \frac{P_{flash]}}{P_{in}}\leq\)\num{1.0} & \num{0.2}\(\leq \frac{P_{evap}}{P_{crit}}\leq\)\num{0.8} \T\B \\
    \multicolumn{1}{|l|}{} &  & \qty{3}{\K}\(\leq \Delta T_{sh}\leq\)\qty{15}{\K} \T\B \\
    \hline
\end{tabular}        
        \end{table}

    \subsubsection{Results}
        Partial dissolution can reduce the power requirements of \ac{NCG} re-pressurisation in single flash \ac{DSC} power plants for geofluid \ce{CO2} contents of up to \qty{8}{\mol\percent}, beyond which an increase in pressurisation power compared to the direct pressurisation case, Figure~\ref{fig:prosim_NCG_CarbFix_Wnet}. 

        \begin{figure}[H]
            \centering
            % This file was created with tikzplotlib v0.10.1.
\begin{tikzpicture}

\definecolor{burlywood253194140}{RGB}{253,194,140}
\definecolor{chocolate2369815}{RGB}{236,98,15}
\definecolor{darkgray176}{RGB}{176,176,176}
\definecolor{lightblue182212233}{RGB}{182,212,233}
\definecolor{lightgray204}{RGB}{204,204,204}
\definecolor{midnightblue848107}{RGB}{8,48,107}
\definecolor{saddlebrown127394}{RGB}{127,39,4}
\definecolor{steelblue59139194}{RGB}{59,139,194}

\begin{groupplot}[group style={group size=1 by 2}]
\nextgroupplot[
legend cell align={left},
legend style={
  fill opacity=0.8,
  draw opacity=1,
  text opacity=1,
  at={(0.97,0.03)},
  anchor=south east,
  draw=lightgray204
},
tick align=outside,
tick pos=left,
title={n-Butane},
x grid style={darkgray176},
xlabel={Drilling Cost/\unit{\mega\USD\of{2023}}},
xmin=0, xmax=63,
xtick style={color=black},
y grid style={darkgray176},
ylabel={Net electrical power/\unit{\mega\watt}},
ymin=0, ymax=9,
ytick style={color=black}
]
\addplot [semithick, burlywood253194140]
table {%
0 6.52143371948183
1 6.52143371948183
2 6.52143371948183
4 6.52143371948183
8 6.52143371948183
12 6.52143371948183
16 6.52143371948183
20 6.52143371948183
25 6.52143371948183
30 6.52143371948183
35 6.52143371948183
40 6.52143371948183
50 6.52143371948183
60 6.52143371948183
};
\addlegendentry{Thermodynamic Opt.}
\addplot [semithick, saddlebrown127394]
table {%
0 3.35312380103824
1 3.35312380103824
2 3.35312380103824
4 3.35312380103824
8 3.35312380103824
12 3.35312380103824
16 3.35312380103824
20 3.35312380103824
25 3.35312380103824
30 3.35312380103824
35 3.35312380103824
40 3.35312380103824
50 3.35312380103824
60 3.35312380103824
};
\addlegendentry{Techno-economic Opt. (excluding drilling costs)}
\addplot [semithick, chocolate2369815]
table {%
0 3.41586816872176
1 3.45484599362514
2 3.59827794491314
4 5.44747672618843
8 5.71741855543081
12 6.13905459187174
16 6.22879829708411
20 6.33475894364317
25 6.35406610908275
30 6.39668208743175
35 6.4680377937982
40 6.40986537612564
50 6.50452302914855
60 6.47623187635999
};
\addlegendentry{Techno-economic Opt. (including drilling costs)}

\nextgroupplot[
legend cell align={left},
legend style={fill opacity=0.8, draw opacity=1, text opacity=1, draw=lightgray204},
tick align=outside,
tick pos=left,
title={Cyclopentane},
x grid style={darkgray176},
xlabel={Drilling Cost/\unit{\mega\USD\of{2023}}},
xmin=0, xmax=63,
xtick style={color=black},
y grid style={darkgray176},
ylabel={Net electrical power/\unit{\mega\watt}},
ymin=0, ymax=9,
ytick style={color=black}
]
\addplot [semithick, lightblue182212233]
table {%
0 8.67312917722771
1 8.67312917722771
2 8.67312917722771
4 8.67312917722771
8 8.67312917722771
12 8.67312917722771
16 8.67312917722771
20 8.67312917722771
25 8.67312917722771
30 8.67312917722771
35 8.67312917722771
40 8.67312917722771
50 8.67312917722771
60 8.67312917722771
};
\addlegendentry{Thermodynamic Opt.}
\addplot [semithick, midnightblue848107]
table {%
0 3.24768016868321
1 3.24768016868321
2 3.24768016868321
4 3.24768016868321
8 3.24768016868321
12 3.24768016868321
16 3.24768016868321
20 3.24768016868321
25 3.24768016868321
30 3.24768016868321
35 3.24768016868321
40 3.24768016868321
50 3.24768016868321
60 3.24768016868321
};
\addlegendentry{Techno-economic Opt. (excluding drilling costs)}
\addplot [semithick, steelblue59139194]
table {%
0 3.14095734400058
1 6.21228907960342
2 6.29436173932615
4 6.55325520823824
8 6.84360915751591
12 7.168577206856
16 7.28400752660467
20 7.26659096414527
25 7.38675295828168
30 7.51110970957674
35 7.43616663970955
40 7.57638400329378
50 7.75201144638119
60 8.36888127426749
};
\addlegendentry{Techno-economic Opt. (including drilling costs)}
\end{groupplot}

\end{tikzpicture}

            \caption[The net electrical power of thermodynamically optimised binary \ac{ORC} and single flash \ac{DSC} geothermal power plants partially dissolving and re-injecting \ac{NCG} into the reservoir.]{The net electrical power of thermodynamically optimised binary \ac{ORC} (using n-Butane as the working fluid) and single flash \ac{DSC} geothermal power plants with partial \ac{NCG} re-dissolution and re-pressurisation to \qty{75}{\bar} for re-injection into the reservoir.}
            \label{fig:prosim_NCG_CarbFix_Wnet}
        \end{figure}

        Partial dissolution is only effective at reducing the pressurisation power where 1) a significant portion of the \ac{NCG} stream can be absorbed by the brine and 2) the compression ratio for the secondary compression (i.e. from the absorption pressure \(P_{absorb}\) to the saturation pressure \(P_{sat}\) is large. This is illustrated in Equation~\ref{eq:power_partial_dis}, which can be derived assuming that the pressurisation power is approximately proportional to the product of the mass flow rate and the logarithm of the compression ratio.

        \begin{align}
            \Dot{W}_{compr}^{tot} \propto \Dot{m}_{NCG} * \log \frac{P_{out}}{P_{in}} - \Dot{m}_{absorbed} * \log \frac{P_{out}}{P_{absorb}} \label{eq:power_partial_dis}
        \end{align}

        However, in reality the fraction of the total \ac{NCG} mass flow rate that can be dissolved in the geothermal brine diminishes as the geofluid \ce{CO2} content increases. This is because the brine can only accommodate a few \unit{\mol\percent} of \ce{CO2}, Figure~\ref{fig:prosim_NCG_CarbFix_Solubility}, which represents a fraction of the total \ac{NCG} contained within the geofluid. For example, at a temperature of \qty{375}{\K} and a pressure of around \qty{30}{\bar} the brine is saturated at just \qty{0.5}{\mol\percent} of \ce{CO2}. For a geofluid \ce{CO2} content of \qty{1}{\mol\percent} this represents a \ac{NCG} mass rate reduction of \qty{50}{\percent}, however this reduces to a reduction of just \qty{3.3}{\percent} for geofluid \ce{CO2} contents of \qty{15}{\mol\percent}. To achieve higher \ce{CO2} concentrations in the brine, higher pressures are required, however this reduces the compression ratio of the secondary pressurisation and hence the overall reduction in pressurisation power.
        
        \begin{figure}[H]
            \centering
            \input{Content/ProSim/NCGHandling/Plots/CarbFix/Solubility_P_vs_T}
            \caption{The bubble point pressure for different water-carbon dioxide mixtures as a function of temperature.}
            \label{fig:prosim_NCG_CarbFix_Solubility}
        \end{figure}

        Another factor to consider is that, the average temperature of the \ac{NCG} is higher in this process, due to its exposure to the hot brine in the absorption unit. In turn, this results in higher power consumption in the compressor, undoing any savings from reducing the \ac{NCG} mass rate. This is particularly evident at higher geofluid \ce{CO2} contents, where the savings from partial dissolution are insignificant.

        As for the binary \ac{ORC}, partial dissolution somewhat reduces the pressurisation power requirement, however, as the overall compression ratio is smaller than for the \ac{DSC}, the potential gains from partial dissolution are small.

        \begin{figure}[H]
            \centering
            % This file was created with tikzplotlib v0.10.1.
\begin{tikzpicture}

\definecolor{crimson2143940}{RGB}{214,39,40}
\definecolor{darkgray176}{RGB}{176,176,176}
\definecolor{darkorange25512714}{RGB}{255,127,14}
\definecolor{forestgreen4416044}{RGB}{44,160,44}
\definecolor{lightgray204}{RGB}{204,204,204}
\definecolor{steelblue31119180}{RGB}{31,119,180}

\begin{axis}[
legend cell align={left},
legend style={
  fill opacity=0.8,
  draw opacity=1,
  text opacity=1,
  at={(0.97,0.03)},
  anchor=south east,
  draw=lightgray204
},
log basis y={10},
tick align=outside,
tick pos=left,
x grid style={darkgray176},
xlabel={Geofluid \ce{CO2} content/\unit{\mol\percent}},
xmin=0.3, xmax=15.7,
xtick style={color=black},
y grid style={darkgray176},
ylabel={Pressure/\unit{\bar}},
ymin=0.142832233002548, ymax=34.1714024287807,
ymode=log,
ytick style={color=black},
ytick={0.01,0.1,1,10,100,1000},
yticklabels={
  \(\displaystyle {10^{-2}}\),
  \(\displaystyle {10^{-1}}\),
  \(\displaystyle {10^{0}}\),
  \(\displaystyle {10^{1}}\),
  \(\displaystyle {10^{2}}\),
  \(\displaystyle {10^{3}}\)
}
]
\addplot [semithick, steelblue31119180]
table {%
1 15.6630536688748
2 16.3198841873615
3 16.9938113190951
4 17.6848295204466
5 18.3935694555122
7 19.8673757145131
9 21.4215479157084
11 23.0630210435187
13 24.799575491809
15 26.6399783604382
};
\addlegendentry{Inlet}
\addplot [semithick, darkorange25512714]
table {%
1 8.91280006879514
2 11.619514683976
3 12.742153323404
4 15.2413970634192
5 17.0756406138661
7 18.4471428119401
9 20.4355582975179
11 22.6041335886886
13 23.6051569686054
15 26.2202186408983
};
\addlegendentry{Flash Sep.}
\addplot [semithick, forestgreen4416044]
table {%
1 0.183212525464347
2 0.298665812602474
3 0.3069901933328
4 0.385668843640606
5 0.42807608145105
7 0.465726351278403
9 0.503653005999805
11 0.769740739905919
13 1.02082952300736
15 0.777502984609375
};
\addlegendentry{Condenser}
\addplot [semithick, crimson2143940]
table {%
1 1.1
2 1.1
3 1.1
4 1.1
5 1.1
7 1.1
9 1.1
11 1.1
13 1.1
15 1.1
};
\addlegendentry{Outlet}
\end{axis}

\end{tikzpicture}

            \caption{The pressures of the geofluid at different points in the power plant.}
            \label{fig:prosim_NCG_CarbFix_OPs}
        \end{figure}

\subsection{Conclusion}  
    Single flash DSC geothermal power plants are particularly vulnerable to the presence of \ac{NCG} because power is directly generated from the geofluid. On the other hand, binary \ac{ORC} geothermal power plants are virtually unaffected by the presence of \ac{NCG} in the geofluid.

    With regards to \ac{NCG} disposal, venting \ac{NCG} to atmosphere is by far the least energy intensive option. While binary \ac{ORC} do not require an re-pressurisation to vent \ac{NCG} to the atmosphere, in single flash \ac{DSC} power plants the re-pressurisation power requirements can be mitigated by raising the condensation pressure, however a net plant power reduced for NCG contents less than \qty{7}{\mol\percent}

    \ac{NCG} disposal by re-injection into the geothermal reservoir requires wellhead pressures in excess of \qty{60}{\bar} depending on the injection temperature and whether the NCG can be liquefied. For single flash \acp{DSC}, the re-pressurisation power requirements are significant, and increase with \ac{NCG} content, ultimately resulting in zero net plant power for \ac{NCG} contents higher than \qty{14}{\mol\percent}. Binary \ac{ORC} plants are not as strongly affected due to the smaller pressure differential across the power plant.
    
    Partial dissolution of NCG, akin to the CarbFix process, could help reduce the re-pressurisation power requirements in single flash \ac{DSC} geothermal power plants for geofluid \ac{NCG} contents less than about \qty{8}{\mol\percent}. For binary \ac{ORC} plants no significant power savings can be realised.

\clearpage