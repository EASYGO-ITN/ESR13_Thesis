This chapter aims to provide a summary of the main conclusions of this work. The main achievements are the development of \emph{GeoProp} and \emph{PowerCycle}. \emph{GeoProp} can be used to compute the thermophysical properties and phase behaviour of geothermal geofluids. \emph{PowerCycle} can be used to simulate different configurations and types of geothermal power plants for a variety of heat sources and both ideal and real geofluids.

\section{Overall Motivation}
    Two-phase geothermal heat sources are traditionally exploited using \ac{DSC} power plants. However, due to the higher temperatures and/or geological setting of these two-phase heat sources, co-production of \ac{NCG} is more prevalent. The need for treatment and/or re-pressurisation prior to disposal is not only an operational challenge but also raises the cost of the power plant. By virtue of their closed-loop nature, binary \ac{ORC} geothermal power plants could be better suited to handle and dispose \ac{NCG}. Thus the overall objective of this work was to establish the thermodynamic and techno-economic application envelope of binary \ac{ORC} power plants for two-phase sources, as well as the effect of impurities, such \ac{NCG}, have on power production and plant operation.

\section{Thermophysical Property Modelling}
    The simulation of geothermal power systems requires reliable and robust models for predicting the thermophysical properties and phase behaviour of the geofluid. Given the site-specificity of the composition of geothermal fluids, such models must be capable of capturing a broad range of compositions, temperatures and pressures to be holistically applicable.

    A review of the literature identified three approaches for modelling geofluids: \ac{EOS} that provide high-fidelity predictions of thermophysical properties and phase behaviour of pure fluids and mixtures, but require extensive data for calibration; Chemically active systems, which allow virtually any arbitrary geofluid composition to be modelled, but require extensive input data for the underlying models; and empirical models that are computationally inexpensive but are restricted to specific applications, e.g. phase behaviour.

    % A review of the literature identified three approaches for modelling geofluids. \ac{EOS} provide high-fidelity predictions of thermophysical properties and phase behaviour of fluids and mixtures. However, due to the large quantities of experimental data required to calibrate an \ac{EOS}, they primarily exist for commercially or industrially important fluids, such as water, carbon dioxide, hydrocarbons, etc. Although mixtures, can be captured, the need for calibration data can be prohibitive, and due to their complexity convergence is not guaranteed. 

    % Chemically active systems are a promising alternative, allowing geofluids of virtually arbitrary composition to be modelled, and is not limited to traditional vapour-liquid phase behaviour, but can also capture the formation of solids (i.e. scaling). The main draw-back of this approach is the need for appropriate models for capturing the thermodynamic properties and activity of species, which can lead to discrepancies for quasi-pure fluids. 
    
    % The chemically active system approach, e.g. as implemented in \emph{Reaktoro}, is perhaps the prime candidate for a holistic model of geofluids, but is computationally complex and requires specialised algorithms.

    % Empirical models are a computationally inexpensive alternative to the above approaches, however, they do not offer the same versatility, and are intended for specific applications. For instance, \ac{SP2009} was originally intended to model the phase behaviour of water carbon dioxide in reservoir simulations, but does not provide any thermophysical properties or capabilities for including other common gases such as \ce{H2S} or \ce{CH4}.

    Recognising the synergies between the approaches presented above, \emph{GeoProp} was developed, which couples existing partitioning and property models, extending their individual capabilities. The approach presented, allows the thermophysical properties and phase behaviour of geothermal brines for a wide range of temperatures, pressures and compositions to be calculated, which are required for the accurate simulation, and the proper design and evaluation, of direct steam cycle or binary cycle geothermal power plants. \emph{GeoProp} is available on GitHub (https://github.com/EASYGO-ITN/GeoProp) under an Apache License 2.0 license.

\section{Power Plant Modelling}
    Commercial softwares, such as \emph{Aspen Plus v11}, provide a convenient basis for simulating simple geothermal power plants. Nevertheless, capturing non-standard physics, such as wet expansion effects in turbines, or complex fluids, such as geothermal brines, requires bespoke solutions. 
    
    \emph{PowerCycle} was specifically developed with the challenges of geothermal power plant simulations in mind. The fluid properties module currently enables the use of both \emph{CoolProp} and \emph{GeoProp}, as well as the means for generating look-up tables, but is arbitrarily extendable. Component models for turbines, pumps/compressors/fans, heat exchangers, separators, joints/mixers, etc. are included, each calculating the component thermodynamic performance, exergy loss and cost. The geothermal power plant can then be calculated by means of a script, specifying the order in which the component are calculated. A number of power plant models for various \ac{DSC} and binary \ac{ORC} geothermal power plant configurations were formulated as part of this work. Although \emph{PowerCycle} was created with geothermal power applications in mind, as the fluid properties and base component modules can be extended arbitrarily, in principles it could also be used in non-geothermal applications. \emph{PowerCycle} is entirely written in Python and the codebase will be made available on GitHub under an open-source license.
    
\section{Power Plant Simulations}
    The thermodynamic and techno-economic performance of single flash \ac{DSC} and binary \ac{ORC} geothermal power plants was investigated for both ideal (pure water) and real (water and carbon dioxide mixtures) two-phase geofluids using both commercial and bespoke modelling tools, i.e \emph{Aspen Plus v11} and \emph{PowerCycle} respectively. The conclusions from the different studies are summarised below.

    \subsection{Commercial Software}
        The first comparative study of binary \ac{ORC} and single flash \ac{DSC} geothermal power plants was conducted using \emph{Aspen Plus v11}, with each plant optimised to deliver the maximum specific net power. With increasing geofluid inlet temperature, the binary \ac{ORC} was found to deliver comparable specific net power at increasingly higher geofluid inlet vapour qualities. At the lower temperatures, the single \ac{DSC} outperforms the binary \ac{ORC} across the full range of geofluid vapour qualities, which can be attributed to the critical temperatures of the selected working fluids being poorly aligned with the heat source.

        The effect of impurities such as salinity and \ac{NCG} on the performance was also investigated. The performance of both binary \ac{ORC} and single flash \ac{DSC} was found to be only weakly dependent on the salinity, as the presence of salts does not alter the heat flow to the power plant. On the other hand, the presence of \ac{NCG} has a significant effect on the single flash \ac{DSC}, while the binary \ac{ORC} is virtually unaffected.

        Several issues were encountered (i.e. crashes, difficulties in debugging, scenario initialisation, etc.) as part of the modelling, as well as doubts in the reliability accuracy of the underlying property model for water-carbon dioxide mixtures, ultimately led to the development of a bespoke alternative, \emph{PowerCycle}. 
        
    \subsection{Thermodynamic Optimisation}
        Optimising for the thermodynamic performance (i.e. maximum net electrical power), the performance of binary \ac{ORC} and single flash \ac{DSC} was investigated for a geofluid comprised of pure water and inlet temperatures of \qtyrange{423}{548}{\K} and vapour qualities of \qtyrange{0}{100}{\percent} and a geofluid mass rate of \qty{50}{\kg\per\s}.

        The net electrical power for both binary \ac{ORC} and single flash \ac{DSC} was found to increase with both geofluid inlet temperature and vapour quality. For the single flash \ac{DSC}, the optimum condensation pressure was found to be independent of geofluid inlet condition, but the optimum degree of flashing was found to increase with increasing geofluid inlet temperature and decreasing geofluid vapour quality, a result of the curvature of the quality lines in the pressure-enthalpy domain. 

        For the binary \ac{ORC}, at low geofluid inlet temperatures and vapour quality the net electrical power is limited by temperature approach of the working fluid and geofluid in the \ac{PHE} performance. However, as the location of  pinch-point shifts towards the cold-side inlet of the heat exchanger with increasing geofluid inlet temperature and vapour quality, the net electrical power is only limited by the heat flow entering the power plant. While lighter working fluids (e.g. n-Butane) dominate at pinch-point constraint geofluid inlet conditions, the more complex working fluids (e.g. Cyclopentane) achieve the highest net electrical power. Super-heating is only used with the lighter working fluids to increase the cycle temperature, but the net power is not strongly dependent on the degree of super-heating. There is a significant spread in specific cost of the power plant as well as the cost contributions of the power plant equipment between light and heavy working fluids considered, which is almost solely attributed to size and cost of the turbine. Moreover, the heaviest working fluids (e.g. n-Heptane) require \acp{ORC} with sub-atmospheric condensation pressures, which can lead to operational issues due to air ingress into the system.

        In line with results generated using \emph{Aspen Plus}, it can be seen that binary \acp{ORC} can compete with single flash \acp{DSC}. When the focus is on net electrical power, binary \acp{ORC} can compete up to geofluid inlet vapour qualities of around \qty{70}{\percent} (depending on the inlet temperature), although at a higher specific cost. Similarly, when the focus is on specific cost of the power plant, binary \acp{ORC} can achieve lower specific costs (and more net electrical power) for geofluid inlet vapour qualities around \qtyrange{10}{20}{\percent} across the range of temperatures investigated.

    
    \subsection{Techno-economic optimisation}
        Optimising for the techno-economic performance (i.e. minimum specific cost of the power plant), the performance of binary \ac{ORC} and single flash \ac{DSC} was investigated for a geofluid comprised of pure water and inlet temperatures of \qtyrange{423}{548}{\K} and vapour qualities of \qtyrange{0}{100}{\percent} and a geofluid mass rate of \qty{50}{\kg\per\s}.

        For the single flash \ac{DSC}, specific cost reductions of about \qtyrange{6}{10}{\percent}, at the expense of around \qtyrange{10}{20}{\percent} of net electrical power were observed. The cost reductions are almost entirely a result of a reduction in condenser size, by raising the minimum approach temperature difference and condensation temperature.

        For the binary \ac{ORC}, larger cost reductions of about \qtyrange{20}{30}{\percent} were observed, at the expense of up to \qty{65}{\percent} of net electrical power. For all fluids the condenser costs are reduced by raising the minimum approach temperature difference and condensation temperature, for the heavier working fluids simultaneously reducing the number of turbine stages from a maximum of \num{5} to a minimum of \num{1}. Nevertheless, n-Butane achieves the lowest specific net electrical power across a broad range of geofluid inlet conditions.

        From a specific cost perspective, techno-economically optimised binary \acp{ORC} can compete with single flash \acp{DSC} for geofluid inlet vapour qualities as high as \qty{80}{\percent}, but produce significantly less power. Here, the envelope for competitive specific power and net electrical reduces from a geofluid inlet vapour quality of \qty{20}{\percent} at a geofluid inlet temperature of \qty{423}{\K} to \qty{0}{\percent} at \qty{548}{\K}. 
        
        When drilling costs are considered as part of the optimisation, the techno-economic and thermodynamic optimisations tend converge where drilling and plant costs are of comparable magnitude.
    
    \subsection{Impact and handling of NCG}
        The impact of the presence of \ac{NCG} in geofluid on the performance of single flash \ac{DSC} and binary \ac{ORC} geothermal power plants was investigated for a series of \ac{NCG} disposal scenarios and \ac{NCG} contents.

        Single flash \ac{DSC} geothermal power plants are particularly vulnerable to the presence of \ac{NCG} because power is directly generated from the geofluid. As such, the power generation and subsequent \ac{NCG} handling process are closely coupled. For instance, the power required to re-pressurise is directly dependent on the geofluid condensation pressure (i.e. turbine discharge pressure), which drives the overall power generated in the power plant. On the other hand, binary \ac{ORC} geothermal power plants are virtually unaffected by the presence of \ac{NCG} in the geofluid.

        As can be expected, venting \ac{NCG} to atmosphere is by far the least energy intensive option. In fact, due to minimal pressure losses in the \ac{PHE}, binary \ac{ORC} geothermal power plants would not require any additional re-pressurisation power to vent the \ac{NCG} to atmosphere. For single flash \ac{DSC} a reduction in net plant power is observed for \ac{NCG} content less than \qty{7}{\mol\percent}, however at higher \ac{NCG} contents, with the assumptions considered in this work, the increase in geofluid inlet pressure leads to an increase in net plant power. This is in part facilitated by raising the condensation pressure thus reducing the need for subsequent \ac{NCG} re-pressurisation. 

        \ac{NCG} disposal by re-injection into the geothermal reservoir requires wellhead pressures in excess of \qty{60}{\bar} depending on the injection temperature and whether the \ac{NCG} can be liquefied. For single flash \acp{DSC}, the re-pressurisation power requirements are significant, and increase with \ac{NCG} content, ultimately resulting in zero net plant power for \ac{NCG} contents higher than \qty{14}{\mol\percent}. Again, binary \ac{ORC} plants are not as strongly affected because the pressure differential across the power plant is smaller, meaning that lower compression ratios are required to re-inject the \ac{NCG}. Partial dissolution of \ac{NCG}, akin to the \emph{CarbFix} process, could help reduce the re-pressurisation power requirements in single flash \ac{DSC} geothermal power plants for geofluid \ac{NCG} contents less than about \qty{8}{\mol\percent}. For binary \ac{ORC} plants no significant power savings can be realised. 

        Depending on the \ac{NCG} composition, different handling and disposal approaches can be used. Given the power requirements of re-pressurisation, harmful components, such as \ce{H2S}, are best stripped from the \ac{NCG} stream, for example using the \emph{AMIS} process. The preferred method of \ce{CO2} disposal is primarily a function of whether geothermal plant changes the overall emissions of the region (i.e. natural emissions and emissions from the power plant), relative to natural emissions prior to geothermal operations. Venting is a suitable option where there is no net increase in the overall emissions of \ce{CO2} in the region. Where geothermal operations increase the overall emissions of \ce{CO2} in the region, re-injection can be used to reduce the carbon footprint of the geothermal plant. For sufficiently high carbon prices, re-injection of \ce{CO2} could be an additional income stream for the geothermal power plant, although this was not considered in this study.

        The preferred method of \ce{CO2} disposal is primarily a function of whether geothermal plant changes the overall emissions of the region (i.e. natural emissions and emissions from the power plant), relative to natural emissions prior to geothermal operations. Venting is a suitable option where there is no net increase in the overall emissions of \ce{CO2} in the region. Where geothermal operations increase the overall emissions of \ce{CO2} in the region, re-injection can be used to reduce the carbon footprint of the geothermal plant. For sufficiently high carbon prices, re-injection of \ce{CO2} could be an additional income stream for the geothermal power plant, although this was not considered in this study.
        
\section{Future Work}
    The component models in \emph{PowerCycle} make a number of simplifying assumptions that could be constrained in future studies. For instance, for \ac{ORC} turbines the isentropic efficiency is currently assumed to be constant for all working fluids and expansion conditions. This could be improved by also considering the effects of the volume ratio across the turbine and the specific rotational speed of the turbine. Similarly for steam turbines, the effect of wetness on the isentropic turbine efficiency is currently captured using the empirical Baumann rule, which was originally developed for pure water, but in this context is also being used for mixtures of water and \ac{NCG}. Perhaps more detailed turbine performance calculations could aid in the development of an analogous empirical model for such working fluid mixtures. Finally, the \ac{ORC} evaporator unit was assumed to be capable of handling the simultaneous condensation of steam on the hot-side and evaporation on the cold side. Whether this can be realised using conventional heat exchanger designs (e.g. kettle reboiler or condensers) or, requires specialised heat exchanger designs, requires more detailed design calculations accounting for the heat transfer coefficients and two-phase behaviour of the fluids.

    With respect to the handling of the \ac{NCG}, the calculation of the required wellhead pressure for re-injecting \ac{NCG} into the reservoir is also based on many simplifying assumptions (i.e. static pressures and no heat transfer). To provide more reliable estimates, as well as to explore alternative re-injection strategies (e.g. mixing in the wellbore at depth) requires more detailed wellbore simulations, including frictional pressure losses; heat transfer between surrounding formation, the brine and the \ac{NCG}; mass transfer resistances (to capture the dissolution of \ac{NCG} in the brine); distribution of phases within the wellbore; and flow instability detection. Particularly, simulations of "worst case scenarios", such as infinite mass transfer resistances may aid establish whether simultaneous injection of water and \ac{NCG} is possible.

    Given the particular focus on \ac{NCG}, the work underpinning \emph{GeoProp}, could be expanded to provide a tighter coupling between the empirical \ac{SP2009} partitioning model and the pure component \ac{HEOS} models for \ce{H2O} and \ac{CO2}. Moreover to capture a wider range of \ac{NCG} compositions, as opposed to just \ce{CO2}, further work could focus on expanding the \ac{SP2009} model to include other constituents such as \ce{H2S} or \ce{CH4}.
