\chapter*{Abstract} 
Historically, power generation from two-phase and single-phase (liquid) geothermal resources has been confined to direct steam plants and binary plants respectively. Binary plant technology tends to provide a number of operational advantages over direct steam plant technology: ease of maintenance, effective surface handling and re-injection of possible co-produced \acf{NCG}, which can be harmful or have a \ac{GWP}. \textbf{The present work focuses on the comparison between the two technologies, with particular emphasis on two-phase geothermal fluids}. The design of geothermal power plants is a demanding task; different technologies, plant configurations and working fluids must be considered, and the optimum plant design is strongly dependent on the inlet conditions, composition, thermophysical properties and phase behaviour of the geofluid. Moreover, the impact and cost of handling and re-injecting non-condensable gases must be accounted for. 

Geofluid modelling has historically been focused on two research fields: 1) partitioning the geofluid into separate phases, and 2) the estimation of the thermophysical properties of these phases. Recognising potential synergies between these models, \emph{GeoProp} is introduced, a novel geofluid modelling framework, which addresses this niche by coupling existing state-of-the-art fluid partitioning simulators, such as \emph{Reaktoro}, with high-accuracy thermophysical fluid property computation engines, like \emph{CoolProp} and \emph{ThermoFun}.

Simulation of geothermal power plants using commercial software, such as \emph{Aspen Plus v11} faces some drawbacks; 1) accuracy of the thermophysical properties and phase behaviour models, particularly for geofluids with a multi-component chemical composition, and 2) important phenomena, such as wet expansion effects in steam turbines, are difficult to capture. With this in mind, \emph{PowerCycle} has been developed, a novel power plant simulation tool, which allows the use of virtually any thermophysical property model (e.g. \emph{CoolProp} and \emph{GeoProp}), and provides geothermal power generation specific component models for turbines, heat exchangers, pumps/compressors etc. 

With these tools the thermodynamic and techno-economic performance of binary \ac{ORC} and single flash \ac{DSC} geothermal power plants is compared for two-phase geothermal resources in the range of \qtyrange{423}{548}{\K}; a parametric study on the effect and handling of \ac{NCG} integrates the investigation.
For thermodynamic optimised power plants, binary \acp{ORC} were found to produce equal power with respect to single flash \ac{DSC} plants at geofluid vapour qualities as high as \qty{85}{\percent}. For techno-economically optimised  plants equal specific power plant cost were obtained for vapour qualities as high as \qty{80}{\percent}. With regards to the handling of \ac{NCG}, binary \acp{ORC} are better suited than single flash \acp{DSC}, for all considered scenarios (venting to the atmosphere, re-injection into the reservoir and partial dissolution). \acp{DSC} spend significant power on re-pressurising \ac{NCG} for disposal, be it for venting to the atmosphere or re-injection into the reservoir. While the latter is virtually uneconomical for single flash \ac{DSC} geothermal power plants it could be easier to realise with binary \ac{ORC} power plants, due to the higher \ac{NCG} pressures at the plant outlet and thus lower re-pressurisation power requirements. 
\\
\\
\textbf{Keywords:} geothermal energy, organic Rankine cycle, direct steam cycle, geofluids, non-condensable gases, thermophysical properties % Keywords

% ABSTRACT IN ITALIAN
\chapter*{Abstract in lingua italiana}
Storicamente, la generazione di energia elettrica da risorse geotermiche con fluido bifase e monofase liquido è stata limitata rispettivamente agli impianti a vapore diretto e agli impianti binari. Gli impianti binari offrono una serie di vantaggi operativi rispetto agli impianti a vapore diretto, tra cui la facilità di manutenzione e il trattamento e reiniezione ottimale degli eventuali \acf{NCG} co-prodotti, che possono essere dannosi o avere un \acf{GWP}. \textbf{Il presente lavoro è incentrato sul confronto tra le due tecnologie, con particolare riferimento all’ambito dei fluidi bifase}. La progettazione di queste centrali geotermiche è un compito impegnativo; devono essere prese in considerazione diverse tecnologie, configurazioni dell’impianto e fluidi di lavoro, e la progettazione ottimale dell’impianto dipende fortemente dalle condizioni di ingresso, dalla composizione, dalle proprietà termofisiche e dal comportamento del geofluido. Inoltre, è necessario tenere conto dell’impatto e del costo del trattamento e reiniezione dei gas incondensabili. 

La modellazione dei geofluidi si è storicamente concentrata su due ambiti di ricerca: 1) la ripartizione del geofluido in fasi separate e 2) la stima delle proprietà termofisiche di queste fasi. Riconoscendo le potenziali sinergie tra questi modelli, viene introdotto \emph{GeoProp}, un nuovo framework per la modellazione dei geofluidi, che affronta questa occorrenza accoppiando simulatori di ripartizione dei fluidi rappresentativi dello stato dell’arte, come \emph{Reaktoro}, con motori di calcolo delle proprietà termofisiche dei fluidi ad alta precisione, come \emph{CoolProp} e \emph{ThermoFun}.

La simulazione di centrali geotermiche utilizzando software commerciali, come \emph{Aspen Plus v11}, presenta alcuni inconvenienti: 1) l’accuratezza delle proprietà termofisiche e dei modelli di comportamento di fase, in particolare per i geofluidi con composizione chimica multicomponente, e fenomeni importanti, come gli effetti legati all’espansione all’interno della curva limite nelle turbine a vapore, sono difficili da riprodurre. Per questo motivo, è stato sviluppato \emph{PowerCycle}, un nuovo strumento di simulazione per impianti geotermoelettrici, che consente di utilizzare virtualmente qualsiasi modello di proprietà termofisiche (ad esempio \emph{CoolProp} e \emph{GeoProp}), e fornisce modelli dei singoli componenti specifici dell’impianto geotermico (turbine, scambiatori di calore, pompe/compressori ecc).

Con questi strumenti sono state confrontate le prestazioni termodinamiche e tecno-economiche delle centrali geotermiche binarie \ac{ORC} e a singolo flash \ac{DSC} per le risorse geotermiche bifase nell’intervallo \qtyrange{423}{548}{\K}: uno studio parametrico sull’effetto e la gestione dei \ac{NCG} completa l’analisi.

Per le centrali elettriche ottimizzate dal punto di vista termodinamico, si è riscontrato che gli \ac{ORC} binari producono la stessa potenza dei \ac{DSC} con titolo di vapore del geofluido fino a \qty{85}{\percent}. Per gli impianti ottimizzati dal punto di vista tecno-economico, sono stati ottenuti gli stessi costi specifici della centrale per titolo del vapore fino a \qty{80}{\percent}. Per quanto riguarda la gestione dei \ac{NCG}, gli \ac{ORC} binari sono più adatti dei \ac{DSC} single flash, per tutti gli scenari considerati (sfiato in atmosfera e reinieizione). I \ac{DSC} consumano una notevole quantità di energia per ripressurizzare i \ac{NCG} per lo smaltimento, sia per lo sfiato nell’atmosfera che per la re-iniezione nel giacimento. Quest’ultimo aspetto rende virtualmente impraticabile dal punto di vista economico la reiniezione nel caso di impianti a flash singolo \ac{DSC}, ma potrebbe essere più facile da realizzare con le centrali binarie \ac{ORC}, grazie alla minor richiesta di potenza di ripressurizzazione.
\\
\\
\textbf{Parole chiave:} energia geotermica, ciclo Rankine a fluido organico, ciclo a vapore diretto,
geofluidi, gas incondensabili, proprietà termofisiche
 % Keywords (italian)
