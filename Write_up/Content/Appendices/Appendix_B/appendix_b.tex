In their 2009 paper, Spycher and Pruess \cite{Spycher2009} provide a model for computing the mutual solubilities of water and carbon dioxide. This model is based on a simplified chemically active system approach, where two reversible reactions are taking place, water migrating to the carbon dioxide-rich phase, and carbon dioxide migrating to the water-rich phase. Salts may be present but are considered nonreactive and are thus confined to the water-rich phase.

\begin{align} 
    H_2O^{(aq)} \rightleftharpoons H_2O^{(g)}
\end{align}
\begin{align} 
    CO_2^{(g)} \rightleftharpoons CO_2^{(aq)}
\end{align}

For each reaction an equilibrium constant is defined based on the species activity in the aqueous (water-rich phase) and the fugacity in the gaseous (carbon dioxide-rich phase), see Equations ~\eqref{eq:eqConst_water} and ~\eqref{eq:eqConst_CO2}.

\begin{align} 
    K_{H_2 O}=f_{H_2 O} / a_{H_2 O} \label{eq:eqConst_water}
\end{align}
\begin{align} 
    K_{CO_2}=f_{CO_2} / a_{CO_2} \label{eq:eqConst_CO2}
\end{align}

Where \(f_i\) is taken to be the ratio of the component fugacity and the reference fugacity - \(f_i = f_i (P,T,y)/f_i^{o}\) - and the reference fugacity, \(f_i^{o}\), is defined to be 1 bar. In turn, \(f_i\) can be calculated from the component’s fugacity coefficient and mole fraction in the carbon dioxide-rich phase as well as the total pressure, see Equation ~\eqref{eq:fugacity}. Similarly, the component’s activity is calculated from its activity coefficient and water rich phase mole fraction, see Equations ~\eqref{eq:activity_water} and ~\eqref{eq:activity_CO2}. 

\begin{align} 
    f_i= \Phi_i*y_i*P_{tot} \label{eq:fugacity}
\end{align}
\begin{align} 
   a_{H_2 O}= \gamma_{H_2 O}*x_{H_2 O} \label{eq:activity_water}
\end{align}
\begin{align} 
   a_{CO_2}= \gamma_{CO_2}^{'} * \gamma_{CO_2}*x_{CO_2} \label{eq:activity_CO2}
\end{align}

Substituting Equations ~\eqref{eq:fugacity}, ~\eqref{eq:activity_water} and ~\eqref{eq:activity_CO2} into the respective expression for the equilibrium constant (i.e. Equations ~\eqref{eq:eqConst_water} and ~\eqref{eq:eqConst_CO2}) and using the fact that that mole fractions ought to sum to one, expressions for the mole fraction of water in the carbon dioxide rich phase and the mole fraction of carbon dioxide in the water rich phase can be obtained, see Equations~\eqref{eq:yH2O} and ~\eqref{eq:xCO2}.

\begin{align} 
    y_{H_2 O} = A*(1 - x_{CO_2} - x_{salt}) \label{eq:yH2O}
\end{align}
\begin{align} 
    x_{CO_2} = B^{'} * (1 - y_{H_2 O}) \label{eq:xCO2}
\end{align}
\begin{align} 
    A = \frac{K_{H_2 O} * \gamma_{H_2 O}}{\Phi_{H_2 O} * P_{tot}} \label{eq:defA}
\end{align}
\begin{align} 
    B^{'}= \frac{\Phi_{CO_2} * P_{tot}}{55.508*\gamma_{CO_2}^{'}* \gamma_{CO_2} * K_{CO_2}} \label{eq:defBdash}
\end{align}

Parameters \(x_{salt}\) and \(m_{CO_2}\) can be calculated from \eqref{eq:defx_salt} and \eqref{eq:defmolal_CO2} respectively.
\begin{align} 
    x_{salt}= \frac{v*m_{salt}}{55.508 + v*m_{salt}+m_{CO_2}} \label{eq:defx_salt}
\end{align}
\begin{align} 
    m_{CO_2} = \frac{55.508 * x_{CO_2}}{x_{H_2 O}}  \label{eq:defmolal_CO2}
\end{align}

Equations~\eqref{eq:yH2O} and ~\eqref{eq:xCO2} can either be solved by iteration or by substitution using Equation~\eqref{eq:solution_yH2O} below, followed by substituting the result into Equation~\eqref{eq:xCO2} to obtain \(x_{CO_2}\).

\begin{align} 
    y_{H_2 O}=\frac{55.508*(1-B^{'})}{(\frac{1}{A} - B^{'} )(v*m_{salt} + 55.508) + v*m_{salt} B^{'}}  \label{eq:solution_yH2O}
\end{align}

Thus, to calculate the mutual solubilities, a total of six parameters, three for water and three for carbon dioxide must be calculated:

\begin{itemize}
    \item The equilibrium constant
    \item The fugacity coefficient
    \item The activity coefficient
\end{itemize}

Subsequently, a mole balance can be used to estimate vapour fraction, \(\alpha\), see Equation~\eqref{eq:mole_balance}. Note, \(z_{H_2O}\) is the mole fraction across all phases.

\begin{align} 
    \alpha = \frac{z_{H_2 O} - x_{H_2 O}}{y_{H_2 O} - x_{H_2 O}}  \label{eq:mole_balance}
\end{align}

If \(\alpha \leq 0\), the geofluid is entirely liquid, i.e. all carbon dioxide is contained within the water-rich phase, whereas if \(\alpha \geq 1 \), the geofluid is entirely vapour, i.e. all water is contained within the carbon dioxide rich phase. Only when \(0 < \alpha < 1\) can the water and carbon dioxide rich phases coexist.

\begin{notes}{Note}
    If the pressure is below the saturation pressure of pure water then the geofluid is also assumed to be fully vapour.
\end{notes}

A Python implementation of this model is available from the following Github repository: \url{https://github.com/EASYGO-ITN/GeoProp}.

\section{The Equilibrium Constant}
\label{sec:equib_constant}

The equilibrium constant is calculated as per Equation~\eqref{eq:defEquilibConst}.  \(K_i^{o}\) and \(P_i^{ref}\) are calculated from polynomials of the form of \eqref{eq:Polynomial1}, while \(\bar{V_i}\) is calculated using \eqref{eq:Polynomial2}. The coefficients can be found in Tables~\ref{table:EquibDataLow} and ~\ref{table:EquibDataHigh}.
\begin{align} 
    K_i(T,P) = K_i^{o}(T)*\exp{\frac{(P-P_i^{ref})\bar V_i}{RT}}   \label{eq:defEquilibConst}
\end{align}
\begin{align} 
    F(T) = a + b(T-273.15) + c(T-273.15)^2 + d(T-273.15)^3 + e(T-273.15)^4   \label{eq:Polynomial1}
\end{align}
\begin{align} 
    \bar{V_i} = a + b(T-373.15)  \label{eq:Polynomial2}
\end{align}

\begin{table}[H]
    \caption{Low   temperature parameters: 285.15 K to 382.15 K , 1 bar to 600 bar \cite{Spycher2003}.}
    \centering
    \label{table:EquibDataLow}
    \begin{tabular}{|p{6em} c c c c c c|}
        \hline
        \rowcolor{bluepoli!40} % comment this line to remove the color
        \textbf{Parameter}& \textbf{Units} & \textbf{a} & \textbf{b} & \textbf{c} & \textbf{d} & \textbf{e} \T\B \\
        \hline \hline
        \(\log_{10}K_{H_2O}^{o}\) & bar & -2.21 & 3.097E-02 & -1.098E-04 & 2.048E-07 & 0.000E+00 \T\B \\
        \(\log_{10}K_{CO_2}^{o}\)(L) & bar mol\textsuperscript{-1} & 1.169 & 1.368E-02 & -5.380E-05 & - & - \T\B\\
        \(\log_{10}K_{CO_2}^{o}\) & bar mol\textsuperscript{-1} & 1.189 & 1.304E-02 & -5.446E-05 & - & - \B\\
        \(\Bar{V}_{CO_2}\) & cm\textsuperscript{3} mol\textsuperscript{-1} & 32.6 & - & - & - & - \B\\
        \(\Bar{V}_{H_2O}\) & cm\textsuperscript{3} mol\textsuperscript{-1} & 18.1 & - & - & - & - \B\\
        \(P_{ref}\) & bar & 1 & - & - & - & - \B\\
        \hline
    \end{tabular}
    \\[10pt]

    \caption{High   temperature parameters: 372.15 K to 573.15 K, 1 bar  to 600 bar \cite{Spycher2009}.}
    \centering 
    \label{table:EquibDataHigh}
    \begin{tabular}{|p{8.25em} c c c c c c|}
        \hline
        \rowcolor{bluepoli!40} % comment this line to remove the color
        \textbf{Parameter}& \textbf{Units} & \textbf{a} & \textbf{b} & \textbf{c} & \textbf{d} & \textbf{e} \T\B \\
        \hline \hline
        \(\log_{10}K_{H_2O}^{o}\) & bar & -2.11 & 2.813E-02 & -8.430E-05 & 1.497E-07 & -1.1812E-10 \T\B \\
        \(\log_{10}K_{CO_2}^{o}\) & bar mol\textsuperscript{-1} & 1.668 & 3.992E-03 & -1.156E-05 & 1.593E-09 & 0.000E+00 \T\B\\
        \(\Bar{V}_{CO_2}\) & cm\textsuperscript{3} mol\textsuperscript{-1} & 32.6 & 3.413E-02 & - & - & - \B\\
        \(\Bar{V}_{H_2O}\) & cm\textsuperscript{3} mol\textsuperscript{-1} & 18.1 & 3.137E-02 & - & - & - \B\\
        \(P_{ref}, T\leq373.15 K\) & bar & 1 & - & - & - & - \B\\
        \(P_{ref}, T>373.15 K\) & bar & 0.19906 & 2.0471E-03 & 1.0152E-04 & -1.4234E-06 & 1.4168E-08 \B\\
        \hline
    \end{tabular}
    \\[10pt]
\end{table}

\section{The Fugacity Coefficient}
\label{sec:fug_constant}
A Soave-Redlich-Kwong stlye cubic equation of state with asymmetric binary interaction coefficients is used to calculate the fugacity of water and carbon dioxide in the carbon dioxide-rich phase, see Equation~\eqref{eq:SRKEOS}

\begin{align} 
    P = \frac{RT}{V-b_{mix}} - \frac{a_{mix}}{\sqrt{T}*V(V-b_{mix}}  \label{eq:SRKEOS}
\end{align}

\(a_{mix}\) and \(b_{mix}\) are calculate using the following mixing rules:

\begin{align} 
    b_{mix} = \sum_{i=1}^N y_i * b_i \label{eq:bmix}
\end{align}
\begin{align} 
    a_{mix} = \sum_{i=1}^N \sum_{j=1}^N y_i y_j * a_{ij} \label{eq:amix}
\end{align}
\begin{align} 
    a_{ij} = \sqrt{a_i a_j} (1 - k_{ij}) \label{eq:aij}
\end{align}
\begin{align} 
    k_{ij} = K_{ij}y_i + K_{ji}y_j) \label{eq:kij}
\end{align}

Parameters \(a_i\), \(b_i\) and \(K_{ij}\) are calculated from polynomials of the form of Equation~\eqref{eq:Polynomial3}. The coefficients can be found in Tables~\ref{table:SRKDataLow} and ~\ref{table:SRKDataHigh}
\begin{align} 
    F(T) = a + bT \label{eq:Polynomial3}
\end{align}

\begin{table}[H]
    \caption{Low   temperature parameters: 285.15 K to 382.15 K , 1 bar to 600 bar \cite{Spycher2003}.}
    \centering
    \label{table:SRKDataLow}
    \begin{tabular}{|p{6em} c c c |}
        \hline
        \rowcolor{bluepoli!40} % comment this line to remove the color
        \textbf{Parameter}& \textbf{Units} & \textbf{a} & \textbf{b} \T\B \\
        \hline \hline
        \(a_{CO_2}\) & bar cm\textsuperscript{6} K\textsuperscript{0.5} mol\textsuperscript{-2} & 7.54E+07 & -4.13E+04 \T\B \\
        \(a_{H_2O}\) & bar cm\textsuperscript{6} K\textsuperscript{0.5} mol\textsuperscript{-2} & 0 & - \T\B\\
        \(a_{CO_2-H_2O}\) & bar cm\textsuperscript{6} K\textsuperscript{0.5} mol\textsuperscript{-2} & 7.89E+07 & - \B\\
        \(b_{CO_2}\) & cm\textsuperscript{3} mol\textsuperscript{-1} & 27.8 & - \B\\
        \(a_{H_2O}\) & cm\textsuperscript{3} mol\textsuperscript{-1} & 18.18 & - \B\\
        \hline
    \end{tabular}
    \\[10pt]

    \caption{High   temperature parameters: 372.15 K to 573.15 K, 1 bar  to 600 bar \cite{Spycher2009}.}
    \centering 
    \label{table:SRKDataHigh}
    \begin{tabular}{|p{6em} c c c |}
        \hline
        \rowcolor{bluepoli!40} % comment this line to remove the color
        \textbf{Parameter}& \textbf{Units} & \textbf{a} & \textbf{b} \T\B \\
        \hline \hline
        \(a_{CO_2}\) & bar cm\textsuperscript{6} K\textsuperscript{0.5} mol\textsuperscript{-2} & 8.008E+07 & -4.984E+07 \T\B \\
        \(a_{H_2O}\) & bar cm\textsuperscript{6} K\textsuperscript{0.5} mol\textsuperscript{-2} & 1.337E+08 & -1.400E+04 \T\B\\
        \(K_{H_2O-CO_2}\) & - & 1.427E-02 & -4.037E-04 \B\\
        \(K_{CO_2-H_2O}\) & - & 4.228E-01 & -7.422E-04 \B\\
        \(b_{CO_2}\) & cm\textsuperscript{3} mol\textsuperscript{-1} & 28.25 & - \B\\
        \(b_{H_2O}\) & cm\textsuperscript{3} mol\textsuperscript{-1} & 15.70 & - \B\\
        \hline
    \end{tabular}
    \\[10pt]
\end{table}

Determining the fugacity coefficient requires the volume to be determined, which can be achieved by recasting Equation~\eqref{eq:SRKEOS} into its cubic form, ~\eqref{eq:DepressedCubic}, and then applying the Cadorna method (or similar) to obtain the roots.

\begin{align} 
    V^3 + a_1 V^2 + a_2 V^3 + a_3= 0 \label{eq:DepressedCubic}
\end{align}
\begin{align} 
    a_1 = - \frac{RT}{P}
\end{align}
\begin{align} 
    a_2 = - \left( \frac{RTb_mix}{P} - \frac{a_{mix}}{P\sqrt{T}}+b_{mix}^2 \right)
\end{align}
\begin{align} 
    a_3 = - \frac{a_{mix}b_{mix}}{P\sqrt{T}}
\end{align}

If a single real root is found then this corresponds to the stable phase, however if three or more roots are found, the smallest root corresponds to the liquid phase and the largest root to the vapour phase. To determine, which is stable the following auxiliary parameters need to be calculated, Equation~\eqref{eq:Auxillary1} and ~\eqref{eq:Auxillary2}. If \(w_2 - w_1 \geq 0\), then largest root is accepted (i.e. stable vapour), otherwise the smallest root is taken (i.e. stable liquid).

\begin{align} 
    V_{liq} = \min V_1, V_2, V3
\end{align}
\begin{align} 
    V_{vap} = \max V_1, V_2, V3
\end{align}
\begin{align} 
    w_1 = P(V_{gas}-V_{liq})  \label{eq:Auxillary1}
\end{align}
\begin{align} 
    w_2 = RT * \ln \frac{V_{gas}-b_{mix}}{V_{liq}-b_{mix}} + \frac{a_{mix}}{b_{mix}\sqrt{T}} \ln \frac{(V_{gas} + b_{mix})*V_{liq}}{(V_{liq} + b_{mix})*V_{gas}}  \label{eq:Auxillary2}
\end{align}

With this in place the fugacity coefficient can be calculated using Equation~\eqref{eq:FugacityCoeff}. 

\begin{equation}
    \label{eq:FugacityCoeff}
    \begin{split}
        \ln \Phi_i =& \frac{b_k}{b_{mix}} \left( \frac{PV}{RT} - 1 \right) - \ln \frac{P(V-b_{mix})}{RT} \\
        & + \frac{a_{mix}}{b_{mix}RT^{1.5}} \ln \frac{V}{V+b_{mix}}\left(
        \begin{aligned}
            & \frac{1}{a_{mix}}\sum_{i=1}^N y_i(a_{ik} - a_{ki}) \\
            & \qquad - \frac{1}{a_{mix}}\sum_{i=1}^N\sum_{j=1}^N y_1^2 y_j (\mathbf{K_{ij}} - \mathbf{K_{ji}})\sqrt{a_i a_j} \\
            & \qquad \qquad + \frac{1}{a_{mix}} y_k \sum_{i=1}^N y_i(\mathbf{K_{ki}} - \mathbf{K_{ik}})\sqrt{a_i a_k} \\
            & \qquad \qquad \qquad- \frac{b_k}{b_{mix}}
        \end{aligned}
        \right)
    \end{split}
\end{equation}

\begin{notes}{Note}
    There appears to be a misprint in the original paper by Spycher and Pruess, whereby the \(K_{ij}\) highlighted above are instead written as \(k_{ij}\). This has the effect that since \(k_{ij} = k_{ji}\) by definition, most terms cancel out.
\end{notes}

\section{The Activity Coefficient}
\label{sec:activ_constant}

The activity coefficients for water and carbon dioxide in the water-rich phase are defined as per Equations~\eqref{eq:GammaH2O} and ~\eqref{eq:GammaCO2}	

\begin{align} 
    \ln \gamma_{H_2O} = \left( A_M -2A_M * x_{H_2O} \right) * x_{CO_2}^2 \label{eq:GammaH2O}
\end{align}
\begin{align} 
    \ln \gamma_{CO_2} = 2A_M * x_{CO_2} * x_{H_2O}^2  \label{eq:GammaCO2}
\end{align}
\begin{equation}
    \label{eq:activityAM}
    A_M = \left\{
    \begin{aligned}
    0 , & \ T \leq 373.15 K \\
    a * (T - 373.15) + b * (T373.15)^2 , & \ T > 373.15 K
    \end{aligned}
    \right.
\end{equation}

To account for salinity effects, a correction term for the activity of carbon dioxide in the water-rich phase is provided, see Equation ~\eqref{eq:activityCO2_corr}, where parameters \(\lambda\) and \(\xi\) are calculated using Equation~\eqref{eq:Polynomial4}.

\begin{equation}
    \label{eq:activityCO2_corr}
    \gamma_{CO_2}°{'} = \left(1 +\frac{\sum m_{i\neq CO_2}}{55.508} \right) \exp \left(
    \begin{aligned}
    &2\gamma (m_{Na^{+}} + m_{K^{+}} + 2m_{Ca^{+2}} + 2m_{Mg^{+2}}) \\
    &\qquad \quad+ \xi m_{Cl^{-}}(m_{Na^{+}} + m_{K^{+}} + m_{Ca^{+2}} + m_{Mg^{+2}}) \\
    &\qquad \qquad \quad - 0.07m_{SO_4^{-2}}
    \end{aligned}
    \right)
\end{equation}
\begin{align} 
    F(T) = aT +\frac{b}{T} + \frac{c}{T^2} \label{eq:Polynomial4}
\end{align}

The coefficients for Equations~\eqref{eq:activityAM} and ~\eqref{eq:Polynomial4} can be found in Table~\ref{table:ActDataHigh}

\begin{table}[H]
    \caption{High   temperature parameters: 372.15 K to 573.15 K, 1 bar  to 600 bar \cite{Spycher2009}.}
    \centering 
    \label{table:ActDataHigh}
    \begin{tabular}{|p{6em} c c c c |}
        \hline
        \rowcolor{bluepoli!40} % comment this line to remove the color
        \textbf{Parameter}& \textbf{Units} & \textbf{a} & \textbf{b} & \textbf{c}\T\B \\
        \hline \hline
        \(A_M\) & - & -3.084E-02 & 1.927E-05 & - \T\B\\
        \(\lambda\) & - & 2.217E-04 & 1.074 & 2648 \T\B \\

        \(\xi\) & - & 1.300E-05 & -20.12 & 5259 \B\\
        \hline
    \end{tabular}
    \\[10pt]
\end{table}

\section{Validation}
\label{sec:appb_validation}
The above equations were implemented in Python and then validated by digitalising the plots of the equilibrium mole fraction of water/carbon dioxide for different temperatures as calculated by \citeauthor{Spycher2009}. The plots were digitalised using WebPlotDigitizer \cite{Rohatgi2024}. From Figures~\ref{fig:SP2009_validation_yH2O} and \ref{fig:SP2009_validation_xCO2} it can be seen that our implementation of the model presented by \citeauthor{Spycher2009} matches their calculations for a wide range of temperatures and pressures

    \begin{figure}[H]
        \centering
        % This file was created with tikzplotlib v0.10.1.
\begin{tikzpicture}

\definecolor{crimson2143940}{RGB}{214,39,40}
\definecolor{darkgray176}{RGB}{176,176,176}
\definecolor{darkorange25512714}{RGB}{255,127,14}
\definecolor{forestgreen4416044}{RGB}{44,160,44}
\definecolor{lightgray204}{RGB}{204,204,204}
\definecolor{mediumpurple148103189}{RGB}{148,103,189}
\definecolor{orchid227119194}{RGB}{227,119,194}
\definecolor{sienna1408675}{RGB}{140,86,75}
\definecolor{steelblue31119180}{RGB}{31,119,180}

\begin{axis}[
legend cell align={left},
legend style={
  % fill opacity=0.8,
  % draw opacity=1,
  % text opacity=1,
  % at={(0.5,0.91)},
  % anchor=north,
  % draw=lightgray204,
  at={(1.03, 0.5)},
  anchor=west,
},
log basis y={10},
% tick align=outside,
% tick pos=left,
% x grid style={darkgray176},
xlabel={Pressure/\unit{\bar}},
xmin=0, xmax=600,
% xtick style={color=black},
% y grid style={darkgray176},
ylabel={\(y_{H_2O}\)/\unit{\percent}},
ymin=0.06, ymax=100,
ymode=log,
% ytick style={color=black},
% ytick={0.001,0.01,0.1,1,10,100,1000,10000,100000},
% yticklabels={
%   \(\displaystyle {10^{-3}}\),
%   \(\displaystyle {10^{-2}}\),
%   \(\displaystyle {10^{-1}}\),
%   \(\displaystyle {10^{0}}\),
%   \(\displaystyle {10^{1}}\),
%   \(\displaystyle {10^{2}}\),
%   \(\displaystyle {10^{3}}\),
%   \(\displaystyle {10^{4}}\),
%   \(\displaystyle {10^{5}}\)
% }
ylabel near ticks,
xlabel near ticks
]
\addplot [semithick, black, mark=x, mark size=3, mark options={solid}, only marks]
table {%
-1 1000
-1 1000
};
\addlegendentry{Samples}
\addplot [semithick, black]
table {%
-1 1000
-1 1000
};
\addlegendentry{This Study}
\addplot [semithick, orchid227119194]
table {%
1 2.35264971957208
2 1.18661188500407
3 0.798042909436359
4 0.603843305887396
5 0.487393248140123
6 0.409819523614363
7 0.354462241156084
8 0.312991508211641
9 0.280779656774538
10 0.255050133984396
11 0.234036088984049
13 0.201805473247432
15 0.17828960125294
17 0.160419461398019
21 0.135189088372316
23 0.126014099160879
25 0.118407897546522
27 0.112029314173569
29 0.10663225186952
31 0.102035204580275
33 0.0981019116914533
35 0.0947287122752502
37 0.091836090587067
39 0.0893629467939429
41 0.0872627205157718
43 0.0855008552146039
45 0.0840533362794944
47 0.0829062432368411
49 0.0820565085528615
51 0.081514521397142
54 0.0813526744282246
57 0.0822426012889671
60 0.0853166534215115
63 0.260240859772041
66 0.263243171610668
69 0.266038659674326
72 0.268656045336382
75 0.271118010367697
80 0.2749240461933
85 0.278413679086521
90 0.281635983014775
95 0.28462867004342
100 0.287421460584148
125 0.299091809975646
150 0.30807355014898
200 0.321147373390524
250 0.33016837792347
300 0.336627312675522
350 0.341316439505947
400 0.344713586944738
450 0.347131366124806
500 0.348786274810539
550 0.349834691526042
600 0.350393285369793
};
\addlegendentry{\qty{20}{\degreeCelsius}}
\addplot [semithick, sienna1408675]
table {%
1 4.51683471711622
2 2.27670675425374
3 1.53017271297495
4 1.15704018560931
5 0.933270777901633
6 0.784185101544104
7 0.67777777332292
8 0.598046165107499
9 0.536099987375552
10 0.486605122814147
11 0.4461671872364
13 0.384106940303503
15 0.338779488157144
17 0.304288013838946
21 0.255461044327703
23 0.237636518417898
25 0.222810265218576
27 0.210324850981041
29 0.199705100025449
31 0.190599436999602
33 0.182742587090295
35 0.175931098462732
37 0.170006844832548
39 0.164845655985083
41 0.16034934184477
43 0.156440027799543
45 0.153056113644619
47 0.150149417372258
49 0.147683231185634
51 0.145631139351113
54 0.143295892203336
57 0.141845726774367
60 0.141335921738369
63 0.141931447920552
66 0.144034909992659
69 0.148808095651535
72 0.16821009142237
75 0.320010303606056
80 0.333501876413481
85 0.344080753070529
90 0.352894816263377
95 0.360496155659726
100 0.367200071047013
125 0.392474485018069
150 0.409982140846556
200 0.433644145666457
250 0.449092452262799
300 0.459818598357877
350 0.467458663503529
400 0.472927960804077
450 0.476795608341932
500 0.479441370018738
550 0.481131829307522
600 0.482061393794867
};
\addlegendentry{\qty{31}{\degreeCelsius}}
\addplot [semithick, mediumpurple148103189]
table {%
1 19.9873187323954
2 10.0598646152343
3 6.75119703716956
4 5.09723223239361
5 4.10515357134599
6 3.44402228404596
7 2.97200756355852
8 2.61819409198314
9 2.34318454923113
10 2.12334061664519
11 1.94361979693063
13 1.6675201157532
15 1.46551622196881
17 1.3114722637043
21 1.09250698759603
23 1.01211265345535
25 0.944922551656159
27 0.888015530016043
29 0.839277047625009
31 0.797139041341842
33 0.760414393006203
35 0.728188157944251
37 0.699744078469779
39 0.674513717018478
41 0.652040486298789
43 0.63195372756195
45 0.613949712359501
47 0.597777507104801
49 0.583228312911115
51 0.570127328803914
54 0.55287586158314
57 0.538153149718716
60 0.525643846917749
63 0.515101728629109
66 0.50633574557022
69 0.499200143076213
72 0.493587578088459
75 0.489424538578709
80 0.48560568212928
85 0.48569681015941
90 0.489927368208858
95 0.498787705258638
100 0.513012799866181
125 0.651979826849887
150 0.756648383869949
200 0.860520897313372
250 0.915725684192483
300 0.950381880919357
350 0.97359545237227
400 0.989533245484759
450 1.00047120041752
500 1.00779444492035
550 1.01240991135276
600 1.01494358195364
};
\addlegendentry{\qty{60}{\degreeCelsius}}
\addplot [semithick, crimson2143940]
table {%
1 95.8606316813074
2 48.1757796031571
3 32.2821241340086
4 24.3362788064538
5 19.5695664614579
6 16.3924281601276
7 14.1236244568558
8 12.4225357176395
9 11.0999288928803
10 10.0422642265305
11 9.17728921481984
13 7.84755613362324
15 6.87359414564234
17 6.12986168652296
21 5.07004910767003
23 4.67963419448738
25 4.35248060035318
27 4.07454848949184
29 3.8356716410119
31 3.62830791508808
33 3.44674409564568
35 3.28657335893589
37 3.14434221224398
39 3.0173060672773
41 2.9032563534378
43 2.80039587889069
45 2.70724742896406
47 2.62258570135033
49 2.54538591039606
51 2.47478448458696
54 2.37968117578582
57 2.29575921968184
60 2.22138780492659
63 2.15524893809483
66 2.0962664220923
69 2.04355335369373
72 1.99637273562772
75 1.95410752742201
80 1.89321210656189
85 1.84260872883118
90 1.80079359753984
95 1.76658999103182
100 1.73906428613391
125 1.67935765329994
150 1.71044117131609
200 1.87552652280567
250 2.01822869760922
300 2.11276919497652
350 2.1746996649175
400 2.21554171070386
450 2.24218342684361
500 2.25884950997315
550 2.2682625210935
600 2.27225878878612
};
\addlegendentry{\qty{99}{\degreeCelsius}}
\addplot [semithick, forestgreen4416044]
table {%
1 450.35786229226
2 229.449350106173
3 154.215813867376
4 116.359756241962
5 93.5693149779796
6 78.343708239149
7 67.452957992816
8 59.2769033650954
9 52.9133622712664
10 47.8200665006706
11 43.6514543709304
13 37.2365578361712
15 32.5320068253014
17 28.935173041047
21 23.8004984433452
23 21.9050649979064
25 20.3144250783058
27 18.9609431992183
29 17.7956244137679
31 16.782106642357
33 15.8928357662703
35 15.106551073667
37 14.4065856218961
39 13.7796891909833
41 13.2151955038036
43 12.7044216987639
45 12.2402278434159
47 11.8166888454963
49 11.4288466660984
51 11.0725208032811
54 10.5895041390204
57 10.1596358952072
60 9.77499820083833
63 9.42918155842476
66 9.11694214469342
69 8.83394846871651
72 8.57659133444167
75 8.34183938844646
80 7.9940284723662
85 7.69237766564122
90 7.42937188548925
95 7.19907889372545
100 6.99675267027732
125 6.29152858973381
150 5.91800549819175
200 5.68206601314327
250 5.75163272098748
300 5.90449093842759
350 6.05642570334671
400 6.1850943393286
450 6.28859782216541
500 6.37021269826036
550 6.43387939057336
600 6.48306419504824
};
\addlegendentry{\qty{150}{\degreeCelsius}}
\addplot [semithick, darkorange25512714]
table {%
1 1156.51829327808
2 652.623858395477
3 458.830571630523
4 351.871315593291
5 284.987706905881
6 239.529580785241
7 206.716806503633
8 181.944082162249
9 162.586585006466
10 147.04581982364
11 134.294854299503
13 114.615288190719
15 100.135573867999
17 89.036063355412
21 73.1423030454507
23 67.2589954907123
25 62.3140616499141
27 58.100288332983
29 54.4673120114626
31 51.3033700415453
33 48.5236585798501
35 46.062665039542
37 43.8689770363057
39 41.9016821836461
41 40.127817517342
43 38.5205280321057
45 37.057714485989
47 35.7210252270183
49 34.4950940740325
51 33.3669569240847
54 31.8344670713569
57 30.4668770116671
60 29.2395854662059
63 28.1326640017945
66 27.1297978585099
69 26.2175023718406
72 25.3845348024306
75 24.6214469873916
80 23.4837413378863
85 22.4882570175331
90 21.6115235242842
95 20.8350008587108
100 20.143848505633
125 17.6150226082096
150 16.0739860889773
200 14.4980817743486
250 13.9225059261508
300 13.799580045365
350 13.8719266377027
400 14.0146184372955
450 14.1707891918517
500 14.3168943960795
550 14.4446397400657
600 14.5523487339373
};
\addlegendentry{\qty{200}{\degreeCelsius}}
\addplot [semithick, steelblue31119180]
table {%
1 3373.19628441501
2 1638.35814509514
3 929.438254932963
4 714.066105009353
5 601.538317778862
6 520.234131916313
7 457.802146052199
8 408.337966914968
9 368.251267183253
10 335.180855846863
11 307.49411434631
13 263.898057882313
15 231.264869966687
17 206.000394594738
21 169.534057082698
23 155.96385183873
25 144.530537718346
27 134.767787719483
29 126.335317167584
31 118.979222751117
33 112.50630948205
35 106.767013676241
37 101.643761915768
39 97.0428584203852
41 92.8887176073862
43 89.1196893914688
45 85.684987653441
47 82.5423965971268
49 79.6565346753605
51 76.9975241994237
54 73.3798022129648
57 70.1452390721801
60 67.2369628135922
63 64.6088513008231
66 62.2231048688591
69 60.0484487078598
72 58.0587821872011
75 56.2321505675229
80 53.5007968028779
85 51.1015597449896
90 48.9797284948602
95 47.0920370156331
100 45.4038185724287
125 39.1277015673981
150 35.1558048327492
200 30.7255372465483
250 28.6777010621692
300 27.8119322448473
350 27.5664846647821
400 27.635432569002
450 27.846152943384
500 28.104082810654
550 28.3601988366968
600 28.5909404160945
};
\addlegendentry{\qty{250}{\degreeCelsius}}
\addplot [semithick, orchid227119194, mark=x, mark size=3, mark options={solid}, only marks, forget plot]
table {%
5.9908 0.7985
7.8341 0.597
7.8341 0.3978
11.5207 0.297
15.2074 0.1985
18.894 0.1604
24.424 0.1269
31.7972 0.1022
41.0138 0.0821
52.0737 0.0776
56.682 0.0799
57.6037 0.1201
59.447 0.1649
58.5253 0.1985
61.2903 0.2433
81.5668 0.2612
114.7465 0.2769
147.9263 0.2836
176.4977 0.2881
212.4424 0.297
245.6221 0.306
283.4101 0.3127
311.9816 0.3149
352.5346 0.3172
398.6175 0.3194
455.7604 0.3239
500.9217 0.3284
540.553 0.3284
588.4793 0.3306
};
\addplot [semithick, sienna1408675, mark=x, mark size=3, mark options={solid}, only marks, forget plot]
table {%
7.9727 0.7816
7.9727 0.7
8.8838 0.6206
9.795 0.5456
11.6173 0.4706
13.4396 0.4022
18.9066 0.3118
23.4624 0.2632
29.8405 0.1993
35.3075 0.1728
41.6856 0.1574
49.8861 0.1441
58.9977 0.1397
64.4647 0.1375
69.9317 0.1441
71.754 0.1529
73.5763 0.1926
73.5763 0.2368
75.3986 0.2765
76.3098 0.314
89.0661 0.3449
100 0.3603
120.9567 0.3846
146.4692 0.4
174.7153 0.4199
195.672 0.4265
242.1412 0.4441
281.3212 0.4529
326.8793 0.4618
374.2597 0.4662
428.9294 0.475
486.3326 0.4772
557.4032 0.4816
};
\addplot [semithick, mediumpurple148103189, mark=x, mark size=3, mark options={solid}, only marks, forget plot]
table {%
20.7373 1.178
22.5806 1.1152
22.5806 1.0437
27.1889 0.9679
28.1106 0.8964
30.8756 0.8271
34.5622 0.7708
36.4055 0.7101
40.0922 0.669
44.7005 0.6191
52.0737 0.5693
59.447 0.526
66.8203 0.5065
71.4286 0.4892
79.7235 0.4848
87.0968 0.4848
94.47 0.4935
99.0783 0.5065
105.53 0.5282
111.0599 0.5628
118.4332 0.604
127.6498 0.6581
135.023 0.6971
147.0046 0.7383
158.9862 0.7751
176.4977 0.8141
199.5392 0.8531
223.5023 0.8834
255.7604 0.9159
284.3318 0.9375
315.6682 0.9549
356.2212 0.9722
429.9539 0.9917
474.1935 1.0004
525.8065 1.0047
570.0461 1.009
};
\addplot [semithick, crimson2143940, mark=x, mark size=3, mark options={solid}, only marks, forget plot]
table {%
30.7865 3.9189
32.5843 3.6689
36.1798 3.3919
39.7753 3.1351
42.4719 2.9595
47.8652 2.6824
55.0562 2.4257
61.3483 2.2703
69.4382 2.1216
81.1236 1.9662
93.7079 1.8581
104.4944 1.7838
120.6742 1.7432
144.0449 1.7568
166.5169 1.7973
190.7865 1.8784
217.7528 1.9865
246.5169 2.0473
274.382 2.1081
301.3483 2.1486
331.9101 2.2027
359.7753 2.2297
390.3371 2.2568
426.2921 2.2838
476.6292 2.2973
517.9775 2.3041
};
\addplot [semithick, forestgreen4416044, mark=x, mark size=3, mark options={solid}, only marks, forget plot]
table {%
10.3139 59.6
10.3139 53.6364
12.1076 46.5091
13.9013 41.5636
16.5919 35.6
19.2825 30.0727
22.87 23.5273
30.0448 18
40.8072 13.2
60.5381 9.8545
85.6502 7.6727
113.4529 6.6545
157.3991 5.7818
213.0045 5.9273
265.0224 5.6364
317.0404 5.9273
392.3767 6.0727
476.6816 6.3636
541.2556 6.5091
};
\addplot [semithick, darkorange25512714, mark=x, mark size=3, mark options={solid}, only marks, forget plot]
table {%
19.2661 79.7674
24.7706 71.9934
26.6055 65.0166
28.4404 57.6412
32.1101 51.8605
37.6147 46.0797
42.2018 39.701
51.3761 33.3223
64.2202 28.1395
77.9817 23.9535
97.2477 20.5648
114.6789 18.5714
144.0367 16.1794
174.3119 14.7841
206.422 14.186
236.6972 13.7874
273.3945 13.588
331.1927 13.588
367.8899 13.588
408.2569 13.7874
448.6239 13.9867
481.6514 13.9867
527.5229 14.186
563.3028 14.186
599.0826 14.3854
};
\addplot [semithick, steelblue31119180, mark=x, mark size=3, mark options={solid}, only marks, forget plot]
table {%
50.8121 79.9668
60.0928 68.6047
76.7981 56.0465
93.5035 48.0731
112.065 42.4917
132.4826 37.907
152.9002 34.7176
170.5336 32.9236
199.3039 30.7309
228.0742 29.5349
269.8376 28.3389
318.0974 27.7409
364.5012 27.1429
414.6172 27.5415
470.3016 27.7409
518.5615 27.7409
563.109 28.7375
599.3039 28.3389
};
\end{axis}

\end{tikzpicture}
        
        \caption{Comparison of the equilibrium mole fraction of water in a carbon dioxide-rich phase as calculated by \citeauthor{Spycher2003} and \citeauthor{Spycher2009} against our implementation of their model}
        \label{fig:SP2009_validation_yH2O}
    \end{figure}
    \begin{figure}[H]
        \centering
        % This file was created with tikzplotlib v0.10.1.
\begin{tikzpicture}

\definecolor{crimson2143940}{RGB}{214,39,40}
\definecolor{darkgray176}{RGB}{176,176,176}
\definecolor{darkorange25512714}{RGB}{255,127,14}
\definecolor{forestgreen4416044}{RGB}{44,160,44}
\definecolor{lightgray204}{RGB}{204,204,204}
\definecolor{mediumpurple148103189}{RGB}{148,103,189}
\definecolor{orchid227119194}{RGB}{227,119,194}
\definecolor{sienna1408675}{RGB}{140,86,75}
\definecolor{steelblue31119180}{RGB}{31,119,180}

\begin{axis}[
legend cell align={left},
legend style={
  % fill opacity=0.8,
  % draw opacity=1,
  % text opacity=1,
  % at={(0.03,0.97)},
  % anchor=north west,
  % draw=lightgray204,
  at={(1.03, 0.5)},
  anchor=west,
},
% tick align=outside,
% tick pos=left,
% x grid style={darkgray176},
xlabel={Pressure/\unit{\bar}},
xmin=0, xmax=600,
% xtick style={color=black},
% y grid style={darkgray176},
ylabel={\(x_{CO_2}\)/\unit{\percent}},
ymin=0, ymax=9,
% ytick style={color=black}
ylabel near ticks,
xlabel near ticks
]
\addplot [semithick, black, mark=x, mark size=3, mark options={solid}, only marks]
table {%
-1 0
-1 0
};
\addlegendentry{Samples}
\addplot [semithick, black]
table {%
-1 0
-1 0
};
\addlegendentry{This Study}
\addplot [semithick, orchid227119194, mark=x, mark size=3, mark options={solid}, only marks, forget plot]
table {%
3.9648 0.2566
12.7753 0.7743
24.2291 1.2655
32.1586 1.7168
41.8502 2.0619
50.6608 2.3274
55.9471 2.4602
64.7577 2.5398
73.5683 2.5531
89.4273 2.6062
103.5242 2.6195
125.5507 2.6858
147.5771 2.6991
170.4846 2.7655
207.489 2.8053
245.3744 2.8982
276.2115 2.9381
309.6916 2.9912
332.5991 3.0044
356.3877 3.0442
388.9868 3.0973
415.4185 3.1239
445.3744 3.177
486.7841 3.2168
542.2907 3.2832
};
\addplot [semithick, orchid227119194]
table {%
1 0.0653247926336365
2 0.131360247318469
3 0.196540672844325
4 0.260870725067352
5 0.324354999479936
6 0.3869980298946
7 0.448804287030193
8 0.509778176992118
9 0.569924039637462
10 0.629246146814943
11 0.687748700468461
13 0.802311593020832
15 0.913644852836863
17 1.02177934772966
21 1.22856859671631
23 1.32727781705191
25 1.42289691708249
27 1.5154486203101
29 1.60495347262642
31 1.6914295612775
33 1.77489216647402
35 1.85535332411346
37 1.93282126909016
39 2.00729971489879
41 2.07878690349778
43 2.14727432386141
45 2.21274493712173
47 2.2751706379597
49 2.33450847652741
51 2.39069474575702
54 2.46884604789077
57 2.53911916860705
60 2.60019599323272
63 2.51441493832889
66 2.52219759386243
69 2.52984060875024
72 2.53735511134764
75 2.54475056853287
80 2.55683345629869
85 2.56863954506079
90 2.58019457609512
95 2.59151990430326
100 2.60263353829511
125 2.6554999674246
150 2.70472701965934
200 2.79510892268245
250 2.87738460949771
300 2.95343535127034
350 3.02441618976896
400 3.09110719549519
450 3.15406988156671
500 3.21372750000453
550 3.27041047261781
600 3.32438399595852
};
\addlegendentry{\qty{20}{\degreeCelsius}}
\addplot [semithick, sienna1408675, mark=x, mark size=3, mark options={solid}, only marks, forget plot]
table {%
2.0225 0.0633
11.9101 0.6063
26.2921 1.1222
38.8764 1.5294
54.1573 1.991
63.1461 2.1674
68.5393 2.2353
73.9326 2.3032
85.618 2.3167
117.9775 2.3846
142.2472 2.4389
171.9101 2.5339
197.0787 2.5611
223.1461 2.6018
261.7978 2.6968
293.2584 2.7511
334.6067 2.8054
382.2472 2.8869
452.3596 2.9683
525.1685 3.0769
593.4831 3.1584
};
\addplot [semithick, sienna1408675]
table {%
1 0.0492827466363969
2 0.100290118453331
3 0.150699066946316
4 0.200512743348626
5 0.249734266316697
6 0.298366721434451
7 0.346413160684667
8 0.393876601885051
9 0.440760028086468
10 0.487066386930558
11 0.532798589963713
13 0.62255198985819
15 0.710042769087295
17 0.795292838911824
21 0.959155148732422
23 1.03780774870784
25 1.11430026766752
27 1.18865087723116
29 1.26087683456559
31 1.33099440365012
33 1.39901876210206
35 1.46496389016083
37 1.52884243740475
39 1.59066556136614
41 1.65044273024263
43 1.70818147910462
45 1.76388710493679
47 1.8175622798233
49 1.86920655240391
51 1.91881569332855
54 1.98939253443561
57 2.05531242943015
60 2.11647359474034
63 2.17270057074217
66 2.22368164135647
69 2.26878917620921
72 2.30584807264241
75 2.29600128843308
80 2.31145177865509
85 2.32606166293033
90 2.34002568841602
95 2.35346465853231
100 2.366461262498
125 2.42650252604883
150 2.48073142450865
200 2.57799601199996
250 2.66503754558897
300 2.74474923496738
350 2.81873684422587
400 2.8880181030512
450 2.95329362235444
500 3.01507407528993
550 3.07374773147589
600 3.12961957499625
};
\addlegendentry{\qty{31}{\degreeCelsius}}
\addplot [semithick, mediumpurple148103189, mark=x, mark size=3, mark options={solid}, only marks, forget plot]
table {%
6.1503 0.1818
23.4624 0.5636
43.508 1.0136
55.3531 1.2455
72.6651 1.5455
88.1549 1.7091
110.0228 1.8727
129.1572 1.9682
154.6697 2.0773
253.9863 2.2818
342.369 2.4727
444.4191 2.6091
520.0456 2.7182
595.672 2.8136
};
\addplot [semithick, mediumpurple148103189]
table {%
1 0.0240980682246103
2 0.0539282236908063
3 0.0834840626515818
4 0.112766906803538
5 0.141778069866682
6 0.170518857564195
7 0.198990567600705
8 0.227194489638992
9 0.255131905275075
10 0.282804088011631
11 0.310212303229661
13 0.364241851843101
15 0.417230510537358
17 0.469188106657489
21 0.570048752496079
23 0.61897078448649
25 0.666899715160711
27 0.713844689560489
29 0.759814710683541
31 0.804818637153197
33 0.84886518065199
35 0.891962903098731
37 0.934120213546735
39 0.975345364778864
41 1.01564644957297
43 1.0550313966092
45 1.09350796598841
47 1.13108374432907
49 1.16776613940797
51 1.20356237430902
54 1.25561050490897
57 1.30570344043661
60 1.35386300902356
63 1.400109911549
66 1.4444636487019
69 1.48694244141512
72 1.52756314660185
75 1.56634117249993
80 1.62691336651363
85 1.68245947457602
90 1.73302272385372
95 1.77863992774758
100 1.81935769866457
125 1.95945622148193
150 2.04612502273426
200 2.17676564827454
250 2.28371754263978
300 2.37786910403317
350 2.463384999714
400 2.54242614590541
450 2.61629882353388
500 2.68587051039568
550 2.75175556138895
600 2.81441003966434
};
\addlegendentry{\qty{60}{\degreeCelsius}}
\addplot [semithick, crimson2143940, mark=x, mark size=3, mark options={solid}, only marks, forget plot]
table {%
26.9406 0.4533
47.032 0.7617
68.9498 1.0981
85.3881 1.2664
112.7854 1.5187
137.4429 1.6729
165.7534 1.8271
191.3242 1.9813
233.3333 2.1215
273.516 2.2757
334.7032 2.4159
383.105 2.5421
446.1187 2.6542
518.2648 2.7944
598.6301 2.9065
};
\addplot [semithick, crimson2143940]
table {%
1 0.000842091372508321
2 0.0210139570098535
3 0.0410476956525778
4 0.0609438987978718
5 0.0807031559452749
6 0.100326054609207
7 0.119813180331676
8 0.139165116695176
9 0.158382445335765
10 0.177465745956357
11 0.196415596340209
13 0.233917248014918
15 0.270891984759432
17 0.307344361160281
21 0.378700105070782
23 0.413612437039206
25 0.448020338621845
27 0.48192822281512
29 0.515340475936113
31 0.548261458224602
33 0.580695504472868
35 0.612646924685472
37 0.644120004771363
39 0.675119007270853
41 0.705648172120167
43 0.735711717456504
45 0.765313840466724
47 0.794458718283002
49 0.823150508929045
51 0.851393352320671
54 0.892924855690442
57 0.933469349192333
60 0.973040610177722
63 1.01165235463741
66 1.0493182481707
69 1.08605191799506
72 1.12186696609924
75 1.15677698364687
80 1.21298610437518
85 1.26678214802429
90 1.31822865910022
95 1.36738989165389
100 1.41433113270418
125 1.61814201534475
150 1.77747593487639
200 2.00598203845093
250 2.17160118515993
300 2.30761918369489
350 2.42658564914289
400 2.53413423695859
450 2.63327998505786
500 2.72584501247257
550 2.81302759281723
600 2.89566559681683
};
\addlegendentry{\qty{99}{\degreeCelsius}}
\addplot [semithick, forestgreen4416044, mark=x, mark size=3, mark options={solid}, only marks, forget plot]
table {%
5.9406 0.0044
24.7525 0.2832
42.5743 0.5619
58.4158 0.7876
78.2178 1.0398
96.0396 1.2655
114.8515 1.4779
140.5941 1.7035
162.3762 1.8894
186.1386 2.0885
213.8614 2.2743
237.6238 2.4336
263.3663 2.5398
289.1089 2.6858
313.8614 2.7788
338.6139 2.8982
365.3465 3.0177
390.099 3.1239
412.8713 3.2035
437.6238 3.2832
459.4059 3.3628
487.1287 3.4425
507.9208 3.5354
531.6832 3.6018
555.4455 3.6814
578.2178 3.7345
598.0198 3.8142
};
\addplot [semithick, forestgreen4416044]
table {%
1 -0.0609944522738537
2 -0.0450239450828673
3 -0.0282821716908178
4 -0.0113677177321072
5 0.00557777432155286
6 0.0225049629355507
7 0.039392321806293
8 0.0562290127699571
9 0.0730090138005047
10 0.0897287322702979
11 0.106385908950199
13 0.139507209209479
15 0.172365928872223
17 0.204958741227428
21 0.269341261773402
23 0.301130204362337
25 0.33265110939945
27 0.363904404284206
29 0.394890663945583
31 0.425610569714571
33 0.456064883115899
35 0.486254428618943
37 0.516180081962328
39 0.54584276205432
41 0.575243425230086
43 0.604383061100091
45 0.633262689495958
47 0.661883358188064
49 0.690246141155386
51 0.718352137256959
54 0.760032118275156
57 0.801140745980103
60 0.841682018139158
63 0.881660041275716
66 0.92107902821152
69 0.959943296244722
72 0.998257265781078
75 1.03602545929171
80 1.09777221670985
85 1.1580378409528
90 1.2168451285062
95 1.2742176409104
100 1.33017969974697
125 1.58973676139832
150 1.81810031810904
200 2.19666609082599
250 2.49735923208106
300 2.74686771662808
350 2.96299418345139
400 3.15633601632836
450 3.33318082407821
500 3.49742663083469
550 3.6516210888643
600 3.79751201982417
};
\addlegendentry{\qty{150}{\degreeCelsius}}
\addplot [semithick, darkorange25512714, mark=x, mark size=3, mark options={solid}, only marks, forget plot]
table {%
25.7426 0.136
50.495 0.536
79.2079 0.952
106.9307 1.304
129.703 1.624
162.3762 2.008
192.0792 2.328
219.802 2.6
248.5149 2.872
274.2574 3.08
300 3.288
331.6832 3.528
362.3762 3.768
393.0693 3.992
432.6733 4.216
462.3762 4.408
494.0594 4.6
522.7723 4.76
555.4455 4.92
587.1287 5.112
};
\addplot [semithick, darkorange25512714]
table {%
1 -0.209830680589779
2 -0.196192370565988
3 -0.184165282072564
4 -0.170960331838576
5 -0.156819285309495
6 -0.14204804380314
7 -0.126863830642195
8 -0.111406385982936
9 -0.0957653235864359
10 -0.0799994737128217
11 -0.0641486180210685
13 -0.032294794281159
15 -0.000345378909574973
17 0.0316212542844356
21 0.0954361845356526
23 0.127235132346014
25 0.158941527098648
27 0.190545607840569
29 0.222040184910984
31 0.253419828970211
33 0.284680349656216
35 0.315818450158069
37 0.346831492241519
39 0.37771733266618
41 0.408474206944814
43 0.439100645219942
45 0.469595410377783
47 0.499957451843782
49 0.530185870621655
51 0.560279892516352
54 0.605167489678274
57 0.649749301971555
60 0.694023779734938
63 0.737989661648074
66 0.781645926345186
69 0.824991756734955
72 0.868026513327649
75 0.910749714047461
80 0.981261837575989
85 1.05090684576567
90 1.11968484313151
95 1.18759675272537
100 1.2546442624723
125 1.57702429580435
150 1.87843899555021
200 2.42243363504952
250 2.89758542429717
300 3.31734951101392
350 3.69465896122736
400 4.04003161191148
450 4.36122490138826
500 4.66372269607187
550 4.95137032345634
600 5.22688825336974
};
\addlegendentry{\qty{200}{\degreeCelsius}}
\addplot [semithick, steelblue31119180, mark=x, mark size=3, mark options={solid}, only marks, forget plot]
table {%
38.9313 -0.0435
64.3766 0.4348
105.0891 1.087
149.8728 1.8043
191.6031 2.4565
230.2799 3.0435
307.6336 4.1304
345.2926 4.6957
386.0051 5.2609
427.7354 5.8261
476.5903 6.4783
519.3384 7.0435
551.9084 7.5
586.514 7.9565
};
\addplot [semithick, steelblue31119180]
table {%
1 -0.653841086378044
2 -0.587318523834475
3 -0.527281299132599
4 -0.506015076503974
5 -0.493764302302543
6 -0.482140363523801
7 -0.470356117000907
8 -0.458236300156135
9 -0.445714655475876
10 -0.432776119439212
11 -0.419434502395587
13 -0.391664375281455
15 -0.362685201804279
17 -0.332765308382717
21 -0.270921271293673
23 -0.239288020715552
25 -0.207316001819702
27 -0.175076269618617
29 -0.142623276698775
31 -0.109999283136248
33 -0.0772375075615523
35 -0.0443643864785528
37 -0.0114012077192016
39 0.0216346960746392
41 0.0547290654310063
43 0.0878700546399711
45 0.12104773317503
47 0.154253704608061
49 0.187480812314075
51 0.220722909979323
54 0.270602723554086
57 0.320489293744996
60 0.370370119834738
63 0.420234761156227
66 0.470074388206378
69 0.5198814466336
72 0.569649402483058
75 0.619372546783056
80 0.702131450227207
85 0.784732039760643
90 0.867157434785795
95 0.949393166279459
100 1.03142662525023
125 1.43820715596948
150 1.83864431033483
200 2.61804066756336
250 3.36722783281498
300 4.08756180335536
350 4.78365190525344
400 5.46224457543851
450 6.13083340315622
500 6.79657752970924
550 7.4657308112937
600 8.14344582145374
};
\addlegendentry{\qty{250}{\degreeCelsius}}
\end{axis}

\end{tikzpicture}
        
        \caption{Comparison of the equilibrium mole fraction of carbon dioxide in a water-rich phase as calculated by \citeauthor{Spycher2003} and \citeauthor{Spycher2009} against our implementation of their model}
        \label{fig:SP2009_validation_xCO2}
    \end{figure}
