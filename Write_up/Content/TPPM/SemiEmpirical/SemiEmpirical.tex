\todo{perhaps this could be framed as future work with potential}

In specific cases it is also possible to tightly couple different models to extend their respective capabilities or to improve speed of calculations. For example, mixtures of water and carbon dioxide can be modelled using their respective high-fidelity pure fluid formulation; however, convergence is not guaranteed for all conditions, particularly for quasi-pure water, see Section~\ref{sec:excess_properties}. On the other hand, the \acf{SP2009} model can be used to partition mixtures of water and carbon dioxide, providing stable convergence for a broad range of compositions, temperatures and pressure.

The following section aims to investigate the coupling of the \acf{SP2009} model with the \acf{WP} \ac{EOS} for water and the \acf{SW} \ac{EOS} for pure carbon dioxide.

\subsection{Formulation}
\label{sec_semiempirical_formulation}
    Assuming a system comprised of two phases, one water-rich phase, hereinafter \emph{wp}, and one carbon dioxide-rich, hereinafter \emph{cp}, where the only reactions taking place are the migrations of water and carbon dioxide molecules from one phase to another, see Reaction~\ref{eq:reaction_1} and ~\ref{eq:reaction_2}.

    \begin{align}
        H_2O^{(wp)} \rightleftharpoons H_2O^{(cp)} \label{eq:reaction_1}
    \end{align}
    \begin{align}
        CO_2^{(cp)} \rightleftharpoons CO_2^{(wp)} \label{eq:reaction_2}
    \end{align}

    At equilibrium the chemical potential of a species \(i\) is equal across all phases, see Equation~\ref{eq:chempot_at_equib}, where the chemical potential of species \(i\) at a given temperature, pressure and composition can be calculated from its chemical potential at some reference conditions and the activity of species \(i\), as previously shown in Equations~\ref{eq:chemical_potential} and ~\ref{eq:activity_def}. By convention, the reference conditions depend on the phase that the species occupies. 

    \begin{align}
        \mu_i^{(x)}= \mu_i^{(y)}= \mu_i^{(z)}= ... \label{eq:chempot_at_equib}
    \end{align}
    \begin{align}
        \ln a_i^{j}(P, T, \mathbf{z}_j)= \frac{\mu_i^{j} (P, T, \mathbf{z}_j) - \mu_i^{j} (P^o, T^o, \mathbf{z}_j^o)}{RT} \tag{\ref{eq:activity_def}}
    \end{align}
    \begin{align}
        \mu_i^{j} (P, T, \mathbf{z}_j) = \mu_i^{j} (P^o, T^o, \mathbf{z}_j^o) +RT*\ln a_i (P, T, \mathbf{z}_j) \tag{\ref{eq:chemical_potential}}
    \end{align}

    It is then possible to calculate the thermodynamic properties of species \(i\) for the partial derivaties of chemical potential. The property \(\Psi\), a placeholder for the molar enthalpy, molar entropy or molar volume, can be calculated via Equation~\ref{eq:props_from_mu}. The choice of \(x\) and \(f(x)\) depends on the property to be calculated: for the partial molar enthalpy \(x=1/T\) and \(f(x)=x\), for the partial molar entropy \(x=T\) and \(f(x)=-1\), and for the partial molar volume \(x=P\) and \(f(x)=1\), with the unused properties being constant.

    \begin{align}
        \Psi_i^{j} (P, T, \mathbf{z}_j) = \frac{\partial \left( \mu_i^{j} (P, T, \mathbf{z}_j)*f(x) \right)}{\partial x} \label{eq:props_from_mu}
    \end{align}

    Substituting Equation~\ref{eq:chemical_potential} into Equation~\ref{eq:props_from_mu}, yields Equation~\ref{eq:props_from_mu_expandend}, which simplifies to Equation~\ref{eq:props_from_mu_expandend_simp} by recognising that \(\Psi_i^{j} (P^o, T^o, \mathbf{z}_j^o)=\frac{\partial \left( \mu_i (P^o, T^o, \mathbf{z}_j^o)*f(x) \right)}{\partial x}\).

    \begin{align}
        \Psi_i^{j} (P, T, \mathbf{z}_j) = \frac{\partial \left( \mu_i^{j} (P^o, T^o, \mathbf{z}_j^o)*f(x) \right)}{\partial x} + \frac{\partial \left(RT * f(x)* \ln a_i^{j} (P, T, \mathbf{z}_j) \right)}{\partial x} \label{eq:props_from_mu_expandend}
    \end{align}
    \begin{align}
        \Psi_i^{j} (P, T, \mathbf{z}_j) = \Psi_i^{j} (P^o, T^o, \mathbf{z}_j^o) + \frac{\partial \left(RT * f(x)* \ln a_i^{j} (P, T, \mathbf{y}_j) \right)}{\partial x} \label{eq:props_from_mu_expandend_simp}
    \end{align}

    Equation~\ref{eq:props_from_mu_expandend_simp} provides the basis for coupling the \ac{WP} and \ac{SW} \ac{EOS} with the \ac{SP2009} model. The \ac{WP} and \ac{SW} \ac{EOS} can be used to evaluate the properties of water and carbon dioxide at their respective reference conditions, i.e. \(\Psi_i^{j} (P^o, T^o, \mathbf{z}_j^o)\). Meanwhile, the \ac{SP2009} model is used to 1) partition the fluid into water-rich and carbon dioxide-rich phase and 2) to provide the species activity (i.e. \(\ln a_i^{j} (P, T, \mathbf{z}_j)\), which accounts for the deviation from the reference conditions and composition.

\subsection{Change of Reference Conditions}
\label{sec:change_ref_cond}

    However, Equation~\ref{eq:props_from_mu_expandend_simp} gives undue importance to the the activity model as it not only corrects for the presence of other species but also the temperature and pressure being different to the reference conditions. For example, considering the boundary case where the fluid is pure (either water or carbon dioxide) then it would be best to simply use the corresponding \ac{HEOS} to calculate the fluid's properties. However, with the current formulation it would be the activity model's responsibility to correct the properties at the reference conditions to the temperature and pressure of interest, which can introduce inconsistencies in the pure component properties, as the activity model does not have the same level of accuracy as the \ac{HEOS}. This has previously been illustrated in Section~\ref{sec:chemically_active_system_limitations}.

    An alternative approach is to change the reference conditions to be the current temperature and pressure and composition corresponding to a pure substance, i.e. \(P, T, \mathbf{z}_j^o\), by adding and subtracting \(\mu_i (P, T, \mathbf{z}_j^o)\) to Equation~\ref{eq:chemical_potential}, yielding Equation~\ref{eq:chem_pot_change_ref}. The definition of the species activity allows this expression to be simplified to Equation~\ref{eq:chem_pot_change_ref_simp}, where \(i\) is the species index, and \(j\) is the phase index

    \begin{align}
        \mu_i^{j} (P, T, \mathbf{z}_j) = \mu_i^{j} (P^o, T^o, \mathbf{z}_j^o) +RT*\ln a_i^{j} (P, T, \mathbf{z}_j) + \mu_i^{j} (P, T, \mathbf{z}_j^o) - \mu_i^{j} (P, T, \mathbf{z}_j^o) \label{eq:chem_pot_change_ref}
    \end{align}
    \begin{align}
        \mu_i^{j} (P, T, \mathbf{z}_j) = \mu_i^{j} (P, T, \mathbf{z}_j^o) + RT*\ln a_i^{j} (P, T, \mathbf{z}_j) -RT*\ln a_i^{j} (P, T, \mathbf{z}_j^o)
    \end{align}
    \begin{align}
        \mu_i^{j} (P, T, \mathbf{z}_j) = \mu_i^{j} (P, T, \mathbf{z}_j^o) + RT*\ln \frac{a_i^{j} (P, T, \mathbf{z}_j)}{a_i^{j} (P, T, \mathbf{z}_j^0)} \label{eq:chem_pot_change_ref_simp}
    \end{align}

    From Equation~\ref{eq:chem_pot_change_ref_simp}, the thermodynamic properties can then be obtained as outlined above, yielding Equation~\ref{eq:props_chempot_changed_ref}, where \(A_i^{j} (P, T, \mathbf{z}_j) = \frac{a_i^{j} (P, T, \mathbf{z}_j)}{a_i^{j} (P, T, \mathbf{z}_j^o)}\)

    \begin{align}
        \Psi_i^{j} (P, T, \mathbf{z}_j) = \Psi_i^{j} (P, T, \mathbf{z}_{j}^o) + \frac{\partial \left(RT * f(x)* \ln A_i^{j} (P, T, \mathbf{z}_j) \right)}{\partial x} \label{eq:props_chempot_changed_ref}
    \end{align}

    This formulation allows the properties to be evaluated at the temperature and pressure of interest directly from the corresponding \ac{HEOS} and the activity model only corrects for the difference in composition. The additional benefit being that if the fluid is pure, the properties are entirely consistent with those predicted by the \ac{HEOS} as \(A_i^{j} (P, T, \mathbf{z}_j) = 1\) and hence the activity contribution is zero.

    The properties of species \(i\) in phase \(j\) can then be aggregated by phase or by all phases to obtain the phase or total properties, Equations~\ref{eq:props_phase} and \ref{eq:props_total}.

    \begin{align}
        \Psi_i^j (P, T, \mathbf{z}_j) = \Psi_i^j (P, T, \mathbf{z}_j^o) + \frac{\partial \left(RT * f(x)* \ln A_i^j (P, T, \mathbf{z}_j) \right)}{\partial x} \label{eq:props_comp_phase}
    \end{align}
    \begin{align}
        \Psi^j (P, T, \mathbf{z}_j) = \sum_{i=0}^N n_i^j\Psi_i^j (P, T, \mathbf{z}_j) \label{eq:props_phase}
    \end{align}
    \begin{align}
        \Psi (P, T, \mathbf{z}) = \sum_{i=0}^N n^j\Psi^j (P, T, \mathbf{z}_j) \label{eq:props_total}
    \end{align}

\subsection{Extrapolation}
\label{sec:extrapolation}
    \todo{After speaking with Allan, this section needs some more work... essentially the extrapolation should be treated as tuning to reduce the differences to the CoolProp model}
    One drawback of the above formulation is that the models used to obtain \(\Psi_i^j (P, T, \mathbf{z}_j)\) are not necessarily continuous over the fully temperature and pressure domain of interest. For example, considering pure water, if the temperature is below the saturation temperature at a given pressure (but above the melting point), water can only exist in its liquid state - a vapour or solid state is not feasible. However, the presence of impurities changes the saturation point. Specifically, for a mixture of water and carbon dioxide, some water may exist in its vapour state despite the temperature being below the saturation temperature of pure water. This poses an obstacle, because the \ac{WP} \ac{EOS} is designed to model physical states of pure water, but such equilibrium states are non-physical for pure water. 

    \todo{insert the chemical potential diagram}

    As an alternative, the desired properties of water vapour could be obtained by extrapolating from the saturation point. The following methods were considered, also see Table~\ref{table:SemiEmpirical_ExtrapolationFuncs}:

    \begin{itemize}
        \item \emph{Gibbs Energy}: the Gibbs energy is linearly extrapolated from the saturation point, with the remaining properties being determined from the partial derivatives. This has the unfortunate consequence that the molar enthalpy and entropy are constant with temperature.
        \item \emph{Enthalpy \& Entropy}: both the molar enthalpy and molar entropy are linearly extrapolated from the saturation point, with the Gibbs energy being back-calculated from the resulting values of the molar enthalpy and entropy; the volume is not computed.
        \item \emph{Density}: the ideal gas law is used to compute the density of the water vapour at the current temperature. In turn, the temperature and density are then used to compute the properties from the \ac{WP} \ac{EOS}.
        \item \emph{Power Law}: it is assumed that \(v \propto T^a\), with the value of \(a\)  being back-calculated from the saturation point.
    \end{itemize}

    \begin{table}[H]
        \caption{The extrapolation schemes investigated for obtaining water vapour properties for sub-saturation conditions}
        \centering 
        \label{table:SemiEmpirical_ExtrapolationFuncs}
        \begin{tabular}{|p{7em} | c | c | c | c|}
    \hline
    \rowcolor{bluepoli!40}
    \textbf{Extrapolation} & \textbf{Gibbs Energy} & \textbf{Enthalpy} & \textbf{Entropy} & \textbf{Volume} \T\B \\
    \hline \hline 
    Gibbs Energy & \(G_{sat} - \Delta T*\frac{dG}{dT}\) & \(\frac{d(G/T)}{d(1/T)}\) & \(\frac{dG}{dT}\) & \(\frac{dG}{dP}\) \T\B\\
    \hline
    Enthalpy \& Entropy  & H - TS & \(H_{sat} - \Delta T*\frac{dH}{dT}\) & \(S_{sat} - \Delta T*\frac{dS}{dT}\) & - \T\B\\
    \hline
    Density & \multicolumn{4}{c|}{\(\rho=\rho _{sat}*\frac{T_{sat}}{T}\)} \T\B\\
     & \(G(T,\rho)\) & \(H(T,\rho)\) & \(S(T,\rho)\) & \(1/\rho\)\T\B\\
    \hline
    Ideal Gas & - & - & - & \(v_{sat}*\frac{T}{T_{sat}}\)\T\B\\
    \hline
    Power Law & - & - & - & \(a=\frac{T_{sat}}{v_{sat}}*\frac{dv}{dT}\)\T\B\\
     & - & - & - & \(v_{sat} * \left(\frac{T}{T_{sat}} \right)^a\)\T\B\\
    \hline
\end{tabular}        
        \\[10pt]
    \end{table}

    While the selection of extrapolation method is arbitrary, provided it allows existing property data to be reproduced, care was taken to ensure that the selected methods display smoothly around the saturation point, and that when extrapolating backwards (i.e. to conditions where \(T>T_{sat}\) and water vapour can exist) the extrapolated properties are close to those predicted by the \ac{WP} \ac{EOS}, see Figures~\ref{fig:SemiEmpirical_extrapolation1} and \ref{fig:SemiEmpirical_extrapolation2}. Based on these criteria the \emph{Enthalpy \& Entropy} method was selected for the molar enthalpy and molar entropy, and the \emph{Power Law} method was chosen for the molar volume.

    \begin{notes}{Note:}
        While it was attempted to also extrapolate the derivatives of the species activity from the saturation point, it was ultimately decided to simply use the derivatives evaluated at the saturation point to ensure model stability.
    \end{notes}

    \begin{figure}[H]
        \centering
        \input{Content/TPPM/SemiEmpirical/Plots/ExtrapolationFuncs_part1}
        \caption{The molar Gibbs energy, molar enthalpy and molar entropy of water vapour at sub-saturation conditions, as extrapolated by the various methods considered, see Table~\ref{table:SemiEmpirical_ExtrapolationFuncs}}
        \label{fig:SemiEmpirical_extrapolation1}
    \end{figure}

    \begin{figure}[H]
        \centering
        % This file was created with tikzplotlib v0.10.1.
\begin{tikzpicture}

\definecolor{darkgray176}{RGB}{176,176,176}
\definecolor{darkorange25512714}{RGB}{255,127,14}
\definecolor{forestgreen4416044}{RGB}{44,160,44}
\definecolor{lightgray204}{RGB}{204,204,204}
\definecolor{steelblue31119180}{RGB}{31,119,180}

\begin{axis}[
legend cell align={left},
legend style={
  % fill opacity=0.8,
  % draw opacity=1,
  % text opacity=1,
  at={(1.03,0.5)},
  anchor=west,
  % draw=lightgray204
},
% log basis y={10},
% tick align=outside,
% tick pos=left,
% x grid style={darkgray176},
xlabel={Temperature/\unit{\K}},
xmin=298, xmax=700,
% xtick style={color=black},
% y grid style={darkgray176},
ylabel={Volume/\unit{\cubic\m \per\mole}},
ymin=1e-05, ymax=0.1,
ymode=log,
% ytick style={color=black},
% ytick={1e-07,1e-05,0.001,0.1,10,1000},
% yticklabels={
%   \(\displaystyle {10^{-7}}\),
%   \(\displaystyle {10^{-5}}\),
%   \(\displaystyle {10^{-3}}\),
%   \(\displaystyle {10^{-1}}\),
%   \(\displaystyle {10^{1}}\),
%   \(\displaystyle {10^{3}}\)
% }
ylabel near ticks,
xlabel near ticks,
width = 7.5cm, height=7cm
]
    \addplot [semithick, black]
    table {%
    0 1
    };
    \addlegendentry{WP EOS}
    \addplot [semithick, black, mark=*, mark size=3, mark options={solid}, only marks]
    table {%
    0 1
    };
    \addlegendentry{Saturation}
    \addplot [semithick, black, dashed]
    table {%
    0 1
    };
    \addlegendentry{G Extrap.}
    \addplot [semithick, black, dotted]
    table {%
    0 1
    };
    \addlegendentry{IdealGas Extrap.}
    \addplot [semithick, black, dash pattern=on 1pt off 3pt on 3pt off 3pt]
    table {%
    0 1
    };
    \addlegendentry{Power Extrap.}
    \addplot [semithick, steelblue31119180]
    table {%
    373.755928897123 0.0306051041977436
    378.607980593138 0.0310332038999433
    383.460032289153 0.0314591088609619
    388.312083985168 0.0318831453330401
    393.164135681182 0.0323055539817563
    398.016187377197 0.0327265241658882
    402.868239073212 0.033146211586911
    407.720290769227 0.0335647479603254
    412.572342465241 0.0339822467328254
    417.424394161256 0.03439880675304
    422.276445857271 0.0348145148218167
    427.128497553286 0.0352294475853638
    431.980549249301 0.0356436730125589
    436.832600945315 0.036057251588772
    441.68465264133 0.036470237303519
    446.536704337345 0.0368826784803917
    451.38875603336 0.0372946184818288
    456.240807729375 0.0377060963120227
    461.092859425389 0.0381171471354881
    465.944911121404 0.0385278027249808
    };
    \addlegendentry{P=1 bar}
    \addplot [semithick, steelblue31119180, mark=*, mark size=3, mark options={solid}, only marks, forget plot]
    table {%
    372.755928897123 0.0305165608645238
    };
    \addplot [semithick, steelblue31119180, forget plot]
    table {%
    279.566946672842 1.80165658036436e-05
    284.418998368857 1.80228443189366e-05
    289.271050064872 1.80346424918743e-05
    294.123101760887 1.80513579732168e-05
    298.975153456902 1.80725192775695e-05
    303.827205152916 1.80977517447863e-05
    308.679256848931 1.81267541561107e-05
    313.531308544946 1.81592822172187e-05
    318.383360240961 1.81951366145866e-05
    323.235411936975 1.82341542071955e-05
    328.08746363299 1.82762014231897e-05
    332.939515329005 1.8321169242286e-05
    337.79156702502 1.83689693414718e-05
    342.643618721035 1.84195311094006e-05
    347.495670417049 1.84727993201621e-05
    352.347722113064 1.85287323152513e-05
    357.199773809079 1.85873005830908e-05
    362.051825505094 1.86484856542136e-05
    366.903877201108 1.87122792510062e-05
    371.755928897123 1.87786826461874e-05
    };
    \addplot [semithick, steelblue31119180, dashed, forget plot]
    table {%
    279.566946672842 0.0222598827740029
    284.418998368857 0.0226897815450415
    289.271050064872 0.0231196803160802
    294.123101760887 0.0235495790871188
    298.975153456902 0.0239794778581575
    303.827205152916 0.0244093766291961
    308.679256848931 0.0248392754002347
    313.531308544946 0.0252691741712734
    318.383360240961 0.025699072942312
    323.235411936975 0.0261289717133507
    328.08746363299 0.0265588704843893
    332.939515329005 0.0269887692554279
    337.79156702502 0.0274186680264666
    342.643618721035 0.0278485667975052
    347.495670417049 0.0282784655685438
    352.347722113064 0.0287083643395825
    357.199773809079 0.0291382631106211
    362.051825505094 0.0295681618816598
    366.903877201108 0.0299980606526984
    371.755928897123 0.030427959423737
    373.755928897123 0.0306051623053105
    378.607980593138 0.0310350610763492
    383.460032289153 0.0314649598473878
    388.312083985168 0.0318948586184264
    393.164135681182 0.0323247573894651
    398.016187377197 0.0327546561605037
    402.868239073212 0.0331845549315423
    407.720290769227 0.033614453702581
    412.572342465241 0.0340443524736196
    417.424394161256 0.0344742512446583
    422.276445857271 0.0349041500156969
    427.128497553286 0.0353340487867355
    431.980549249301 0.0357639475577742
    436.832600945315 0.0361938463288128
    441.68465264133 0.0366237450998515
    446.536704337345 0.0370536438708901
    451.38875603336 0.0374835426419287
    456.240807729375 0.0379134414129674
    461.092859425389 0.038343340184006
    465.944911121404 0.0387732389550447
    };
    \addplot [semithick, steelblue31119180, dotted, forget plot]
    table {%
    279.566946672842 0.0228874206483928
    284.418998368857 0.0232846455331514
    289.271050064872 0.0236818704179101
    294.123101760887 0.0240790953026687
    298.975153456902 0.0244763201874273
    303.827205152916 0.0248735450721859
    308.679256848931 0.0252707699569445
    313.531308544946 0.0256679948417031
    318.383360240961 0.0260652197264618
    323.235411936975 0.0264624446112204
    328.08746363299 0.026859669495979
    332.939515329005 0.0272568943807376
    337.79156702502 0.0276541192654962
    342.643618721035 0.0280513441502548
    347.495670417049 0.0284485690350135
    352.347722113064 0.0288457939197721
    357.199773809079 0.0292430188045307
    362.051825505094 0.0296402436892893
    366.903877201108 0.0300374685740479
    371.755928897123 0.0304346934588065
    373.755928897123 0.030598428270241
    378.607980593138 0.0309956531549996
    383.460032289153 0.0313928780397582
    388.312083985168 0.0317901029245169
    393.164135681182 0.0321873278092755
    398.016187377197 0.0325845526940341
    402.868239073212 0.0329817775787927
    407.720290769227 0.0333790024635513
    412.572342465241 0.0337762273483099
    417.424394161256 0.0341734522330686
    422.276445857271 0.0345706771178272
    427.128497553286 0.0349679020025858
    431.980549249301 0.0353651268873444
    436.832600945315 0.035762351772103
    441.68465264133 0.0361595766568616
    446.536704337345 0.0365568015416203
    451.38875603336 0.0369540264263789
    456.240807729375 0.0373512513111375
    461.092859425389 0.0377484761958961
    465.944911121404 0.0381457010806547
    };
    \addplot [semithick, steelblue31119180, dash pattern=on 1pt off 3pt on 3pt off 3pt, forget plot]
    table {%
    279.566946672842 0.0223521841923462
    284.418998368857 0.0227723276211109
    289.271050064872 0.0231930610521656
    294.123101760887 0.0236143753960784
    298.975153456902 0.0240362618512102
    303.827205152916 0.024458711890075
    308.679256848931 0.0248817172465554
    313.531308544946 0.0253052699039049
    318.383360240961 0.0257293620834796
    323.235411936975 0.0261539862341419
    328.08746363299 0.0265791350222888
    332.939515329005 0.0270048013224581
    337.79156702502 0.0274309782084711
    342.643618721035 0.0278576589450746
    347.495670417049 0.0282848369800468
    352.347722113064 0.0287125059367348
    357.199773809079 0.0291406596069961
    362.051825505094 0.0295692919445148
    366.903877201108 0.0299983970584698
    371.755928897123 0.0304279692075312
    373.755928897123 0.030605172073059
    378.607980593138 0.0310353942661578
    383.460032289153 0.0314660702264989
    388.312083985168 0.0318971946816662
    393.164135681182 0.0323287624856601
    398.016187377197 0.0327607686143423
    402.868239073212 0.0331932081610989
    407.720290769227 0.0336260763327083
    412.572342465241 0.0340593684454009
    417.424394161256 0.034493079921102
    422.276445857271 0.0349272062838446
    427.128497553286 0.0353617431563442
    431.980549249301 0.0357966862567266
    436.832600945315 0.036232031395399
    441.68465264133 0.0366677744720578
    446.536704337345 0.037103911472825
    451.38875603336 0.037540438467507
    456.240807729375 0.0379773516069682
    461.092859425389 0.0384146471206151
    465.944911121404 0.0388523213139828
    };
    \addplot [semithick, darkorange25512714]
    table {%
    454.028007881676 0.00351235661124847
    459.936271143277 0.00357546402804142
    465.844534404878 0.00363702644052686
    471.752797666479 0.00369740442685451
    477.66106092808 0.00375681626266118
    483.569324189681 0.00381540884795562
    489.477587451282 0.00387329010002339
    495.385850712883 0.00393054451663538
    501.294113974484 0.00398724113522943
    507.202377236085 0.00404343792506486
    513.110640497686 0.0040991844270412
    519.018903759287 0.00415452348112089
    524.927167020888 0.00420949244407219
    530.835430282489 0.00426412409938716
    536.74369354409 0.00431844736633356
    542.651956805691 0.00437248786868834
    548.560220067292 0.00442626839999029
    554.468483328893 0.00447980930936753
    560.376746590494 0.0045331288246579
    566.285009852095 0.00458624332503592
    };
    \addlegendentry{P=10 bar}
    \addplot [semithick, darkorange25512714, mark=*, mark size=3, mark options={solid}, only marks, forget plot]
    table {%
    453.028007881676 0.00350148206214618
    };
    \addplot [semithick, darkorange25512714, forget plot]
    table {%
    339.771005911257 1.83818456197703e-05
    345.679269172858 1.84450180255277e-05
    351.587532434459 1.8512143164282e-05
    357.49579569606 1.85831621943783e-05
    363.404058957661 1.86580406680686e-05
    369.312322219262 1.87367662331484e-05
    375.220585480863 1.88193470027022e-05
    381.128848742464 1.89058104627799e-05
    387.037112004065 1.89962028268197e-05
    392.945375265666 1.90905887744168e-05
    398.853638527267 1.91890515340157e-05
    404.761901788868 1.92916932866154e-05
    410.670165050469 1.93986358822562e-05
    416.57842831207 1.95100218741922e-05
    422.486691573671 1.96260158882238e-05
    428.394954835272 1.97468063575961e-05
    434.303218096873 1.98726076680406e-05
    440.211481358474 2.00036627739409e-05
    446.119744620075 2.01402463663932e-05
    452.028007881676 2.02826686986277e-05
    };
    \addplot [semithick, darkorange25512714, dashed, forget plot]
    table {%
    339.771005911257 0.00226619512968994
    345.679269172858 0.00233063618210693
    351.587532434459 0.00239507723452392
    357.49579569606 0.00245951828694091
    363.404058957661 0.0025239593393579
    369.312322219262 0.00258840039177489
    375.220585480863 0.00265284144419189
    381.128848742464 0.00271728249660888
    387.037112004065 0.00278172354902587
    392.945375265666 0.00284616460144286
    398.853638527267 0.00291060565385985
    404.761901788868 0.00297504670627684
    410.670165050469 0.00303948775869383
    416.57842831207 0.00310392881111082
    422.486691573671 0.00316836986352781
    428.394954835272 0.0032328109159448
    434.303218096873 0.0032972519683618
    440.211481358474 0.00336169302077879
    446.119744620075 0.00342613407319578
    452.028007881676 0.00349057512561277
    454.028007881676 0.0035123889986796
    459.936271143277 0.00357683005109659
    465.844534404878 0.00364127110351358
    471.752797666479 0.00370571215593057
    477.66106092808 0.00377015320834756
    483.569324189681 0.00383459426076455
    489.477587451282 0.00389903531318154
    495.385850712883 0.00396347636559853
    501.294113974484 0.00402791741801553
    507.202377236085 0.00409235847043252
    513.110640497686 0.00415679952284951
    519.018903759287 0.0042212405752665
    524.927167020888 0.00428568162768349
    530.835430282489 0.00435012268010048
    536.74369354409 0.00441456373251747
    542.651956805691 0.00447900478493446
    548.560220067292 0.00454344583735145
    554.468483328893 0.00460788688976844
    560.376746590494 0.00467232794218544
    566.285009852095 0.00473676899460243
    };
    \addplot [semithick, darkorange25512714, dotted, forget plot]
    table {%
    339.771005911257 0.00262611154660964
    345.679269172858 0.00267177688621119
    351.587532434459 0.00271744222581275
    357.49579569606 0.00276310756541431
    363.404058957661 0.00280877290501586
    369.312322219262 0.00285443824461742
    375.220585480863 0.00290010358421897
    381.128848742464 0.00294576892382053
    387.037112004065 0.00299143426342209
    392.945375265666 0.00303709960302364
    398.853638527267 0.0030827649426252
    404.761901788868 0.00312843028222676
    410.670165050469 0.00317409562182831
    416.57842831207 0.00321976096142987
    422.486691573671 0.00326542630103142
    428.394954835272 0.00331109164063298
    434.303218096873 0.00335675698023454
    440.211481358474 0.00340242231983609
    446.119744620075 0.00344808765943765
    452.028007881676 0.0034937529990392
    454.028007881676 0.00350921112525316
    459.936271143277 0.00355487646485472
    465.844534404878 0.00360054180445627
    471.752797666479 0.00364620714405783
    477.66106092808 0.00369187248365939
    483.569324189681 0.00373753782326094
    489.477587451282 0.0037832031628625
    495.385850712883 0.00382886850246405
    501.294113974484 0.00387453384206561
    507.202377236085 0.00392019918166717
    513.110640497686 0.00396586452126872
    519.018903759287 0.00401152986087028
    524.927167020888 0.00405719520047184
    530.835430282489 0.00410286054007339
    536.74369354409 0.00414852587967495
    542.651956805691 0.0041941912192765
    548.560220067292 0.00423985655887806
    554.468483328893 0.00428552189847962
    560.376746590494 0.00433118723808117
    566.285009852095 0.00437685257768273
    };
    \addplot [semithick, darkorange25512714, dash pattern=on 1pt off 3pt on 3pt off 3pt, forget plot]
    table {%
    339.771005911257 0.00233315445563773
    345.679269172858 0.00239061072905434
    351.587532434459 0.00244847220993624
    357.49579569606 0.0025067348744086
    363.404058957661 0.00256539480463296
    369.312322219262 0.00262444818432837
    375.220585480863 0.00268389129455054
    381.128848742464 0.00274372050971037
    387.037112004065 0.0028039322938148
    392.945375265666 0.00286452319691418
    398.853638527267 0.00292548985174211
    404.761901788868 0.00298682897053441
    410.670165050469 0.00304853734201527
    416.57842831207 0.00311061182853956
    422.486691573671 0.00317304936338101
    428.394954835272 0.0032358469481569
    434.303218096873 0.00329900165038063
    440.211481358474 0.0033625106011341
    446.119744620075 0.0034263709928526
    452.028007881676 0.00349058007721521
    454.028007881676 0.0035123939459932
    459.936271143277 0.00357706555669724
    465.844534404878 0.00364207965316137
    471.752797666479 0.00370743367080838
    477.66106092808 0.00377312509622894
    483.569324189681 0.00383915146553871
    489.477587451282 0.00390551036280773
    495.385850712883 0.00397219941855803
    501.294113974484 0.00403921630832569
    507.202377236085 0.0041065587512839
    513.110640497686 0.00417422450892369
    519.018903759287 0.00424221138378937
    524.927167020888 0.00431051721826573
    530.835430282489 0.00437913989341432
    536.74369354409 0.00444807732785641
    542.651956805691 0.00451732747670005
    548.560220067292 0.00458688833050932
    554.468483328893 0.00465675791431336
    560.376746590494 0.00472693428665349
    566.285009852095 0.00479741553866643
    };
    \addplot [semithick, forestgreen4416044]
    table {%
    585.147146966519 0.000327519412666254
    592.780662058184 0.000346311310697216
    600.414177149848 0.000362802123852569
    608.047692241513 0.00037775357288756
    615.681207333178 0.000391576987791443
    623.314722424843 0.000404528199999452
    630.948237516507 0.000416780743285698
    638.581752608172 0.000428459290607768
    646.215267699837 0.000439657127347321
    653.848782791501 0.000450446205264266
    661.482297883166 0.000460883351805354
    669.115812974831 0.000471014315760819
    676.749328066496 0.000480876515295257
    684.38284315816 0.00049050096783296
    692.016358249825 0.000499913682965948
    699.64987334149 0.000509136691111628
    707.283388433154 0.000518188818169375
    714.916903524819 0.000527086278839013
    722.550418616484 0.000535843137817035
    730.183933708149 0.000544471673001094
    };
    \addlegendentry{P=100 bar}
    \addplot [semithick, forestgreen4416044, mark=*, mark size=3, mark options={solid}, only marks, forget plot]
    table {%
    584.147146966519 0.000324815468175749
    };
    \addplot [semithick, forestgreen4416044, forget plot]
    table {%
    438.110360224889 1.98376085719536e-05
    445.743875316554 2.00053256920947e-05
    453.377390408219 2.01818721839686e-05
    461.010905499883 2.03678739811543e-05
    468.644420591548 2.05640575954332e-05
    476.277935683213 2.07712677731207e-05
    483.911450774877 2.09904899316858e-05
    491.544965866542 2.12228789350273e-05
    499.178480958207 2.14697964240414e-05
    506.811996049872 2.17328599118282e-05
    514.445511141536 2.20140083863679e-05
    522.079026233201 2.23155915958625e-05
    529.712541324866 2.2640494166552e-05
    537.346056416531 2.29923124216735e-05
    544.979571508195 2.3375613583208e-05
    552.61308659986 2.37963287889712e-05
    560.246601691525 2.42623736815989e-05
    567.880116783189 2.4784678434187e-05
    575.513631874854 2.5379008743034e-05
    583.147146966519 2.60694648284117e-05
    };
    \addplot [semithick, forestgreen4416044, dashed, forget plot]
    table {%
    438.110360224889 -7.51804604765381e-05
    445.743875316554 -5.42722014953695e-05
    453.377390408219 -3.3363942514201e-05
    461.010905499883 -1.24556835330324e-05
    468.644420591548 8.45257544813613e-06
    476.277935683213 2.93608344293047e-05
    483.911450774877 5.02690934104732e-05
    491.544965866542 7.11773523916418e-05
    499.178480958207 9.20856113728103e-05
    506.811996049872 0.000112993870353979
    514.445511141536 0.000133902129335147
    522.079026233201 0.000154810388316316
    529.712541324866 0.000175718647297485
    537.346056416531 0.000196626906278653
    544.979571508195 0.000217535165259822
    552.61308659986 0.00023844342424099
    560.246601691525 0.000259351683222159
    567.880116783189 0.000280259942203327
    575.513631874854 0.000301168201184496
    583.147146966519 0.000322076460165665
    585.147146966519 0.000327554476185833
    592.780662058184 0.000348462735167002
    600.414177149848 0.00036937099414817
    608.047692241513 0.000390279253129339
    615.681207333178 0.000411187512110507
    623.314722424843 0.000432095771091676
    630.948237516507 0.000453004030072844
    638.581752608172 0.000473912289054013
    646.215267699837 0.000494820548035181
    653.848782791501 0.00051572880701635
    661.482297883166 0.000536637065997519
    669.115812974831 0.000557545324978687
    676.749328066496 0.000578453583959856
    684.38284315816 0.000599361842941024
    692.016358249825 0.000620270101922193
    699.64987334149 0.000641178360903361
    707.283388433154 0.00066208661988453
    714.916903524819 0.000682994878865698
    722.550418616484 0.000703903137846867
    730.183933708149 0.000724811396828036
    };
    \addplot [semithick, forestgreen4416044, dotted, forget plot]
    table {%
    438.110360224889 0.000243611601131812
    445.743875316554 0.000247856223041208
    453.377390408219 0.000252100844950605
    461.010905499883 0.000256345466860002
    468.644420591548 0.000260590088769399
    476.277935683213 0.000264834710678796
    483.911450774877 0.000269079332588192
    491.544965866542 0.000273323954497589
    499.178480958207 0.000277568576406986
    506.811996049872 0.000281813198316383
    514.445511141536 0.00028605782022578
    522.079026233201 0.000290302442135176
    529.712541324866 0.000294547064044573
    537.346056416531 0.00029879168595397
    544.979571508195 0.000303036307863367
    552.61308659986 0.000307280929772764
    560.246601691525 0.00031152555168216
    567.880116783189 0.000315770173591557
    575.513631874854 0.000320014795500954
    583.147146966519 0.000324259417410351
    585.147146966519 0.000325371518941147
    592.780662058184 0.000329616140850544
    600.414177149848 0.00033386076275994
    608.047692241513 0.000338105384669337
    615.681207333178 0.000342350006578734
    623.314722424843 0.000346594628488131
    630.948237516507 0.000350839250397528
    638.581752608172 0.000355083872306924
    646.215267699837 0.000359328494216321
    653.848782791501 0.000363573116125718
    661.482297883166 0.000367817738035115
    669.115812974831 0.000372062359944512
    676.749328066496 0.000376306981853908
    684.38284315816 0.000380551603763305
    692.016358249825 0.000384796225672702
    699.64987334149 0.000389040847582099
    707.283388433154 0.000393285469491495
    714.916903524819 0.000397530091400892
    722.550418616484 0.000401774713310289
    730.183933708149 0.000406019335219686
    };
    \addplot [semithick, forestgreen4416044, dash pattern=on 1pt off 3pt on 3pt off 3pt, forget plot]
    table {%
    438.110360224889 7.87427393757862e-05
    445.743875316554 8.57360290468106e-05
    453.377390408219 9.32156264571468e-05
    461.010905499883 0.000101206299993923
    468.644420591548 0.000109733634289132
    476.277935683213 0.00011882404293552
    483.911450774877 0.000128504781186872
    491.544965866542 0.000138803958642948
    499.178480958207 0.000149750551919356
    506.811996049872 0.000161374417302603
    514.445511141536 0.000173706303390572
    522.079026233201 0.000186777863718673
    529.712541324866 0.000200621669371887
    537.346056416531 0.00021527122158294
    544.979571508195 0.000230760964316817
    552.61308659986 0.000247126296841828
    560.246601691525 0.000264403586287438
    567.880116783189 0.000282630180189051
    575.513631874854 0.000301844419019944
    583.147146966519 0.000322085648710547
    585.147146966519 0.00032756369546383
    592.780662058184 0.000349158730797641
    600.414177149848 0.000371873532577751
    608.047692241513 0.000395750813413037
    615.681207333178 0.000420834344588929
    623.314722424843 0.000447168968522958
    630.948237516507 0.000474800611208723
    638.581752608172 0.0005037762946484
    646.215267699837 0.000534144149273979
    653.848782791501 0.000565953426357345
    661.482297883166 0.000599254510409364
    669.115812974831 0.000634098931568115
    676.749328066496 0.000670539377976381
    684.38284315816 0.00070862970814857
    692.016358249825 0.000748424963327153
    699.64987334149 0.000789981379828788
    707.283388433154 0.000833356401380216
    714.916903524819 0.000878608691444078
    722.550418616484 0.000925798145534747
    730.183933708149 0.000974985903524325
    };
        
    \end{axis}
    

\end{tikzpicture}

        \caption{The molar volume of water vapour at sub-saturation conditions, as extrapolated by the various methods considered, see Table~\ref{table:SemiEmpirical_ExtrapolationFuncs}}
        \label{fig:SemiEmpirical_extrapolation2}
    \end{figure}


\subsection{Algorithm}
\label{sec:algorithm}
    Similarly to \emph{GeoProp}, the fluid is first equilibrated to determine the composition of the water and carbon dioxide-rich phases. Using these compositions, the composition contribution to the component properties are then evaluated (i.e. the partial derivatives of the component activity, \(A_i\)). For the carbon dioxide-rich phase the properties of pure carbon dioxide are then determined from the \ac{SW} \ac{HEOS}, while the properties of pure water are determined from the \ac{WP} \ac{HEOS} or extrapolated from the saturated properties if the temperature is below the saturation temperature of pure water. For the water-rich phase, the properties of pure water are obtained from the \ac{WP} \ac{HEOS}, while the properties of aqueous carbon dioxide is calculated using \emph{ThermoFun} using the "CO2@" component in the \emph{slop98-inorganic} database \cite{Johnson1992}. The overall fluid properties are then obtained by aggregating the component and phase properties, see Figure~\ref{fig:coupled_model}.

    \todo{maybe I should just write a wee routine to calculate the CO2@ properties directly instead of using ThermoFUN- for speed}

    \begin{figure}[H]
        \centering
        \begin{tikzpicture} [node distance=1.5cm]
    \node (start) [startstop] {Start};
    \node (input) [io, right of=start, xshift=2cm] {\(P, T, \mathbf{z}\)};
    \node (SP2009) [process, right of=input, text width=4.5cm, xshift=3cm] {\(\mathbf{x}, \mathbf{y}=SP2009(P,T,\mathbf{z})\)};
    \node (cp_derivs) [process, below of=SP2009, xshift=-2.5cm, yshift=-0.2cm, text width=3.5cm] {\(\frac{\partial RT*f(x)*A_i(P,T,\mathbf{y})}{\partial x}\)};
    \node (wp_derivs) [process, right of=cp_derivs, xshift=3cm, yshift=-0.2cm, text width=3.5cm] {\(\frac{\partial RT*f(x)*A_i(P,T,\mathbf{x})}{\partial x}\)};
    \node (cp_CO2) [process, below of=cp_derivs, text width=3.5cm] {\(\Psi_{CO_2}^{cp}=SW(P,T)\)};
    \node (Tsat) [decision, left of=cp_CO2, xshift=-3.7cm] {\(T<T_{sat}^{wat}\)};        

    \node (extrap) [process, below of=Tsat, text width=4cm, xshift=2.5cm] {\(\Psi_{H_2O}^{cp}=Extrap(P,T)\)};
    \node (cp_H2O) [process, below of=Tsat, text width=3.5cm, yshift=-1.5cm] {\(\Psi_{H_2O}^{cp}=WP(P,T)\)};
    \node (cp_props) [process, right of=cp_H2O, xshift=3.5cm] {\(\Psi^{cp}(P,T, \mathbf{y})\)};

    \node (wp_CO2) [process, below of=wp_derivs, text width=3.5cm] {\(\Psi_{CO_2}^{wp}=SW(P,T)\)};
    \node (wp_H2O) [process, below of=wp_CO2, text width=3.5cm] {\(\Psi_{H_2O}^{wp}=WP(P,T)\)};
    \node (wp_props) [process, below of=wp_H2O, text width=3.5cm] {\(\Psi^{wp}(P,T, \mathbf{x})\)};
    
    \node (out1) [io, below of=wp_props] {\(\Psi(P,T\mathbf{z})\)};
    \node (stop) [startstop, right of=out1, xshift=2cm] {Stop};
    
    % \node (stop) [startstop, below of=out1] {Stop};
    
    \draw [arrow] (start) -- (input);
    \draw [arrow] (input) -- (SP2009);
    \draw [arrow] (SP2009) |- ($0.5*(SP2009)+0.5*(cp_derivs)$) -| (cp_derivs);
    \draw [arrow] (cp_derivs) -- (cp_CO2);
    \draw [arrow] (cp_CO2) -- (Tsat);
    \draw [arrow] (Tsat) -- node[anchor=west] {yes} (extrap);
    \draw [arrow] (Tsat) -- node[anchor=east] {no} (cp_H2O);
    \draw [arrow] (cp_H2O) -- (cp_props);
    \draw [arrow] (extrap) -| (cp_props);
    \draw [arrow] (cp_props) -| ($0.5*(cp_props)+0.5*(wp_derivs)$) |- (wp_derivs);
    \draw [arrow] (wp_derivs) -- (wp_CO2);
    \draw [arrow] (wp_CO2) -- (wp_H2O);
    \draw [arrow] (wp_H2O) -- (wp_props);
    \draw [arrow] (wp_props) -- (out1);
    \draw [arrow] (out1) -- (stop);

\end{tikzpicture}
        \caption{Calculation steps for equilibrating a mixture of water and carbon dioxide and determining the thermophysical properties}
        \label{fig:coupled_model}
    \end{figure}

\subsection{Validation}
\label{sec:SemiEmpirical_validation}
    The proposed model was validated against the \ac{HEOS} mixture model for water and carbon dioxide implemented in \emph{CoolProp}. An extract of the validation plots is shown in Figures~\ref{fig:SemiEmpirical_properties_maintext} and \ref{fig:SemiEmpirical_ratios_maintext} for a 50:50 mixture of carbon dioxide and water, further validation plots are provided in \nameref{ch:appendix_e}.

    For temperatures up to \qty{466}{\K} (\qty{193}{\degreeCelsius}) the differences between the coupled model and the \ac{HEOS} mixture are below \qty{10}{\percent}, see Figure~\ref{fig:SemiEmpirical_ratios_maintext}, particularly for the density and molar volume, as well as the vapour quality. Comparing the relative differences in molar enthalpy and molar enthalpy is difficult at low temperatures, as the values are small in magnitude and transition from positive to negative, which can lead to \emph{infinite} relative differences. In absolute terms, the differences are small, see Figure~\ref{fig:SemiEmpirical_properties_maintext}.

    \begin{figure}[H]
        \centering
        \input{Content/TPPM/SemiEmpirical/Plots/SemiEmpirical_properties}
        \caption{The molar density/volume/enthalpy/entropy/vapour quality for a carbon dioxide mole fraction of \num{0.5} as a function of temperature and pressure, calculated using the \ac{HEOS} mixture in \emph{CoolProp} and the coupled model}
        \label{fig:SemiEmpirical_properties_maintext}
    \end{figure}

    \begin{figure}[H]
        \centering
        \input{Content/TPPM/SemiEmpirical/Plots/SemiEmpirical_ratios}
        \caption{The ratio of molar density/volume/enthalpy/entropy/vapour quality, as a function of temperature and pressure for a mole fraction of carbon dioxide of \num{0.5}, as calculated using the \ac{HEOS} mixture in \emph{CoolProp} and the coupled model}
        \label{fig:SemiEmpirical_ratios_maintext}
    \end{figure}

\subsection{Performance}
\label{sec:SemiEmpirical_performance}
    The performance of the coupled model, the \ac{HEOS} mixture as well as the pure component \ac{HEOS} was compared for randomised values of pressure, temperature and composition. The temperatures were obtained by linear sampling of temperatures between \qty{298}{\K} and \qty{573}{\K}.The pressures were obtained by linear sampling \(\log P\) for pressures between \qty{1}{\bar} to \qty{100}{\bar}. The compositions were obtained by linear sampling the standard deviation for values between \num{-3.5} and \num{+3.5} of the Normal distribution, to ensure more representative sampling of the tails (i.e. quasi pure compositions). The results are provided in Table~\ref{table:SemiEmpirical_Performance}

    \begin{table}[H]
        \caption{The computational performance over 10000 randomised calculations}
        \centering 
        \label{table:SemiEmpirical_Performance}
        \begin{tabular}{|p{7em} | c | c | c |}
    \hline
    \rowcolor{bluepoli!40}
    \textbf{Model} & \textbf{Runtime}/\unit{\s} & \textbf{Speed-up} & \textbf{Failure Rate}/\unit{\percent} \T\B \\
    \hline \hline 
    \ac{HEOS} mixture & 252.97 & 1 & 5.69 \T\B\\
    \ac{WP} \ac{EOS}  & 0.34 & 744 & 0 \T\B\\
    \ac{SW} \ac{EOS} & 0.27 & 937 & 0 \T\B\\
    Coupled Model & 14.28 & 17 & 0 \T\B\\
    \hline
\end{tabular}    
        \\[10pt]
    \end{table}



\subsection{Conclusions}
\label{sec:tppm_semi_conclusions}
    The approach of tightly coupling the \ac{WP} \ac{EOS} and \ac{SW} \ac{EOS} with the \ac{SP2009} model has yielded a useful model for evaluating the properties of water and carbon dioxide for a wide range of temperatures, pressures and compositions. The calculated properties are similar to the values obtained from a \ac{HEOS} mixture of water and carbon dioxide in \emph{CoolProp}, while being computationally cheaper, speed-ups of \num{17} times have been observed, and stable, converging for all conditions tested. The performance could be further improved by implementing the model in C++ to streamline the evaluations and reduce computational overheads. Moreover, extending the \ac{SP2009} to sub-atmospheric conditions (i.e. below pressures of \qty{1}{bar} and temperatures below \qty{298}{\K}) would allow this model to be used geothermal power plant simulations, particularly direct steam cycles. 