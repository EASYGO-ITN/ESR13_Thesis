\todo{perhaps this could be framed as future work with potential}

In specific cases it is also possible to tightly couple different models to extend their respective capabilities or to improve speed of calculations. For example, mixtures of water and carbon dioxide can be modelled using their respective high-fidelity pure fluid formulation; however, convergence is not guaranteed for all conditions, particularly for quasi-pure water, see Section~\ref{sec:excess_properties}. On the other hand, the \acf{SP2009} model can be used to partition mixtures of water and carbon dioxide, providing stable convergence for a broad range of compositions, temperatures and pressure.

The following section aims to investigate the coupling of the \acf{SP2009} model with the \acf{WP} \ac{EOS} for water and the \acf{SW} \ac{EOS} for pure carbon dioxide.

\subsection{Formulation}
\label{sec_semiempirical_formulation}
    Assuming a system comprised of two phases, one water-rich phase, hereinafter \emph{wp}, and one carbon dioxide-rich, hereinafter \emph{cp}, where the only reactions taking place are the migrations of water and carbon dioxide molecules from one phase to another, see Reaction~\ref{eq:reaction_1} and ~\ref{eq:reaction_2}.

    \begin{align}
        H_2O^{(wp)} \rightleftharpoons H_2O^{(cp)} \label{eq:reaction_1}
    \end{align}
    \begin{align}
        CO_2^{(cp)} \rightleftharpoons CO_2^{(wp)} \label{eq:reaction_2}
    \end{align}

    At equilibrium the chemical potential of a species \(i\) is equal across all phases, see Equation~\ref{eq:chempot_at_equib}, where the chemical potential of species \(i\) at a given temperature, pressure and composition can be calculated from its chemical potential at some reference conditions and the activity of species \(i\), as previously shown in Equations~\ref{eq:chemical_potential} and ~\ref{eq:activity_def}. By convention, the reference conditions depend on the phase that the species occupies. 

    \begin{align}
        \mu_i^{(x)}= \mu_i^{(y)}= \mu_i^{(z)}= ... \label{eq:chempot_at_equib}
    \end{align}
    \begin{align}
        \ln a_i^{j}(P, T, \mathbf{z}_j)= \frac{\mu_i^{j} (P, T, \mathbf{z}_j) - \mu_i^{j} (P^o, T^o, \mathbf{z}_j^o)}{RT} \tag{\ref{eq:activity_def}}
    \end{align}
    \begin{align}
        \mu_i^{j} (P, T, \mathbf{z}_j) = \mu_i^{j} (P^o, T^o, \mathbf{z}_j^o) +RT*\ln a_i (P, T, \mathbf{z}_j) \tag{\ref{eq:chemical_potential}}
    \end{align}

    It is then possible to calculate the thermodynamic properties of species \(i\) for the partial derivaties of chemical potential. The property \(\Psi\), a placeholder for the molar enthalpy, molar entropy or molar volume, can be calculated via Equation~\ref{eq:props_from_mu}. The choice of \(x\) and \(f(x)\) depends on the property to be calculated: for the partial molar enthalpy \(x=1/T\) and \(f(x)=x\), for the partial molar entropy \(x=T\) and \(f(x)=-1\), and for the partial molar volume \(x=P\) and \(f(x)=1\), with the unused properties being constant.

    \begin{align}
        \Psi_i^{j} (P, T, \mathbf{z}_j) = \frac{\partial \left( \mu_i^{j} (P, T, \mathbf{z}_j)*f(x) \right)}{\partial x} \label{eq:props_from_mu}
    \end{align}

    Substituting Equation~\ref{eq:chemical_potential} into Equation~\ref{eq:props_from_mu}, yields Equation~\ref{eq:props_from_mu_expandend}, which simplifies to Equation~\ref{eq:props_from_mu_expandend_simp} by recognising that \(\Psi_i^{j} (P^o, T^o, \mathbf{z}_j^o)=\frac{\partial \left( \mu_i (P^o, T^o, \mathbf{z}_j^o)*f(x) \right)}{\partial x}\).

    \begin{align}
        \Psi_i^{j} (P, T, \mathbf{z}_j) = \frac{\partial \left( \mu_i^{j} (P^o, T^o, \mathbf{z}_j^o)*f(x) \right)}{\partial x} + \frac{\partial \left(RT * f(x)* \ln a_i^{j} (P, T, \mathbf{z}_j) \right)}{\partial x} \label{eq:props_from_mu_expandend}
    \end{align}
    \begin{align}
        \Psi_i^{j} (P, T, \mathbf{z}_j) = \Psi_i^{j} (P^o, T^o, \mathbf{z}_j^o) + \frac{\partial \left(RT * f(x)* \ln a_i^{j} (P, T, \mathbf{y}_j) \right)}{\partial x} \label{eq:props_from_mu_expandend_simp}
    \end{align}

    Equation~\ref{eq:props_from_mu_expandend_simp} provides the basis for coupling the \ac{WP} and \ac{SW} \ac{EOS} with the \ac{SP2009} model. The \ac{WP} and \ac{SW} \ac{EOS} can be used to evaluate the properties of water and carbon dioxide at their respective reference conditions, i.e. \(\Psi_i^{j} (P^o, T^o, \mathbf{z}_j^o)\). Meanwhile, the \ac{SP2009} model is used to 1) partition the fluid into water-rich and carbon dioxide-rich phase and 2) to provide the species activity (i.e. \(\ln a_i^{j} (P, T, \mathbf{z}_j)\), which accounts for the deviation from the reference conditions and composition.

\subsection{Change of Reference Conditions}
\label{sec:change_ref_cond}

    However, Equation~\ref{eq:props_from_mu_expandend_simp} gives undue importance to the the activity model as it not only corrects for the presence of other species but also the temperature and pressure being different to the reference conditions. For example, considering the boundary case where the fluid is pure (either water or carbon dioxide) then it would be best to simply use the corresponding \ac{HEOS} to calculate the fluid's properties. However, with the current formulation it would be the activity model's responsibility to correct the properties at the reference conditions to the temperature and pressure of interest, which can introduce inconsistencies in the pure component properties, as the activity model does not have the same level of accuracy as the \ac{HEOS}. This has previously been illustrated in Section~\ref{sec:chemically_active_system_limitations}.

    An alternative approach is to change the reference conditions to be the current temperature and pressure and composition corresponding to a pure substance, i.e. \(P, T, \mathbf{z}_j^o\), by adding and subtracting \(\mu_i (P, T, \mathbf{z}_j^o)\) to Equation~\ref{eq:chemical_potential}, yielding Equation~\ref{eq:chem_pot_change_ref}. The definition of the species activity allows this expression to be simplified to Equation~\ref{eq:chem_pot_change_ref_simp}, where \(i\) is the species index, and \(j\) is the phase index

    \begin{align}
        \mu_i^{j} (P, T, \mathbf{z}_j) = \mu_i^{j} (P^o, T^o, \mathbf{z}_j^o) +RT*\ln a_i^{j} (P, T, \mathbf{z}_j) + \mu_i^{j} (P, T, \mathbf{z}_j^o) - \mu_i^{j} (P, T, \mathbf{z}_j^o) \label{eq:chem_pot_change_ref}
    \end{align}
    \begin{align}
        \mu_i^{j} (P, T, \mathbf{z}_j) = \mu_i^{j} (P, T, \mathbf{z}_j^o) + RT*\ln a_i^{j} (P, T, \mathbf{z}_j) -RT*\ln a_i^{j} (P, T, \mathbf{z}_j^o)
    \end{align}
    \begin{align}
        \mu_i^{j} (P, T, \mathbf{z}_j) = \mu_i^{j} (P, T, \mathbf{z}_j^o) + RT*\ln \frac{a_i^{j} (P, T, \mathbf{z}_j)}{a_i^{j} (P, T, \mathbf{z}_j^0)} \label{eq:chem_pot_change_ref_simp}
    \end{align}

    From Equation~\ref{eq:chem_pot_change_ref_simp}, the thermodynamic properties can then be obtained as outlined above, yielding Equation~\ref{eq:props_chempot_changed_ref}, where \(A_i^{j} (P, T, \mathbf{z}_j) = \frac{a_i^{j} (P, T, \mathbf{z}_j)}{a_i^{j} (P, T, \mathbf{z}_j^o)}\)

    \begin{align}
        \Psi_i^{j} (P, T, \mathbf{z}_j) = \Psi_i^{j} (P, T, \mathbf{z}_{j}^o) + \frac{\partial \left(RT * f(x)* \ln A_i^{j} (P, T, \mathbf{z}_j) \right)}{\partial x} \label{eq:props_chempot_changed_ref}
    \end{align}

    This formulation allows the properties to be evaluated at the temperature and pressure of interest directly from the corresponding \ac{HEOS} and the activity model only corrects for the difference in composition. The additional benefit being that if the fluid is pure, the properties are entirely consistent with those predicted by the \ac{HEOS} as \(A_i^{j} (P, T, \mathbf{z}_j) = 1\) and hence the activity contribution is zero.

    The properties of species \(i\) in phase \(j\) can then be aggregated by phase or by all phases to obtain the phase or total properties, Equations~\ref{eq:props_phase} and \ref{eq:props_total}.

    \begin{align}
        \Psi_i^j (P, T, \mathbf{z}_j) = \Psi_i^j (P, T, \mathbf{z}_j^o) + \frac{\partial \left(RT * f(x)* \ln A_i^j (P, T, \mathbf{z}_j) \right)}{\partial x} \label{eq:props_comp_phase}
    \end{align}
    \begin{align}
        \Psi^j (P, T, \mathbf{z}_j) = \sum_{i=0}^N n_i^j\Psi_i^j (P, T, \mathbf{z}_j) \label{eq:props_phase}
    \end{align}
    \begin{align}
        \Psi (P, T, \mathbf{z}) = \sum_{i=0}^N n^j\Psi^j (P, T, \mathbf{z}_j) \label{eq:props_total}
    \end{align}

\subsection{Extrapolation}
\label{sec:extrapolation}
    \todo{After speaking with Allan, this section needs some more work... essentially the extrapolation should be treated as tuning to reduce the differences to the CoolProp model}
    One drawback of the above formulation is that the models used to obtain \(\Psi_i^j (P, T, \mathbf{z}_j)\) are not necessarily continuous over the fully temperature and pressure domain of interest. For example, considering pure water, if the temperature is below the saturation temperature at a given pressure (but above the melting point), water can only exist in its liquid state - a vapour or solid state is not feasible. However, the presence of impurities changes the saturation point. Specifically, for a mixture of water and carbon dioxide, some water may exist in its vapour state despite the temperature being below the saturation temperature of pure water. This poses an obstacle, because the \ac{WP} \ac{EOS} is designed to model physical states of pure water, but such equilibrium states are non-physical for pure water. 

    \todo{insert the chemical potential diagram}

    As an alternative, the desired properties of water vapour could be obtained by extrapolating from the saturation point. The following methods were considered, also see Table~\ref{table:SemiEmpirical_ExtrapolationFuncs}:

    \begin{itemize}
        \item \emph{Gibbs Energy}: the Gibbs energy is linearly extrapolated from the saturation point, with the remaining properties being determined from the partial derivatives. This has the unfortunate consequence that the molar enthalpy and entropy are constant with temperature.
        \item \emph{Enthalpy \& Entropy}: both the molar enthalpy and molar entropy are linearly extrapolated from the saturation point, with the Gibbs energy being back-calculated from the resulting values of the molar enthalpy and entropy; the volume is not computed.
        \item \emph{Density}: the ideal gas law is used to compute the density of the water vapour at the current temperature. In turn, the temperature and density are then used to compute the properties from the \ac{WP} \ac{EOS}.
        \item \emph{Power Law}: it is assumed that \(v \propto T^a\), with the value of \(a\)  being back-calculated from the saturation point.
    \end{itemize}

    \begin{table}[H]
        \caption{The extrapolation schemes investigated for obtaining water vapour properties for sub-saturation conditions}
        \centering 
        \label{table:SemiEmpirical_ExtrapolationFuncs}
        \begin{tabular}{|p{7em} | c | c | c | c|}
    \hline
    \rowcolor{bluepoli!40}
    \textbf{Extrapolation} & \textbf{Gibbs Energy} & \textbf{Enthalpy} & \textbf{Entropy} & \textbf{Volume} \T\B \\
    \hline \hline 
    Gibbs Energy & \(G_{sat} - \Delta T*\frac{dG}{dT}\) & \(\frac{d(G/T)}{d(1/T)}\) & \(\frac{dG}{dT}\) & \(\frac{dG}{dP}\) \T\B\\
    \hline
    Enthalpy \& Entropy  & H - TS & \(H_{sat} - \Delta T*\frac{dH}{dT}\) & \(S_{sat} - \Delta T*\frac{dS}{dT}\) & - \T\B\\
    \hline
    Density & \multicolumn{4}{c|}{\(\rho=\rho _{sat}*\frac{T_{sat}}{T}\)} \T\B\\
     & \(G(T,\rho)\) & \(H(T,\rho)\) & \(S(T,\rho)\) & \(1/\rho\)\T\B\\
    \hline
    Ideal Gas & - & - & - & \(v_{sat}*\frac{T}{T_{sat}}\)\T\B\\
    \hline
    Power Law & - & - & - & \(a=\frac{T_{sat}}{v_{sat}}*\frac{dv}{dT}\)\T\B\\
     & - & - & - & \(v_{sat} * \left(\frac{T}{T_{sat}} \right)^a\)\T\B\\
    \hline
\end{tabular}        
        \\[10pt]
    \end{table}

    While the selection of extrapolation method is arbitrary, provided it allows existing property data to be reproduced, care was taken to ensure that the selected methods display smoothly around the saturation point, and that when extrapolating backwards (i.e. to conditions where \(T>T_{sat}\) and water vapour can exist) the extrapolated properties are close to those predicted by the \ac{WP} \ac{EOS}, see Figures~\ref{fig:SemiEmpirical_extrapolation1} and \ref{fig:SemiEmpirical_extrapolation2}. Based on these criteria the \emph{Enthalpy \& Entropy} method was selected for the molar enthalpy and molar entropy, and the \emph{Power Law} method was chosen for the molar volume.

    \begin{notes}{Note:}
        While it was attempted to also extrapolate the derivatives of the species activity from the saturation point, it was ultimately decided to simply use the derivatives evaluated at the saturation point to ensure model stability.
    \end{notes}

    \begin{figure}[H]
        \centering
        % This file was created with tikzplotlib v0.10.1.
\begin{tikzpicture}

\definecolor{darkgray176}{RGB}{176,176,176}
\definecolor{darkorange25512714}{RGB}{255,127,14}
\definecolor{forestgreen4416044}{RGB}{44,160,44}
\definecolor{lightgray204}{RGB}{204,204,204}
\definecolor{steelblue31119180}{RGB}{31,119,180}

\begin{groupplot}[group style={group size=2 by 2, horizontal sep=2cm, vertical sep=2cm}]
\nextgroupplot[
% tick align=outside,
% tick pos=left,
% x grid style={darkgray176},
xlabel={Temperature/\unit{\K}},
xmin=298, xmax=700,
% xtick style={color=black},
% y grid style={darkgray176},
ylabel={Gibbs Energy/\unit{\joule\per\mole}},
ymin=-20000, ymax=10000,
% ytick style={color=black},
ylabel near ticks,
xlabel near ticks,
width = 7.5cm, height=7cm
]
\addplot [semithick, steelblue31119180]
table {%
373.755928897123 -1359.58578761505
378.607980593138 -2004.48428941232
383.460032289153 -2651.68555111103
388.312083985168 -3301.14418692346
393.164135681182 -3952.81900623516
398.016187377197 -4606.67202649461
402.868239073212 -5262.66787001705
407.720290769227 -5920.77336061466
412.572342465241 -6580.95723124622
417.424394161256 -7243.18989832351
422.276445857271 -7907.44327946554
427.128497553286 -8573.69064181843
431.980549249301 -9241.90647326822
436.832600945315 -9912.0663716169
441.68465264133 -10584.1469483189
446.536704337345 -11258.1257442814
451.38875603336 -11933.9811558083
456.240807729375 -12611.6923691614
461.092859425389 -13291.2393024968
465.944911121404 -13972.6025541493
};
\addplot [semithick, steelblue31119180, mark=*, mark size=3, mark options={solid}, only marks]
table {%
372.755928897123 -1226.96400816125
};
\addplot [semithick, steelblue31119180]
table {%
279.566946672842 -3.85746806290671
284.418998368857 -15.5563394783069
289.271050064872 -33.508747661138
294.123101760887 -57.6010369088503
298.975153456902 -87.7262232800561
303.827205152916 -123.782921370445
308.679256848931 -165.674551772509
313.531308544946 -213.308745201964
318.383360240961 -266.59688988904
323.235411936975 -325.453784543228
328.08746363299 -389.797370128043
332.939515329005 -459.548520589643
337.79156702502 -534.63087755822
342.643618721035 -614.97071793906
347.495670417049 -700.496845438535
352.347722113064 -791.140500424307
357.199773809079 -886.835282649609
362.051825505094 -987.517083847837
366.903877201108 -1093.12402755582
371.755928897123 -1203.59641444446
};
\addplot [semithick, steelblue31119180, dashed]
table {%
279.566946672842 11127.2518177744
284.418998368857 10484.0073867225
289.271050064872 9840.76295567063
294.123101760887 9197.51852461872
298.975153456902 8554.27409356681
303.827205152916 7911.0296625149
308.679256848931 7267.785231463
313.531308544946 6624.54080041109
318.383360240961 5981.29636935918
323.235411936975 5338.05193830727
328.08746363299 4694.80750725536
332.939515329005 4051.56307620346
337.79156702502 3408.31864515155
342.643618721035 2765.07421409963
347.495670417049 2121.82978304773
352.347722113064 1478.58535199582
357.199773809079 835.340920943913
362.051825505094 192.096489891999
366.903877201108 -451.147941159908
371.755928897123 -1094.39237221181
373.755928897123 -1359.53564411068
378.607980593138 -2002.78007516259
383.460032289153 -2646.0245062145
388.312083985168 -3289.26893726641
393.164135681182 -3932.51336831832
398.016187377197 -4575.75779937023
402.868239073212 -5219.00223042213
407.720290769227 -5862.24666147404
412.572342465241 -6505.49109252595
417.424394161256 -7148.73552357786
422.276445857271 -7791.97995462977
427.128497553286 -8435.22438568167
431.980549249301 -9078.46881673358
436.832600945315 -9721.71324778549
441.68465264133 -10364.9576788374
446.536704337345 -11008.2021098893
451.38875603336 -11651.4465409412
456.240807729375 -12294.6909719931
461.092859425389 -12937.935403045
465.944911121404 -13581.1798340969
};
\addplot [semithick, steelblue31119180, dotted]
table {%
279.566946672842 10254.9138672852
284.418998368857 9700.14425425362
289.271050064872 9140.64490997519
294.123101760887 8576.4158344499
298.975153456902 8007.45702767777
303.827205152916 7433.76848965877
308.679256848931 6855.35022039291
313.531308544946 6272.20221988019
318.383360240961 5684.3244881206
323.235411936975 5091.71702511417
328.08746363299 4494.37983086087
332.939515329005 3892.31290536071
337.79156702502 3285.51624861369
342.643618721035 2673.9898606198
347.495670417049 2057.73374137907
352.347722113064 1436.74789089147
357.199773809079 811.032309157024
362.051825505094 180.586996175698
366.903877201108 -454.588048052479
371.755928897123 -1094.49282352751
373.755928897123 -1359.63609542638
378.607980593138 -2006.22018205516
383.460032289153 -2657.5339999308
388.312083985168 -3313.57754905331
393.164135681182 -3974.35082942266
398.016187377197 -4639.85384103888
402.868239073212 -5310.08658390195
407.720290769227 -5985.04905801189
412.572342465241 -6664.74126336871
417.424394161256 -7349.16319997236
422.276445857271 -8038.31486782287
427.128497553286 -8732.19626692025
431.980549249301 -9430.80739726448
436.832600945315 -10134.1482588556
441.68465264133 -10842.2188516935
446.536704337345 -11555.0191757783
451.38875603336 -12272.54923111
456.240807729375 -12994.8090176886
461.092859425389 -13721.798535514
465.944911121404 -14453.5177845862
};
\addplot [semithick, steelblue31119180, dash pattern=on 1pt off 3pt on 3pt off 3pt]
table {%
279.566946672842 -5.64182339829699
284.418998368857 -17.3345756490633
289.271050064872 -35.2791960390268
294.123101760887 -59.3613704599179
298.975153456902 -89.4733454353322
303.827205152916 -125.51283404293
308.679256848931 -167.382202286098
313.531308544946 -214.987852295142
318.383360240961 -268.239748192887
323.235411936975 -327.051047262155
328.08746363299 -391.337809672488
332.939515329005 -461.018766892474
337.79156702502 -536.015134004001
342.643618721035 -616.250454714835
347.495670417049 -701.650470677159
352.347722113064 -792.143008825973
357.199773809079 -887.657882073381
362.051825505094 -988.12679995982
366.903877201108 -1093.48328675715
371.755928897123 -1203.66260530443
373.755928897123 -1358.92876266099
378.607980593138 -2000.78263488456
383.460032289153 -2645.14725822694
388.312083985168 -3291.94713910633
393.164135681182 -3941.11885503473
398.016187377197 -4592.60689032229
402.868239073212 -5246.36139953823
407.720290769227 -5902.33691504593
412.572342465241 -6560.49153439794
417.424394161256 -7220.78636422439
422.276445857271 -7883.18511038215
427.128497553286 -8547.65375808163
431.980549249301 -9214.1603119162
436.832600945315 -9882.67457875538
441.68465264133 -10553.1679831586
446.536704337345 -11225.6134085502
451.38875603336 -11899.9850594199
456.240807729375 -12576.2583410325
461.092859425389 -13254.4097539175
465.944911121404 -13934.4168009476
};
\addplot [semithick, darkorange25512714]
table {%
454.028007881676 -3831.41540215831
459.936271143277 -4534.77466869594
465.844534404878 -5241.65717949339
471.752797666479 -5951.91270149131
477.66106092808 -6665.42192234616
483.569324189681 -7382.08449881147
489.477587451282 -8101.81301659991
495.385850712883 -8824.52966232246
501.294113974484 -9550.16420469171
507.202377236085 -10278.652649624
513.110640497686 -11009.936270409
519.018903759287 -11743.9608655429
524.927167020888 -12480.6761669843
530.835430282489 -13220.0353553009
536.74369354409 -13961.9946551438
542.651956805691 -14706.5129935302
548.560220067292 -15453.5517085879
554.468483328893 -16203.0742995892
560.376746590494 -16955.0462112032
566.285009852095 -17709.4346463658
};
\addplot [semithick, darkorange25512714, mark=*, mark size=3, mark options={solid}, only marks]
table {%
453.028007881676 -3712.73087414544
};
\addplot [semithick, darkorange25512714]
table {%
339.771005911257 -550.228426964695
345.679269172858 -651.272674519286
351.587532434459 -759.939279196773
357.49579569606 -876.107280635836
363.404058957661 -999.660758837645
369.312322219262 -1130.48867385499
375.220585480863 -1268.48472275542
381.128848742464 -1413.54720947488
387.037112004065 -1565.57892501876
392.945375265666 -1724.48703674579
398.853638527267 -1890.18298631246
404.761901788868 -2062.58239644874
410.670165050469 -2241.60498712509
416.57842831207 -2427.17450193128
422.486691573671 -2619.21864570679
428.394954835272 -2817.66903459696
434.303218096873 -3022.4611599363
440.211481358474 -3233.53436749362
446.119744620075 -3450.8318539115
452.028007881676 -3674.30068245653
};
\addplot [semithick, darkorange25512714, dashed]
table {%
339.771005911257 9723.04078614547
345.679269172858 9022.13863695659
351.587532434459 8321.2364877677
357.49579569606 7620.33433857881
363.404058957661 6919.43218938992
369.312322219262 6218.53004020104
375.220585480863 5517.62789101215
381.128848742464 4816.72574182327
387.037112004065 4115.82359263438
392.945375265666 3414.92144344549
398.853638527267 2714.0192942566
404.761901788868 2013.11714506771
410.670165050469 1312.21499587883
416.57842831207 611.312846689938
422.486691573671 -89.5893024989427
428.394954835272 -790.49145168783
434.303218096873 -1491.39360087672
440.211481358474 -2192.29575006561
446.119744620075 -2893.19789925449
452.028007881676 -3594.10004844338
454.028007881676 -3831.36169984751
459.936271143277 -4532.26384903639
465.844534404878 -5233.16599822528
471.752797666479 -5934.06814741417
477.66106092808 -6634.97029660306
483.569324189681 -7335.87244579194
489.477587451282 -8036.77459498083
495.385850712883 -8737.67674416971
501.294113974484 -9438.5788933586
507.202377236085 -10139.4810425475
513.110640497686 -10840.3831917364
519.018903759287 -11541.2853409253
524.927167020888 -12242.1874901142
530.835430282489 -12943.089639303
536.74369354409 -13643.9917884919
542.651956805691 -14344.8939376808
548.560220067292 -15045.7960868697
554.468483328893 -15746.6982360586
560.376746590494 -16447.6003852475
566.285009852095 -17148.5025344364
};
\addplot [semithick, darkorange25512714, dotted]
table {%
339.771005911257 8339.9985124021
345.679269172858 7779.63054856023
351.587532434459 7211.73501121345
357.49579569606 6636.31190036175
363.404058957661 6053.36121600513
369.312322219262 5462.8829581436
375.220585480863 4864.87712677714
381.128848742464 4259.34372190577
387.037112004065 3646.28274352949
392.945375265666 3025.69419164828
398.853638527267 2397.57806626215
404.761901788868 1761.93436737111
410.670165050469 1118.76309497516
416.57842831207 468.06424907429
422.486691573671 -190.162170331489
428.394954835272 -855.916163242204
434.303218096873 -1529.19772965783
440.211481358474 -2210.00686957836
446.119744620075 -2898.34358300383
452.028007881676 -3594.2078699342
454.028007881676 -3831.46952133833
459.936271143277 -4537.40953278573
465.844534404878 -5250.87711773805
471.752797666479 -5971.87227619527
477.66106092808 -6700.39500815743
483.569324189681 -7436.44531362449
489.477587451282 -8180.02319259648
495.385850712883 -8931.12864507338
501.294113974484 -9689.76167105519
507.202377236085 -10455.9222705419
513.110640497686 -11229.6104435336
519.018903759287 -12010.8261900302
524.927167020888 -12799.5695100316
530.835430282489 -13595.840403538
536.74369354409 -14399.6388705494
542.651956805691 -15210.9649110656
548.560220067292 -16029.8185250868
554.468483328893 -16856.1997126128
560.376746590494 -17690.1084736438
566.285009852095 -18531.5448081797
};
\addplot [semithick, darkorange25512714, dash pattern=on 1pt off 3pt on 3pt off 3pt]
table {%
339.771005911257 -568.119446024164
345.679269172858 -669.080148959988
351.587532434459 -777.631634868937
357.49579569606 -893.646233624317
363.404058957661 -1017.00047587981
369.312322219262 -1147.57487253487
375.220585480863 -1285.25371276273
381.128848742464 -1429.92487595696
387.037112004065 -1581.47965471216
392.945375265666 -1739.81258711602
398.853638527267 -1904.82129739656
404.761901788868 -2076.40634441668
410.670165050469 -2254.47107778195
416.57842831207 -2438.92150145268
422.486691573671 -2629.66614479197
428.394954835272 -2826.61594097677
434.303218096873 -3029.68411264953
440.211481358474 -3238.78606464093
446.119744620075 -3453.83928353273
452.028007881676 -3674.76324377168
454.028007881676 -3828.52277603413
459.936271143277 -4515.66223329303
465.844534404878 -5207.51895630354
471.752797666479 -5903.68031951033
477.66106092808 -6603.86846745038
483.569324189681 -7307.8771390531
489.477587451282 -8015.54133902211
495.385850712883 -8726.72219907046
501.294113974484 -9441.29899887949
507.202377236085 -10159.1646497509
513.110640497686 -10880.2229139542
519.018903759287 -11604.386538091
524.927167020888 -12331.575899852
530.835430282489 -13061.7179661692
536.74369354409 -13794.7454561508
542.651956805691 -14530.5961490784
548.560220067292 -15269.2123015459
554.468483328893 -16010.5401504242
560.376746590494 -16754.5294854088
566.285009852095 -17501.1332791357
};
\addplot [semithick, forestgreen4416044]
table {%
585.147146966519 -10100.6379397474
592.780662058184 -10880.3976538
600.414177149848 -11670.4844140183
608.047692241513 -12469.5307467608
615.681207333178 -13276.5917749761
623.314722424843 -14090.9617550156
630.948237516507 -14912.0878591949
638.581752608172 -15739.5229539064
646.215267699837 -16572.8968177816
653.848782791501 -17411.8973102969
661.482297883166 -18256.2573809941
669.115812974831 -19105.7457516577
676.749328066496 -19960.1600282683
684.38284315816 -20819.3214848151
692.016358249825 -21683.0710340908
699.64987334149 -22551.2660630663
707.283388433154 -23423.7779116062
714.916903524819 -24300.4898386703
722.550418616484 -25181.295363736
730.183933708149 -26066.0969010072
};
\addplot [semithick, forestgreen4416044, mark=*, mark size=3, mark options={solid}, only marks]
table {%
584.147146966519 -9999.35614644462
};
\addplot [semithick, forestgreen4416044]
table {%
438.110360224889 -2978.6871496364
445.743875316554 -3256.21371712073
453.377390408219 -3543.96444503334
461.010905499883 -3841.82806549846
468.644420591548 -4149.70338617183
476.277935683213 -4467.49946908798
483.911450774877 -4795.13592429908
491.544965866542 -5132.54334187674
499.178480958207 -5479.66389395323
506.811996049872 -5836.45215013548
514.445511141536 -6202.87616658684
522.079026233201 -6578.91893424702
529.712541324866 -6964.58030991992
537.346056416531 -7359.87961371234
544.979571508195 -7764.85917271512
552.61308659986 -8179.58925247088
560.246601691525 -8604.17510159869
567.880116783189 -9038.76736216001
575.513631874854 -9483.57814927125
583.147146966519 -9938.90739952843
};
\addplot [semithick, forestgreen4416044, dashed]
table {%
438.110360224889 4775.60210740792
445.743875316554 4003.29761258881
453.377390408219 3230.99311776971
461.010905499883 2458.68862295061
468.644420591548 1686.3841281315
476.277935683213 914.079633312396
483.911450774877 141.775138493293
491.544965866542 -630.529356325811
499.178480958207 -1402.83385114491
506.811996049872 -2175.13834596402
514.445511141536 -2947.44284078312
522.079026233201 -3719.74733560223
529.712541324866 -4492.05183042133
537.346056416531 -5264.35632524044
544.979571508195 -6036.66082005954
552.61308659986 -6808.96531487865
560.246601691525 -7581.26980969775
567.880116783189 -8353.57430451686
575.513631874854 -9125.87879933595
583.147146966519 -9898.18329415507
585.147146966519 -10100.5289987342
592.780662058184 -10872.8334935533
600.414177149848 -11645.1379883724
608.047692241513 -12417.4424831915
615.681207333178 -13189.7469780106
623.314722424843 -13962.0514728297
630.948237516507 -14734.3559676488
638.581752608172 -15506.6604624679
646.215267699837 -16278.964957287
653.848782791501 -17051.2694521061
661.482297883166 -17823.5739469252
669.115812974831 -18595.8784417443
676.749328066496 -19368.1829365634
684.38284315816 -20140.4874313825
692.016358249825 -20912.7919262016
699.64987334149 -21685.0964210207
707.283388433154 -22457.4009158398
714.916903524819 -23229.705410659
722.550418616484 -24002.0099054781
730.183933708149 -24774.3144002972
};
\addplot [semithick, forestgreen4416044, dotted]
table {%
438.110360224889 78.9192662100686
445.743875316554 -215.215559795288
453.377390408219 -535.015655340423
461.010905499883 -880.481020425344
468.644420591548 -1251.61165505006
476.277935683213 -1648.40755921455
483.911450774877 -2070.86873291883
491.544965866542 -2518.99517616288
499.178480958207 -2992.78688894672
506.811996049872 -3492.24387127035
514.445511141536 -4017.36612313376
522.079026233201 -4568.15364453696
529.712541324866 -5144.60643547993
537.346056416531 -5746.7244959627
544.979571508195 -6374.50782598524
552.61308659986 -7027.95642554757
560.246601691525 -7707.07029464968
567.880116783189 -8411.84943329159
575.513631874854 -9142.29384147326
583.147146966519 -9898.40351919473
585.147146966519 -10100.7492237738
592.780662058184 -10889.2485356906
600.414177149848 -11703.4131171471
608.047692241513 -12543.2429681434
615.681207333178 -13408.7380886795
623.314722424843 -14299.8984787554
630.948237516507 -15216.7241383711
638.581752608172 -16159.2150675265
646.215267699837 -17127.3712662217
653.848782791501 -18121.1927344568
661.482297883166 -19140.6794722316
669.115812974831 -20185.8314795461
676.749328066496 -21256.6487564005
684.38284315816 -22353.1313027946
692.016358249825 -23475.2791187286
699.64987334149 -24623.0922042023
707.283388433154 -25796.5705592158
714.916903524819 -26995.7141837691
722.550418616484 -28220.5230778622
730.183933708149 -29470.997241495
};
\addplot [semithick, forestgreen4416044, dash pattern=on 1pt off 3pt on 3pt off 3pt]
table {%
438.110360224889 -3163.73824751846
445.743875316554 -3439.98242050964
453.377390408219 -3726.03514357091
461.010905499883 -4021.72715858398
468.644420591548 -4326.89422607691
476.277935683213 -4641.37685692495
483.911450774877 -4965.02006628169
491.544965866542 -5297.67314708741
499.178480958207 -5639.18946051662
506.811996049872 -5989.4262408382
514.445511141536 -6348.24441234663
522.079026233201 -6715.50841623888
529.712541324866 -7091.08604556018
537.346056416531 -7474.84828637474
544.979571508195 -7866.66916315029
552.61308659986 -8266.42558539872
560.246601691525 -8673.99719003116
567.880116783189 -9089.26616777405
575.513631874854 -9512.11704910331
583.147146966519 -9942.43640088088
585.147146966519 -10088.8939430601
592.780662058184 -10780.0921922954
600.414177149848 -11483.6682552545
608.047692241513 -12198.3419208358
615.681207333178 -12923.1985689488
623.314722424843 -13657.5248777787
630.948237516507 -14400.731167483
638.581752608172 -15152.3122271412
646.215267699837 -15911.8257764692
653.848782791501 -16678.8793891372
661.482297883166 -17453.1217451652
669.115812974831 -18234.236285708
676.749328066496 -19021.9363274785
684.38284315816 -19815.9611449924
692.016358249825 -20616.0727434343
699.64987334149 -21422.0531524456
707.283388433154 -22233.7021285384
714.916903524819 -23050.835186836
722.550418616484 -23873.2819032619
730.183933708149 -24700.8844418427
};

\nextgroupplot[
% tick align=outside,
% tick pos=left,
% x grid style={darkgray176},
xlabel={Temperature/\unit{\K}},
xmin=298, xmax=700,
% xtick style={color=black},
% y grid style={darkgray176},
ylabel={Entropy,\unit{\joule\per\mole\K}},
ymin=0, ymax=150,
% ytick style={color=black},
ylabel near ticks,
xlabel near ticks,
width = 7.5cm, height=7cm
]
\addplot [semithick, steelblue31119180]
table {%
373.755928897123 132.671841510161
378.607980593138 133.151480162978
383.460032289153 133.621231207086
388.312083985168 134.082104105031
393.164135681182 134.534843839552
398.016187377197 134.980040315862
402.868239073212 135.418183910192
407.720290769227 135.849695369187
412.572342465241 136.274943112748
417.424394161256 136.694254094325
422.276445857271 137.107921186436
427.128497553286 137.516208564784
431.980549249301 137.919355850744
436.832600945315 138.317581423872
441.68465264133 138.711085141364
446.536704337345 139.100050610529
451.38875603336 139.484647110839
456.240807729375 139.865031233547
461.092859425389 140.241348289352
465.944911121404 140.613733523065
};
\addplot [semithick, steelblue31119180, mark=*, mark size=3, mark options={solid}, only marks]
table {%
372.755928897123 132.571635949435
};
\addplot [semithick, steelblue31119180]
table {%
279.566946672842 1.75848710958586
284.418998368857 3.05957255441018
289.271050064872 4.33645294494203
294.123101760887 5.5906394332962
298.975153456902 6.82338497115196
303.827205152916 8.03575379396882
308.679256848931 9.22866930668582
313.531308544946 10.4029483918701
318.383360240961 11.5593265919043
323.235411936975 12.6984768342017
328.08746363299 13.8210234277359
332.939515329005 14.9275525271191
337.79156702502 16.0186199340335
342.643618721035 17.0947568891212
347.495670417049 18.1564743531808
352.347722113064 19.2042661621836
357.199773809079 20.2386113519307
362.051825505094 21.2599758799516
366.903877201108 22.2688139182244
371.755928897123 23.2655688482785
};
\addplot [semithick, steelblue31119180, dashed]
table {%
279.566946672842 132.571635949769
284.418998368857 132.571635948682
289.271050064872 132.571635953021
294.123101760887 132.571635951916
298.975153456902 132.571635950846
303.827205152916 132.571635949811
308.679256848931 132.571635948809
313.531308544946 132.571635950738
318.383360240961 132.571635949752
323.235411936975 132.571635948795
328.08746363299 132.571635947075
332.939515329005 132.571635948917
337.79156702502 132.571635948013
342.643618721035 132.57163594979
347.495670417049 132.571635951517
352.347722113064 132.571635949324
357.199773809079 132.571635948464
362.051825505094 132.571635949512
366.903877201108 132.571635949513
371.755928897123 132.571635949602
373.755928897123 132.571635949009
378.607980593138 132.571635949403
383.460032289153 132.571635949194
388.312083985168 132.571635949576
393.164135681182 132.571635948791
398.016187377197 132.571635950311
402.868239073212 132.571635948892
407.720290769227 132.571635950375
412.572342465241 132.571635949618
417.424394161256 132.571635951057
422.276445857271 132.57163595031
427.128497553286 132.571635947451
431.980549249301 132.571635950972
436.832600945315 132.57163595025
441.68465264133 132.571635948956
446.536704337345 132.571635948272
451.38875603336 132.571635951633
456.240807729375 132.571635946948
461.092859425389 132.571635950252
465.944911121404 132.571635949584
};
\addplot [semithick, steelblue31119180, dotted]
table {%
279.566946672842 123.210680076027
284.418998368857 123.698075052748
289.271050064872 124.185470029469
294.123101760887 124.672865006191
298.975153456902 125.160259982912
303.827205152916 125.647654959633
308.679256848931 126.135049936355
313.531308544946 126.622444913076
318.383360240961 127.109839889797
323.235411936975 127.597234866519
328.08746363299 128.08462984324
332.939515329005 128.572024819961
337.79156702502 129.059419796683
342.643618721035 129.546814773404
347.495670417049 130.034209750125
352.347722113064 130.521604726847
357.199773809079 131.008999703568
362.051825505094 131.496394680289
366.903877201108 131.983789657011
371.755928897123 132.471184633732
373.755928897123 132.672087265137
378.607980593138 133.159482241858
383.460032289153 133.64687721858
388.312083985168 134.134272195301
393.164135681182 134.621667172022
398.016187377197 135.109062148744
402.868239073212 135.596457125465
407.720290769227 136.083852102186
412.572342465241 136.571247078908
417.424394161256 137.058642055629
422.276445857271 137.54603703235
427.128497553286 138.033432009072
431.980549249301 138.520826985793
436.832600945315 139.008221962514
441.68465264133 139.495616939236
446.536704337345 139.983011915957
451.38875603336 140.470406892678
456.240807729375 140.9578018694
461.092859425389 141.445196846121
465.944911121404 141.932591822842
};
\addplot [semithick, steelblue31119180, dash pattern=on 1pt off 3pt on 3pt off 3pt]
table {%
279.566946672842 3.27687098053479
284.418998368857 5.12270253502825
289.271050064872 7.10593692112846
294.123101760887 9.2658503178421
298.975153456902 11.6477396510995
303.827205152916 14.3035351389085
308.679256848931 17.2923632834443
313.531308544946 20.6810592266681
318.383360240961 24.5446247760162
323.235411936975 28.9666278325183
328.08746363299 34.0395394070078
332.939515329005 39.8650053877226
337.79156702502 46.5540514550455
342.643618721035 54.2272208646582
347.495670417049 63.014646133672
352.347722113064 73.0560568929014
357.199773809079 84.5007272609128
362.051825505094 97.507367019406
366.903877201108 112.243961610556
371.755928897123 128.887566527655
373.755928897123 132.669941887624
378.607980593138 133.140984029881
383.460032289153 133.603028457057
388.312083985168 134.056942666447
393.164135681182 134.503366395375
398.016187377197 134.94280672236
402.868239073212 135.375686300222
407.720290769227 135.80236916373
412.572342465241 136.223175514911
417.424394161256 136.638390891155
422.276445857271 137.048272328885
427.128497553286 137.453052818194
431.980549249301 137.852944712725
436.832600945315 138.248142450951
441.68465264133 138.638824790921
446.536704337345 139.025156680934
451.38875603336 139.407290845831
456.240807729375 139.785369144262
461.092859425389 140.159523737661
465.944911121404 140.529878102213
};
\addplot [semithick, darkorange25512714]
table {%
454.028007881676 118.738028262178
459.936271143277 119.349697732158
465.844534404878 119.932186823332
471.752797666479 120.49233034063
477.66106092808 121.034112971161
483.569324189681 121.560127145461
489.477587451282 122.072228967566
495.385850712883 122.57184730304
501.294113974484 123.060138043836
507.202377236085 123.538066837616
513.110640497686 124.006457330114
519.018903759287 124.466021853833
524.927167020888 124.917382559085
530.835430282489 125.361086922598
536.74369354409 125.797619674937
542.651956805691 126.227412275559
548.560220067292 126.650850606473
554.468483328893 127.068281313656
560.376746590494 127.480017089584
566.285009852095 127.886341108603
};
\addplot [semithick, darkorange25512714, mark=*, mark size=3, mark options={solid}, only marks]
table {%
453.028007881676 118.630825702062
};
\addplot [semithick, darkorange25512714]
table {%
339.771005911257 16.4500883932153
345.679269172858 17.7507349114986
351.587532434459 19.0304718024749
357.49579569606 20.290166978835
363.404058957661 21.53065940941
369.312322219262 22.7527625469372
375.220585480863 23.9572668190021
381.128848742464 25.1449415523284
387.037112004065 26.3165365876775
392.945375265666 27.4727837615108
398.853638527267 28.6143983733792
404.761901788868 29.7420807188853
410.670165050469 30.8565177423914
416.57842831207 31.9583848479865
422.486691573671 33.0483478991467
428.394954835272 34.1270654351748
434.303218096873 35.1951911349047
440.211481358474 36.2533765644685
446.119744620075 37.3022742562306
452.028007881676 38.3425411803491
};
\addplot [semithick, darkorange25512714, dashed]
table {%
339.771005911257 118.630825701334
345.679269172858 118.630825703246
351.587532434459 118.630825701435
357.49579569606 118.630825703283
363.404058957661 118.63082570153
369.312322219262 118.630825703317
375.220585480863 118.630825701619
381.128848742464 118.630825703349
387.037112004065 118.630825701703
392.945375265666 118.630825701065
398.853638527267 118.630825701781
404.761901788868 118.630825703408
410.670165050469 118.630825701855
416.57842831207 118.630825701252
422.486691573671 118.630825701925
428.394954835272 118.630825702399
434.303218096873 118.630825701991
440.211481358474 118.630825703484
446.119744620075 118.630825702054
452.028007881676 118.630825701495
454.028007881676 118.630825701727
459.936271143277 118.63082570234
465.844534404878 118.630825701793
471.752797666479 118.63082570239
477.66106092808 118.630825699953
483.569324189681 118.630825702438
489.477587451282 118.630825700059
495.385850712883 118.630825702484
501.294113974484 118.63082570016
507.202377236085 118.630825702527
513.110640497686 118.630825703802
519.018903759287 118.630825701543
524.927167020888 118.630825701364
530.835430282489 118.630825699182
536.74369354409 118.630825700437
542.651956805691 118.630825701665
548.560220067292 118.630825704809
554.468483328893 118.630825702683
560.376746590494 118.630825707095
566.285009852095 118.630825701778
};
\addplot [semithick, darkorange25512714, dotted]
table {%
339.771005911257 106.419286903741
345.679269172858 107.056324656767
351.587532434459 107.693362409793
357.49579569606 108.33040016282
363.404058957661 108.967437915846
369.312322219262 109.604475668873
375.220585480863 110.241513421899
381.128848742464 110.878551174925
387.037112004065 111.515588927952
392.945375265666 112.152626680978
398.853638527267 112.789664434004
404.761901788868 113.426702187031
410.670165050469 114.063739940057
416.57842831207 114.700777693084
422.486691573671 115.33781544611
428.394954835272 115.974853199136
434.303218096873 116.611890952163
440.211481358474 117.248928705189
446.119744620075 117.885966458215
452.028007881676 118.523004211242
454.028007881676 118.738647192883
459.936271143277 119.375684945909
465.844534404878 120.012722698936
471.752797666479 120.649760451962
477.66106092808 121.286798204989
483.569324189681 121.923835958015
489.477587451282 122.560873711041
495.385850712883 123.197911464068
501.294113974484 123.834949217094
507.202377236085 124.47198697012
513.110640497686 125.109024723147
519.018903759287 125.746062476173
524.927167020888 126.383100229199
530.835430282489 127.020137982226
536.74369354409 127.657175735252
542.651956805691 128.294213488279
548.560220067292 128.931251241305
554.468483328893 129.568288994331
560.376746590494 130.205326747358
566.285009852095 130.842364500384
};
\addplot [semithick, darkorange25512714, dash pattern=on 1pt off 3pt on 3pt off 3pt]
table {%
339.771005911257 19.5651289957303
345.679269172858 21.6853057210853
351.587532434459 23.9536944788216
357.49579569606 26.3959707879601
363.404058957661 29.0397365262767
369.312322219262 31.9144835368805
375.220585480863 35.0515424383616
381.128848742464 38.4840192008605
387.037112004065 42.2467220293171
392.945375265666 46.3760810363833
398.853638527267 50.9100631014246
404.761901788868 55.8880842102854
410.670165050469 61.3509214618998
416.57842831207 67.3406268187165
422.486691573671 73.9004445722421
428.394954835272 81.0747343956946
434.303218096873 88.9089017661416
440.211481358474 97.4493374635227
446.119744620075 106.743367801223
452.028007881676 116.839217222492
454.028007881676 118.72909665117
459.936271143277 119.293309249409
465.844534404878 119.835235341736
471.752797666479 120.359879445696
477.66106092808 120.87016490307
483.569324189681 121.367980150848
489.477587451282 121.85466712764
495.385850712883 122.331255789054
501.294113974484 122.798581260221
507.202377236085 123.257345642848
513.110640497686 123.708153008757
519.018903759287 124.151530908552
524.927167020888 124.587944744865
530.835430282489 125.017808115116
536.74369354409 125.441490697217
542.651956805691 125.85932451467
548.560220067292 126.271609053754
554.468483328893 126.678615519871
560.376746590494 127.080590420891
566.285009852095 127.477758609126
};
\addplot [semithick, forestgreen4416044]
table {%
585.147146966519 101.389590554741
592.780662058184 102.862416805307
600.414177149848 104.113306389199
608.047692241513 105.218550069562
615.681207333178 106.218218740629
623.314722424843 107.136774856313
630.948237516507 107.990610897584
638.581752608172 108.791405070582
646.215267699837 109.547839163012
653.848782791501 110.266572540148
661.482297883166 110.952837891107
669.115812974831 111.610827001383
676.749328066496 112.24395175281
684.38284315816 112.855026916369
692.016358249825 113.446401800015
699.64987334149 114.020057279325
707.283388433154 114.577678722437
714.916903524819 115.120711722623
722.550418616484 115.650405315603
730.183933708149 116.167845923187
};
\addplot [semithick, forestgreen4416044, mark=*, mark size=3, mark options={solid}, only marks]
table {%
584.147146966519 101.172852289558
};
\addplot [semithick, forestgreen4416044]
table {%
438.110360224889 35.6814514792716
445.743875316554 37.0285540816132
453.377390408219 38.3604250369148
461.010905499883 39.6783769366197
468.644420591548 40.9837391117164
476.277935683213 42.2778713979085
483.911450774877 43.5621805702178
491.544965866542 44.8381403336621
499.178480958207 46.1073161344866
506.811996049872 47.3713966174297
514.445511141536 48.632234409363
522.079026233201 49.8919002487464
529.712541324866 51.1527566421881
537.346056416531 52.4175608400251
544.979571508195 53.6896131934934
552.61308659986 54.9729783580307
560.246601691525 56.272828711208
567.880116783189 57.5960043640385
575.513631874854 58.9519848341755
583.147146966519 60.3547192349192
};
\addplot [semithick, forestgreen4416044, dashed]
table {%
438.110360224889 101.172852286797
445.743875316554 101.172852290205
453.377390408219 101.172852289486
461.010905499883 101.172852289429
468.644420591548 101.172852292627
476.277935683213 101.172852288085
483.911450774877 101.172852288053
491.544965866542 101.172852291124
499.178480958207 101.172852290457
506.811996049872 101.17285229039
514.445511141536 101.172852290325
522.079026233201 101.172852289699
529.712541324866 101.172852290808
537.346056416531 101.172852290193
544.979571508195 101.172852289595
552.61308659986 101.17285229066
560.246601691525 101.172852289498
567.880116783189 101.172852290535
575.513631874854 101.172852289965
583.147146966519 101.172852289409
585.147146966519 101.172852288705
592.780662058184 101.172852288174
600.414177149848 101.172852287657
608.047692241513 101.17285228812
615.681207333178 101.172852290571
623.314722424843 101.172852290044
630.948237516507 101.172852289529
638.581752608172 101.172852289027
646.215267699837 101.172852288537
653.848782791501 101.172852286175
661.482297883166 101.172852288479
669.115812974831 101.17285229073
676.749328066496 101.172852287555
684.38284315816 101.172852289766
692.016358249825 101.172852286671
699.64987334149 101.172852289684
707.283388433154 101.172852286657
714.916903524819 101.172852288784
722.550418616484 101.172852290865
730.183933708149 101.172852292903
};
\addplot [semithick, forestgreen4416044, dotted]
table {%
438.110360224889 69.0118951362941
445.743875316554 70.6929863001675
453.377390408219 72.3740774640409
461.010905499883 74.0551686279143
468.644420591548 75.7362597917877
476.277935683213 77.4173509556611
483.911450774877 79.0984421195346
491.544965866542 80.779533283408
499.178480958207 82.4606244472814
506.811996049872 84.1417156111548
514.445511141536 85.8228067750282
522.079026233201 87.5038979389017
529.712541324866 89.1849891027751
537.346056416531 90.8660802666485
544.979571508195 92.5471714305219
552.61308659986 94.2282625943953
560.246601691525 95.9093537582687
567.880116783189 97.5904449221422
575.513631874854 99.2715360860156
583.147146966519 100.952627249889
585.147146966519 101.393077329226
592.780662058184 103.0741684931
600.414177149848 104.755259656973
608.047692241513 106.436350820847
615.681207333178 108.11744198472
623.314722424843 109.798533148593
630.948237516507 111.479624312467
638.581752608172 113.16071547634
646.215267699837 114.841806640214
653.848782791501 116.522897804087
661.482297883166 118.203988967961
669.115812974831 119.885080131834
676.749328066496 121.566171295707
684.38284315816 123.247262459581
692.016358249825 124.928353623454
699.64987334149 126.609444787328
707.283388433154 128.290535951201
714.916903524819 129.971627115074
722.550418616484 131.652718278948
730.183933708149 133.333809442821
};
\addplot [semithick, forestgreen4416044, dash pattern=on 1pt off 3pt on 3pt off 3pt]
table {%
438.110360224889 39.7678200475064
445.743875316554 41.8197702620298
453.377390408219 43.9467529445613
461.010905499883 46.1570248129241
468.644420591548 48.458982882234
476.277935683213 50.8611718696057
483.911450774877 53.3722983628706
491.544965866542 56.0012529936905
499.178480958207 58.757142209558
506.811996049872 61.6493317198342
514.445511141536 64.6875043648395
522.079026233201 67.8817361264017
529.712541324866 71.2425953527198
537.346056416531 74.7812720898301
544.979571508195 78.5097469029663
552.61308659986 82.441012232719
560.246601691525 86.5893650600105
567.880116783189 90.9707989744996
575.513631874854 95.6035395722381
583.147146966519 100.508795588675
585.147146966519 101.293236187089
592.780662058184 102.172787311212
600.414177149848 102.99580217737
608.047692241513 103.774350126921
615.681207333178 104.515917228968
623.314722424843 105.225670043031
630.948237516507 105.907489421395
638.581752608172 106.564466068465
646.215267699837 107.199154907941
653.848782791501 107.813717490609
661.482297883166 108.410009718798
669.115812974831 108.989640966438
676.749328066496 109.554016937877
684.38284315816 110.104372428218
692.016358249825 110.64179728088
699.64987334149 111.167257450745
707.283388433154 111.681612371238
714.916903524819 112.185629434488
722.550418616484 112.679996163848
730.183933708149 113.165330511922
};

\nextgroupplot[
legend cell align={left},
legend style={at={(1.55, 0.5)},
anchor=west},
% tick align=outside,
% tick pos=left,
% x grid style={darkgray176},
xlabel={Temperature/\unit{\K}},
xmin=298, xmax=700,
% xtick style={color=black},
% y grid style={darkgray176},
ylabel={Enthalpy/\unit{\joule\per\mole}},
ymin=0, ymax=60000,
% ytick style={color=black},
ylabel near ticks,
xlabel near ticks,
width = 7.5cm, height=7cm
]
\addplot [semithick, black]
table {%
0 0
};
\addlegendentry{WP EOS}
\addplot [semithick, black, mark=*, mark size=3, mark options={solid}, only marks]
table {%
0 0
};
\addlegendentry{Saturation}
\addplot [semithick, black, dashed]
table {%
0 0
};
\addlegendentry{G Extrap.}
\addplot [semithick, black, dotted]
table {%
0 0
};
\addlegendentry{H\&S Extrap.}
\addplot [semithick, black, dash pattern=on 1pt off 3pt on 3pt off 3pt]
table {%
0 0
};
\addlegendentry{Rho Extrap.}
\addplot [semithick, steelblue31119180]
table {%
373.755928897123 48227.3015745069
378.607980593138 48407.7287280802
383.460032289153 48586.7160820743
388.312083985168 48764.5570832172
393.164135681182 48941.4565909452
398.016187377197 49117.5689920451
402.868239073212 49293.0174203743
407.720290769227 49467.903936221
412.572342465241 49642.3152680979
417.424394161256 49816.3263023248
422.276445857271 49990.0023780216
427.128497553286 50163.4009116822
431.980549249301 50336.5726192461
436.832600945315 50509.5624782387
441.68465264133 50682.4105098463
446.536704337345 50855.152428502
451.38875603336 51027.8201893053
456.240807729375 51200.4424539261
461.092859425389 51373.0449899126
465.944911121404 51545.6510147042
};
\addlegendentry{P=1 bar}
\addplot [semithick, steelblue31119180, mark=*, mark size=3, mark options={solid}, only marks, forget plot]
table {%
372.755928897123 48189.8992955815
};
\addplot [semithick, steelblue31119180, forget plot]
table {%
279.566946672842 487.757403927566
284.418998368857 854.644221883883
289.271050064872 1220.90154927915
294.123101760887 1586.73517403895
298.975153456902 1952.29634556562
303.827205152916 2317.69769514805
308.679256848931 2683.02423151981
313.531308544946 3048.34127682661
318.383360240961 3413.70035256413
323.235411936975 3779.14360593211
328.08746363299 4144.70715108996
332.939515329005 4510.42358283765
337.79156702502 4876.32385153718
342.643618721035 5242.43864370577
347.495670417049 5608.79938232999
352.347722113064 5975.43893667409
357.199773809079 6342.3921144699
362.051825505094 6709.6959936829
366.903877201108 7077.3901397107
371.755928897123 7445.51674406729
};
\addplot [semithick, steelblue31119180, dashed, forget plot]
table {%
279.566946672842 48189.8992957452
284.418998368857 48189.8992954937
289.271050064872 48189.8992991337
294.123101760887 48189.8992930107
298.975153456902 48189.8992944609
303.827205152916 48189.8992946634
308.679256848931 48189.8992994082
313.531308544946 48189.899291857
318.383360240961 48189.8992970429
323.235411936975 48189.8992979903
328.08746363299 48189.8993011857
332.939515329005 48189.8992967056
337.79156702502 48189.8992939207
342.643618721035 48189.8993003667
347.495670417049 48189.8992954729
352.347722113064 48189.8993014607
357.199773809079 48189.8992893949
362.051825505094 48189.8993010596
366.903877201108 48189.8992987485
371.755928897123 48189.8992960651
373.755928897123 48189.8992925359
378.607980593138 48189.8992930547
383.460032289153 48189.8993024603
388.312083985168 48189.899290708
393.164135681182 48189.8992988965
398.016187377197 48189.8992991161
402.868239073212 48189.8992964916
407.720290769227 48189.8992988287
412.572342465241 48189.8993014621
417.424394161256 48189.8992931408
422.276445857271 48189.8992976201
427.128497553286 48189.8992991188
431.980549249301 48189.899291465
436.832600945315 48189.8992927291
441.68465264133 48189.8992958571
446.536704337345 48189.8992980745
451.38875603336 48189.8992918338
456.240807729375 48189.8992954872
461.092859425389 48189.8992907284
465.944911121404 48189.8992928044
};
\addplot [semithick, steelblue31119180, dotted, forget plot]
table {%
279.566946672842 44700.5474936244
284.418998368857 44882.226860912
289.271050064872 45063.9062281995
294.123101760887 45245.5855954871
298.975153456902 45427.2649627746
303.827205152916 45608.9443300622
308.679256848931 45790.6236973497
313.531308544946 45972.3030646373
318.383360240961 46153.9824319248
323.235411936975 46335.6617992124
328.08746363299 46517.3411664999
332.939515329005 46699.0205337875
337.79156702502 46880.699901075
342.643618721035 47062.3792683626
347.495670417049 47244.0586356501
352.347722113064 47425.7380029377
357.199773809079 47607.4173702252
362.051825505094 47789.0967375128
366.903877201108 47970.7761048003
371.755928897123 48152.4554720879
373.755928897123 48227.3431190751
378.607980593138 48409.0224863627
383.460032289153 48590.7018536502
388.312083985168 48772.3812209378
393.164135681182 48954.0605882253
398.016187377197 49135.7399555129
402.868239073212 49317.4193228004
407.720290769227 49499.098690088
412.572342465241 49680.7780573755
417.424394161256 49862.4574246631
422.276445857271 50044.1367919506
427.128497553286 50225.8161592382
431.980549249301 50407.4955265257
436.832600945315 50589.1748938133
441.68465264133 50770.8542611008
446.536704337345 50952.5336283884
451.38875603336 51134.2129956759
456.240807729375 51315.8923629635
461.092859425389 51497.571730251
465.944911121404 51679.2510975386
};
\addplot [semithick, steelblue31119180, dash pattern=on 1pt off 3pt on 3pt off 3pt, forget plot]
table {%
279.566946672842 910.462991270658
284.418998368857 1439.65934830528
289.271050064872 2020.26263883055
294.123101760887 2665.9392654759
298.975153456902 3392.91140417818
303.827205152916 4220.29027101816
308.679256848931 5170.41164520923
313.531308544946 6269.17170913765
318.383360240961 7546.3603638487
323.235411936975 9035.98883260694
328.08746363299 10776.6083376079
332.939515329005 12811.6168054841
337.79156702502 15189.5508583592
342.643618721035 17964.3607355364
347.495670417049 21195.6662336363
352.347722113064 24948.9922239502
357.199773809079 29295.9827822273
362.051825505094 34314.5934296113
366.903877201108 40089.2614205684
371.755928897123 46711.0544124737
373.755928897123 48227.2486042753
378.607980593138 48407.4564628521
383.460032289153 48586.2743478447
388.312083985168 48763.983640382
393.164135681182 48940.7809400124
398.016187377197 49116.8145552896
402.868239073212 49292.2029535599
407.720290769227 49467.0445275401
412.572342465241 49641.4230858426
417.424394161256 49815.4111726847
422.276445857271 49989.0722395388
427.128497553286 50162.4621762661
431.980549249301 50335.63046072
436.832600945315 50508.621063952
441.68465264133 50681.4731872214
446.536704337345 50854.221875737
451.38875603336 51026.8985374606
456.240807729375 51199.5313860944
461.092859425389 51372.1458219812
465.944911121404 51544.7647612896
};
\addplot [semithick, darkorange25512714]
table {%
454.028007881676 50078.9750295165
459.936271143277 50358.4802683102
465.844534404878 50628.0965513803
471.752797666479 50890.6812340543
477.66106092808 51147.8608879475
483.569324189681 51400.6640333309
489.477587451282 51649.8071132446
495.385850712883 51895.8291873439
501.294113974484 52139.1586615706
507.202377236085 52380.148529565
513.110640497686 52619.0964760949
519.018903759287 52856.2573523131
524.927167020888 53091.8515714206
530.835430282489 53326.0711619368
536.74369354409 53559.0843682365
542.651956805691 53791.0392803205
548.560220067292 54022.0667718089
554.468483328893 54252.2829196027
560.376746590494 54481.7910207585
566.285009852095 54710.6832882679
};
\addlegendentry{P=10 bar}
\addplot [semithick, darkorange25512714, mark=*, mark size=3, mark options={solid}, only marks, forget plot]
table {%
453.028007881676 50030.3557670182
};
\addplot [semithick, darkorange25512714, forget plot]
table {%
339.771005911257 5039.03465372718
345.679269172858 5484.7883969687
351.587532434459 5930.93734289893
357.49579569606 6377.54210826869
363.404058957661 6824.66826257691
369.312322219262 7272.38689925782
375.220585480863 7720.77495959179
381.128848742464 8169.91541606058
387.037112004065 8619.89739382525
392.945375265666 9070.81628801358
398.853638527267 9522.77391917854
404.761901788868 9975.87875848529
410.670165050469 10430.2462470255
416.57842831207 10885.9992294352
422.486691573671 11343.2685201794
428.394954835272 11802.1936211651
434.303218096873 12262.9236114873
440.211481358474 12725.6182341976
446.119744620075 13190.4492110261
452.028007881676 13657.6018244178
};
\addplot [semithick, darkorange25512714, dashed, forget plot]
table {%
339.771005911257 50030.3557675893
345.679269172858 50030.355765357
351.587532434459 50030.3557706994
357.49579569606 50030.3557676629
363.404058957661 50030.3557666233
369.312322219262 50030.3557672555
375.220585480863 50030.3557641905
381.128848742464 50030.3557710335
387.037112004065 50030.3557735199
392.945375265666 50030.3557659889
398.853638527267 50030.3557623379
404.761901788868 50030.3557649483
410.670165050469 50030.3557630075
416.57842831207 50030.3557643776
422.486691573671 50030.3557671115
428.394954835272 50030.3557705085
434.303218096873 50030.3557655567
440.211481358474 50030.3557677269
446.119744620075 50030.3557672199
452.028007881676 50030.3557688748
454.028007881676 50030.3557675509
459.936271143277 50030.3557688576
465.844534404878 50030.3557712994
471.752797666479 50030.3557678748
477.66106092808 50030.355769669
483.569324189681 50030.3557678134
489.477587451282 50030.3557670351
495.385850712883 50030.3557686135
501.294113974484 50030.355772343
507.202377236085 50030.3557626262
513.110640497686 50030.3557661913
519.018903759287 50030.3557618318
524.927167020888 50030.3557583146
530.835430282489 50030.3557741809
536.74369354409 50030.3557683734
542.651956805691 50030.3557741495
548.560220067292 50030.3557728739
554.468483328893 50030.3557690013
560.376746590494 50030.3557660859
566.285009852095 50030.3557615969
};
\addplot [semithick, darkorange25512714, dotted, forget plot]
table {%
339.771005911257 44498.1866720447
345.679269172858 44786.7826162437
351.587532434459 45075.3785604427
357.49579569606 45363.9745046416
363.404058957661 45652.5704488406
369.312322219262 45941.1663930395
375.220585480863 46229.7623372385
381.128848742464 46518.3582814374
387.037112004065 46806.9542256364
392.945375265666 47095.5501698354
398.853638527267 47384.1461140343
404.761901788868 47672.7420582333
410.670165050469 47961.3380024322
416.57842831207 48249.9339466312
422.486691573671 48538.5298908301
428.394954835272 48827.1258350291
434.303218096873 49115.7217792281
440.211481358474 49404.317723427
446.119744620075 49692.913667626
452.028007881676 49981.5096118249
454.028007881676 50079.2019222115
459.936271143277 50367.7978664105
465.844534404878 50656.3938106094
471.752797666479 50944.9897548084
477.66106092808 51233.5856990073
483.569324189681 51522.1816432063
489.477587451282 51810.7775874053
495.385850712883 52099.3735316042
501.294113974484 52387.9694758032
507.202377236085 52676.5654200021
513.110640497686 52965.1613642011
519.018903759287 53253.7573084
524.927167020888 53542.353252599
530.835430282489 53830.9491967979
536.74369354409 54119.5451409969
542.651956805691 54408.1410851959
548.560220067292 54696.7370293948
554.468483328893 54985.3329735938
560.376746590494 55273.9289177927
566.285009852095 55562.5248619917
};
\addplot [semithick, darkorange25512714, dash pattern=on 1pt off 3pt on 3pt off 3pt, forget plot]
table {%
339.771005911257 6079.54411363861
345.679269172858 6827.08048449478
351.587532434459 7644.18869962888
357.49579569606 8542.80234638745
363.404058957661 9536.15764883021
369.312322219262 10638.8371548989
375.220585480863 11866.8065629666
381.128848742464 13237.4450570499
387.037112004065 14769.5696311533
392.945375265666 16483.4539790765
398.853638527267 18400.8426082594
404.761901788868 20544.9609078749
410.670165050469 22940.5219649748
416.57842831207 25613.7309802379
422.486691573671 28592.288188358
428.394954835272 31905.3912387485
434.303218096873 35583.7380418445
440.211481358474 39659.5311375782
446.119744620075 44166.4846998355
452.028007881676 49139.8353597657
454.028007881676 50077.8124540874
459.936271143277 50351.657595222
465.844534404878 50617.0704567664
471.752797666479 50876.4295357967
477.66106092808 51131.1027347019
483.569324189681 51381.9550007592
489.477587451282 51629.587146294
495.385850712883 51874.4510187656
501.294113974484 52116.9069912868
507.202377236085 52357.2540721117
513.110640497686 52595.746711155
519.018903759287 52832.6049341029
524.927167020888 53068.0209800251
530.835430282489 53302.163997592
536.74369354409 53535.1835843503
542.651956805691 53767.2125810498
548.560220067292 53998.3693492325
554.468483328893 54228.7596670824
560.376746590494 54458.4783294491
566.285009852095 54687.6105107564
};
\addplot [semithick, forestgreen4416044]
table {%
585.147146966519 49227.1917054628
592.780662058184 50094.4538809546
600.414177149848 50840.6207720028
608.047692241513 51508.3658040344
615.681207333178 52119.969380034
623.314722424843 52688.9673260399
630.948237516507 53224.3977549669
638.581752608172 53732.6831647716
646.215267699837 54218.589392883
653.848782791501 54685.7669276699
661.482297883166 55137.0807838735
669.115812974831 55574.8234941658
676.749328066496 56000.8588999741
684.38284315816 56416.7227009002
692.016358249825 56823.694796102
699.64987334149 57222.8525708027
707.283388433154 57615.1109340044
714.916903524819 58001.2529176407
722.550418616484 58381.9534102194
730.183933708149 58757.7978055877
};
\addlegendentry{P=100 bar}
\addplot [semithick, forestgreen4416044, mark=*, mark size=3, mark options={solid}, only marks, forget plot]
table {%
584.147146966519 49100.4768689655
};
\addplot [semithick, forestgreen4416044, forget plot]
table {%
438.110360224889 12653.7264112942
445.743875316554 13249.0374765861
453.377390408219 13847.7849531532
461.010905499883 14450.3364148183
468.644420591548 15057.0972835137
476.277935683213 15668.5178453882
483.911450774877 16285.1020743522
491.544965866542 16907.4188179524
499.178480958207 17536.1161351196
506.811996049872 18171.9399252142
514.445511141536 18815.7585220929
522.079026233201 19468.5957645425
529.712541324866 20131.6764067859
537.346056416531 20806.4899906487
544.979571508195 21494.8832199156
552.61308659986 22199.1979975478
560.246601691525 22922.4859514248
567.880116783189 23668.8583223353
575.513631874854 24444.0927488764
583.147146966519 25256.77492828
};
\addplot [semithick, forestgreen4416044, dashed, forget plot]
table {%
438.110360224889 49100.4768694571
445.743875316554 49100.4768685629
453.377390408219 49100.4768642683
461.010905499883 49100.4768694902
468.644420591548 49100.4768714143
476.277935683213 49100.476869432
483.911450774877 49100.4768672424
491.544965866542 49100.4768663501
499.178480958207 49100.476867187
506.811996049872 49100.4768724287
514.445511141536 49100.4768668222
522.079026233201 49100.4768689115
529.712541324866 49100.4768719003
537.346056416531 49100.4768752754
544.979571508195 49100.4768690354
552.61308659986 49100.4768704223
560.246601691525 49100.4768650072
567.880116783189 49100.4768647248
575.513631874854 49100.4768676943
583.147146966519 49100.4768740616
585.147146966519 49100.4768679878
592.780662058184 49100.4768652586
600.414177149848 49100.4768654982
608.047692241513 49100.4768674224
615.681207333178 49100.4768700213
623.314722424843 49100.4768749623
630.948237516507 49100.4768646514
638.581752608172 49100.476874097
646.215267699837 49100.4768661671
653.848782791501 49100.4768719293
661.482297883166 49100.4768734056
669.115812974831 49100.4768752838
676.749328066496 49100.476873665
684.38284315816 49100.4768640306
692.016358249825 49100.4768646204
699.64987334149 49100.4768717919
707.283388433154 49100.4768652933
714.916903524819 49100.4768748914
722.550418616484 49100.4768666853
730.183933708149 49100.4768580429
};
\addplot [semithick, forestgreen4416044, dotted, forget plot]
table {%
438.110360224889 30313.7455041741
445.743875316554 31295.7501113414
453.377390408219 32277.7547185087
461.010905499883 33259.759325676
468.644420591548 34241.7639328433
476.277935683213 35223.7685400105
483.911450774877 36205.7731471778
491.544965866542 37187.7777543451
499.178480958207 38169.7823615124
506.811996049872 39151.7869686797
514.445511141536 40133.7915758469
522.079026233201 41115.7961830142
529.712541324866 42097.8007901815
537.346056416531 43079.8053973488
544.979571508195 44061.8100045161
552.61308659986 45043.8146116834
560.246601691525 46025.8192188506
567.880116783189 47007.8238260179
575.513631874854 47989.8284331852
583.147146966519 48971.8330403525
585.147146966519 49229.1206975786
592.780662058184 50211.1253047459
600.414177149848 51193.1299119132
608.047692241513 52175.1345190804
615.681207333178 53157.1391262477
623.314722424843 54139.143733415
630.948237516507 55121.1483405823
638.581752608172 56103.1529477496
646.215267699837 57085.1575549168
653.848782791501 58067.1621620841
661.482297883166 59049.1667692514
669.115812974831 60031.1713764187
676.749328066496 61013.175983586
684.38284315816 61995.1805907533
692.016358249825 62977.1851979205
699.64987334149 63959.1898050878
707.283388433154 64941.1944122551
714.916903524819 65923.1990194224
722.550418616484 66905.2036265897
730.183933708149 67887.2082337569
};
\addplot [semithick, forestgreen4416044, dash pattern=on 1pt off 3pt on 3pt off 3pt, forget plot]
table {%
438.110360224889 14258.9557188532
445.743875316554 15200.9240409355
453.377390408219 16198.429023349
461.010905499883 17257.1646456027
468.644420591548 18383.1377292234
476.277935683213 19582.67708756
483.911450774877 20862.4462656847
491.544965866542 22229.4608441798
499.178480958207 23691.1115330959
506.811996049872 25255.1946232316
514.445511141536 26929.9518350936
522.079026233201 28724.122279652
529.712541324866 30647.0101893081
537.346056416531 32708.573364907
544.979571508195 34919.5390632451
552.61308659986 37291.5566469409
560.246601691525 39837.4003274665
567.880116783189 42571.2417777248
575.513631874854 45509.0232302068
583.147146966519 48668.9809916959
585.147146966519 49182.5542188205
592.780662058184 49785.9603143749
600.414177149848 50356.4715589595
608.047692241513 50901.412187701
615.681207333178 51425.2875361164
623.314722424843 51931.1844370614
630.948237516507 52421.4126227445
638.581752608172 52897.8112806134
646.215267699837 53361.9048095619
653.848782791501 53814.9885603241
661.482297883166 54258.180597162
669.115812974831 54692.455935385
676.749328066496 55118.6710222151
684.38284315816 55537.5823015767
692.016358249825 55949.8608810954
699.64987334149 56356.1044426887
707.283388433154 56756.8470950691
714.916903524819 57152.5676284511
722.550418616484 57543.6964946299
730.183933708149 57930.6217507351
};

\end{groupplot}

\end{tikzpicture}

        \caption{The molar Gibbs energy, molar enthalpy and molar entropy of water vapour at sub-saturation conditions, as extrapolated by the various methods considered, see Table~\ref{table:SemiEmpirical_ExtrapolationFuncs}}
        \label{fig:SemiEmpirical_extrapolation1}
    \end{figure}

    \begin{figure}[H]
        \centering
        % This file was created with tikzplotlib v0.10.1.
\begin{tikzpicture}

\definecolor{darkgray176}{RGB}{176,176,176}
\definecolor{darkorange25512714}{RGB}{255,127,14}
\definecolor{forestgreen4416044}{RGB}{44,160,44}
\definecolor{lightgray204}{RGB}{204,204,204}
\definecolor{steelblue31119180}{RGB}{31,119,180}

\begin{axis}[
legend cell align={left},
legend style={
  % fill opacity=0.8,
  % draw opacity=1,
  % text opacity=1,
  at={(1.03,0.5)},
  anchor=west,
  % draw=lightgray204
},
% log basis y={10},
% tick align=outside,
% tick pos=left,
% x grid style={darkgray176},
xlabel={Temperature/\unit{\K}},
xmin=298, xmax=700,
% xtick style={color=black},
% y grid style={darkgray176},
ylabel={Volume/\unit{\cubic\m \per\mole}},
ymin=1e-05, ymax=0.1,
ymode=log,
% ytick style={color=black},
% ytick={1e-07,1e-05,0.001,0.1,10,1000},
% yticklabels={
%   \(\displaystyle {10^{-7}}\),
%   \(\displaystyle {10^{-5}}\),
%   \(\displaystyle {10^{-3}}\),
%   \(\displaystyle {10^{-1}}\),
%   \(\displaystyle {10^{1}}\),
%   \(\displaystyle {10^{3}}\)
% }
ylabel near ticks,
xlabel near ticks,
width = 7.5cm, height=7cm
]
    \addplot [semithick, black]
    table {%
    0 1
    };
    \addlegendentry{WP EOS}
    \addplot [semithick, black, mark=*, mark size=3, mark options={solid}, only marks]
    table {%
    0 1
    };
    \addlegendentry{Saturation}
    \addplot [semithick, black, dashed]
    table {%
    0 1
    };
    \addlegendentry{G Extrap.}
    \addplot [semithick, black, dotted]
    table {%
    0 1
    };
    \addlegendentry{IdealGas Extrap.}
    \addplot [semithick, black, dash pattern=on 1pt off 3pt on 3pt off 3pt]
    table {%
    0 1
    };
    \addlegendentry{Power Extrap.}
    \addplot [semithick, steelblue31119180]
    table {%
    373.755928897123 0.0306051041977436
    378.607980593138 0.0310332038999433
    383.460032289153 0.0314591088609619
    388.312083985168 0.0318831453330401
    393.164135681182 0.0323055539817563
    398.016187377197 0.0327265241658882
    402.868239073212 0.033146211586911
    407.720290769227 0.0335647479603254
    412.572342465241 0.0339822467328254
    417.424394161256 0.03439880675304
    422.276445857271 0.0348145148218167
    427.128497553286 0.0352294475853638
    431.980549249301 0.0356436730125589
    436.832600945315 0.036057251588772
    441.68465264133 0.036470237303519
    446.536704337345 0.0368826784803917
    451.38875603336 0.0372946184818288
    456.240807729375 0.0377060963120227
    461.092859425389 0.0381171471354881
    465.944911121404 0.0385278027249808
    };
    \addlegendentry{P=1 bar}
    \addplot [semithick, steelblue31119180, mark=*, mark size=3, mark options={solid}, only marks, forget plot]
    table {%
    372.755928897123 0.0305165608645238
    };
    \addplot [semithick, steelblue31119180, forget plot]
    table {%
    279.566946672842 1.80165658036436e-05
    284.418998368857 1.80228443189366e-05
    289.271050064872 1.80346424918743e-05
    294.123101760887 1.80513579732168e-05
    298.975153456902 1.80725192775695e-05
    303.827205152916 1.80977517447863e-05
    308.679256848931 1.81267541561107e-05
    313.531308544946 1.81592822172187e-05
    318.383360240961 1.81951366145866e-05
    323.235411936975 1.82341542071955e-05
    328.08746363299 1.82762014231897e-05
    332.939515329005 1.8321169242286e-05
    337.79156702502 1.83689693414718e-05
    342.643618721035 1.84195311094006e-05
    347.495670417049 1.84727993201621e-05
    352.347722113064 1.85287323152513e-05
    357.199773809079 1.85873005830908e-05
    362.051825505094 1.86484856542136e-05
    366.903877201108 1.87122792510062e-05
    371.755928897123 1.87786826461874e-05
    };
    \addplot [semithick, steelblue31119180, dashed, forget plot]
    table {%
    279.566946672842 0.0222598827740029
    284.418998368857 0.0226897815450415
    289.271050064872 0.0231196803160802
    294.123101760887 0.0235495790871188
    298.975153456902 0.0239794778581575
    303.827205152916 0.0244093766291961
    308.679256848931 0.0248392754002347
    313.531308544946 0.0252691741712734
    318.383360240961 0.025699072942312
    323.235411936975 0.0261289717133507
    328.08746363299 0.0265588704843893
    332.939515329005 0.0269887692554279
    337.79156702502 0.0274186680264666
    342.643618721035 0.0278485667975052
    347.495670417049 0.0282784655685438
    352.347722113064 0.0287083643395825
    357.199773809079 0.0291382631106211
    362.051825505094 0.0295681618816598
    366.903877201108 0.0299980606526984
    371.755928897123 0.030427959423737
    373.755928897123 0.0306051623053105
    378.607980593138 0.0310350610763492
    383.460032289153 0.0314649598473878
    388.312083985168 0.0318948586184264
    393.164135681182 0.0323247573894651
    398.016187377197 0.0327546561605037
    402.868239073212 0.0331845549315423
    407.720290769227 0.033614453702581
    412.572342465241 0.0340443524736196
    417.424394161256 0.0344742512446583
    422.276445857271 0.0349041500156969
    427.128497553286 0.0353340487867355
    431.980549249301 0.0357639475577742
    436.832600945315 0.0361938463288128
    441.68465264133 0.0366237450998515
    446.536704337345 0.0370536438708901
    451.38875603336 0.0374835426419287
    456.240807729375 0.0379134414129674
    461.092859425389 0.038343340184006
    465.944911121404 0.0387732389550447
    };
    \addplot [semithick, steelblue31119180, dotted, forget plot]
    table {%
    279.566946672842 0.0228874206483928
    284.418998368857 0.0232846455331514
    289.271050064872 0.0236818704179101
    294.123101760887 0.0240790953026687
    298.975153456902 0.0244763201874273
    303.827205152916 0.0248735450721859
    308.679256848931 0.0252707699569445
    313.531308544946 0.0256679948417031
    318.383360240961 0.0260652197264618
    323.235411936975 0.0264624446112204
    328.08746363299 0.026859669495979
    332.939515329005 0.0272568943807376
    337.79156702502 0.0276541192654962
    342.643618721035 0.0280513441502548
    347.495670417049 0.0284485690350135
    352.347722113064 0.0288457939197721
    357.199773809079 0.0292430188045307
    362.051825505094 0.0296402436892893
    366.903877201108 0.0300374685740479
    371.755928897123 0.0304346934588065
    373.755928897123 0.030598428270241
    378.607980593138 0.0309956531549996
    383.460032289153 0.0313928780397582
    388.312083985168 0.0317901029245169
    393.164135681182 0.0321873278092755
    398.016187377197 0.0325845526940341
    402.868239073212 0.0329817775787927
    407.720290769227 0.0333790024635513
    412.572342465241 0.0337762273483099
    417.424394161256 0.0341734522330686
    422.276445857271 0.0345706771178272
    427.128497553286 0.0349679020025858
    431.980549249301 0.0353651268873444
    436.832600945315 0.035762351772103
    441.68465264133 0.0361595766568616
    446.536704337345 0.0365568015416203
    451.38875603336 0.0369540264263789
    456.240807729375 0.0373512513111375
    461.092859425389 0.0377484761958961
    465.944911121404 0.0381457010806547
    };
    \addplot [semithick, steelblue31119180, dash pattern=on 1pt off 3pt on 3pt off 3pt, forget plot]
    table {%
    279.566946672842 0.0223521841923462
    284.418998368857 0.0227723276211109
    289.271050064872 0.0231930610521656
    294.123101760887 0.0236143753960784
    298.975153456902 0.0240362618512102
    303.827205152916 0.024458711890075
    308.679256848931 0.0248817172465554
    313.531308544946 0.0253052699039049
    318.383360240961 0.0257293620834796
    323.235411936975 0.0261539862341419
    328.08746363299 0.0265791350222888
    332.939515329005 0.0270048013224581
    337.79156702502 0.0274309782084711
    342.643618721035 0.0278576589450746
    347.495670417049 0.0282848369800468
    352.347722113064 0.0287125059367348
    357.199773809079 0.0291406596069961
    362.051825505094 0.0295692919445148
    366.903877201108 0.0299983970584698
    371.755928897123 0.0304279692075312
    373.755928897123 0.030605172073059
    378.607980593138 0.0310353942661578
    383.460032289153 0.0314660702264989
    388.312083985168 0.0318971946816662
    393.164135681182 0.0323287624856601
    398.016187377197 0.0327607686143423
    402.868239073212 0.0331932081610989
    407.720290769227 0.0336260763327083
    412.572342465241 0.0340593684454009
    417.424394161256 0.034493079921102
    422.276445857271 0.0349272062838446
    427.128497553286 0.0353617431563442
    431.980549249301 0.0357966862567266
    436.832600945315 0.036232031395399
    441.68465264133 0.0366677744720578
    446.536704337345 0.037103911472825
    451.38875603336 0.037540438467507
    456.240807729375 0.0379773516069682
    461.092859425389 0.0384146471206151
    465.944911121404 0.0388523213139828
    };
    \addplot [semithick, darkorange25512714]
    table {%
    454.028007881676 0.00351235661124847
    459.936271143277 0.00357546402804142
    465.844534404878 0.00363702644052686
    471.752797666479 0.00369740442685451
    477.66106092808 0.00375681626266118
    483.569324189681 0.00381540884795562
    489.477587451282 0.00387329010002339
    495.385850712883 0.00393054451663538
    501.294113974484 0.00398724113522943
    507.202377236085 0.00404343792506486
    513.110640497686 0.0040991844270412
    519.018903759287 0.00415452348112089
    524.927167020888 0.00420949244407219
    530.835430282489 0.00426412409938716
    536.74369354409 0.00431844736633356
    542.651956805691 0.00437248786868834
    548.560220067292 0.00442626839999029
    554.468483328893 0.00447980930936753
    560.376746590494 0.0045331288246579
    566.285009852095 0.00458624332503592
    };
    \addlegendentry{P=10 bar}
    \addplot [semithick, darkorange25512714, mark=*, mark size=3, mark options={solid}, only marks, forget plot]
    table {%
    453.028007881676 0.00350148206214618
    };
    \addplot [semithick, darkorange25512714, forget plot]
    table {%
    339.771005911257 1.83818456197703e-05
    345.679269172858 1.84450180255277e-05
    351.587532434459 1.8512143164282e-05
    357.49579569606 1.85831621943783e-05
    363.404058957661 1.86580406680686e-05
    369.312322219262 1.87367662331484e-05
    375.220585480863 1.88193470027022e-05
    381.128848742464 1.89058104627799e-05
    387.037112004065 1.89962028268197e-05
    392.945375265666 1.90905887744168e-05
    398.853638527267 1.91890515340157e-05
    404.761901788868 1.92916932866154e-05
    410.670165050469 1.93986358822562e-05
    416.57842831207 1.95100218741922e-05
    422.486691573671 1.96260158882238e-05
    428.394954835272 1.97468063575961e-05
    434.303218096873 1.98726076680406e-05
    440.211481358474 2.00036627739409e-05
    446.119744620075 2.01402463663932e-05
    452.028007881676 2.02826686986277e-05
    };
    \addplot [semithick, darkorange25512714, dashed, forget plot]
    table {%
    339.771005911257 0.00226619512968994
    345.679269172858 0.00233063618210693
    351.587532434459 0.00239507723452392
    357.49579569606 0.00245951828694091
    363.404058957661 0.0025239593393579
    369.312322219262 0.00258840039177489
    375.220585480863 0.00265284144419189
    381.128848742464 0.00271728249660888
    387.037112004065 0.00278172354902587
    392.945375265666 0.00284616460144286
    398.853638527267 0.00291060565385985
    404.761901788868 0.00297504670627684
    410.670165050469 0.00303948775869383
    416.57842831207 0.00310392881111082
    422.486691573671 0.00316836986352781
    428.394954835272 0.0032328109159448
    434.303218096873 0.0032972519683618
    440.211481358474 0.00336169302077879
    446.119744620075 0.00342613407319578
    452.028007881676 0.00349057512561277
    454.028007881676 0.0035123889986796
    459.936271143277 0.00357683005109659
    465.844534404878 0.00364127110351358
    471.752797666479 0.00370571215593057
    477.66106092808 0.00377015320834756
    483.569324189681 0.00383459426076455
    489.477587451282 0.00389903531318154
    495.385850712883 0.00396347636559853
    501.294113974484 0.00402791741801553
    507.202377236085 0.00409235847043252
    513.110640497686 0.00415679952284951
    519.018903759287 0.0042212405752665
    524.927167020888 0.00428568162768349
    530.835430282489 0.00435012268010048
    536.74369354409 0.00441456373251747
    542.651956805691 0.00447900478493446
    548.560220067292 0.00454344583735145
    554.468483328893 0.00460788688976844
    560.376746590494 0.00467232794218544
    566.285009852095 0.00473676899460243
    };
    \addplot [semithick, darkorange25512714, dotted, forget plot]
    table {%
    339.771005911257 0.00262611154660964
    345.679269172858 0.00267177688621119
    351.587532434459 0.00271744222581275
    357.49579569606 0.00276310756541431
    363.404058957661 0.00280877290501586
    369.312322219262 0.00285443824461742
    375.220585480863 0.00290010358421897
    381.128848742464 0.00294576892382053
    387.037112004065 0.00299143426342209
    392.945375265666 0.00303709960302364
    398.853638527267 0.0030827649426252
    404.761901788868 0.00312843028222676
    410.670165050469 0.00317409562182831
    416.57842831207 0.00321976096142987
    422.486691573671 0.00326542630103142
    428.394954835272 0.00331109164063298
    434.303218096873 0.00335675698023454
    440.211481358474 0.00340242231983609
    446.119744620075 0.00344808765943765
    452.028007881676 0.0034937529990392
    454.028007881676 0.00350921112525316
    459.936271143277 0.00355487646485472
    465.844534404878 0.00360054180445627
    471.752797666479 0.00364620714405783
    477.66106092808 0.00369187248365939
    483.569324189681 0.00373753782326094
    489.477587451282 0.0037832031628625
    495.385850712883 0.00382886850246405
    501.294113974484 0.00387453384206561
    507.202377236085 0.00392019918166717
    513.110640497686 0.00396586452126872
    519.018903759287 0.00401152986087028
    524.927167020888 0.00405719520047184
    530.835430282489 0.00410286054007339
    536.74369354409 0.00414852587967495
    542.651956805691 0.0041941912192765
    548.560220067292 0.00423985655887806
    554.468483328893 0.00428552189847962
    560.376746590494 0.00433118723808117
    566.285009852095 0.00437685257768273
    };
    \addplot [semithick, darkorange25512714, dash pattern=on 1pt off 3pt on 3pt off 3pt, forget plot]
    table {%
    339.771005911257 0.00233315445563773
    345.679269172858 0.00239061072905434
    351.587532434459 0.00244847220993624
    357.49579569606 0.0025067348744086
    363.404058957661 0.00256539480463296
    369.312322219262 0.00262444818432837
    375.220585480863 0.00268389129455054
    381.128848742464 0.00274372050971037
    387.037112004065 0.0028039322938148
    392.945375265666 0.00286452319691418
    398.853638527267 0.00292548985174211
    404.761901788868 0.00298682897053441
    410.670165050469 0.00304853734201527
    416.57842831207 0.00311061182853956
    422.486691573671 0.00317304936338101
    428.394954835272 0.0032358469481569
    434.303218096873 0.00329900165038063
    440.211481358474 0.0033625106011341
    446.119744620075 0.0034263709928526
    452.028007881676 0.00349058007721521
    454.028007881676 0.0035123939459932
    459.936271143277 0.00357706555669724
    465.844534404878 0.00364207965316137
    471.752797666479 0.00370743367080838
    477.66106092808 0.00377312509622894
    483.569324189681 0.00383915146553871
    489.477587451282 0.00390551036280773
    495.385850712883 0.00397219941855803
    501.294113974484 0.00403921630832569
    507.202377236085 0.0041065587512839
    513.110640497686 0.00417422450892369
    519.018903759287 0.00424221138378937
    524.927167020888 0.00431051721826573
    530.835430282489 0.00437913989341432
    536.74369354409 0.00444807732785641
    542.651956805691 0.00451732747670005
    548.560220067292 0.00458688833050932
    554.468483328893 0.00465675791431336
    560.376746590494 0.00472693428665349
    566.285009852095 0.00479741553866643
    };
    \addplot [semithick, forestgreen4416044]
    table {%
    585.147146966519 0.000327519412666254
    592.780662058184 0.000346311310697216
    600.414177149848 0.000362802123852569
    608.047692241513 0.00037775357288756
    615.681207333178 0.000391576987791443
    623.314722424843 0.000404528199999452
    630.948237516507 0.000416780743285698
    638.581752608172 0.000428459290607768
    646.215267699837 0.000439657127347321
    653.848782791501 0.000450446205264266
    661.482297883166 0.000460883351805354
    669.115812974831 0.000471014315760819
    676.749328066496 0.000480876515295257
    684.38284315816 0.00049050096783296
    692.016358249825 0.000499913682965948
    699.64987334149 0.000509136691111628
    707.283388433154 0.000518188818169375
    714.916903524819 0.000527086278839013
    722.550418616484 0.000535843137817035
    730.183933708149 0.000544471673001094
    };
    \addlegendentry{P=100 bar}
    \addplot [semithick, forestgreen4416044, mark=*, mark size=3, mark options={solid}, only marks, forget plot]
    table {%
    584.147146966519 0.000324815468175749
    };
    \addplot [semithick, forestgreen4416044, forget plot]
    table {%
    438.110360224889 1.98376085719536e-05
    445.743875316554 2.00053256920947e-05
    453.377390408219 2.01818721839686e-05
    461.010905499883 2.03678739811543e-05
    468.644420591548 2.05640575954332e-05
    476.277935683213 2.07712677731207e-05
    483.911450774877 2.09904899316858e-05
    491.544965866542 2.12228789350273e-05
    499.178480958207 2.14697964240414e-05
    506.811996049872 2.17328599118282e-05
    514.445511141536 2.20140083863679e-05
    522.079026233201 2.23155915958625e-05
    529.712541324866 2.2640494166552e-05
    537.346056416531 2.29923124216735e-05
    544.979571508195 2.3375613583208e-05
    552.61308659986 2.37963287889712e-05
    560.246601691525 2.42623736815989e-05
    567.880116783189 2.4784678434187e-05
    575.513631874854 2.5379008743034e-05
    583.147146966519 2.60694648284117e-05
    };
    \addplot [semithick, forestgreen4416044, dashed, forget plot]
    table {%
    438.110360224889 -7.51804604765381e-05
    445.743875316554 -5.42722014953695e-05
    453.377390408219 -3.3363942514201e-05
    461.010905499883 -1.24556835330324e-05
    468.644420591548 8.45257544813613e-06
    476.277935683213 2.93608344293047e-05
    483.911450774877 5.02690934104732e-05
    491.544965866542 7.11773523916418e-05
    499.178480958207 9.20856113728103e-05
    506.811996049872 0.000112993870353979
    514.445511141536 0.000133902129335147
    522.079026233201 0.000154810388316316
    529.712541324866 0.000175718647297485
    537.346056416531 0.000196626906278653
    544.979571508195 0.000217535165259822
    552.61308659986 0.00023844342424099
    560.246601691525 0.000259351683222159
    567.880116783189 0.000280259942203327
    575.513631874854 0.000301168201184496
    583.147146966519 0.000322076460165665
    585.147146966519 0.000327554476185833
    592.780662058184 0.000348462735167002
    600.414177149848 0.00036937099414817
    608.047692241513 0.000390279253129339
    615.681207333178 0.000411187512110507
    623.314722424843 0.000432095771091676
    630.948237516507 0.000453004030072844
    638.581752608172 0.000473912289054013
    646.215267699837 0.000494820548035181
    653.848782791501 0.00051572880701635
    661.482297883166 0.000536637065997519
    669.115812974831 0.000557545324978687
    676.749328066496 0.000578453583959856
    684.38284315816 0.000599361842941024
    692.016358249825 0.000620270101922193
    699.64987334149 0.000641178360903361
    707.283388433154 0.00066208661988453
    714.916903524819 0.000682994878865698
    722.550418616484 0.000703903137846867
    730.183933708149 0.000724811396828036
    };
    \addplot [semithick, forestgreen4416044, dotted, forget plot]
    table {%
    438.110360224889 0.000243611601131812
    445.743875316554 0.000247856223041208
    453.377390408219 0.000252100844950605
    461.010905499883 0.000256345466860002
    468.644420591548 0.000260590088769399
    476.277935683213 0.000264834710678796
    483.911450774877 0.000269079332588192
    491.544965866542 0.000273323954497589
    499.178480958207 0.000277568576406986
    506.811996049872 0.000281813198316383
    514.445511141536 0.00028605782022578
    522.079026233201 0.000290302442135176
    529.712541324866 0.000294547064044573
    537.346056416531 0.00029879168595397
    544.979571508195 0.000303036307863367
    552.61308659986 0.000307280929772764
    560.246601691525 0.00031152555168216
    567.880116783189 0.000315770173591557
    575.513631874854 0.000320014795500954
    583.147146966519 0.000324259417410351
    585.147146966519 0.000325371518941147
    592.780662058184 0.000329616140850544
    600.414177149848 0.00033386076275994
    608.047692241513 0.000338105384669337
    615.681207333178 0.000342350006578734
    623.314722424843 0.000346594628488131
    630.948237516507 0.000350839250397528
    638.581752608172 0.000355083872306924
    646.215267699837 0.000359328494216321
    653.848782791501 0.000363573116125718
    661.482297883166 0.000367817738035115
    669.115812974831 0.000372062359944512
    676.749328066496 0.000376306981853908
    684.38284315816 0.000380551603763305
    692.016358249825 0.000384796225672702
    699.64987334149 0.000389040847582099
    707.283388433154 0.000393285469491495
    714.916903524819 0.000397530091400892
    722.550418616484 0.000401774713310289
    730.183933708149 0.000406019335219686
    };
    \addplot [semithick, forestgreen4416044, dash pattern=on 1pt off 3pt on 3pt off 3pt, forget plot]
    table {%
    438.110360224889 7.87427393757862e-05
    445.743875316554 8.57360290468106e-05
    453.377390408219 9.32156264571468e-05
    461.010905499883 0.000101206299993923
    468.644420591548 0.000109733634289132
    476.277935683213 0.00011882404293552
    483.911450774877 0.000128504781186872
    491.544965866542 0.000138803958642948
    499.178480958207 0.000149750551919356
    506.811996049872 0.000161374417302603
    514.445511141536 0.000173706303390572
    522.079026233201 0.000186777863718673
    529.712541324866 0.000200621669371887
    537.346056416531 0.00021527122158294
    544.979571508195 0.000230760964316817
    552.61308659986 0.000247126296841828
    560.246601691525 0.000264403586287438
    567.880116783189 0.000282630180189051
    575.513631874854 0.000301844419019944
    583.147146966519 0.000322085648710547
    585.147146966519 0.00032756369546383
    592.780662058184 0.000349158730797641
    600.414177149848 0.000371873532577751
    608.047692241513 0.000395750813413037
    615.681207333178 0.000420834344588929
    623.314722424843 0.000447168968522958
    630.948237516507 0.000474800611208723
    638.581752608172 0.0005037762946484
    646.215267699837 0.000534144149273979
    653.848782791501 0.000565953426357345
    661.482297883166 0.000599254510409364
    669.115812974831 0.000634098931568115
    676.749328066496 0.000670539377976381
    684.38284315816 0.00070862970814857
    692.016358249825 0.000748424963327153
    699.64987334149 0.000789981379828788
    707.283388433154 0.000833356401380216
    714.916903524819 0.000878608691444078
    722.550418616484 0.000925798145534747
    730.183933708149 0.000974985903524325
    };
        
    \end{axis}
    

\end{tikzpicture}

        \caption{The molar volume of water vapour at sub-saturation conditions, as extrapolated by the various methods considered, see Table~\ref{table:SemiEmpirical_ExtrapolationFuncs}}
        \label{fig:SemiEmpirical_extrapolation2}
    \end{figure}


\subsection{Algorithm}
\label{sec:algorithm}
    Similarly to \emph{GeoProp}, the fluid is first equilibrated to determine the composition of the water and carbon dioxide-rich phases. Using these compositions, the composition contribution to the component properties are then evaluated (i.e. the partial derivatives of the component activity, \(A_i\)). For the carbon dioxide-rich phase the properties of pure carbon dioxide are then determined from the \ac{SW} \ac{HEOS}, while the properties of pure water are determined from the \ac{WP} \ac{HEOS} or extrapolated from the saturated properties if the temperature is below the saturation temperature of pure water. For the water-rich phase, the properties of pure water are obtained from the \ac{WP} \ac{HEOS}, while the properties of aqueous carbon dioxide is calculated using \emph{ThermoFun} using the "CO2@" component in the \emph{slop98-inorganic} database \cite{Johnson1992}. The overall fluid properties are then obtained by aggregating the component and phase properties, see Figure~\ref{fig:coupled_model}.

    \todo{maybe I should just write a wee routine to calculate the CO2@ properties directly instead of using ThermoFUN- for speed}

    \begin{figure}[H]
        \centering
        \begin{tikzpicture} [node distance=1.5cm]
    \node (start) [startstop] {Start};
    \node (input) [io, right of=start, xshift=2cm] {\(P, T, \mathbf{z}\)};
    \node (SP2009) [process, right of=input, text width=4.5cm, xshift=3cm] {\(\mathbf{x}, \mathbf{y}=SP2009(P,T,\mathbf{z})\)};
    \node (cp_derivs) [process, below of=SP2009, xshift=-2.5cm, yshift=-0.2cm, text width=3.5cm] {\(\frac{\partial RT*f(x)*A_i(P,T,\mathbf{y})}{\partial x}\)};
    \node (wp_derivs) [process, right of=cp_derivs, xshift=3cm, yshift=-0.2cm, text width=3.5cm] {\(\frac{\partial RT*f(x)*A_i(P,T,\mathbf{x})}{\partial x}\)};
    \node (cp_CO2) [process, below of=cp_derivs, text width=3.5cm] {\(\Psi_{CO_2}^{cp}=SW(P,T)\)};
    \node (Tsat) [decision, left of=cp_CO2, xshift=-3.7cm] {\(T<T_{sat}^{wat}\)};        

    \node (extrap) [process, below of=Tsat, text width=4cm, xshift=2.5cm] {\(\Psi_{H_2O}^{cp}=Extrap(P,T)\)};
    \node (cp_H2O) [process, below of=Tsat, text width=3.5cm, yshift=-1.5cm] {\(\Psi_{H_2O}^{cp}=WP(P,T)\)};
    \node (cp_props) [process, right of=cp_H2O, xshift=3.5cm] {\(\Psi^{cp}(P,T, \mathbf{y})\)};

    \node (wp_CO2) [process, below of=wp_derivs, text width=3.5cm] {\(\Psi_{CO_2}^{wp}=SW(P,T)\)};
    \node (wp_H2O) [process, below of=wp_CO2, text width=3.5cm] {\(\Psi_{H_2O}^{wp}=WP(P,T)\)};
    \node (wp_props) [process, below of=wp_H2O, text width=3.5cm] {\(\Psi^{wp}(P,T, \mathbf{x})\)};
    
    \node (out1) [io, below of=wp_props] {\(\Psi(P,T\mathbf{z})\)};
    \node (stop) [startstop, right of=out1, xshift=2cm] {Stop};
    
    % \node (stop) [startstop, below of=out1] {Stop};
    
    \draw [arrow] (start) -- (input);
    \draw [arrow] (input) -- (SP2009);
    \draw [arrow] (SP2009) |- ($0.5*(SP2009)+0.5*(cp_derivs)$) -| (cp_derivs);
    \draw [arrow] (cp_derivs) -- (cp_CO2);
    \draw [arrow] (cp_CO2) -- (Tsat);
    \draw [arrow] (Tsat) -- node[anchor=west] {yes} (extrap);
    \draw [arrow] (Tsat) -- node[anchor=east] {no} (cp_H2O);
    \draw [arrow] (cp_H2O) -- (cp_props);
    \draw [arrow] (extrap) -| (cp_props);
    \draw [arrow] (cp_props) -| ($0.5*(cp_props)+0.5*(wp_derivs)$) |- (wp_derivs);
    \draw [arrow] (wp_derivs) -- (wp_CO2);
    \draw [arrow] (wp_CO2) -- (wp_H2O);
    \draw [arrow] (wp_H2O) -- (wp_props);
    \draw [arrow] (wp_props) -- (out1);
    \draw [arrow] (out1) -- (stop);

\end{tikzpicture}
        \caption{Calculation steps for equilibrating a mixture of water and carbon dioxide and determining the thermophysical properties}
        \label{fig:coupled_model}
    \end{figure}

\subsection{Validation}
\label{sec:SemiEmpirical_validation}
    The proposed model was validated against the \ac{HEOS} mixture model for water and carbon dioxide implemented in \emph{CoolProp}. An extract of the validation plots is shown in Figures~\ref{fig:SemiEmpirical_properties_maintext} and \ref{fig:SemiEmpirical_ratios_maintext} for a 50:50 mixture of carbon dioxide and water, further validation plots are provided in \nameref{ch:appendix_e}.

    For temperatures up to \qty{466}{\K} (\qty{193}{\degreeCelsius}) the differences between the coupled model and the \ac{HEOS} mixture are below \qty{10}{\percent}, see Figure~\ref{fig:SemiEmpirical_ratios_maintext}, particularly for the density and molar volume, as well as the vapour quality. Comparing the relative differences in molar enthalpy and molar enthalpy is difficult at low temperatures, as the values are small in magnitude and transition from positive to negative, which can lead to \emph{infinite} relative differences. In absolute terms, the differences are small, see Figure~\ref{fig:SemiEmpirical_properties_maintext}.

    \begin{figure}[H]
        \centering
        % This file was created with tikzplotlib v0.10.1.
\begin{tikzpicture}

\definecolor{crimson2143940}{RGB}{214,39,40}
\definecolor{darkgray176}{RGB}{176,176,176}
\definecolor{darkorange25512714}{RGB}{255,127,14}
\definecolor{forestgreen4416044}{RGB}{44,160,44}
\definecolor{lightgray204}{RGB}{204,204,204}
\definecolor{mediumpurple148103189}{RGB}{148,103,189}
\definecolor{steelblue31119180}{RGB}{31,119,180}

\begin{groupplot}[
    group style={
        group size=2 by 3,
        vertical sep=2cm,
        horizontal sep=2.5cm
        },
    width=7cm,
    height=6.5cm,
    legend style={
        at={(1.55,0.5)},
        anchor=west
        },
    legend cell align={left},
    unbounded coords=jump,
    xlabel={Pressure/\unit{\bar}},
    xmin=0, xmax=100,
    ylabel near ticks,
    xlabel near ticks
    ]
\nextgroupplot[
ylabel={Density/\unit{\mole\per\cubic\m}},
ymin=10, ymax=100000,
ymode=log,
]
\addplot [semithick, steelblue31119180]
table {%
1 78.6181231807981
6.21052631578947 514.622589769306
11.4210526315789 974.985519021075
16.6315789473684 1463.15545586596
21.8421052631579 1983.59576324449
27.0526315789474 2542.25329903563
32.2631578947368 3147.31225867986
37.4736842105263 3810.52201975274
42.6842105263158 4549.73206626009
47.8947368421053 5394.3208088903
53.1052631578947 6398.98001483221
58.3157894736842 7689.92400808856
63.5263157894737 9759.92566203363
68.7368421052632 25958.2656723849
73.9473684210526 26510.9773428999
79.1578947368421 26931.8772757895
84.3684210526316 27276.5685972453
89.5789473684211 27570.4754912051
94.7894736842105 27827.6301300102
100 28056.7811032421
};
\addplot [semithick, steelblue31119180, mark=x, mark size=3, mark options={solid}, only marks]
table {%
1 78.5898262771933
6.21052631578947 514.82292905089
11.4210526315789 975.793187577952
16.6315789473684 1464.71493842127
21.8421052631579 1985.91323578295
27.0526315789474 2545.25915374247
32.2631578947368 3150.91613832405
37.4736842105263 3814.67294657867
42.6842105263158 4554.51213166329
47.8947368421053 5400.1498911377
53.1052631578947 6407.19044357623
58.3157894736842 7705.22301303514
63.5263157894737 9818.13030472255
68.7368421052632 nan
73.9473684210526 nan
79.1578947368421 nan
84.3684210526316 nan
89.5789473684211 nan
94.7894736842105 27911.7039015851
100 28131.4106447992
};
\addplot [semithick, darkorange25512714]
table {%
1 34.8066078194273
6.21052631578947 395.82480749396
11.4210526315789 765.94041123069
16.6315789473684 1145.80186456717
21.8421052631579 1535.99920584729
27.0526315789474 1937.16153026038
32.2631578947368 2349.97036368346
37.4736842105263 2775.1696874478
42.6842105263158 3213.57326807184
47.8947368421053 3666.06806793108
53.1052631578947 4133.62028541041
58.3157894736842 4617.27664309938
63.5263157894737 5118.16280827484
68.7368421052632 5637.48001033265
73.9473684210526 6176.48961741407
79.1578947368421 6736.49231021913
84.3684210526316 7318.78823653631
89.5789473684211 7924.61832724859
94.7894736842105 8555.06644057043
100 9210.92562681569
};
\addplot [semithick, darkorange25512714, mark=x, mark size=3, mark options={solid}, only marks]
table {%
1 34.183163488496
6.21052631578947 393.709692763133
11.4210526315789 762.234244526454
16.6315789473684 1140.61804746057
21.8421052631579 1529.43155633068
27.0526315789474 1929.29845828602
32.2631578947368 2340.90510509946
37.4736842105263 2765.0073070839
42.6842105263158 3202.43669687345
47.8947368421053 3654.10671776687
53.1052631578947 4121.01785489631
58.3157894736842 4604.26141461595
63.5263157894737 5105.02075860801
68.7368421052632 5624.56832568942
73.9473684210526 6164.25593596656
79.1578947368421 6725.49465466176
84.3684210526316 7309.71875993487
89.5789473684211 7918.32598449698
94.7894736842105 8552.58317818906
100 9213.48325188865
};
\addplot [semithick, forestgreen4416044]
table {%
1 29.455491285494
6.21052631578947 189.021571737763
11.4210526315789 491.178746333363
16.6315789473684 812.065893823356
21.8421052631579 1137.19569781502
27.0526315789474 1466.62508378298
32.2631578947368 1800.40613584438
37.4736842105263 2138.57989184851
42.6842105263158 2481.17469071288
47.8947368421053 2828.20457548431
53.1052631578947 3179.66630829352
58.3157894736842 3535.5386577688
63.5263157894737 3895.77926346434
68.7368421052632 4260.32151846909
73.9473684210526 4629.07204079565
79.1578947368421 5001.90688402734
84.3684210526316 5378.66871634548
89.5789473684211 5759.16313526844
94.7894736842105 6143.15508862893
100 6530.36746161477
};
\addplot [semithick, forestgreen4416044, mark=x, mark size=3, mark options={solid}, only marks]
table {%
1 29.4078976760542
6.21052631578947 186.489181483051
11.4210526315789 478.716090700305
16.6315789473684 792.787115437114
21.8421052631579 1110.91904551143
27.0526315789474 1433.23981197277
32.2631578947368 1759.82351705616
37.4736842105263 2090.72478629783
42.6842105263158 2425.98705974478
47.8947368421053 2765.64445192434
53.1052631578947 3109.72145284401
58.3157894736842 3458.23167782197
63.5263157894737 3811.1760806705
68.7368421052632 4168.54079767839
73.9473684210526 4530.29470400293
79.1578947368421 4896.38673448684
84.3684210526316 5266.74301378634
89.5789473684211 5641.26384409261
94.7894736842105 6019.82060763931
100 6402.25265340539
};
\addplot [semithick, crimson2143940]
table {%
1 25.8439165666813
6.21052631578947 163.36547752645
11.4210526315789 306.469194238175
16.6315789473684 456.632024820297
21.8421052631579 614.234449863661
27.0526315789474 779.61981828486
32.2631578947368 1054.94669375207
37.4736842105263 1350.29992057475
42.6842105263158 1648.24601741846
47.8947368421053 1948.67599529701
53.1052631578947 2251.48862529711
58.3157894736842 2556.57627299398
63.5263157894737 2863.81864930327
68.7368421052632 3173.07683650953
73.9473684210526 3484.18849215078
79.1578947368421 3796.96509428599
84.3684210526316 4111.18721061734
89.5789473684211 4426.60042843191
94.7894736842105 4742.91367736579
100 5059.79343786605
};
\addplot [semithick, crimson2143940, mark=x, mark size=3, mark options={solid}, only marks]
table {%
1 25.8228373674791
6.21052631578947 162.390542752225
11.4210526315789 302.483037762943
16.6315789473684 446.305094707113
21.8421052631579 594.079293017459
27.0526315789474 746.04802281413
32.2631578947368 975.635685010843
37.4736842105263 1252.74355401234
42.6842105263158 1531.44633216032
47.8947368421053 1811.71297374055
53.1052631578947 2093.51701584741
58.3157894736842 2376.82474056036
63.5263157894737 2661.59291833818
68.7368421052632 2947.76898372636
73.9473684210526 3235.29164126462
79.1578947368421 3524.09138915215
84.3684210526316 3814.09087386066
89.5789473684211 4105.20510137776
94.7894736842105 4397.34155186076
100 4690.40024123136
};
\addplot [semithick, mediumpurple148103189]
table {%
1 23.0361826190002
6.21052631578947 144.613988747703
11.4210526315789 268.974803158886
16.6315789473684 396.410891669675
21.8421052631579 527.27385741685
27.0526315789474 662.001472241784
32.2631578947368 801.165897762212
37.4736842105263 945.572372565025
42.6842105263158 1096.39086712809
47.8947368421053 1253.4104543006
53.1052631578947 1416.75959852651
58.3157894736842 1586.87336202841
63.5263157894737 1764.21412381928
68.7368421052632 1949.2692559799
73.9473684210526 2142.54992638886
79.1578947368421 2344.58458309772
84.3684210526316 2555.91255517219
89.5789473684211 2807.24766462329
94.7894736842105 3110.11509528931
100 3416.00591901316
};
\addplot [semithick, mediumpurple148103189, mark=x, mark size=3, mark options={solid}, only marks]
table {%
1 23.0247136537989
6.21052631578947 144.131002358246
11.4210526315789 267.184609130274
16.6315789473684 392.244929686455
21.8421052631579 519.373570914644
27.0526315789474 648.634401784888
32.2631578947368 780.093597522793
37.4736842105263 913.819675303201
42.6842105263158 1049.88351934842
47.8947368421053 1188.3583929545
53.1052631578947 1329.31993469357
58.3157894736842 1472.84613525723
63.5263157894737 1619.01729145387
68.7368421052632 1767.91593273705
73.9473684210526 1919.62671541197
79.1578947368421 2074.23627881802
84.3684210526316 2231.83305708681
89.5789473684211 2392.50703927727
94.7894736842105 2568.0187119573
100 2815.51011282874
};

\nextgroupplot[
ylabel={Volume/\unit{\cubic\m\per\mole}},
ymin=1e-05, ymax=0.1,
ymode=log,
]
\addplot [semithick, steelblue31119180]
table {%
1 0.0127197134647987
6.21052631578947 0.00194317159774949
11.4210526315789 0.0010256562589812
16.6315789473684 0.000683454376628871
21.8421052631579 0.000504134974741193
27.0526315789474 0.000393351834917212
32.2631578947368 0.000317731422181621
37.4736842105263 0.000262431235094894
42.6842105263158 0.000219793162638258
47.8947368421053 0.000185380149870196
53.1052631578947 0.000156274905950964
58.3157894736842 0.000130040296750418
63.5263157894737 0.00010245979678821
68.7368421052632 3.85233748903273e-05
73.9473684210526 3.77202238554143e-05
79.1578947368421 3.71307202152949e-05
84.3684210526316 3.66615029465616e-05
89.5789473684211 3.6270683845079e-05
94.7894736842105 3.59355070959337e-05
100 3.56420074106236e-05
};
\addplot [semithick, steelblue31119180, mark=x, mark size=3, mark options={solid}, only marks]
table {%
1 0.0127242933006737
6.21052631578947 0.00194241542785121
11.4210526315789 0.00102480731852836
16.6315789473684 0.000682726702492596
21.8421052631579 0.000503546671617679
27.0526315789474 0.000392887301290964
32.2631578947368 0.000317368014920224
37.4736842105263 0.000262145671202793
42.6842105263158 0.000219562484650756
47.8947368421053 0.000185180045028217
53.1052631578947 0.000156074649069092
58.3157894736842 0.000129782096937139
63.5263157894737 0.000101852386244965
68.7368421052632 nan
73.9473684210526 nan
79.1578947368421 nan
84.3684210526316 nan
89.5789473684211 nan
94.7894736842105 3.58272645599113e-05
100 3.55474530810589e-05
};
\addplot [semithick, darkorange25512714]
table {%
1 0.0287301769016931
6.21052631578947 0.00252637020486711
11.4210526315789 0.00130558459292314
16.6315789473684 0.000872751241662321
21.8421052631579 0.000651042003272637
27.0526315789474 0.000516219212687745
32.2631578947368 0.000425537281428754
37.4736842105263 0.000360338326165437
42.6842105263158 0.000311180084155979
47.8947368421053 0.000272771803870064
53.1052631578947 0.000241918688934611
58.3157894736842 0.000216577882872693
63.5263157894737 0.000195382608459278
68.7368421052632 0.000177384220993627
73.9473684210526 0.000161904263091544
79.1578947368421 0.000148445207676259
84.3684210526316 0.000136634640555369
89.5789473684211 0.000126189042639634
94.7894736842105 0.000116889799389252
100 0.000108566721794899
};
\addplot [semithick, darkorange25512714, mark=x, mark size=3, mark options={solid}, only marks]
table {%
1 0.029254167781649
6.21052631578947 0.00253994254746893
11.4210526315789 0.00131193266004634
16.6315789473684 0.00087671767269189
21.8421052631579 0.000653837692743268
27.0526315789474 0.000518323121912612
32.2631578947368 0.000427185193377376
37.4736842105263 0.000361662697034478
42.6842105263158 0.000312262222380947
47.8947368421053 0.000273664694886396
53.1052631578947 0.000242658497296213
58.3157894736842 0.00021719010063711
63.5263157894737 0.000195885589360986
68.7368421052632 0.000177791421864793
73.9473684210526 0.000162225580895385
79.1578947368421 0.000148687948076333
84.3684210526316 0.000136804168921119
89.5789473684211 0.000126289319479631
94.7894736842105 0.000116923738613875
100 0.000108536584119259
};
\addplot [semithick, forestgreen4416044]
table {%
1 0.0339495271121984
6.21052631578947 0.00529040146479863
11.4210526315789 0.00203591871078497
16.6315789473684 0.00123142716324634
21.8421052631579 0.000879356123067806
27.0526315789474 0.000681837513252279
32.2631578947368 0.0005554302332629
37.4736842105263 0.000467600019906499
42.6842105263158 0.000403034902678571
47.8947368421053 0.000353581211440038
53.1052631578947 0.000314498410538144
58.3157894736842 0.000282842332328245
63.5263157894737 0.000256688054525642
68.7368421052632 0.000234724068515688
73.9473684210526 0.000216026017998225
79.1578947368421 0.00019992375371747
84.3684210526316 0.000185919611847641
89.5789473684211 0.000173636338563865
94.7894736842105 0.000162782802252708
100 0.0001531307397138
};
\addplot [semithick, forestgreen4416044, mark=x, mark size=3, mark options={solid}, only marks]
table {%
1 0.0340044708743074
6.21052631578947 0.00536224134851966
11.4210526315789 0.00208892080175772
16.6315789473684 0.00126137266931821
21.8421052631579 0.000900155600032613
27.0526315789474 0.000697719943059327
32.2631578947368 0.000568238797986293
37.4736842105263 0.000478303029912778
42.6842105263158 0.000412203352851025
47.8947368421053 0.000361579377748358
53.1052631578947 0.000321572209975734
58.3157894736842 0.000289165126331215
63.5263157894737 0.000262386197549831
68.7368421052632 0.000239892098586857
73.9473684210526 0.00022073619164696
79.1578947368421 0.000204232233732004
84.3684210526316 0.000189870665301568
89.5789473684211 0.000177265241909785
94.7894736842105 0.000166117907023836
100 0.000156195022929639
};
\addplot [semithick, crimson2143940]
table {%
1 0.0386938255825059
6.21052631578947 0.0061212443114739
11.4210526315789 0.0032629706959155
16.6315789473684 0.00218994714703495
21.8421052631579 0.00162804284295999
27.0526315789474 0.0012826764745411
32.2631578947368 0.000947915194125458
37.4736842105263 0.000740576211819927
42.6842105263158 0.000606705546036284
47.8947368421053 0.000513168942612023
53.1052631578947 0.000444150589420828
58.3157894736842 0.000391148118897665
63.5263157894737 0.000349184121782043
68.7368421052632 0.00031515152374943
73.9473684210526 0.000287010878502357
79.1578947368421 0.000263368236253972
84.3684210526316 0.000243238740726146
89.5789473684211 0.000225906994807354
94.7894736842105 0.000210840860286413
100 0.000197636526526218
};
\addplot [semithick, crimson2143940, mark=x, mark size=3, mark options={solid}, only marks]
table {%
1 0.0387254113778908
6.21052631578947 0.00615799407435811
11.4210526315789 0.00330597050133999
16.6315789473684 0.00224061972820689
21.8421052631579 0.00168327698297778
27.0526315789474 0.00134039628739709
32.2631578947368 0.00102497275915947
37.4736842105263 0.000798247970861358
42.6842105263158 0.000652977501725026
47.8947368421053 0.000551963812421872
53.1052631578947 0.000477665092965686
58.3157894736842 0.00042072938022525
63.5263157894737 0.00037571485598345
68.7368421052632 0.000339239609861107
73.9473684210526 0.00030909114567771
79.1578947368421 0.000283761086071206
84.3684210526316 0.000262185677549888
89.5789473684211 0.000243593188477815
94.7894736842105 0.000227410126824659
100 0.000213201421748493
};
\addplot [semithick, mediumpurple148103189]
table {%
1 0.043409970156045
6.21052631578947 0.00691496036213083
11.4210526315789 0.00371782036181764
16.6315789473684 0.00252263502596515
21.8421052631579 0.00189654765912171
27.0526315789474 0.0015105706587232
32.2631578947368 0.00124818093580014
37.4736842105263 0.00105756050939531
42.6842105263158 0.000912083482252472
47.8947368421053 0.000797823248217595
53.1052631578947 0.000705836050830389
58.3157894736842 0.000630170008476137
63.5263157894737 0.000566824619811534
68.7368421052632 0.000513012759490376
73.9473684210526 0.000466733581179805
79.1578947368421 0.000426514789532045
84.3684210526316 0.000391249692003893
89.5789473684211 0.000356220796833112
94.7894736842105 0.000321531509079724
100 0.000292739539599184
};
\addplot [semithick, mediumpurple148103189, mark=x, mark size=3, mark options={solid}, only marks]
table {%
1 0.0434315933321067
6.21052631578947 0.00693813255745242
11.4210526315789 0.00374273055343701
16.6315789473684 0.00254942747328655
21.8421052631579 0.0019253963928872
27.0526315789474 0.00154170052844597
32.2631578947368 0.00128189745842746
37.4736842105263 0.0010943078016658
42.6842105263158 0.000952486615487227
47.8947368421053 0.000841496980985504
53.1052631578947 0.000752264352546943
58.3157894736842 0.000678957547609242
63.5263157894737 0.000617658628649977
68.7368421052632 0.00056563775544
73.9473684210526 0.00052093461294916
79.1578947368421 0.000482105153695335
84.3684210526316 0.000448062186741372
89.5789473684211 0.000417971601998747
94.7894736842105 0.000389405262252866
100 0.000355175424674749
};

\nextgroupplot[
ylabel={Enthalpy/\unit{\joule\per\mole}},
ymin=-10000, ymax=35000,
]
\addplot [semithick, steelblue31119180]
table {%
1 708.71390286158
6.21052631578947 7.09942345749788
11.4210526315789 -152.749804525569
16.6315789473684 -286.42282500426
21.8421052631579 -417.856111770729
27.0526315789474 -553.695010661354
32.2631578947368 -697.718907471252
37.4736842105263 -853.491024723231
42.6842105263158 -1025.5612977933
47.8947368421053 -1220.9179024951
53.1052631578947 -1452.19988913228
58.3157894736842 -1748.20123828285
63.5263157894737 -2220.5649865141
68.7368421052632 -4983.83168026514
73.9473684210526 -5058.1212510902
79.1578947368421 -5113.26855894831
84.3684210526316 -5157.4157798679
89.5789473684211 -5194.31990428447
94.7894736842105 -5226.08170635963
100 -5254.02170491015
};
\addplot [semithick, steelblue31119180, mark=x, mark size=3, mark options={solid}, only marks]
table {%
1 705.395869383656
6.21052631578947 -40.6537083137584
11.4210526315789 -238.249345991862
16.6315789473684 -404.137505588901
21.8421052631579 -563.363938019849
27.0526315789474 -723.369771880287
32.2631578947368 -888.51756918407
37.4736842105263 -1062.81324819917
42.6842105263158 -1251.15904486175
47.8947368421053 -1460.85442282885
53.1052631578947 -1704.90976966169
58.3157894736842 -2012.98013663158
63.5263157894737 -2503.85304714544
68.7368421052632 nan
73.9473684210526 nan
79.1578947368421 nan
84.3684210526316 nan
89.5789473684211 nan
94.7894736842105 -5508.04321249885
100 -5532.83035498024
};
\addplot [semithick, darkorange25512714]
table {%
1 23335.3702855107
6.21052631578947 4926.48126615134
11.4210526315789 4009.00670612832
16.6315789473684 3639.56203619884
21.8421052631579 3415.27667457717
27.0526315789474 3249.36187960706
32.2631578947368 3111.90387419746
37.4736842105263 2989.78320822287
42.6842105263158 2876.29443242532
47.8947368421053 2767.61236981195
53.1052631578947 2661.35984422822
58.3157894736842 2555.94907129339
63.5263157894737 2450.25146499306
68.7368421052632 2343.41959436776
73.9473684210526 2234.78578029334
79.1578947368421 2123.80454655109
84.3684210526316 2010.0189864968
89.5789473684211 1893.04567037736
94.7894736842105 1772.57010815252
100 1648.35477896071
};
\addplot [semithick, darkorange25512714, mark=x, mark size=3, mark options={solid}, only marks]
table {%
1 23991.4413008831
6.21052631578947 4982.08768711739
11.4210526315789 4044.82645717823
16.6315789473684 3661.44117948127
21.8421052631579 3425.27577107421
27.0526315789474 3248.74890456664
32.2631578947368 3101.66096334113
37.4736842105263 2970.75786110119
42.6842105263158 2849.26070022392
47.8947368421053 2733.30176293403
53.1052631578947 2620.47875910023
58.3157894736842 2509.19135436311
63.5263157894737 2398.30771799119
68.7368421052632 2286.98531692753
73.9473684210526 2174.57024110867
79.1578947368421 2060.53988421771
84.3684210526316 1944.47184906588
89.5789473684211 1826.0307025911
94.7894736842105 1704.96882527082
100 1581.14017817866
};
\addplot [semithick, forestgreen4416044]
table {%
1 26082.750285621
6.21052631578947 25582.2873166182
11.4210526315789 14404.5055639428
16.6315789473684 11213.7791922272
21.8421052631579 9789.4567270344
27.0526315789474 8967.90509460991
32.2631578947368 8422.95386419974
37.4736842105263 8027.36947842961
42.6842105263158 7721.2405606467
47.8947368421053 7472.63661318472
53.1052631578947 7262.96427648222
58.3157894736842 7080.64418317181
63.5263157894737 6918.06991275491
68.7368421052632 6770.01980067873
73.9473684210526 6632.76913921448
79.1578947368421 6503.56847477225
84.3684210526316 6380.32052747216
89.5789473684211 6261.37235398758
94.7894736842105 6145.37604701453
100 6031.19089087394
};
\addplot [semithick, forestgreen4416044, mark=x, mark size=3, mark options={solid}, only marks]
table {%
1 26123.7843325479
6.21052631578947 25852.3446205405
11.4210526315789 14857.1477386746
16.6315789473684 11546.3087903372
21.8421052631579 10074.9400224092
27.0526315789474 9228.10588364075
32.2631578947368 8667.53897147035
37.4736842105263 8261.76646867109
42.6842105263158 7949.0370206123
47.8947368421053 7696.51541644035
53.1052631578947 7485.1329424674
58.3157894736842 7303.05964135004
63.5263157894737 7142.56992323432
68.7368421052632 6998.40775129014
73.9473684210526 6866.87629908273
79.1578947368421 6745.30335826711
84.3684210526316 6631.7132621125
89.5789473684211 6524.61792894134
94.7894736842105 6422.87948114745
100 6325.61740374568
};
\addplot [semithick, crimson2143940]
table {%
1 28289.6519271448
6.21052631578947 28008.5498132446
11.4210526315789 27686.8124211914
16.6315789473684 27305.8869908899
21.8421052631579 26904.6421839262
27.0526315789474 26493.2030181292
32.2631578947368 22438.6612196441
37.4736842105263 19614.496748936
42.6842105263158 17767.7105765724
47.8947368421053 16457.0600269805
53.1052631578947 15471.5284011718
58.3157894736842 14697.4282144113
63.5263157894737 14068.0201166646
68.7368421052632 13541.4783983364
73.9473684210526 13090.1873882664
79.1578947368421 12695.1071259359
84.3684210526316 12342.6071040814
89.5789473684211 12022.5912278414
94.7894736842105 11727.334448477
100 11450.7278276945
};
\addplot [semithick, crimson2143940, mark=x, mark size=3, mark options={solid}, only marks]
table {%
1 28314.2012883851
6.21052631578947 28126.3283050289
11.4210526315789 27935.280832301
16.6315789473684 27740.9214174641
21.8421052631579 27543.1032436704
27.0526315789474 27341.6693131726
32.2631578947368 24259.5140647931
37.4736842105263 21106.9127861002
42.6842105263158 19076.5655502645
47.8947368421053 17655.5589695989
53.1052631578947 16601.9592698062
58.3157894736842 15786.8447371348
63.5263157894737 15135.2586490784
68.7368421052632 14600.664772009
73.9473684210526 14152.6518889145
79.1578947368421 13770.5109359907
84.3684210526316 13439.654736664
89.5789473684211 13149.5138601027
94.7894736842105 12892.2444100687
100 12661.9043912856
};
\addplot [semithick, mediumpurple148103189]
table {%
1 30546.7322242544
6.21052631578947 30359.540283888
11.4210526315789 30161.3150804211
16.6315789473684 29950.8361339489
21.8421052631579 29726.4506504769
27.0526315789474 29485.7785836934
32.2631578947368 29225.1110693597
37.4736842105263 28938.1335776383
42.6842105263158 28616.5702957424
47.8947368421053 28278.4795958639
53.1052631578947 27928.1034161847
58.3157894736842 27564.5235665577
63.5263157894737 27186.5969200893
68.7368421052632 26792.9647923019
73.9473684210526 26382.0416182051
79.1578947368421 25951.9896113252
84.3684210526316 25500.68308696
89.5789473684211 24628.9755699646
94.7894736842105 23306.2150875022
100 22168.4399995689
};
\addplot [semithick, mediumpurple148103189, mark=x, mark size=3, mark options={solid}, only marks]
table {%
1 30564.8871063627
6.21052631578947 30426.9833083227
11.4210526315789 30287.9169759124
16.6315789473684 30147.66899045
21.8421052631579 30006.2203087588
27.0526315789474 29863.5520376789
32.2631578947368 29719.6455194869
37.4736842105263 29574.4824294432
42.6842105263158 29428.0448868851
47.8947368421053 29280.3155814371
53.1052631578947 29131.277915919
58.3157894736842 28980.916168038
63.5263157894737 28829.2156725163
68.7368421052632 28676.1630260607
73.9473684210526 28521.7463173112
79.1578947368421 28365.9553842092
84.3684210526316 28208.7821012926
89.5789473684211 28050.2206995007
94.7894736842105 27716.9423181545
100 26472.1708687376
};

\nextgroupplot[
ylabel={Entropy/\unit{\joule\per\mole\K}},
ymin=-40, ymax=90,
]
\addplot [semithick, steelblue31119180]
table {%
1 1.98427686460376
6.21052631578947 -7.49870848649087
11.4210526315789 -10.3944190345447
16.6315789473684 -12.2520192710663
21.8421052631579 -13.6774198860905
27.0526315789474 -14.8753905851583
32.2631578947368 -15.9437786429574
37.4736842105263 -16.941113184192
42.6842105263158 -17.9107418440397
47.8947368421053 -18.8941013522973
53.1052631578947 -19.9454278204458
58.3157894736842 -21.1687748340141
63.5263157894737 -22.9415696799664
68.7368421052632 -32.3195869319795
73.9473684210526 -32.6299737606445
79.1578947368421 -32.8744028808955
84.3684210526316 -33.0804817144859
89.5789473684211 -33.2609692926471
94.7894736842105 -33.4229958731404
100 -33.571028091012
};
\addplot [semithick, steelblue31119180, mark=x, mark size=3, mark options={solid}, only marks]
table {%
1 2.55773795553476
6.21052631578947 -7.53959286764862
11.4210526315789 -10.6353629306802
16.6315789473684 -12.6475746891573
21.8421052631579 -14.204875273138
27.0526315789474 -15.5187238721704
32.2631578947368 -16.6900900390198
37.4736842105263 -17.7793344392938
42.6842105263158 -18.8310388365236
47.8947368421053 -19.8876041864562
53.1052631578947 -21.0043321979625
58.3157894736842 -22.2878252040422
63.5263157894737 -24.138534648813
68.7368421052632 nan
73.9473684210526 nan
79.1578947368421 nan
84.3684210526316 nan
89.5789473684211 nan
94.7894736842105 -34.6153103278712
100 -34.7608831342685
};
\addplot [semithick, darkorange25512714]
table {%
1 63.9515769474803
6.21052631578947 6.371645145571
11.4210526315789 1.62276313652221
16.6315789473684 -0.779590262843959
21.8421052631579 -2.3996091344461
27.0526315789474 -3.63759987341111
32.2631578947368 -4.65253980803272
37.4736842105263 -5.52311617072836
42.6842105263158 -6.29393328202975
47.8947368421053 -6.99274381324586
53.1052631578947 -7.63804104996815
58.3157894736842 -8.24282328147595
63.5263157894737 -8.81662593741155
68.7368421052632 -9.36669505301108
73.9473684210526 -9.89870074982685
79.1578947368421 -10.4171835419801
84.3684210526316 -10.9258414826345
89.5789473684211 -11.4277084494979
94.7894736842105 -11.9252612457481
100 -12.420465763248
};
\addplot [semithick, darkorange25512714, mark=x, mark size=3, mark options={solid}, only marks]
table {%
1 71.4107260032673
6.21052631578947 7.7366623572252
11.4210526315789 2.4509511491046
16.6315789473684 -0.199248076412646
21.8421052631579 -1.97610735242852
27.0526315789474 -3.32908718614225
32.2631578947368 -4.43554752481536
37.4736842105263 -5.38268358567202
42.6842105263158 -6.21960302805551
47.8947368421053 -6.9766742644194
53.1052631578947 -7.67401911707537
58.3157894736842 -8.32570333843011
63.5263157894737 -8.94199167115213
68.7368421052632 -9.53064422272666
73.9473684210526 -10.0976998250546
79.1578947368421 -10.6479659376129
84.3684210526316 -11.1853295717544
89.5789473684211 -11.7129510552654
94.7894736842105 -12.2333742214954
100 -12.748570688155
};
\addplot [semithick, forestgreen4416044]
table {%
1 71.1634251118613
6.21052631578947 55.0450389214607
11.4210526315789 25.7585703830658
16.6315789473684 16.889735725512
21.8421052631579 12.5909067041405
27.0526315789474 9.92311874281181
32.2631578947368 8.03960126303883
37.4736842105263 6.60022624857664
42.6842105263158 5.43993520109075
47.8947368421053 4.46804920252061
53.1052631578947 3.63015636025445
58.3157894736842 2.89136736774434
63.5263157894737 2.22811830155234
68.7368421052632 1.62381190660765
73.9473684210526 1.06633893261103
79.1578947368421 0.546594841239898
84.3684210526316 0.0575476370559134
89.5789473684211 -0.406370150348597
94.7894736842105 -0.849673060608588
100 -1.27611114651806
};
\addplot [semithick, forestgreen4416044, mark=x, mark size=3, mark options={solid}, only marks]
table {%
1 76.9948282753147
6.21052631578947 61.318893294471
11.4210526315789 30.3022660934073
16.6315789473684 20.1989427600002
21.8421052631579 15.268876015014
27.0526315789474 12.2030083848909
32.2631578947368 10.0398199071579
37.4736842105263 8.39059399478692
42.6842105263158 7.06583211333704
47.8947368421053 5.96112501049997
53.1052631578947 5.01373035060921
58.3157894736842 4.1834179624741
63.5263157894737 3.4430611212755
68.7368421052632 2.77361975297068
73.9473684210526 2.16128614560563
79.1578947368421 1.59577322356888
84.3684210526316 1.06924225592088
89.5789473684211 0.575605993199932
94.7894736842105 0.110061347678494
100 -0.331232620564567
};
\addplot [semithick, crimson2143940]
table {%
1 76.1996230019678
6.21052631578947 60.5539372038357
11.4210526315789 54.9470992016016
16.6315789473684 51.1680507994292
21.8421052631579 48.2365039154457
27.0526315789474 45.7936342851306
32.2631578947368 36.529896281145
37.4736842105263 30.0922743250855
42.6842105263158 25.7975627639123
47.8947368421053 22.68959317356
53.1052631578947 20.3091391418035
58.3157894736842 18.4075950254143
63.5263157894737 16.8382650041088
68.7368421052632 15.5087294851501
73.9473684210526 14.3576582407354
79.1578947368421 13.342576453581
84.3684210526316 12.4329745169549
89.5789473684211 11.6062186687175
94.7894736842105 10.8450098198444
100 10.1357353863686
};
\addplot [semithick, crimson2143940, mark=x, mark size=3, mark options={solid}, only marks]
table {%
1 81.9929809847755
6.21052631578947 66.5089404422558
11.4210526315789 61.1376729457515
16.6315789473684 57.7002548697588
21.8421052631579 55.1150135768525
27.0526315789474 53.0098304639514
32.2631578947368 45.0809867774759
37.4736842105263 37.32122172565
42.6842105263158 32.167810537288
47.8947368421053 28.4547754565906
53.1052631578947 25.6253977526914
58.3157894736842 23.3794352776546
63.5263157894737 21.5402461237533
68.7368421052632 19.9968739733874
73.9473684210526 18.6759658279702
79.1578947368421 17.5269953233548
84.3684210526316 16.5139662138722
89.5789473684211 15.6105014711468
94.7894736842105 14.7968058893651
100 14.0577164251039
};
\addplot [semithick, mediumpurple148103189]
table {%
1 80.7631984553152
6.21052631578947 65.3087194290777
11.4210526315789 59.9539120614097
16.6315789473684 56.5179436228241
21.8421052631579 53.9164380750021
27.0526315789474 51.7732445750001
32.2631578947368 49.9086884684741
37.4736842105263 48.2164933411128
42.6842105263158 46.6251730311479
47.8947368421053 45.1373330968571
53.1052631578947 43.735007827716
58.3157894736842 42.3972643723918
63.5263157894737 41.1078298070483
68.7368421052632 39.8535272154755
73.9473684210526 38.6232564183311
79.1578947368421 37.4072940950858
84.3684210526316 36.1967889845729
89.5789473684211 34.2964710250325
94.7894736842105 31.6651181425146
100 29.4003213869728
};
\addplot [semithick, mediumpurple148103189, mark=x, mark size=3, mark options={solid}, only marks]
table {%
1 86.5435237898226
6.21052631578947 71.1609823026524
11.4210526315789 65.8953233000991
16.6315789473684 62.5678663462205
21.8421052631579 60.0972015391196
27.0526315789474 58.1114944167459
32.2631578947368 56.437863382714
37.4736842105263 54.9817004956845
42.6842105263158 53.6855517607292
47.8947368421053 52.5118907007625
53.1052631578947 51.4348468530823
58.3157894736842 50.4358256634995
63.5263157894737 49.5010107148194
68.7368421052632 48.61985404547
73.9473684210526 47.7841197962726
79.1578947368421 46.987254445932
84.3684210526316 46.2239584090555
89.5789473684211 45.489886451945
94.7894736842105 44.45024820926
100 41.7005920627908
};

\nextgroupplot[
ylabel={Quality/\unit{\mole\per\mole}},
ymin=-0.05, ymax=1.05,
]
\addplot [semithick, black]
table {%
0 1.1
0 1.1
};
\addlegendentry{Coupled Model}
\addplot [semithick, black, mark=x, mark size=3, mark options={solid}, only marks]
table {%
0 1.1
0 1.1
};
\addlegendentry{HEOS mixture}
\addplot [semithick, steelblue31119180]
table {%
1 0.516011533585925
6.21052631578947 0.500887378406771
11.4210526315789 0.498326501377695
16.6315789473684 0.496581959361251
21.8421052631579 0.495131243204229
27.0526315789474 0.493854379011677
32.2631578947368 0.492708773965796
37.4736842105263 0.491675674574458
42.6842105263158 0.490746059229303
47.8947368421053 0.489916090628951
53.1052631578947 0.489185995275397
58.3157894736842 0.488561677405177
63.5263157894737 0.488066431622808
68.7368421052632 0.489071006697961
73.9473684210526 0.489033554808858
79.1578947368421 0.48899353286716
84.3684210526316 0.488951940168384
89.5789473684211 0.488909351930561
94.7894736842105 0.488866126134767
100 0.488822498196382
};
\addlegendentry{\qty{298}{\K}}
\addplot [semithick, steelblue31119180, mark=x, mark size=3, mark options={solid}, only marks, forget plot]
table {%
1 0.51590924677535
6.21052631578947 0.500546625557502
11.4210526315789 0.497804240572895
16.6315789473684 0.495967209820122
21.8421052631579 0.494490024298483
27.0526315789474 0.493234840328572
32.2631578947368 0.492145733641909
37.4736842105263 0.491193947294814
42.6842105263158 0.49036292731391
47.8947368421053 0.489643258901035
53.1052631578947 0.489031416314211
58.3157894736842 0.488532224268826
63.5263157894737 0.488179346400957
68.7368421052632 nan
73.9473684210526 nan
79.1578947368421 nan
84.3684210526316 nan
89.5789473684211 nan
94.7894736842105 0.488810289297036
100 0.488779824764678
};
\addplot [semithick, darkorange25512714]
table {%
1 0.98573719557845
6.21052631578947 0.54387985666053
11.4210526315789 0.522809657411654
16.6315789473684 0.515023987260721
21.8421052631579 0.510814189733477
27.0526315789474 0.508089615933051
32.2631578947368 0.506130940588933
37.4736842105263 0.504624137012261
42.6842105263158 0.50341023904244
47.8947368421053 0.502400141074518
53.1052631578947 0.501540026741673
58.3157894736842 0.500795491706181
63.5263157894737 0.500143543812563
68.7368421052632 0.499568270487665
73.9473684210526 0.499058351987659
79.1578947368421 0.498605563509695
84.3684210526316 0.498203833877555
89.5789473684211 0.497848628842174
94.7894736842105 0.497536526592628
100 0.497264904249163
};
\addlegendentry{\qty{354}{\K}}
\addplot [semithick, darkorange25512714, mark=x, mark size=3, mark options={solid}, only marks, forget plot]
table {%
1 0
6.21052631578947 0.544367581250852
11.4210526315789 0.523189358287048
16.6315789473684 0.515364580932782
21.8421052631579 0.511128095188173
27.0526315789474 0.508379564391254
32.2631578947368 0.506397059964113
37.4736842105263 0.504865802528504
42.6842105263158 0.503626688502333
47.8947368421053 0.502590696526588
53.1052631578947 0.501704164252288
58.3157894736842 0.500932843488841
63.5263157894737 0.500253866051834
68.7368421052632 0.499651391108772
73.9473684210526 0.499114104702003
79.1578947368421 0.498633711084365
84.3684210526316 0.498203982445991
89.5789473684211 0.497820135498961
94.7894736842105 0.497478404516778
100 0.497175733564423
};
\addplot [semithick, forestgreen4416044]
table {%
1 1
6.21052631578947 1
11.4210526315789 0.713754303226086
16.6315789473684 0.632325125451636
21.8421052631579 0.596506269221134
27.0526315789474 0.576284597456432
32.2631578947368 0.563248312464286
37.4736842105263 0.554116636660871
42.6842105263158 0.547345227584338
47.8947368421053 0.542111625912264
53.1052631578947 0.537937607216233
58.3157894736842 0.534526153627081
63.5263157894737 0.5316830587319
68.7368421052632 0.52927598850789
73.9473684210526 0.527211659036506
79.1578947368421 0.525422419393079
84.3684210526316 0.523858007735202
89.5789473684211 0.522480293320154
94.7894736842105 0.521259813785085
100 0.520173430309554
};
\addlegendentry{\qty{410}{\K}}
\addplot [semithick, forestgreen4416044, mark=x, mark size=3, mark options={solid}, only marks, forget plot]
table {%
1 0
6.21052631578947 0
11.4210526315789 0.719039173051982
16.6315789473684 0.634998813472266
21.8421052631579 0.598237154687791
27.0526315789474 0.577530797754696
32.2631578947368 0.564191947153812
37.4736842105263 0.55484549757368
42.6842105263158 0.547906793458681
47.8947368421053 0.54253349251601
53.1052631578947 0.538236534087756
58.3157894736842 0.534712544033811
63.5263157894737 0.531763371071014
68.7368421052632 0.52925412048285
73.9473684210526 0.527089783596756
79.1578947368421 0.525201510694682
84.3684210526316 0.523538183484306
89.5789473684211 0.522061044235608
94.7894736842105 0.520740161672579
100 0.519552040117685
};
\addplot [semithick, crimson2143940]
table {%
1 1
6.21052631578947 1
11.4210526315789 1
16.6315789473684 1
21.8421052631579 1
27.0526315789474 1
32.2631578947368 0.888256240048952
37.4736842105263 0.8098173951934
42.6842105263158 0.759054257008111
47.8947368421053 0.723512454412853
53.1052631578947 0.697238087842908
58.3157894736842 0.677026171652174
63.5263157894737 0.660998349808673
68.7368421052632 0.647980866322711
73.9473684210526 0.637202515227218
79.1578947368421 0.628135698426651
84.3684210526316 0.620407276764068
89.5789473684211 0.61374589741645
94.7894736842105 0.607949496738782
100 0.602864503441605
};
\addlegendentry{\qty{466}{\K}}
\addplot [semithick, crimson2143940, mark=x, mark size=3, mark options={solid}, only marks, forget plot]
table {%
1 0
6.21052631578947 0
11.4210526315789 0
16.6315789473684 0
21.8421052631579 0
27.0526315789474 0
32.2631578947368 0.915978877253726
37.4736842105263 0.828124025827601
42.6842105263158 0.771848130999485
47.8947368421053 0.732708878421184
53.1052631578947 0.703898314214745
58.3157894736842 0.681790744827807
63.5263157894737 0.664279260310786
68.7368421052632 0.650056401707074
73.9473684210526 0.638268068689694
79.1578947368421 0.628332803952416
84.3684210526316 0.619841097538277
89.5789473684211 0.612496229539406
94.7894736842105 0.606077952804862
100 0.600419352682368
};
\addplot [semithick, mediumpurple148103189]
table {%
1 1
6.21052631578947 1
11.4210526315789 1
16.6315789473684 1
21.8421052631579 1
27.0526315789474 1
32.2631578947368 1
37.4736842105263 1
42.6842105263158 1
47.8947368421053 1
53.1052631578947 1
58.3157894736842 1
63.5263157894737 1
68.7368421052632 1
73.9473684210526 1
79.1578947368421 1
84.3684210526316 1
89.5789473684211 0.982973748849817
94.7894736842105 0.945405720064505
100 0.914193923886455
};
\addlegendentry{\qty{523}{\K}}
\addplot [semithick, mediumpurple148103189, mark=x, mark size=3, mark options={solid}, only marks, forget plot]
table {%
1 0
6.21052631578947 0
11.4210526315789 0
16.6315789473684 0
21.8421052631579 0
27.0526315789474 0
32.2631578947368 0
37.4736842105263 0
42.6842105263158 0
47.8947368421053 0
53.1052631578947 0
58.3157894736842 0
63.5263157894737 0
68.7368421052632 0
73.9473684210526 0
79.1578947368421 0
84.3684210526316 0
89.5789473684211 0
94.7894736842105 0.993776216893024
100 0.954336140002162
};
\end{groupplot}

\end{tikzpicture}

        \caption{The molar density/volume/enthalpy/entropy/vapour quality for a carbon dioxide mole fraction of \num{0.5} as a function of temperature and pressure, calculated using the \ac{HEOS} mixture in \emph{CoolProp} and the coupled model}
        \label{fig:SemiEmpirical_properties_maintext}
    \end{figure}

    \begin{figure}[H]
        \centering
        % This file was created with tikzplotlib v0.10.1.
\begin{tikzpicture}

\definecolor{crimson2143940}{RGB}{214,39,40}
\definecolor{darkgray176}{RGB}{176,176,176}
\definecolor{darkorange25512714}{RGB}{255,127,14}
\definecolor{forestgreen4416044}{RGB}{44,160,44}
\definecolor{lightgray204}{RGB}{204,204,204}
\definecolor{mediumpurple148103189}{RGB}{148,103,189}
\definecolor{steelblue31119180}{RGB}{31,119,180}

\begin{groupplot}[
    group style={
        group size=2 by 3,
        vertical sep=2cm,
        horizontal sep=2.5cm
        },
    width=7cm,
    height=6.5cm,
    legend style={
        at={(1.55,0.5)},
        anchor=west
        },
    legend cell align={left},
    unbounded coords=jump,
    xlabel={Pressure/\unit{\bar}},
    xmin=0, xmax=100,
    ymin=0.75, ymax=1.25,
    ylabel near ticks,
    xlabel near ticks
    ]
\nextgroupplot[
ylabel={\(\frac{\rho^{HEOS Mix}}{\rho^{Couples}}\)},
]
\addplot [semithick, steelblue31119180]
table {%
1 1.00036005809639
6.21052631578947 0.999610857889401
11.4210526315789 0.999172295352815
16.6315789473684 0.998935299618143
21.8421052631579 0.998833044417282
27.0526315789474 0.99881903785662
32.2631578947368 0.998856243871675
37.4736842105263 0.998911852501387
42.6842105263158 0.998950476963512
47.8947368421053 0.998920570287153
53.1052631578947 0.99871856021527
58.3157894736842 0.998014463057239
63.5263157894737 0.994071718252201
68.7368421052632 nan
73.9473684210526 nan
79.1578947368421 nan
84.3684210526316 nan
89.5789473684211 nan
94.7894736842105 0.996987866743343
100 0.997347109873134
};
\addplot [semithick, darkorange25512714]
table {%
1 1.01823834452604
6.21052631578947 1.00537226988396
11.4210526315789 1.00486224113648
16.6315789473684 1.00454474406537
21.8421052631579 1.00429417680388
27.0526315789474 1.00407561201146
32.2631578947368 1.00387254423957
37.4736842105263 1.00367535388937
42.6842105263158 1.00347753047103
47.8947368421053 1.00327339924222
53.1052631578947 1.00305808685009
58.3157894736842 1.00282677878352
63.5263157894737 1.00257433814389
68.7368421052632 1.00229558677098
73.9473684210526 1.00198461607899
79.1578947368421 1.00163521883823
84.3684210526316 1.00124074220884
89.5789473684211 1.00079465568393
94.7894736842105 1.00029035232155
100 0.999722404111153
};
\addplot [semithick, forestgreen4416044]
table {%
1 1.00161839545096
6.21052631578947 1.01357928766159
11.4210526315789 1.02603350054266
16.6315789473684 1.02431772412357
21.8421052631579 1.02365307569989
27.0526315789474 1.02329356993019
32.2631578947368 1.02306061850625
37.4736842105263 1.02288924197073
42.6842105263158 1.02274852651158
47.8947368421053 1.0226204505405
53.1052631578947 1.0224923217361
58.3157894736842 1.02235448261615
63.5263157894737 1.02219870742806
68.7368421052632 1.02201746971502
73.9473684210526 1.02180373313984
79.1578947368421 1.02155061583999
84.3684210526316 1.02125140760521
89.5789473684211 1.02089944636757
94.7894736842105 1.02048806584246
100 1.02001089540276
};
\addplot [semithick, crimson2143940]
table {%
1 1.00081630062901
6.21052631578947 1.00600364252559
11.4210526315789 1.01317811567727
16.6315789473684 1.02313872329157
21.8421052631579 1.03392671155041
27.0526315789474 1.0449995099499
32.2631578947368 1.08129162337789
37.4736842105263 1.07787417159089
42.6842105263158 1.07626756663233
47.8947368421053 1.0755986314974
53.1052631578947 1.07545752347768
58.3157894736842 1.07562675080352
63.5263157894737 1.07597921135715
68.7368421052632 1.07643334805088
73.9473684210526 1.0769318127722
79.1578947368421 1.07743093890708
84.3684210526316 1.07789440432972
89.5789473684211 1.07828971245992
94.7894736842105 1.07858660087029
100 1.07875515468995
};
\addplot [semithick, mediumpurple148103189]
table {%
1 1.00049811541093
6.21052631578947 1.00335102357024
11.4210526315789 1.00670021386239
16.6315789473684 1.01062081790561
21.8421052631579 1.01521118310484
27.0526315789474 1.0206080195554
32.2631578947368 1.0270125280855
37.4736842105263 1.03474722429952
42.6842105263158 1.04429762623975
47.8947368421053 1.05474111318356
53.1052631578947 1.06577774205073
58.3157894736842 1.07741964619668
63.5263157894737 1.08968207630774
68.7368421052632 1.10258028664265
73.9473684210526 1.11612841656714
79.1578947368421 1.1303363107232
84.3684210526316 1.14520776852512
89.5789473684211 1.17334980351738
94.7894736842105 1.21109518423553
100 1.21328135290121
};

\nextgroupplot[
ylabel={\(\frac{v^{HEOS Mix}}{v^{Couples}}\)},
]
\addplot [semithick, steelblue31119180]
table {%
1 0.999640071497035
6.21052631578947 1.00038929358185
11.4210526315789 1.00082839022971
16.6315789473684 1.00106583502159
21.8421052631579 1.00116831872762
27.0526315789474 1.00118235816349
32.2631578947368 1.00114506544392
37.4736842105263 1.00108933243817
42.6842105263158 1.00105062521408
47.8947368421053 1.00108059555709
53.1052631578947 1.00128308315769
58.3157894736842 1.0019894856103
63.5263157894737 1.00596363000687
68.7368421052632 nan
73.9473684210526 nan
79.1578947368421 nan
84.3684210526316 nan
89.5789473684211 nan
94.7894736842105 1.00302122518226
100 1.00265993919044
};
\addplot [semithick, darkorange25512714]
table {%
1 0.982088334145213
6.21052631578947 0.994656437374543
11.4210526315789 0.995161286220968
16.6315789473684 0.995475817699702
21.8421052631579 0.995724184955872
27.0526315789474 0.995940931959207
32.2631578947368 0.996142395409746
37.4736842105263 0.99633810588233
42.6842105263158 0.996534521947073
47.8947368421053 0.996737282131219
53.1052631578947 0.996951237789039
58.3157894736842 0.997181190667808
63.5263157894737 0.997432273366423
68.7368421052632 0.997709672166055
73.9473684210526 0.998019316041288
79.1578947368421 0.998367451834804
84.3684210526316 0.99876079623036
89.5789473684211 0.999205975920948
94.7894736842105 0.999709732206665
100 1.00027767271392
};
\addplot [semithick, forestgreen4416044]
table {%
1 0.99838421947071
6.21052631578947 0.986602639136874
11.4210526315789 0.974627047425231
16.6315789473684 0.976259590503085
21.8421052631579 0.976893467470068
27.0526315789474 0.977236672552206
32.2631578947368 0.977459190546823
37.4736842105263 0.977622956370302
42.6842105263158 0.977757463919722
47.8947368421053 0.977879921841455
53.1052631578947 0.978002460976432
58.3157894736842 0.978134321048245
63.5263157894737 0.978283381878558
68.7368421052632 0.978456864785047
73.9473684210526 0.978661534931153
79.1578947368421 0.978904025940439
84.3684210526316 0.979190827784086
89.5789473684211 0.979528410309519
94.7894736842105 0.979923279655952
100 0.98038169721074
};
\addplot [semithick, crimson2143940]
table {%
1 0.999184365144759
6.21052631578947 0.99403218615613
11.4210526315789 0.986993288625418
16.6315789473684 0.97738456987034
21.8421052631579 0.967186543097184
27.0526315789474 0.956938250954203
32.2631578947368 0.924819897087617
37.4736842105263 0.927752084663105
42.6842105263158 0.929136994030881
47.8947368421053 0.929714843784574
53.1052631578947 0.929836831239942
58.3157894736842 0.92969054293094
63.5263157894737 0.929386004525794
68.7368421052632 0.928993907813627
73.9473684210526 0.928563919282347
79.1578947368421 0.928133758885135
84.3684210526316 0.9277346887356
89.5789473684211 0.927394577326084
94.7894736842105 0.92713930780685
100 0.926994445937231
};
\addplot [semithick, mediumpurple148103189]
table {%
1 0.999502132564039
6.21052631578947 0.996660168309596
11.4210526315789 0.993344380366648
16.6315789473684 0.989490798796508
21.8421052631579 0.985016730906041
27.0526315789474 0.979808096884444
32.2631578947368 0.973697958619506
37.4736842105263 0.966419604377753
42.6842105263158 0.957581420740249
47.8947368421053 0.948099959281189
53.1052631578947 0.938281941198521
58.3157894736842 0.928143472122456
63.5263157894737 0.917698874021336
68.7368421052632 0.906963447648517
73.9473684210526 0.895954271385549
79.1578947368421 0.884692473817303
84.3684210526316 0.873204024488067
89.5789473684211 0.852260799306887
94.7894736842105 0.82569897656143
100 0.824211183659851
};

\nextgroupplot[
ylabel={\(\frac{h^{HEOS Mix}}{h^{Couples}}\)},
]
\addplot [semithick, steelblue31119180]
table {%
1 1.00470378920154
6.21052631578947 -0.174631661577768
11.4210526315789 0.64113420135861
16.6315789473684 0.708726165604493
21.8421052631579 0.741716115130044
27.0526315789474 0.765438413313219
32.2631578947368 0.78526180174156
37.4736842105263 0.803048913788412
42.6842105263158 0.819688993037906
47.8947368421053 0.835756036502689
53.1052631578947 0.851775216979501
58.3157894736842 0.86846422691317
63.5263157894737 0.886859150513066
68.7368421052632 nan
73.9473684210526 nan
79.1578947368421 nan
84.3684210526316 nan
89.5789473684211 nan
94.7894736842105 0.948809133241624
100 0.949608313967276
};
\addplot [semithick, darkorange25512714]
table {%
1 0.972653955753762
6.21052631578947 0.988838730980453
11.4210526315789 0.991144304602367
16.6315789473684 0.994024444964714
21.8421052631579 0.997080790814403
27.0526315789474 1.00018868033764
32.2631578947368 1.0033023953856
37.4736842105263 1.00640420660478
42.6842105263158 1.00948798128223
47.8947368421053 1.01255280603505
53.1052631578947 1.01560061686073
58.3157894736842 1.01863457596821
63.5263157894737 1.02165849969566
68.7368421052632 1.02467627448145
73.9473684210526 1.02769077678828
79.1578947368421 1.03070295449617
84.3684210526316 1.03370948128084
89.5789473684211 1.03669980337925
94.7894736842105 1.03964957120628
100 1.04251020982653
};
\addplot [semithick, forestgreen4416044]
table {%
1 0.998429245686501
6.21052631578947 0.989553856415125
11.4210526315789 0.969533709789868
16.6315789473684 0.971200354665764
21.8421052631579 0.971664020360276
27.0526315789474 0.971803445660077
32.2631578947368 0.971781481681541
37.4736842105263 0.971628707843298
42.6842105263158 0.971342886019243
47.8947368421053 0.970911667019036
53.1052631578947 0.970318674676437
58.3157894736842 0.969544893637668
63.5263157894737 0.968568734662618
68.7368421052632 0.967365726791688
73.9473684210526 0.965907765097585
79.1578947368421 0.96416248897646
84.3684210526316 0.962092339540876
89.5789473684211 0.959653488099934
94.7894736842105 0.956794544424468
100 0.953454897121844
};
\addplot [semithick, crimson2143940]
table {%
1 0.999132966493055
6.21052631578947 0.995812518061271
11.4210526315789 0.991105569598805
16.6315789473684 0.984317953249938
21.8421052631579 0.976819566986601
27.0526315789474 0.96896801415839
32.2631578947368 0.924942732149922
37.4736842105263 0.929292547317655
42.6842105263158 0.931389380851869
47.8947368421053 0.932117757097683
53.1052631578947 0.931909791477311
58.3157894736842 0.930992130422875
63.5263157894737 0.929486600983309
68.7368421052632 0.927456291193634
73.9473684210526 0.924928238968041
79.1578947368421 0.921905308018315
84.3684210526316 0.918372335153207
89.5789473684211 0.914299293178066
94.7894736842105 0.90964257855742
100 0.904344834231328
};
\addplot [semithick, mediumpurple148103189]
table {%
1 0.999406021620674
6.21052631578947 0.99778344689156
11.4210526315789 0.995820052742888
16.6315789473684 0.993471042270072
21.8421052631579 0.990676277938581
27.0526315789474 0.987350016049794
32.2631578947368 0.983360015186377
37.4736842105263 0.978483178757176
42.6842105263158 0.972425127315313
47.8947368421053 0.965784658886185
53.1052631578947 0.958698190200729
58.3157894736842 0.951126714103211
63.5263157894737 0.943022426588734
68.7368421052632 0.93432879315194
73.9473684210526 0.924979884638677
79.1578947368421 0.914899190240469
84.3684210526316 0.90399801719506
89.5789473684211 0.878031436327521
94.7894736842105 0.840865302533599
100 0.837424331750267
};

\nextgroupplot[
ylabel={\(\frac{s^{HEOS Mix}}{s^{Couples}}\)},
]
\addplot [semithick, steelblue31119180]
table {%
1 0.775793737789358
6.21052631578947 0.994577374760406
11.4210526315789 0.977345021475906
16.6315789473684 0.968724798304194
21.8421052631579 0.962868000314163
27.0526315789474 0.958544701628392
32.2631578947368 0.955284157303321
37.4736842105263 0.952854179940779
42.6842105263158 0.951128716299489
47.8947368421053 0.95004411417282
53.1052631578947 0.949586379707275
58.3157894736842 0.949790954927527
63.5263157894737 0.950412689260209
68.7368421052632 nan
73.9473684210526 nan
79.1578947368421 nan
84.3684210526316 nan
89.5789473684211 nan
94.7894736842105 0.965555285280353
100 0.965770286304572
};
\addplot [semithick, darkorange25512714]
table {%
1 0.895545818271172
6.21052631578947 0.823565127674915
11.4210526315789 0.662095397135782
16.6315789473684 3.91267604463821
21.8421052631579 1.21431122950296
27.0526315789474 1.09267188351959
32.2631578947368 1.04892120553879
37.4736842105263 1.02608970208092
42.6842105263158 1.0119509662578
47.8947368421053 1.00230332541276
53.1052631578947 0.995311704173963
58.3157894736842 0.99004527743304
63.5263157894737 0.98598010908857
68.7368421052632 0.982797680504435
73.9473684210526 0.980292631155325
79.1578947368421 0.978326149927716
84.3684210526316 0.976801031149394
89.5789473684211 0.975647244765699
94.7894736842105 0.974813735331322
100 0.974263393232907
};
\addplot [semithick, forestgreen4416044]
table {%
1 0.924262405432425
6.21052631578947 0.897684809140859
11.4210526315789 0.850054265037812
16.6315789473684 0.836169303018639
21.8421052631579 0.82461255610087
27.0526315789474 0.81316988538074
32.2631578947368 0.800771481621451
37.4736842105263 0.786622075392432
42.6842105263158 0.769893105856499
47.8947368421053 0.749531245380579
53.1052631578947 0.724043054244093
58.3157894736842 0.691149606023291
63.5263157894737 0.647133052809087
68.7368421052632 0.585448787426497
73.9473684210526 0.493381888093462
79.1578947368421 0.342527052490833
84.3684210526316 0.053821837769133
89.5789473684211 -0.705983692257623
94.7894736842105 -7.71991583434679
100 3.85262175254585
};
\addplot [semithick, crimson2143940]
table {%
1 0.929343245695303
6.21052631578947 0.910463118051724
11.4210526315789 0.898743714887764
16.6315789473684 0.886790726109185
21.8421052631579 0.875197172418137
27.0526315789474 0.863870607026396
32.2631578947368 0.810317144368267
37.4736842105263 0.806304647259152
42.6842105263158 0.801968257430522
47.8947368421053 0.797391404855149
53.1052631578947 0.792539477641123
58.3157894736842 0.787341311689697
63.5263157894737 0.781711830294675
68.7368421052632 0.775557706181074
73.9473684210526 0.768777293994362
79.1578947368421 0.761258643946991
84.3684210526316 0.75287635950439
89.5789473684211 0.74348789798211
94.7894736842105 0.732929131327586
100 0.721008687247364
};
\addplot [semithick, mediumpurple148103189]
table {%
1 0.933209037319136
6.21052631578947 0.91776023036831
11.4210526315789 0.909835617294619
16.6315789473684 0.903306234014354
21.8421052631579 0.897153890647503
27.0526315789474 0.890929500328788
32.2631578947368 0.884312154868857
37.4736842105263 0.876955296570756
42.6842105263158 0.868486429467373
47.8947368421053 0.859564046065431
53.1052631578947 0.850299177566153
58.3157894736842 0.840618032401058
63.5263157894737 0.830444255238153
68.7368421052632 0.819696565903043
73.9473684210526 0.808286451120472
79.1578947368421 0.796115770969645
84.3684210526316 0.783074198907369
89.5789473684211 0.753936181118561
94.7894736842105 0.712372139770098
100 0.705033675245414
};

\nextgroupplot[
ylabel={\(\frac{q^{HEOS Mix}}{q^{Couples}}\)},
]
\addplot [semithick, black]
table {%
0 2.1
0 2.1
};
\addlegendentry{Coupled Model}
\addplot [semithick, black, mark=x, mark size=3, mark options={solid}, only marks]
table {%
0 2.1
0 2.1
};
\addlegendentry{HEOS mixture}
\addplot [semithick, steelblue31119180]
table {%
1 1.00019826473929
6.21052631578947 1.00068076009529
11.4210526315789 1.00104912677134
16.6315789473684 1.00123949384045
21.8421052631579 1.00129672506524
27.0526315789474 1.00125606988067
32.2631578947368 1.00114404970184
37.4736842105263 1.00098072523404
42.6842105263158 1.00078132157374
47.8947368421053 1.00055720397606
53.1052631578947 1.00031609144103
58.3157894736842 1.00006028891156
63.5263157894737 0.999768702736641
68.7368421052632 nan
73.9473684210526 nan
79.1578947368421 nan
84.3684210526316 nan
89.5789473684211 nan
94.7894736842105 1.00011422984484
100 1.00008730586296
};
\addlegendentry{\qty{298}{\K}}
\addplot [semithick, darkorange25512714]
table {%
1 nan
6.21052631578947 0.999104054482349
11.4210526315789 0.999274258652937
16.6315789473684 0.999339122198568
21.8421052631579 0.999385858763174
27.0526315789474 0.999429662582498
32.2631578947368 0.999474485792465
37.4736842105263 0.999521328170057
42.6842105263158 0.999570219301212
47.8947368421053 0.999620854356754
53.1052631578947 0.999672840699458
58.3157894736842 0.999725808538503
63.5263157894737 0.999779467933733
68.7368421052632 0.999833643103514
73.9473684210526 0.999888296880182
79.1578947368421 0.999943550711484
84.3684210526316 0.999999701792552
89.5789473684211 1.00005723610587
94.7894736842105 1.00011683312982
100 1.00017935409829
};
\addlegendentry{\qty{354}{\K}}
\addplot [semithick, forestgreen4416044]
table {%
1 nan
6.21052631578947 nan
11.4210526315789 0.992650104926037
16.6315789473684 0.995789466447226
21.8421052631579 0.997106694962376
27.0526315789474 0.997842196216539
32.2631578947368 0.998327461032106
37.4736842105263 0.998686373770049
42.6842105263158 0.998975072300399
47.8947368421053 0.999222415146749
53.1052631578947 0.999444619052757
58.3157894736842 0.999651420075647
63.5263157894737 0.999848970063654
68.7368421052632 1.00004131849498
73.9473684210526 1.00023122286244
79.1578947368421 1.00042061622688
84.3684210526316 1.00061088885223
89.5789473684211 1.0008030637147
94.7894736842105 1.0009979086555
100 1.00119600915381
};
\addlegendentry{\qty{410}{\K}}
\addplot [semithick, crimson2143940]
table {%
1 nan
6.21052631578947 nan
11.4210526315789 nan
16.6315789473684 nan
21.8421052631579 nan
27.0526315789474 nan
32.2631578947368 0.969734447346276
37.4736842105263 0.977893880678336
42.6842105263158 0.98342438505462
47.8947368421053 0.987448751164448
53.1052631578947 0.990538097939096
58.3157894736842 0.993011688375845
63.5263157894737 0.995060954392086
68.7368421052632 0.996807150600988
73.9473684210526 0.998330557573997
79.1578947368421 0.999686304437983
84.3684210526316 1.00091342493199
89.5789473684211 1.00204028331358
94.7894736842105 1.00308795401203
100 1.00407239819289
};
\addlegendentry{\qty{466}{\K}}
\addplot [semithick, mediumpurple148103189]
table {%
1 nan
6.21052631578947 nan
11.4210526315789 nan
16.6315789473684 nan
21.8421052631579 nan
27.0526315789474 nan
32.2631578947368 nan
37.4736842105263 nan
42.6842105263158 nan
47.8947368421053 nan
53.1052631578947 nan
58.3157894736842 nan
63.5263157894737 nan
68.7368421052632 nan
73.9473684210526 nan
79.1578947368421 nan
84.3684210526316 nan
89.5789473684211 nan
94.7894736842105 0.951326619280178
100 0.957937070210203
};
\addlegendentry{\qty{523}{\K}}
\end{groupplot}

\end{tikzpicture}

        \caption{The ratio of molar density/volume/enthalpy/entropy/vapour quality, as a function of temperature and pressure for a mole fraction of carbon dioxide of \num{0.5}, as calculated using the \ac{HEOS} mixture in \emph{CoolProp} and the coupled model}
        \label{fig:SemiEmpirical_ratios_maintext}
    \end{figure}

\subsection{Performance}
\label{sec:SemiEmpirical_performance}
    The performance of the coupled model, the \ac{HEOS} mixture as well as the pure component \ac{HEOS} was compared for randomised values of pressure, temperature and composition. The temperatures were obtained by linear sampling of temperatures between \qty{298}{\K} and \qty{573}{\K}.The pressures were obtained by linear sampling \(\log P\) for pressures between \qty{1}{\bar} to \qty{100}{\bar}. The compositions were obtained by linear sampling the standard deviation for values between \num{-3.5} and \num{+3.5} of the Normal distribution, to ensure more representative sampling of the tails (i.e. quasi pure compositions). The results are provided in Table~\ref{table:SemiEmpirical_Performance}

    \begin{table}[H]
        \caption{The computational performance over 10000 randomised calculations}
        \centering 
        \label{table:SemiEmpirical_Performance}
        \begin{tabular}{|p{7em} | c | c | c |}
    \hline
    \rowcolor{bluepoli!40}
    \textbf{Model} & \textbf{Runtime}/\unit{\s} & \textbf{Speed-up} & \textbf{Failure Rate}/\unit{\percent} \T\B \\
    \hline \hline 
    \ac{HEOS} mixture & 252.97 & 1 & 5.69 \T\B\\
    \ac{WP} \ac{EOS}  & 0.34 & 744 & 0 \T\B\\
    \ac{SW} \ac{EOS} & 0.27 & 937 & 0 \T\B\\
    Coupled Model & 14.28 & 17 & 0 \T\B\\
    \hline
\end{tabular}    
        \\[10pt]
    \end{table}



\subsection{Conclusions}
\label{sec:tppm_semi_conclusions}
    The approach of tightly coupling the \ac{WP} \ac{EOS} and \ac{SW} \ac{EOS} with the \ac{SP2009} model has yielded a useful model for evaluating the properties of water and carbon dioxide for a wide range of temperatures, pressures and compositions. The calculated properties are similar to the values obtained from a \ac{HEOS} mixture of water and carbon dioxide in \emph{CoolProp}, while being computationally cheaper, speed-ups of \num{17} times have been observed, and stable, converging for all conditions tested. The performance could be further improved by implementing the model in C++ to streamline the evaluations and reduce computational overheads. Moreover, extending the \ac{SP2009} to sub-atmospheric conditions (i.e. below pressures of \qty{1}{bar} and temperatures below \qty{298}{\K}) would allow this model to be used geothermal power plant simulations, particularly direct steam cycles. 