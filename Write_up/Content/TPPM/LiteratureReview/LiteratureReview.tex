% The following section aims to provide an overview of the make-up of geothermal geofluids and the various techniques employed for the modelling of thermophysical properties of pure fluids and mixtures.

\section{Geofluids}
\label{sec:pure_fluids}
    Geofluids typically have three main constituents: water, minerals and gases\cite{DiPippo2016}. The compositions of minerals depends not only on the chemical composition of the reservoir rock, but also the composition the waters migrating through the rock and the left over waters from the original deposition of the reservoir rock. \ce{Na}, \ce{Ca}, \ce{K}, \ce{Fe}, \ce{Cl}, \ce{SO4} and \ce{HCO3} are common minerals found in geothermal geofluids. For the gases, \ce{CO2}, \ce{CH4} and \ce{H2S} are common constituents. \ce{CO2} is either produced as part of the dissolution of the reservoir rock (e.g. in case of a carbonate reservoir) or, similarly to \ce{H2S}, migrates into the formation from deeper within the Earth (e.g. magmatic origin). \ce{CH4} is typically created deeper within the Earth in hydrocarbon source rock and then migrates into the reservoir. \citeauthor{kottsova_2022} \cite{kottsova_2022} and \citeauthor{REFLECT_2022} \cite{REFLECT_2022} have collated geofluid compositions from a large number of geothermal wells across Europe. 
    
\section{Equations of State}
\label{sec:equations_of_state}
    In thermodynamics, substances can exist in different \emph{states}, where each \emph{state} is uniquely defined by \emph{state variables}, such as the temperature, pressure, and volume/density of the substance. In this context, an \ac{EOS} is any mathematical expression that correlates the state variables of a substance, allowing the all thermodynamic properties for any given state to be determined. The following sections aim to provide background on the most common \acp{EOS}.
    
    \subsection{Ideal Gas Law}
        The simplest \ac{EOS} is the \emph{Ideal Gas Law}, Equation~\ref{eq:ideal_gas_law}, which links the pressure, molar volume and temperature, and describes the behaviour of a gases, assuming non-interacting, spherical particles of zero volume. 
        
        \begin{align}
            P V_m = R T  \label{eq:ideal_gas_law}
        \end{align}

    \subsection{Real Gas Law}
        Over time, as experimental techniques and equipment improved, temperature and pressure dependent deviations from the ideal gas law were observed. To capture these deviations the \ac{Z} was introduced, leading to the \emph{Real Gas Law}, Equation~\ref{eq:real_gas_law}. Standing and Katz provided graphical methods for evaluating \ac{Z} from the reduced temperature and pressure, based on the law of corresponding states.
    
        \begin{align}
            P V_m = Z R T  \label{eq:real_gas_law}
        \end{align}

    \subsection{Cubic Equations of State}
        \citeauthor{Waals1873} was the first to present a cubic \ac{EOS} based on the the insight that particles did indeed have an irreducible volume and that there are short-ranged attractive interactions between particles. This is now known as the \ac{VDW} \ac{EOS}, Equation~\ref{eq:van_der_Waals_EOS}. One significant advance of the \ac{VDW} \ac{EOS} is that its domain is not restricted to just gases, but also allows the behaviour of liquids to be approximated - though it is considered non-predictive and typically underestimates liquid densities.
    
        \begin{align}
            P = \frac{R T}{V_m - b} + \frac{a}{V_m^2}  \label{eq:van_der_Waals_EOS}
        \end{align}

        To improve on the accuracy of liquid phase predictions modifications to the original \ac{VDW} \ac{EOS} were suggested. These modifications have primarily focused on the definition of the attraction term, for example \citeauthor{Redlich1949} modified the attraction term to include a temperature dependence and the irreducible volume, Equation~\ref{eq:RK_EOS}, \citeauthor{Soave1972} further modified the \ac{EOS} proposed by \citeauthor{Redlich1949} by including a temperature dependent \(\alpha\) term, Equation~\ref{eq:SRK_EOS} and \citeauthor{Peng1976} changed the definition of the denominator in the attraction term, Equation~\ref{eq:PR_EOS}. These \ac{EOS} are commonly referred to as the \ac{RK} \ac{EOS}, \ac{SRK} \ac{EOS} and \ac{PR} \ac{EOS}, respectively.
    
        \begin{align}
            P = \frac{R T}{V_m - b} + \frac{a}{\sqrt{T}V_m(V_m-b)}  \label{eq:RK_EOS}
        \end{align}
        \begin{align}
            P = \frac{R T}{V_m - b} + \frac{a\alpha (T)}{V_m(V_m-b)}  \label{eq:SRK_EOS}
        \end{align}
        \begin{align}
            P = \frac{R T}{V_m - b} + \frac{a\alpha (T)}{V_m^2 + 2V_mb-b^2}  \label{eq:PR_EOS}
        \end{align}

        Cubic \ac{EOS} above can also be used to model mixtures and have been successfully applied in the petroleum sector for many decades. When applied to mixtures, mixing rules are applied to determine the overall values of \(a\), see Equation~\ref{eq:mixing_rule_a}, and \(b\), see Equation~\ref{eq:mixing_rule_b}. Here, \ac{BIC}, \(k_{ij}\), are used to account for differences in the interactions between different species, and are usually obtain by tuning against experimental data of the mixture's phase behaviour.   

        \begin{align}
            a = \sum_{i=0}^N \sum_{j=0}^N x_i x_j \sqrt{a_i a_j} (1-k_{ij})  \label{eq:mixing_rule_a}
        \end{align}
        \begin{align}
            b = \sum_{i=0}^N x_i b_i  \label{eq:mixing_rule_b}
        \end{align}

    \subsection{Excess Property Formulations}
    \label{sec:excess_properties}
        \citeauthor{Peneloux1989} \cite{Peneloux1989} first investigated the use of \emph{Excess Property} formulations, for example in terms of the Helmholtz free energy, as single mathematical formulation to obtain all thermodynamic properties. Here, the \emph{excess} refers to the difference between the actual thermodynamic property or potential at a given state and that of an ideal gas. The equations below shows the \ac{HEOS} for Nitrogen \cite{Span2000}, where \(\delta=\frac{\rho}{\rho_c}\) and \(\tau =\frac{T_c}{T}\) with \(\rho_c\) and \(T_c\) being the critical density and temperature respectively; and \(a_1\) to \(a_7\),the \(N_k\)s, the \(i_k\)s, the \(j_k\)s, the \(l_k\)s, the \(\phi_k\)s, the \(beta_k\)s and the \(\gamma_k\)s as fluid specific correlation parameters - a total of \emph{\num{154}} parameters in the case of Nitrogen.

        \begin{align}
            \frac{a (\delta, \tau)}{R T} = \alpha^0 (\delta, \tau) + \alpha^r (\delta, \tau) \label{eq:helmholtz_EOS}
        \end{align}
        \begin{align}
            \alpha^0 (\delta, \tau) = \ln \delta + a_1 \ln \tau + a_2 + a_3 \tau + \frac{a_4}{\tau} + \frac{a_5}{\tau^2} + \frac{a_6}{\tau^3} + a_7 \ln[1-e^{-a_8\tau}]  \label{eq:HEOS_ig_term}
        \end{align}
        \begin{align}
            \alpha^r (\delta, \tau) = \sum_{k=1}^{6} N_k \delta^{i_k} \tau^{j_k} + \sum_{k=7}^{32} N_k \delta^{i_k} \tau^{j_k} e^{(-\delta^{l_k})} + \sum_{k=33}^{36} N_k \delta^{i_k} \tau^{j_k} e^{-\phi_k(\delta-1)^2-\beta_k(\tau-\gamma_k)^2} \label{eq:HEOS_residual}
        \end{align}

        As a result of their complexity, these \ac{HEOS} are typically only created for fluids where large experimental datasets over a wide range of thermodynamic conditions exist \cite{Lemmon1999}. Given the high-accuracy of \ac{HEOS}, limited only by the experimental uncertainty \cite{Lemmon1999},  they are primarily developed as reference \ac{EOS}s for specific common-use fluids to calculate thermodynamic property charts and look-up tables for scientific and industrial use.

        One such reference \ac{EOS} is the \ac{HEOS} developed by \citeauthor{Wagner2002} for the properties and phase behaviour of water, better known as the IAPWS95 \cite{IAPWS2018} formulation or the \ac{WP} \ac{EOS}. This original model was developed further by \citeauthor{Wagner2000} leading to the IAPWS97 \cite{IAPWS2018} formulation, which primarily improves the speed and is more commonly used in industry than scientific use. Similar reference \ac{EOS} have also been developed for other common geofluid components, such as carbon dioxide \cite{Span1996} (i.e. the \ac{SW} \ac{EOS}), nitrogen \cite{Span2000}, methane \cite{Setzmann1991} and hydrogen sulphide \cite{Lemmon2006}.

        \begin{figure}[H]
            \centering
            \begin{tikzpicture}[baseline]
    \begin{axis}[xlabel = {Temperature/\unit{\K}},
                 ylabel = {Pressure/\unit{\bar}},
                 legend style={at={(0.97, 0.03)}, anchor=south east},
                 ymode = log,
                 ymin = 0.1,
                 xmin = 250,
                 height=7cm,
                 width=8cm]
        \addplot[color=blue]
            coordinates {(273.16,0.006) (282.748,0.012) (292.336,0.022) (301.924,0.04) (311.512,0.068) (321.1,0.111) (330.688,0.178) (340.277,0.275) (349.865,0.414) (359.453,0.609) (369.041,0.874) (378.629,1.229) (388.217,1.696) (397.805,2.298) (407.393,3.064) (416.981,4.025) (426.569,5.214) (436.157,6.67) (445.745,8.431) (455.333,10.542) (464.922,13.047) (474.51,15.996) (484.098,19.441) (493.686,23.434) (503.274,28.034) (512.862,33.3) (522.45,39.294) (532.038,46.081) (541.626,53.73) (551.214,62.313) (560.802,71.906) (570.39,82.59) (579.978,94.452) (589.567,107.586) (599.155,122.099) (608.743,138.108) (618.331,155.754) (627.919,175.208) (637.507,196.692) (647.095,220.637) (647.095,220.637) (637.507,196.692) (627.919,175.208) (618.331,155.754) (608.743,138.108) (599.155,122.099) (589.567,107.586) (579.978,94.452) (570.39,82.59) (560.802,71.906) (551.214,62.313) (541.626,53.73) (532.038,46.081) (522.45,39.294) (512.862,33.3) (503.274,28.034) (493.686,23.434) (484.098,19.441) (474.51,15.996) (464.922,13.047) (455.333,10.542) (445.745,8.431) (436.157,6.67) (426.569,5.214) (416.981,4.025) (407.393,3.064) (397.805,2.298) (388.217,1.696) (378.629,1.229) (369.041,0.874) (359.453,0.609) (349.865,0.414) (340.277,0.275) (330.688,0.178) (321.1,0.111) (311.512,0.068) (301.924,0.04) (292.336,0.022) (282.748,0.012) (273.16,0.006)};
        \addlegendentry{\(Water\)}
        \addplot[color=red]
            coordinates {(216.592,5.18) (218.836,5.704) (221.081,6.268) (223.325,6.872) (225.57,7.519) (227.814,8.211) (230.059,8.949) (232.303,9.735) (234.548,10.572) (236.792,11.461) (239.037,12.403) (241.281,13.402) (243.526,14.459) (245.77,15.576) (248.015,16.754) (250.259,17.997) (252.504,19.307) (254.748,20.685) (256.993,22.133) (259.237,23.654) (261.482,25.25) (263.726,26.924) (265.971,28.678) (268.215,30.513) (270.46,32.434) (272.704,34.442) (274.949,36.54) (277.193,38.732) (279.438,41.019) (281.682,43.406) (283.927,45.895) (286.171,48.491) (288.416,51.198) (290.66,54.02) (292.905,56.962) (295.149,60.03) (297.394,63.232) (299.638,66.577) (301.883,70.08) (304.127,73.771) (301.883,70.08) (299.638,66.577) (297.394,63.232) (295.149,60.03) (292.905,56.962) (290.66,54.02) (288.416,51.198) (286.171,48.491) (283.927,45.895) (281.682,43.406) (279.438,41.019) (277.193,38.732) (274.949,36.54) (272.704,34.442) (270.46,32.434) (268.215,30.513) (265.971,28.678) (263.726,26.924) (261.482,25.25) (259.237,23.654) (256.993,22.133) (254.748,20.685) (252.504,19.307) (250.259,17.997) (248.015,16.754) (245.77,15.576) (243.526,14.459) (241.281,13.402) (239.037,12.403) (236.792,11.461) (234.548,10.572) (232.303,9.735) (230.059,8.949) (227.814,8.211) (225.57,7.519) (223.325,6.872) (221.081,6.268) (218.836,5.704)};
        \addlegendentry{\(CO_2\)}
            
    \end{axis}
\end{tikzpicture}%
~%
%
\begin{tikzpicture}[baseline]
    \begin{axis}[xlabel = {Enthalpy/\unit{\joule \per \mol}},
                 ylabel = {Pressure/\unit{\bar}},
                 legend style={at={(0.5,0.03)}, anchor=south},
                 ymode = log,
                 ymin = 0.1,
                 height=7cm,
                 width=8cm]
        \addplot[color=blue]
            coordinates {(-7549.426,0.006) (-6822.795,0.012) (-6099.057,0.022) (-5376.621,0.04) (-4654.607,0.068) (-3932.499,0.111) (-3209.957,0.178) (-2486.713,0.275) (-1762.516,0.414) (-1037.095,0.609) (-310.149,0.874) (418.663,1.229) (1149.722,1.696) (1883.452,2.298) (2620.319,3.064) (3360.829,4.025) (4105.534,5.214) (4855.034,6.67) (5609.981,8.431) (6371.087,10.542) (7139.135,13.047) (7914.986,15.996) (8699.602,19.441) (9494.058,23.434) (10299.575,28.034) (11117.551,33.3) (11949.607,39.294) (12797.653,46.081) (13663.974,53.73) (14551.362,62.313) (15463.302,71.906) (16404.255,82.59) (17380.11,94.452) (18398.923,107.586) (19472.244,122.099) (20617.874,138.108) (21866.943,155.754) (23286.343,175.208) (25056.85,196.692) (29845.253,220.637) (30157.268,220.637) (36255.986,196.692) (38002.414,175.208) (39179.569,155.754) (40059.906,138.108) (40746.4,122.099) (41292.188,107.586) (41729.218,94.452) (42078.551,82.59) (42354.966,71.906) (42569.309,62.313) (42729.831,53.73) (42842.991,46.081) (42913.975,39.294) (42947.035,33.3) (42945.727,28.034) (42913.07,23.434) (42851.668,19.441) (42763.8,15.996) (42651.488,13.047) (42516.55,10.542) (42360.654,8.431) (42185.353,6.67) (41992.12,5.214) (41782.381,4.025) (41557.526,3.064) (41318.929,2.298) (41067.943,1.696) (40805.889,1.229) (40534.048,0.874) (40253.639,0.609) (39965.808,0.414) (39671.62,0.275) (39372.052,0.178) (39067.993,0.111) (38760.247,0.068) (38449.528,0.04) (38136.453,0.022) (37821.544,0.012) (37505.22,0.006)};
        \addlegendentry{\(Water\)}
        \addplot[color=red]
            coordinates {(-8542.563,5.18) (-8348.728,5.704) (-8154.31,6.268) (-7959.246,6.872) (-7763.469,7.519) (-7566.91,8.211) (-7369.497,8.949) (-7171.153,9.735) (-6971.796,10.572) (-6771.341,11.461) (-6569.696,12.403) (-6366.764,13.402) (-6162.44,14.459) (-5956.611,15.576) (-5749.155,16.754) (-5539.939,17.997) (-5328.82,19.307) (-5115.642,20.685) (-4900.235,22.133) (-4682.411,23.654) (-4461.959,25.25) (-4238.645,26.924) (-4012.199,28.678) (-3782.315,30.513) (-3548.632,32.434) (-3310.734,34.442) (-3068.125,36.54) (-2820.214,38.732) (-2566.284,41.019) (-2305.439,43.406) (-2036.535,45.895) (-1758.065,48.491) (-1468.004,51.198) (-1163.567,54.02) (-840.796,56.962) (-493.647,60.03) (-111.567,63.232) (327.48,66.577) (883.531,70.08) (2688.586,73.771) (4507.633,70.08) (5041.028,66.577) (5404.377,63.232) (5682.78,60.03) (5907.697,56.962) (6094.968,54.02) (6253.895,51.198) (6390.477,48.491) (6508.842,45.895) (6611.968,43.406) (6702.086,41.019) (6780.916,38.732) (6849.818,36.54) (6909.891,34.442) (6962.037,32.434) (7007.011,30.513) (7045.448,28.678) (7077.891,26.924) (7104.809,25.25) (7126.607,23.654) (7143.636,22.133) (7156.207,20.685) (7164.592,19.307) (7169.034,17.997) (7169.752,16.754) (7166.945,15.576) (7160.798,14.459) (7151.48,13.402) (7139.148,12.403) (7123.95,11.461) (7106.02,10.572) (7085.483,9.735) (7062.456,8.949) (7037.046,8.211) (7009.354,7.519) (6979.478,6.872) (6947.506,6.268) (6913.526,5.704)};
        \addlegendentry{\(CO_2\)}
    \end{axis}
\end{tikzpicture}
            \caption[Saturation curves of water and carbon dioxide]{The saturation curves for water and carbon dioxide in Pressure-Temperature and Pressure-Molar Enthalpy domains. Calculated using the Water and Carbon Dioxide \ac{HEOS} implemented in \emph{CoolProp}.}
            \label{fig:SaturationCurves}
        \end{figure}

        \subsubsection{Mixtures}
        Methodologies for modelling mixtures of reference \ac{EOS} have been developed \cite{Lemmon1999b}, but require an additional set of parameters for each component pair to capture their interactions. The equations below outline the general formulation of the mixture model, with \(N_k\), \(d_k\), \(t_k\) common to all mixtures and \(F_{ij}\), \(\xi_{ij}\), \(\beta_{ij}\), \(\phi_{ij}\) and \(\zeta_{ij}\) specific to each component pair.

        \begin{align}
            A = A^{id\;mix} + A^{excess} \label{eq:HEOS_mixture}
        \end{align}
        \begin{align}
            A^{id\;mix} = \sum_{i=1}^n x_i \cdot \left[A^0_i(\rho,T) + A^r_i(\delta, \tau) + R T \ln x_i\right] \label{eq:HEOSmixture_id_term}
        \end{align}
        \begin{align}
            A^{excess} = RT \sum_{i=1}^{n-1} \sum_{j=i+1}^n x_i x_j F_{ij} \sum_{k=1}^{10} N_k \delta^{d_k} \tau^{t_k} \label{eq:HEOS_excess}
        \end{align}
        \begin{align}
            \delta = \frac{\rho}{\rho_{red}} \label{eq:reduced_density}
        \end{align}
        \begin{align}
            \tau = \frac{T_{red}}{T} \label{eq:reduced_temperature}
        \end{align}

        \begin{align}
            \rho_{red} = \left[\sum_{i=1}^n \frac{x_i}{\rho_c} + \sum_{i=1}^{n-1} \sum_{j=i+1}^n x_i x_j \xi_{ij} \right]^{-1} \label{eq:red_density}
        \end{align}
        \begin{align}
            T_{red} = \sum_{i=1}^{n} x_i T_{c_i} + \sum_{i=1}^{n-1} \sum_{j=i+1}{n} x_i^{\beta_{ij}} x_j^{\phi_{ij}}\zeta_{ij} \label{eq:red_temperature}
        \end{align}

    As a result of the complexity of the formulation, the convergence can be challenging and may not be guaranteed for all compositions, temperatures and pressure of interest. The convergence was investigated using binary mixtures of water and carbon dioxide over a range of temperatures (\qtyrange{298}{573}{\K} corresponding to \qtyrange{25}{300}{\degreeCelsius}) and pressures (\qtyrange{1}{300}{\bar}) using \emph{CoolProp} \cite{Bell2014}, see Section~\ref{sec:calc_frameworks}. 

    \begin{figure}[H]
        \centering
        % This file was created with tikzplotlib v0.10.1.
\begin{tikzpicture}

\definecolor{darkgray176}{RGB}{176,176,176}
\definecolor{lightgray204}{RGB}{204,204,204}

\begin{groupplot}[
    group style={
        group size=2 by 2,
        vertical sep=2.5cm, 
        horizontal sep=2.5cm
        },
    legend style={
        fill opacity=0.8,
        draw opacity=1,
        text opacity=1,
        at={(1.15, 1.4)},
        anchor=north,
        draw=lightgray204
    },
    legend columns=-1,
    height=6cm, 
    width=7cm,
    log basis y={10},
    tick align=outside,
    tick pos=left,
    x grid style={darkgray176},
    xlabel={Temperature/\unit{\K}},
    xmin=298, xmax=573,
    xtick style={color=black},
    y grid style={darkgray176},
    ylabel={Pressure/\unit{\bar}},
    ymin=1, ymax=300,
    ymode=log,
    ytick style={color=black}
    ]
\nextgroupplot[
title={\(z_{CO_2}=\)\qty{0.0}{\mol\percent}},
]
\addplot [semithick, red, mark=*, mark size=1, mark options={solid}, only marks]
table {%
298.15 1
302.811016949153 1
307.472033898305 1
312.133050847458 1
316.79406779661 1
321.455084745763 1
298.15 1.21735704750235
302.811016949153 1.21735704750235
307.472033898305 1.21735704750235
312.133050847458 1.21735704750235
316.79406779661 1.21735704750235
321.455084745763 1.21735704750235
326.116101694915 1.21735704750235
330.777118644068 1.21735704750235
298.15 1.48195818110364
302.811016949153 1.48195818110364
307.472033898305 1.48195818110364
312.133050847458 1.48195818110364
316.79406779661 1.48195818110364
321.455084745763 1.48195818110364
326.116101694915 1.48195818110364
330.777118644068 1.48195818110364
335.43813559322 1.48195818110364
298.15 1.80407223587029
302.811016949153 1.80407223587029
307.472033898305 1.80407223587029
312.133050847458 1.80407223587029
316.79406779661 1.80407223587029
321.455084745763 1.80407223587029
326.116101694915 1.80407223587029
330.777118644068 1.80407223587029
335.43813559322 1.80407223587029
340.099152542373 1.80407223587029
344.760169491525 1.80407223587029
298.15 2.19620005054002
302.811016949153 2.19620005054002
307.472033898305 2.19620005054002
312.133050847458 2.19620005054002
316.79406779661 2.19620005054002
321.455084745763 2.19620005054002
326.116101694915 2.19620005054002
330.777118644068 2.19620005054002
335.43813559322 2.19620005054002
340.099152542373 2.19620005054002
344.760169491525 2.19620005054002
349.421186440678 2.19620005054002
354.082203389831 2.19620005054002
298.15 2.67355960924991
302.811016949153 2.67355960924991
307.472033898305 2.67355960924991
312.133050847458 2.67355960924991
316.79406779661 2.67355960924991
321.455084745763 2.67355960924991
326.116101694915 2.67355960924991
330.777118644068 2.67355960924991
335.43813559322 2.67355960924991
340.099152542373 2.67355960924991
344.760169491525 2.67355960924991
349.421186440678 2.67355960924991
354.082203389831 2.67355960924991
358.743220338983 2.67355960924991
298.15 3.25467663223801
302.811016949153 3.25467663223801
307.472033898305 3.25467663223801
312.133050847458 3.25467663223801
316.79406779661 3.25467663223801
321.455084745763 3.25467663223801
326.116101694915 3.25467663223801
330.777118644068 3.25467663223801
335.43813559322 3.25467663223801
340.099152542373 3.25467663223801
344.760169491525 3.25467663223801
349.421186440678 3.25467663223801
354.082203389831 3.25467663223801
358.743220338983 3.25467663223801
363.404237288136 3.25467663223801
368.065254237288 3.25467663223801
298.15 3.96210353559616
302.811016949153 3.96210353559616
307.472033898305 3.96210353559616
312.133050847458 3.96210353559616
316.79406779661 3.96210353559616
321.455084745763 3.96210353559616
326.116101694915 3.96210353559616
330.777118644068 3.96210353559616
335.43813559322 3.96210353559616
340.099152542373 3.96210353559616
344.760169491525 3.96210353559616
349.421186440678 3.96210353559616
354.082203389831 3.96210353559616
358.743220338983 3.96210353559616
363.404237288136 3.96210353559616
368.065254237288 3.96210353559616
372.726271186441 3.96210353559616
377.387288135593 3.96210353559616
298.15 4.82329466199197
302.811016949153 4.82329466199197
307.472033898305 4.82329466199197
312.133050847458 4.82329466199197
316.79406779661 4.82329466199197
321.455084745763 4.82329466199197
326.116101694915 4.82329466199197
330.777118644068 4.82329466199197
335.43813559322 4.82329466199197
340.099152542373 4.82329466199197
344.760169491525 4.82329466199197
349.421186440678 4.82329466199197
354.082203389831 4.82329466199197
358.743220338983 4.82329466199197
363.404237288136 4.82329466199197
368.065254237288 4.82329466199197
372.726271186441 4.82329466199197
377.387288135593 4.82329466199197
382.048305084746 4.82329466199197
298.15 5.87167174895639
302.811016949153 5.87167174895639
307.472033898305 5.87167174895639
312.133050847458 5.87167174895639
316.79406779661 5.87167174895639
321.455084745763 5.87167174895639
326.116101694915 5.87167174895639
330.777118644068 5.87167174895639
335.43813559322 5.87167174895639
340.099152542373 5.87167174895639
344.760169491525 5.87167174895639
349.421186440678 5.87167174895639
354.082203389831 5.87167174895639
358.743220338983 5.87167174895639
363.404237288136 5.87167174895639
368.065254237288 5.87167174895639
372.726271186441 5.87167174895639
377.387288135593 5.87167174895639
382.048305084746 5.87167174895639
386.709322033898 5.87167174895639
391.370338983051 5.87167174895639
298.15 7.14792098421252
302.811016949153 7.14792098421252
307.472033898305 7.14792098421252
312.133050847458 7.14792098421252
316.79406779661 7.14792098421252
321.455084745763 7.14792098421252
326.116101694915 7.14792098421252
330.777118644068 7.14792098421252
335.43813559322 7.14792098421252
340.099152542373 7.14792098421252
344.760169491525 7.14792098421252
349.421186440678 7.14792098421252
354.082203389831 7.14792098421252
358.743220338983 7.14792098421252
363.404237288136 7.14792098421252
368.065254237288 7.14792098421252
372.726271186441 7.14792098421252
377.387288135593 7.14792098421252
382.048305084746 7.14792098421252
386.709322033898 7.14792098421252
391.370338983051 7.14792098421252
396.031355932203 7.14792098421252
400.692372881356 7.14792098421252
298.15 8.70157198512106
302.811016949153 8.70157198512106
307.472033898305 8.70157198512106
312.133050847458 8.70157198512106
316.79406779661 8.70157198512106
321.455084745763 8.70157198512106
326.116101694915 8.70157198512106
330.777118644068 8.70157198512106
335.43813559322 8.70157198512106
340.099152542373 8.70157198512106
344.760169491525 8.70157198512106
349.421186440678 8.70157198512106
354.082203389831 8.70157198512106
358.743220338983 8.70157198512106
363.404237288136 8.70157198512106
368.065254237288 8.70157198512106
372.726271186441 8.70157198512106
377.387288135593 8.70157198512106
382.048305084746 8.70157198512106
386.709322033898 8.70157198512106
391.370338983051 8.70157198512106
396.031355932203 8.70157198512106
400.692372881356 8.70157198512106
405.353389830508 8.70157198512106
410.014406779661 8.70157198512106
298.15 10.5929199804362
302.811016949153 10.5929199804362
307.472033898305 10.5929199804362
312.133050847458 10.5929199804362
316.79406779661 10.5929199804362
321.455084745763 10.5929199804362
326.116101694915 10.5929199804362
330.777118644068 10.5929199804362
335.43813559322 10.5929199804362
340.099152542373 10.5929199804362
344.760169491525 10.5929199804362
349.421186440678 10.5929199804362
354.082203389831 10.5929199804362
358.743220338983 10.5929199804362
363.404237288136 10.5929199804362
368.065254237288 10.5929199804362
372.726271186441 10.5929199804362
377.387288135593 10.5929199804362
382.048305084746 10.5929199804362
386.709322033898 10.5929199804362
391.370338983051 10.5929199804362
396.031355932203 10.5929199804362
400.692372881356 10.5929199804362
405.353389830508 10.5929199804362
410.014406779661 10.5929199804362
414.675423728814 10.5929199804362
419.336440677966 10.5929199804362
298.15 12.8953657918124
302.811016949153 12.8953657918124
307.472033898305 12.8953657918124
312.133050847458 12.8953657918124
316.79406779661 12.8953657918124
321.455084745763 12.8953657918124
326.116101694915 12.8953657918124
330.777118644068 12.8953657918124
335.43813559322 12.8953657918124
340.099152542373 12.8953657918124
344.760169491525 12.8953657918124
349.421186440678 12.8953657918124
354.082203389831 12.8953657918124
358.743220338983 12.8953657918124
363.404237288136 12.8953657918124
368.065254237288 12.8953657918124
372.726271186441 12.8953657918124
377.387288135593 12.8953657918124
382.048305084746 12.8953657918124
386.709322033898 12.8953657918124
391.370338983051 12.8953657918124
396.031355932203 12.8953657918124
400.692372881356 12.8953657918124
405.353389830508 12.8953657918124
410.014406779661 12.8953657918124
414.675423728814 12.8953657918124
419.336440677966 12.8953657918124
423.997457627119 12.8953657918124
428.658474576271 12.8953657918124
298.15 15.6982644267836
302.811016949153 15.6982644267836
307.472033898305 15.6982644267836
312.133050847458 15.6982644267836
316.79406779661 15.6982644267836
321.455084745763 15.6982644267836
326.116101694915 15.6982644267836
330.777118644068 15.6982644267836
335.43813559322 15.6982644267836
340.099152542373 15.6982644267836
344.760169491525 15.6982644267836
349.421186440678 15.6982644267836
354.082203389831 15.6982644267836
358.743220338983 15.6982644267836
363.404237288136 15.6982644267836
368.065254237288 15.6982644267836
372.726271186441 15.6982644267836
377.387288135593 15.6982644267836
382.048305084746 15.6982644267836
386.709322033898 15.6982644267836
391.370338983051 15.6982644267836
396.031355932203 15.6982644267836
400.692372881356 15.6982644267836
405.353389830508 15.6982644267836
410.014406779661 15.6982644267836
414.675423728814 15.6982644267836
419.336440677966 15.6982644267836
423.997457627119 15.6982644267836
428.658474576271 15.6982644267836
433.319491525424 15.6982644267836
437.980508474576 15.6982644267836
298.15 19.1103928335005
302.811016949153 19.1103928335005
307.472033898305 19.1103928335005
312.133050847458 19.1103928335005
316.79406779661 19.1103928335005
321.455084745763 19.1103928335005
326.116101694915 19.1103928335005
330.777118644068 19.1103928335005
335.43813559322 19.1103928335005
340.099152542373 19.1103928335005
344.760169491525 19.1103928335005
349.421186440678 19.1103928335005
354.082203389831 19.1103928335005
358.743220338983 19.1103928335005
363.404237288136 19.1103928335005
368.065254237288 19.1103928335005
372.726271186441 19.1103928335005
377.387288135593 19.1103928335005
382.048305084746 19.1103928335005
386.709322033898 19.1103928335005
391.370338983051 19.1103928335005
396.031355932203 19.1103928335005
400.692372881356 19.1103928335005
405.353389830508 19.1103928335005
410.014406779661 19.1103928335005
414.675423728814 19.1103928335005
419.336440677966 19.1103928335005
423.997457627119 19.1103928335005
428.658474576271 19.1103928335005
433.319491525424 19.1103928335005
437.980508474576 19.1103928335005
442.641525423729 19.1103928335005
447.302542372881 19.1103928335005
451.963559322034 19.1103928335005
298.15 23.2641713964002
302.811016949153 23.2641713964002
307.472033898305 23.2641713964002
312.133050847458 23.2641713964002
316.79406779661 23.2641713964002
321.455084745763 23.2641713964002
326.116101694915 23.2641713964002
330.777118644068 23.2641713964002
335.43813559322 23.2641713964002
340.099152542373 23.2641713964002
344.760169491525 23.2641713964002
349.421186440678 23.2641713964002
354.082203389831 23.2641713964002
358.743220338983 23.2641713964002
363.404237288136 23.2641713964002
368.065254237288 23.2641713964002
372.726271186441 23.2641713964002
377.387288135593 23.2641713964002
382.048305084746 23.2641713964002
386.709322033898 23.2641713964002
391.370338983051 23.2641713964002
396.031355932203 23.2641713964002
400.692372881356 23.2641713964002
405.353389830508 23.2641713964002
410.014406779661 23.2641713964002
414.675423728814 23.2641713964002
419.336440677966 23.2641713964002
423.997457627119 23.2641713964002
428.658474576271 23.2641713964002
433.319491525424 23.2641713964002
437.980508474576 23.2641713964002
442.641525423729 23.2641713964002
447.302542372881 23.2641713964002
451.963559322034 23.2641713964002
456.624576271186 23.2641713964002
461.285593220339 23.2641713964002
298.15 28.3208030037104
302.811016949153 28.3208030037104
307.472033898305 28.3208030037104
312.133050847458 28.3208030037104
316.79406779661 28.3208030037104
321.455084745763 28.3208030037104
326.116101694915 28.3208030037104
330.777118644068 28.3208030037104
335.43813559322 28.3208030037104
340.099152542373 28.3208030037104
344.760169491525 28.3208030037104
349.421186440678 28.3208030037104
354.082203389831 28.3208030037104
358.743220338983 28.3208030037104
363.404237288136 28.3208030037104
368.065254237288 28.3208030037104
372.726271186441 28.3208030037104
377.387288135593 28.3208030037104
382.048305084746 28.3208030037104
386.709322033898 28.3208030037104
391.370338983051 28.3208030037104
396.031355932203 28.3208030037104
400.692372881356 28.3208030037104
405.353389830508 28.3208030037104
410.014406779661 28.3208030037104
414.675423728814 28.3208030037104
419.336440677966 28.3208030037104
423.997457627119 28.3208030037104
428.658474576271 28.3208030037104
433.319491525424 28.3208030037104
437.980508474576 28.3208030037104
442.641525423729 28.3208030037104
447.302542372881 28.3208030037104
451.963559322034 28.3208030037104
456.624576271186 28.3208030037104
461.285593220339 28.3208030037104
465.946610169491 28.3208030037104
470.607627118644 28.3208030037104
298.15 34.4765291274926
302.811016949153 34.4765291274926
307.472033898305 34.4765291274926
312.133050847458 34.4765291274926
316.79406779661 34.4765291274926
321.455084745763 34.4765291274926
326.116101694915 34.4765291274926
330.777118644068 34.4765291274926
335.43813559322 34.4765291274926
340.099152542373 34.4765291274926
344.760169491525 34.4765291274926
349.421186440678 34.4765291274926
354.082203389831 34.4765291274926
358.743220338983 34.4765291274926
363.404237288136 34.4765291274926
368.065254237288 34.4765291274926
372.726271186441 34.4765291274926
377.387288135593 34.4765291274926
382.048305084746 34.4765291274926
386.709322033898 34.4765291274926
391.370338983051 34.4765291274926
396.031355932203 34.4765291274926
400.692372881356 34.4765291274926
405.353389830508 34.4765291274926
410.014406779661 34.4765291274926
414.675423728814 34.4765291274926
419.336440677966 34.4765291274926
423.997457627119 34.4765291274926
428.658474576271 34.4765291274926
433.319491525424 34.4765291274926
437.980508474576 34.4765291274926
442.641525423729 34.4765291274926
447.302542372881 34.4765291274926
451.963559322034 34.4765291274926
456.624576271186 34.4765291274926
461.285593220339 34.4765291274926
465.946610169491 34.4765291274926
470.607627118644 34.4765291274926
475.268644067797 34.4765291274926
479.929661016949 34.4765291274926
484.590677966102 34.4765291274926
298.15 41.9702457067732
302.811016949153 41.9702457067732
307.472033898305 41.9702457067732
312.133050847458 41.9702457067732
316.79406779661 41.9702457067732
321.455084745763 41.9702457067732
326.116101694915 41.9702457067732
330.777118644068 41.9702457067732
335.43813559322 41.9702457067732
340.099152542373 41.9702457067732
344.760169491525 41.9702457067732
349.421186440678 41.9702457067732
354.082203389831 41.9702457067732
358.743220338983 41.9702457067732
363.404237288136 41.9702457067732
368.065254237288 41.9702457067732
372.726271186441 41.9702457067732
377.387288135593 41.9702457067732
382.048305084746 41.9702457067732
386.709322033898 41.9702457067732
391.370338983051 41.9702457067732
396.031355932203 41.9702457067732
400.692372881356 41.9702457067732
405.353389830508 41.9702457067732
410.014406779661 41.9702457067732
414.675423728814 41.9702457067732
419.336440677966 41.9702457067732
423.997457627119 41.9702457067732
428.658474576271 41.9702457067732
433.319491525424 41.9702457067732
437.980508474576 41.9702457067732
442.641525423729 41.9702457067732
447.302542372881 41.9702457067732
451.963559322034 41.9702457067732
456.624576271186 41.9702457067732
461.285593220339 41.9702457067732
465.946610169491 41.9702457067732
470.607627118644 41.9702457067732
475.268644067797 41.9702457067732
479.929661016949 41.9702457067732
484.590677966102 41.9702457067732
489.251694915254 41.9702457067732
493.912711864407 41.9702457067732
498.573728813559 41.9702457067732
298.15 51.0927743965457
302.811016949153 51.0927743965457
307.472033898305 51.0927743965457
312.133050847458 51.0927743965457
316.79406779661 51.0927743965457
321.455084745763 51.0927743965457
326.116101694915 51.0927743965457
330.777118644068 51.0927743965457
335.43813559322 51.0927743965457
340.099152542373 51.0927743965457
344.760169491525 51.0927743965457
349.421186440678 51.0927743965457
354.082203389831 51.0927743965457
358.743220338983 51.0927743965457
363.404237288136 51.0927743965457
368.065254237288 51.0927743965457
372.726271186441 51.0927743965457
377.387288135593 51.0927743965457
382.048305084746 51.0927743965457
386.709322033898 51.0927743965457
391.370338983051 51.0927743965457
396.031355932203 51.0927743965457
400.692372881356 51.0927743965457
405.353389830508 51.0927743965457
410.014406779661 51.0927743965457
414.675423728814 51.0927743965457
419.336440677966 51.0927743965457
423.997457627119 51.0927743965457
428.658474576271 51.0927743965457
433.319491525424 51.0927743965457
437.980508474576 51.0927743965457
442.641525423729 51.0927743965457
447.302542372881 51.0927743965457
451.963559322034 51.0927743965457
456.624576271186 51.0927743965457
461.285593220339 51.0927743965457
465.946610169491 51.0927743965457
470.607627118644 51.0927743965457
475.268644067797 51.0927743965457
479.929661016949 51.0927743965457
484.590677966102 51.0927743965457
489.251694915254 51.0927743965457
493.912711864407 51.0927743965457
498.573728813559 51.0927743965457
503.234745762712 51.0927743965457
507.895762711864 51.0927743965457
512.556779661017 51.0927743965457
298.15 62.1981489880827
302.811016949153 62.1981489880827
307.472033898305 62.1981489880827
312.133050847458 62.1981489880827
316.79406779661 62.1981489880827
321.455084745763 62.1981489880827
326.116101694915 62.1981489880827
330.777118644068 62.1981489880827
335.43813559322 62.1981489880827
340.099152542373 62.1981489880827
344.760169491525 62.1981489880827
349.421186440678 62.1981489880827
354.082203389831 62.1981489880827
358.743220338983 62.1981489880827
363.404237288136 62.1981489880827
368.065254237288 62.1981489880827
372.726271186441 62.1981489880827
377.387288135593 62.1981489880827
382.048305084746 62.1981489880827
386.709322033898 62.1981489880827
391.370338983051 62.1981489880827
396.031355932203 62.1981489880827
400.692372881356 62.1981489880827
405.353389830508 62.1981489880827
410.014406779661 62.1981489880827
414.675423728814 62.1981489880827
419.336440677966 62.1981489880827
423.997457627119 62.1981489880827
428.658474576271 62.1981489880827
433.319491525424 62.1981489880827
437.980508474576 62.1981489880827
442.641525423729 62.1981489880827
447.302542372881 62.1981489880827
451.963559322034 62.1981489880827
456.624576271186 62.1981489880827
461.285593220339 62.1981489880827
465.946610169491 62.1981489880827
470.607627118644 62.1981489880827
475.268644067797 62.1981489880827
479.929661016949 62.1981489880827
484.590677966102 62.1981489880827
489.251694915254 62.1981489880827
493.912711864407 62.1981489880827
498.573728813559 62.1981489880827
503.234745762712 62.1981489880827
507.895762711864 62.1981489880827
512.556779661017 62.1981489880827
517.217796610169 62.1981489880827
521.878813559322 62.1981489880827
526.539830508475 62.1981489880827
298.15 75.7173550122436
302.811016949153 75.7173550122436
307.472033898305 75.7173550122436
312.133050847458 75.7173550122436
316.79406779661 75.7173550122436
321.455084745763 75.7173550122436
326.116101694915 75.7173550122436
330.777118644068 75.7173550122436
335.43813559322 75.7173550122436
340.099152542373 75.7173550122436
344.760169491525 75.7173550122436
349.421186440678 75.7173550122436
354.082203389831 75.7173550122436
358.743220338983 75.7173550122436
363.404237288136 75.7173550122436
368.065254237288 75.7173550122436
372.726271186441 75.7173550122436
377.387288135593 75.7173550122436
382.048305084746 75.7173550122436
386.709322033898 75.7173550122436
391.370338983051 75.7173550122436
396.031355932203 75.7173550122436
400.692372881356 75.7173550122436
405.353389830508 75.7173550122436
410.014406779661 75.7173550122436
414.675423728814 75.7173550122436
419.336440677966 75.7173550122436
423.997457627119 75.7173550122436
428.658474576271 75.7173550122436
433.319491525424 75.7173550122436
437.980508474576 75.7173550122436
442.641525423729 75.7173550122436
447.302542372881 75.7173550122436
451.963559322034 75.7173550122436
456.624576271186 75.7173550122436
461.285593220339 75.7173550122436
465.946610169491 75.7173550122436
470.607627118644 75.7173550122436
475.268644067797 75.7173550122436
479.929661016949 75.7173550122436
484.590677966102 75.7173550122436
489.251694915254 75.7173550122436
493.912711864407 75.7173550122436
498.573728813559 75.7173550122436
503.234745762712 75.7173550122436
507.895762711864 75.7173550122436
512.556779661017 75.7173550122436
517.217796610169 75.7173550122436
521.878813559322 75.7173550122436
526.539830508475 75.7173550122436
531.200847457627 75.7173550122436
535.86186440678 75.7173550122436
540.522881355932 75.7173550122436
298.15 92.1750557423923
302.811016949153 92.1750557423923
307.472033898305 92.1750557423923
312.133050847458 92.1750557423923
316.79406779661 92.1750557423923
321.455084745763 92.1750557423923
326.116101694915 92.1750557423923
330.777118644068 92.1750557423923
335.43813559322 92.1750557423923
340.099152542373 92.1750557423923
344.760169491525 92.1750557423923
349.421186440678 92.1750557423923
354.082203389831 92.1750557423923
358.743220338983 92.1750557423923
363.404237288136 92.1750557423923
368.065254237288 92.1750557423923
372.726271186441 92.1750557423923
377.387288135593 92.1750557423923
382.048305084746 92.1750557423923
386.709322033898 92.1750557423923
391.370338983051 92.1750557423923
396.031355932203 92.1750557423923
400.692372881356 92.1750557423923
405.353389830508 92.1750557423923
410.014406779661 92.1750557423923
414.675423728814 92.1750557423923
419.336440677966 92.1750557423923
423.997457627119 92.1750557423923
428.658474576271 92.1750557423923
433.319491525424 92.1750557423923
437.980508474576 92.1750557423923
442.641525423729 92.1750557423923
447.302542372881 92.1750557423923
451.963559322034 92.1750557423923
456.624576271186 92.1750557423923
461.285593220339 92.1750557423923
465.946610169491 92.1750557423923
470.607627118644 92.1750557423923
475.268644067797 92.1750557423923
479.929661016949 92.1750557423923
484.590677966102 92.1750557423923
489.251694915254 92.1750557423923
493.912711864407 92.1750557423923
498.573728813559 92.1750557423923
503.234745762712 92.1750557423923
507.895762711864 92.1750557423923
512.556779661017 92.1750557423923
517.217796610169 92.1750557423923
521.878813559322 92.1750557423923
526.539830508475 92.1750557423923
531.200847457627 92.1750557423923
535.86186440678 92.1750557423923
540.522881355932 92.1750557423923
545.183898305085 92.1750557423923
549.844915254237 92.1750557423923
554.50593220339 92.1750557423923
559.166949152542 92.1750557423923
298.15 112.209953711923
302.811016949153 112.209953711923
307.472033898305 112.209953711923
312.133050847458 112.209953711923
316.79406779661 112.209953711923
321.455084745763 112.209953711923
326.116101694915 112.209953711923
330.777118644068 112.209953711923
335.43813559322 112.209953711923
340.099152542373 112.209953711923
344.760169491525 112.209953711923
349.421186440678 112.209953711923
354.082203389831 112.209953711923
358.743220338983 112.209953711923
363.404237288136 112.209953711923
368.065254237288 112.209953711923
372.726271186441 112.209953711923
377.387288135593 112.209953711923
382.048305084746 112.209953711923
386.709322033898 112.209953711923
391.370338983051 112.209953711923
396.031355932203 112.209953711923
400.692372881356 112.209953711923
405.353389830508 112.209953711923
410.014406779661 112.209953711923
414.675423728814 112.209953711923
419.336440677966 112.209953711923
423.997457627119 112.209953711923
428.658474576271 112.209953711923
433.319491525424 112.209953711923
437.980508474576 112.209953711923
442.641525423729 112.209953711923
447.302542372881 112.209953711923
451.963559322034 112.209953711923
456.624576271186 112.209953711923
461.285593220339 112.209953711923
465.946610169491 112.209953711923
470.607627118644 112.209953711923
475.268644067797 112.209953711923
479.929661016949 112.209953711923
484.590677966102 112.209953711923
489.251694915254 112.209953711923
493.912711864407 112.209953711923
498.573728813559 112.209953711923
503.234745762712 112.209953711923
507.895762711864 112.209953711923
512.556779661017 112.209953711923
517.217796610169 112.209953711923
521.878813559322 112.209953711923
526.539830508475 112.209953711923
531.200847457627 112.209953711923
535.86186440678 112.209953711923
540.522881355932 112.209953711923
545.183898305085 112.209953711923
549.844915254237 112.209953711923
554.50593220339 112.209953711923
559.166949152542 112.209953711923
563.827966101695 112.209953711923
568.488983050847 112.209953711923
573.15 112.209953711923
298.15 136.599577951123
302.811016949153 136.599577951123
307.472033898305 136.599577951123
312.133050847458 136.599577951123
316.79406779661 136.599577951123
321.455084745763 136.599577951123
326.116101694915 136.599577951123
330.777118644068 136.599577951123
335.43813559322 136.599577951123
340.099152542373 136.599577951123
344.760169491525 136.599577951123
349.421186440678 136.599577951123
354.082203389831 136.599577951123
358.743220338983 136.599577951123
363.404237288136 136.599577951123
368.065254237288 136.599577951123
372.726271186441 136.599577951123
377.387288135593 136.599577951123
382.048305084746 136.599577951123
386.709322033898 136.599577951123
391.370338983051 136.599577951123
396.031355932203 136.599577951123
400.692372881356 136.599577951123
405.353389830508 136.599577951123
410.014406779661 136.599577951123
414.675423728814 136.599577951123
419.336440677966 136.599577951123
423.997457627119 136.599577951123
428.658474576271 136.599577951123
433.319491525424 136.599577951123
437.980508474576 136.599577951123
442.641525423729 136.599577951123
447.302542372881 136.599577951123
451.963559322034 136.599577951123
456.624576271186 136.599577951123
461.285593220339 136.599577951123
465.946610169491 136.599577951123
470.607627118644 136.599577951123
475.268644067797 136.599577951123
479.929661016949 136.599577951123
484.590677966102 136.599577951123
489.251694915254 136.599577951123
493.912711864407 136.599577951123
498.573728813559 136.599577951123
503.234745762712 136.599577951123
507.895762711864 136.599577951123
512.556779661017 136.599577951123
517.217796610169 136.599577951123
521.878813559322 136.599577951123
526.539830508475 136.599577951123
531.200847457627 136.599577951123
535.86186440678 136.599577951123
540.522881355932 136.599577951123
545.183898305085 136.599577951123
549.844915254237 136.599577951123
554.50593220339 136.599577951123
559.166949152542 136.599577951123
563.827966101695 136.599577951123
568.488983050847 136.599577951123
573.15 136.599577951123
298.15 166.290458904646
302.811016949153 166.290458904646
307.472033898305 166.290458904646
312.133050847458 166.290458904646
316.79406779661 166.290458904646
321.455084745763 166.290458904646
326.116101694915 166.290458904646
330.777118644068 166.290458904646
335.43813559322 166.290458904646
340.099152542373 166.290458904646
344.760169491525 166.290458904646
349.421186440678 166.290458904646
354.082203389831 166.290458904646
358.743220338983 166.290458904646
363.404237288136 166.290458904646
368.065254237288 166.290458904646
372.726271186441 166.290458904646
377.387288135593 166.290458904646
382.048305084746 166.290458904646
386.709322033898 166.290458904646
391.370338983051 166.290458904646
396.031355932203 166.290458904646
400.692372881356 166.290458904646
405.353389830508 166.290458904646
410.014406779661 166.290458904646
414.675423728814 166.290458904646
419.336440677966 166.290458904646
423.997457627119 166.290458904646
428.658474576271 166.290458904646
433.319491525424 166.290458904646
437.980508474576 166.290458904646
442.641525423729 166.290458904646
447.302542372881 166.290458904646
451.963559322034 166.290458904646
456.624576271186 166.290458904646
461.285593220339 166.290458904646
465.946610169491 166.290458904646
470.607627118644 166.290458904646
475.268644067797 166.290458904646
479.929661016949 166.290458904646
484.590677966102 166.290458904646
489.251694915254 166.290458904646
493.912711864407 166.290458904646
498.573728813559 166.290458904646
503.234745762712 166.290458904646
507.895762711864 166.290458904646
512.556779661017 166.290458904646
517.217796610169 166.290458904646
521.878813559322 166.290458904646
526.539830508475 166.290458904646
531.200847457627 166.290458904646
535.86186440678 166.290458904646
540.522881355932 166.290458904646
545.183898305085 166.290458904646
549.844915254237 166.290458904646
554.50593220339 166.290458904646
559.166949152542 166.290458904646
563.827966101695 166.290458904646
568.488983050847 166.290458904646
573.15 166.290458904646
298.15 202.434862079971
302.811016949153 202.434862079971
307.472033898305 202.434862079971
312.133050847458 202.434862079971
316.79406779661 202.434862079971
321.455084745763 202.434862079971
326.116101694915 202.434862079971
330.777118644068 202.434862079971
335.43813559322 202.434862079971
340.099152542373 202.434862079971
344.760169491525 202.434862079971
349.421186440678 202.434862079971
354.082203389831 202.434862079971
358.743220338983 202.434862079971
363.404237288136 202.434862079971
368.065254237288 202.434862079971
372.726271186441 202.434862079971
377.387288135593 202.434862079971
382.048305084746 202.434862079971
386.709322033898 202.434862079971
391.370338983051 202.434862079971
396.031355932203 202.434862079971
400.692372881356 202.434862079971
405.353389830508 202.434862079971
410.014406779661 202.434862079971
414.675423728814 202.434862079971
419.336440677966 202.434862079971
423.997457627119 202.434862079971
428.658474576271 202.434862079971
433.319491525424 202.434862079971
437.980508474576 202.434862079971
442.641525423729 202.434862079971
447.302542372881 202.434862079971
451.963559322034 202.434862079971
456.624576271186 202.434862079971
461.285593220339 202.434862079971
465.946610169491 202.434862079971
470.607627118644 202.434862079971
475.268644067797 202.434862079971
479.929661016949 202.434862079971
484.590677966102 202.434862079971
489.251694915254 202.434862079971
493.912711864407 202.434862079971
498.573728813559 202.434862079971
503.234745762712 202.434862079971
507.895762711864 202.434862079971
512.556779661017 202.434862079971
517.217796610169 202.434862079971
521.878813559322 202.434862079971
526.539830508475 202.434862079971
531.200847457627 202.434862079971
535.86186440678 202.434862079971
540.522881355932 202.434862079971
545.183898305085 202.434862079971
549.844915254237 202.434862079971
554.50593220339 202.434862079971
559.166949152542 202.434862079971
563.827966101695 202.434862079971
568.488983050847 202.434862079971
573.15 202.434862079971
298.15 246.435506013219
302.811016949153 246.435506013219
307.472033898305 246.435506013219
312.133050847458 246.435506013219
316.79406779661 246.435506013219
321.455084745763 246.435506013219
326.116101694915 246.435506013219
330.777118644068 246.435506013219
335.43813559322 246.435506013219
340.099152542373 246.435506013219
344.760169491525 246.435506013219
349.421186440678 246.435506013219
354.082203389831 246.435506013219
358.743220338983 246.435506013219
363.404237288136 246.435506013219
368.065254237288 246.435506013219
372.726271186441 246.435506013219
377.387288135593 246.435506013219
382.048305084746 246.435506013219
386.709322033898 246.435506013219
391.370338983051 246.435506013219
396.031355932203 246.435506013219
400.692372881356 246.435506013219
405.353389830508 246.435506013219
410.014406779661 246.435506013219
414.675423728814 246.435506013219
419.336440677966 246.435506013219
423.997457627119 246.435506013219
428.658474576271 246.435506013219
433.319491525424 246.435506013219
437.980508474576 246.435506013219
442.641525423729 246.435506013219
447.302542372881 246.435506013219
451.963559322034 246.435506013219
456.624576271186 246.435506013219
461.285593220339 246.435506013219
465.946610169491 246.435506013219
470.607627118644 246.435506013219
475.268644067797 246.435506013219
479.929661016949 246.435506013219
484.590677966102 246.435506013219
489.251694915254 246.435506013219
493.912711864407 246.435506013219
498.573728813559 246.435506013219
503.234745762712 246.435506013219
507.895762711864 246.435506013219
512.556779661017 246.435506013219
517.217796610169 246.435506013219
521.878813559322 246.435506013219
526.539830508475 246.435506013219
531.200847457627 246.435506013219
535.86186440678 246.435506013219
540.522881355932 246.435506013219
545.183898305085 246.435506013219
549.844915254237 246.435506013219
554.50593220339 246.435506013219
559.166949152542 246.435506013219
563.827966101695 246.435506013219
568.488983050847 246.435506013219
573.15 246.435506013219
298.15 300
302.811016949153 300
307.472033898305 300
312.133050847458 300
316.79406779661 300
321.455084745763 300
326.116101694915 300
330.777118644068 300
335.43813559322 300
340.099152542373 300
344.760169491525 300
349.421186440678 300
354.082203389831 300
358.743220338983 300
363.404237288136 300
368.065254237288 300
372.726271186441 300
377.387288135593 300
382.048305084746 300
386.709322033898 300
391.370338983051 300
396.031355932203 300
400.692372881356 300
405.353389830508 300
410.014406779661 300
414.675423728814 300
419.336440677966 300
423.997457627119 300
428.658474576271 300
433.319491525424 300
437.980508474576 300
442.641525423729 300
447.302542372881 300
451.963559322034 300
456.624576271186 300
461.285593220339 300
465.946610169491 300
470.607627118644 300
475.268644067797 300
479.929661016949 300
484.590677966102 300
489.251694915254 300
493.912711864407 300
498.573728813559 300
503.234745762712 300
507.895762711864 300
512.556779661017 300
517.217796610169 300
521.878813559322 300
526.539830508475 300
531.200847457627 300
535.86186440678 300
540.522881355932 300
545.183898305085 300
549.844915254237 300
554.50593220339 300
559.166949152542 300
563.827966101695 300
568.488983050847 300
573.15 300
};
\addlegendentry{Non-convergence\quad\quad}
\addplot [semithick, black]
table {%
298.15 0.0316992938881239
302.811016949153 0.0416502526748325
307.472033898305 0.0542135031209127
312.133050847458 0.0699379446480481
316.79406779661 0.0894573631288235
321.455084745763 0.11349814624938
326.116101694915 0.142886977832051
330.777118644068 0.178558427224653
335.43813559322 0.221562362278131
340.099152542373 0.273071114093698
344.760169491525 0.334386329262953
349.421186440678 0.406945451585077
354.082203389831 0.492327780973097
358.743220338983 0.592260071535126
363.404237288136 0.708621631256037
368.065254237288 0.843448901418998
372.726271186441 0.998939497096534
377.387288135593 1.17745570205744
382.048305084746 1.38152741613099
386.709322033898 1.6138545617758
391.370338983051 1.87730896217833
396.031355932203 2.17493570898292
400.692372881356 2.50995404318333
405.353389830508 2.88575777644233
410.014406779661 3.30591528414528
414.675423728814 3.77416910593384
419.336440677966 4.29443518921697
423.997457627119 4.87080181690722
428.658474576271 5.50752826050788
433.319491525424 6.20904320199238
437.980508474576 6.97994296809687
442.641525423729 7.82498962397373
447.302542372881 8.74910897141505
451.963559322034 9.75738849941128
456.624576271186 10.8550753353115
461.285593220339 12.0475742454149
465.946610169491 13.3404457346604
470.607627118644 14.7394042977819
475.268644067797 16.2503168748088
479.929661016949 17.8792015668371
484.590677966102 19.6322266713913
489.251694915254 21.5157101004182
493.912711864407 23.53611924974
498.573728813559 25.7000713913868
503.234745762712 28.0143346796471
507.895762711864 30.4858298493258
512.556779661017 33.1216327202799
517.217796610169 35.9289776278006
521.878813559322 38.9152619027171
526.539830508475 42.0880515752612
531.200847457627 45.4550884790218
535.86186440678 49.0242989767954
540.522881355932 52.8038045662037
545.183898305085 56.8019346727322
549.844915254237 61.027242000087
554.50593220339 65.4885208869608
559.166949152542 70.1948292225895
563.827966101695 75.1555146090409
568.488983050847 80.3802456394762
573.15 85.8790494083592
};
\addlegendentry{Sat. curve \ce{H2O}\quad\quad}
\addplot [semithick, black, dashed]
table {%
298.15 64.3424425064156
302.811016949153 71.5805370612018
};
\addlegendentry{Sat. curve \ce{CO2}}

\nextgroupplot[
title={\(z_{CO_2}=\)\qty{0.1}{\mol\percent}},
]
\addplot [semithick, red, mark=*, mark size=1, mark options={solid}, only marks]
table {%
312.133050847458 4.82329466199197
316.79406779661 4.82329466199197
321.455084745763 4.82329466199197
326.116101694915 4.82329466199197
330.777118644068 4.82329466199197
335.43813559322 4.82329466199197
340.099152542373 4.82329466199197
344.760169491525 4.82329466199197
349.421186440678 4.82329466199197
354.082203389831 4.82329466199197
358.743220338983 4.82329466199197
363.404237288136 4.82329466199197
368.065254237288 4.82329466199197
372.726271186441 4.82329466199197
377.387288135593 4.82329466199197
298.15 5.87167174895639
302.811016949153 5.87167174895639
307.472033898305 5.87167174895639
312.133050847458 5.87167174895639
316.79406779661 5.87167174895639
321.455084745763 5.87167174895639
326.116101694915 5.87167174895639
330.777118644068 5.87167174895639
335.43813559322 5.87167174895639
340.099152542373 5.87167174895639
344.760169491525 5.87167174895639
349.421186440678 5.87167174895639
354.082203389831 5.87167174895639
358.743220338983 5.87167174895639
363.404237288136 5.87167174895639
368.065254237288 5.87167174895639
372.726271186441 5.87167174895639
377.387288135593 5.87167174895639
382.048305084746 5.87167174895639
386.709322033898 5.87167174895639
391.370338983051 5.87167174895639
298.15 7.14792098421252
302.811016949153 7.14792098421252
307.472033898305 7.14792098421252
312.133050847458 7.14792098421252
316.79406779661 7.14792098421252
321.455084745763 7.14792098421252
326.116101694915 7.14792098421252
330.777118644068 7.14792098421252
335.43813559322 7.14792098421252
340.099152542373 7.14792098421252
344.760169491525 7.14792098421252
349.421186440678 7.14792098421252
354.082203389831 7.14792098421252
358.743220338983 7.14792098421252
363.404237288136 7.14792098421252
368.065254237288 7.14792098421252
372.726271186441 7.14792098421252
377.387288135593 7.14792098421252
382.048305084746 7.14792098421252
386.709322033898 7.14792098421252
391.370338983051 7.14792098421252
396.031355932203 7.14792098421252
400.692372881356 7.14792098421252
298.15 8.70157198512106
302.811016949153 8.70157198512106
307.472033898305 8.70157198512106
312.133050847458 8.70157198512106
316.79406779661 8.70157198512106
321.455084745763 8.70157198512106
326.116101694915 8.70157198512106
330.777118644068 8.70157198512106
335.43813559322 8.70157198512106
340.099152542373 8.70157198512106
344.760169491525 8.70157198512106
349.421186440678 8.70157198512106
354.082203389831 8.70157198512106
358.743220338983 8.70157198512106
363.404237288136 8.70157198512106
368.065254237288 8.70157198512106
372.726271186441 8.70157198512106
377.387288135593 8.70157198512106
382.048305084746 8.70157198512106
386.709322033898 8.70157198512106
391.370338983051 8.70157198512106
396.031355932203 8.70157198512106
400.692372881356 8.70157198512106
405.353389830508 8.70157198512106
410.014406779661 8.70157198512106
298.15 10.5929199804362
302.811016949153 10.5929199804362
307.472033898305 10.5929199804362
312.133050847458 10.5929199804362
316.79406779661 10.5929199804362
321.455084745763 10.5929199804362
326.116101694915 10.5929199804362
330.777118644068 10.5929199804362
335.43813559322 10.5929199804362
340.099152542373 10.5929199804362
344.760169491525 10.5929199804362
349.421186440678 10.5929199804362
354.082203389831 10.5929199804362
358.743220338983 10.5929199804362
363.404237288136 10.5929199804362
368.065254237288 10.5929199804362
372.726271186441 10.5929199804362
377.387288135593 10.5929199804362
382.048305084746 10.5929199804362
386.709322033898 10.5929199804362
391.370338983051 10.5929199804362
396.031355932203 10.5929199804362
400.692372881356 10.5929199804362
405.353389830508 10.5929199804362
410.014406779661 10.5929199804362
414.675423728814 10.5929199804362
419.336440677966 10.5929199804362
298.15 12.8953657918124
302.811016949153 12.8953657918124
307.472033898305 12.8953657918124
312.133050847458 12.8953657918124
316.79406779661 12.8953657918124
321.455084745763 12.8953657918124
326.116101694915 12.8953657918124
330.777118644068 12.8953657918124
335.43813559322 12.8953657918124
340.099152542373 12.8953657918124
344.760169491525 12.8953657918124
349.421186440678 12.8953657918124
354.082203389831 12.8953657918124
358.743220338983 12.8953657918124
363.404237288136 12.8953657918124
368.065254237288 12.8953657918124
372.726271186441 12.8953657918124
377.387288135593 12.8953657918124
382.048305084746 12.8953657918124
386.709322033898 12.8953657918124
391.370338983051 12.8953657918124
396.031355932203 12.8953657918124
400.692372881356 12.8953657918124
405.353389830508 12.8953657918124
410.014406779661 12.8953657918124
414.675423728814 12.8953657918124
419.336440677966 12.8953657918124
423.997457627119 12.8953657918124
428.658474576271 12.8953657918124
298.15 15.6982644267836
302.811016949153 15.6982644267836
307.472033898305 15.6982644267836
312.133050847458 15.6982644267836
316.79406779661 15.6982644267836
321.455084745763 15.6982644267836
326.116101694915 15.6982644267836
330.777118644068 15.6982644267836
335.43813559322 15.6982644267836
340.099152542373 15.6982644267836
344.760169491525 15.6982644267836
349.421186440678 15.6982644267836
354.082203389831 15.6982644267836
358.743220338983 15.6982644267836
363.404237288136 15.6982644267836
368.065254237288 15.6982644267836
372.726271186441 15.6982644267836
377.387288135593 15.6982644267836
382.048305084746 15.6982644267836
386.709322033898 15.6982644267836
391.370338983051 15.6982644267836
396.031355932203 15.6982644267836
400.692372881356 15.6982644267836
405.353389830508 15.6982644267836
410.014406779661 15.6982644267836
414.675423728814 15.6982644267836
419.336440677966 15.6982644267836
423.997457627119 15.6982644267836
428.658474576271 15.6982644267836
433.319491525424 15.6982644267836
437.980508474576 15.6982644267836
298.15 19.1103928335005
302.811016949153 19.1103928335005
307.472033898305 19.1103928335005
312.133050847458 19.1103928335005
316.79406779661 19.1103928335005
321.455084745763 19.1103928335005
326.116101694915 19.1103928335005
330.777118644068 19.1103928335005
335.43813559322 19.1103928335005
340.099152542373 19.1103928335005
344.760169491525 19.1103928335005
349.421186440678 19.1103928335005
354.082203389831 19.1103928335005
358.743220338983 19.1103928335005
363.404237288136 19.1103928335005
368.065254237288 19.1103928335005
372.726271186441 19.1103928335005
377.387288135593 19.1103928335005
382.048305084746 19.1103928335005
386.709322033898 19.1103928335005
391.370338983051 19.1103928335005
396.031355932203 19.1103928335005
400.692372881356 19.1103928335005
405.353389830508 19.1103928335005
410.014406779661 19.1103928335005
414.675423728814 19.1103928335005
419.336440677966 19.1103928335005
423.997457627119 19.1103928335005
428.658474576271 19.1103928335005
433.319491525424 19.1103928335005
437.980508474576 19.1103928335005
442.641525423729 19.1103928335005
447.302542372881 19.1103928335005
298.15 23.2641713964002
302.811016949153 23.2641713964002
307.472033898305 23.2641713964002
312.133050847458 23.2641713964002
316.79406779661 23.2641713964002
321.455084745763 23.2641713964002
326.116101694915 23.2641713964002
330.777118644068 23.2641713964002
335.43813559322 23.2641713964002
340.099152542373 23.2641713964002
344.760169491525 23.2641713964002
349.421186440678 23.2641713964002
354.082203389831 23.2641713964002
358.743220338983 23.2641713964002
363.404237288136 23.2641713964002
368.065254237288 23.2641713964002
372.726271186441 23.2641713964002
377.387288135593 23.2641713964002
382.048305084746 23.2641713964002
386.709322033898 23.2641713964002
391.370338983051 23.2641713964002
396.031355932203 23.2641713964002
400.692372881356 23.2641713964002
405.353389830508 23.2641713964002
410.014406779661 23.2641713964002
414.675423728814 23.2641713964002
419.336440677966 23.2641713964002
423.997457627119 23.2641713964002
428.658474576271 23.2641713964002
433.319491525424 23.2641713964002
437.980508474576 23.2641713964002
442.641525423729 23.2641713964002
447.302542372881 23.2641713964002
451.963559322034 23.2641713964002
456.624576271186 23.2641713964002
461.285593220339 23.2641713964002
298.15 28.3208030037104
302.811016949153 28.3208030037104
307.472033898305 28.3208030037104
312.133050847458 28.3208030037104
316.79406779661 28.3208030037104
321.455084745763 28.3208030037104
326.116101694915 28.3208030037104
330.777118644068 28.3208030037104
335.43813559322 28.3208030037104
340.099152542373 28.3208030037104
344.760169491525 28.3208030037104
349.421186440678 28.3208030037104
354.082203389831 28.3208030037104
358.743220338983 28.3208030037104
363.404237288136 28.3208030037104
368.065254237288 28.3208030037104
372.726271186441 28.3208030037104
377.387288135593 28.3208030037104
382.048305084746 28.3208030037104
386.709322033898 28.3208030037104
391.370338983051 28.3208030037104
396.031355932203 28.3208030037104
400.692372881356 28.3208030037104
405.353389830508 28.3208030037104
410.014406779661 28.3208030037104
414.675423728814 28.3208030037104
419.336440677966 28.3208030037104
423.997457627119 28.3208030037104
428.658474576271 28.3208030037104
433.319491525424 28.3208030037104
437.980508474576 28.3208030037104
442.641525423729 28.3208030037104
447.302542372881 28.3208030037104
451.963559322034 28.3208030037104
456.624576271186 28.3208030037104
461.285593220339 28.3208030037104
465.946610169491 28.3208030037104
470.607627118644 28.3208030037104
298.15 34.4765291274926
302.811016949153 34.4765291274926
307.472033898305 34.4765291274926
312.133050847458 34.4765291274926
316.79406779661 34.4765291274926
321.455084745763 34.4765291274926
326.116101694915 34.4765291274926
330.777118644068 34.4765291274926
335.43813559322 34.4765291274926
340.099152542373 34.4765291274926
344.760169491525 34.4765291274926
349.421186440678 34.4765291274926
354.082203389831 34.4765291274926
358.743220338983 34.4765291274926
363.404237288136 34.4765291274926
368.065254237288 34.4765291274926
372.726271186441 34.4765291274926
377.387288135593 34.4765291274926
382.048305084746 34.4765291274926
386.709322033898 34.4765291274926
391.370338983051 34.4765291274926
396.031355932203 34.4765291274926
400.692372881356 34.4765291274926
405.353389830508 34.4765291274926
410.014406779661 34.4765291274926
414.675423728814 34.4765291274926
419.336440677966 34.4765291274926
423.997457627119 34.4765291274926
428.658474576271 34.4765291274926
433.319491525424 34.4765291274926
437.980508474576 34.4765291274926
442.641525423729 34.4765291274926
447.302542372881 34.4765291274926
451.963559322034 34.4765291274926
456.624576271186 34.4765291274926
461.285593220339 34.4765291274926
465.946610169491 34.4765291274926
470.607627118644 34.4765291274926
475.268644067797 34.4765291274926
479.929661016949 34.4765291274926
484.590677966102 34.4765291274926
298.15 41.9702457067732
302.811016949153 41.9702457067732
307.472033898305 41.9702457067732
312.133050847458 41.9702457067732
316.79406779661 41.9702457067732
321.455084745763 41.9702457067732
326.116101694915 41.9702457067732
330.777118644068 41.9702457067732
335.43813559322 41.9702457067732
340.099152542373 41.9702457067732
344.760169491525 41.9702457067732
349.421186440678 41.9702457067732
354.082203389831 41.9702457067732
358.743220338983 41.9702457067732
363.404237288136 41.9702457067732
368.065254237288 41.9702457067732
372.726271186441 41.9702457067732
377.387288135593 41.9702457067732
382.048305084746 41.9702457067732
386.709322033898 41.9702457067732
391.370338983051 41.9702457067732
396.031355932203 41.9702457067732
400.692372881356 41.9702457067732
405.353389830508 41.9702457067732
410.014406779661 41.9702457067732
414.675423728814 41.9702457067732
419.336440677966 41.9702457067732
423.997457627119 41.9702457067732
428.658474576271 41.9702457067732
433.319491525424 41.9702457067732
437.980508474576 41.9702457067732
442.641525423729 41.9702457067732
447.302542372881 41.9702457067732
451.963559322034 41.9702457067732
456.624576271186 41.9702457067732
461.285593220339 41.9702457067732
465.946610169491 41.9702457067732
470.607627118644 41.9702457067732
475.268644067797 41.9702457067732
479.929661016949 41.9702457067732
484.590677966102 41.9702457067732
489.251694915254 41.9702457067732
493.912711864407 41.9702457067732
498.573728813559 41.9702457067732
298.15 51.0927743965457
302.811016949153 51.0927743965457
307.472033898305 51.0927743965457
312.133050847458 51.0927743965457
316.79406779661 51.0927743965457
321.455084745763 51.0927743965457
326.116101694915 51.0927743965457
330.777118644068 51.0927743965457
335.43813559322 51.0927743965457
340.099152542373 51.0927743965457
344.760169491525 51.0927743965457
349.421186440678 51.0927743965457
354.082203389831 51.0927743965457
358.743220338983 51.0927743965457
363.404237288136 51.0927743965457
368.065254237288 51.0927743965457
372.726271186441 51.0927743965457
377.387288135593 51.0927743965457
382.048305084746 51.0927743965457
386.709322033898 51.0927743965457
391.370338983051 51.0927743965457
396.031355932203 51.0927743965457
400.692372881356 51.0927743965457
405.353389830508 51.0927743965457
410.014406779661 51.0927743965457
414.675423728814 51.0927743965457
419.336440677966 51.0927743965457
423.997457627119 51.0927743965457
428.658474576271 51.0927743965457
433.319491525424 51.0927743965457
437.980508474576 51.0927743965457
442.641525423729 51.0927743965457
447.302542372881 51.0927743965457
451.963559322034 51.0927743965457
456.624576271186 51.0927743965457
461.285593220339 51.0927743965457
465.946610169491 51.0927743965457
470.607627118644 51.0927743965457
475.268644067797 51.0927743965457
479.929661016949 51.0927743965457
484.590677966102 51.0927743965457
489.251694915254 51.0927743965457
493.912711864407 51.0927743965457
498.573728813559 51.0927743965457
503.234745762712 51.0927743965457
507.895762711864 51.0927743965457
298.15 62.1981489880827
302.811016949153 62.1981489880827
307.472033898305 62.1981489880827
312.133050847458 62.1981489880827
316.79406779661 62.1981489880827
321.455084745763 62.1981489880827
326.116101694915 62.1981489880827
330.777118644068 62.1981489880827
335.43813559322 62.1981489880827
340.099152542373 62.1981489880827
344.760169491525 62.1981489880827
349.421186440678 62.1981489880827
354.082203389831 62.1981489880827
358.743220338983 62.1981489880827
363.404237288136 62.1981489880827
368.065254237288 62.1981489880827
372.726271186441 62.1981489880827
377.387288135593 62.1981489880827
382.048305084746 62.1981489880827
386.709322033898 62.1981489880827
391.370338983051 62.1981489880827
396.031355932203 62.1981489880827
400.692372881356 62.1981489880827
405.353389830508 62.1981489880827
410.014406779661 62.1981489880827
414.675423728814 62.1981489880827
419.336440677966 62.1981489880827
423.997457627119 62.1981489880827
428.658474576271 62.1981489880827
433.319491525424 62.1981489880827
437.980508474576 62.1981489880827
442.641525423729 62.1981489880827
447.302542372881 62.1981489880827
451.963559322034 62.1981489880827
456.624576271186 62.1981489880827
461.285593220339 62.1981489880827
465.946610169491 62.1981489880827
470.607627118644 62.1981489880827
475.268644067797 62.1981489880827
479.929661016949 62.1981489880827
484.590677966102 62.1981489880827
489.251694915254 62.1981489880827
493.912711864407 62.1981489880827
498.573728813559 62.1981489880827
503.234745762712 62.1981489880827
507.895762711864 62.1981489880827
512.556779661017 62.1981489880827
517.217796610169 62.1981489880827
521.878813559322 62.1981489880827
526.539830508475 62.1981489880827
298.15 75.7173550122436
302.811016949153 75.7173550122436
307.472033898305 75.7173550122436
312.133050847458 75.7173550122436
316.79406779661 75.7173550122436
321.455084745763 75.7173550122436
326.116101694915 75.7173550122436
330.777118644068 75.7173550122436
335.43813559322 75.7173550122436
340.099152542373 75.7173550122436
344.760169491525 75.7173550122436
349.421186440678 75.7173550122436
354.082203389831 75.7173550122436
358.743220338983 75.7173550122436
363.404237288136 75.7173550122436
368.065254237288 75.7173550122436
372.726271186441 75.7173550122436
377.387288135593 75.7173550122436
382.048305084746 75.7173550122436
386.709322033898 75.7173550122436
391.370338983051 75.7173550122436
396.031355932203 75.7173550122436
400.692372881356 75.7173550122436
405.353389830508 75.7173550122436
410.014406779661 75.7173550122436
414.675423728814 75.7173550122436
419.336440677966 75.7173550122436
423.997457627119 75.7173550122436
428.658474576271 75.7173550122436
433.319491525424 75.7173550122436
437.980508474576 75.7173550122436
442.641525423729 75.7173550122436
447.302542372881 75.7173550122436
451.963559322034 75.7173550122436
456.624576271186 75.7173550122436
461.285593220339 75.7173550122436
465.946610169491 75.7173550122436
470.607627118644 75.7173550122436
475.268644067797 75.7173550122436
479.929661016949 75.7173550122436
484.590677966102 75.7173550122436
489.251694915254 75.7173550122436
493.912711864407 75.7173550122436
498.573728813559 75.7173550122436
503.234745762712 75.7173550122436
507.895762711864 75.7173550122436
512.556779661017 75.7173550122436
517.217796610169 75.7173550122436
521.878813559322 75.7173550122436
526.539830508475 75.7173550122436
531.200847457627 75.7173550122436
535.86186440678 75.7173550122436
540.522881355932 75.7173550122436
298.15 92.1750557423923
302.811016949153 92.1750557423923
307.472033898305 92.1750557423923
312.133050847458 92.1750557423923
316.79406779661 92.1750557423923
321.455084745763 92.1750557423923
326.116101694915 92.1750557423923
330.777118644068 92.1750557423923
335.43813559322 92.1750557423923
340.099152542373 92.1750557423923
344.760169491525 92.1750557423923
349.421186440678 92.1750557423923
354.082203389831 92.1750557423923
358.743220338983 92.1750557423923
363.404237288136 92.1750557423923
368.065254237288 92.1750557423923
372.726271186441 92.1750557423923
377.387288135593 92.1750557423923
382.048305084746 92.1750557423923
386.709322033898 92.1750557423923
391.370338983051 92.1750557423923
396.031355932203 92.1750557423923
400.692372881356 92.1750557423923
405.353389830508 92.1750557423923
410.014406779661 92.1750557423923
414.675423728814 92.1750557423923
419.336440677966 92.1750557423923
423.997457627119 92.1750557423923
428.658474576271 92.1750557423923
433.319491525424 92.1750557423923
437.980508474576 92.1750557423923
442.641525423729 92.1750557423923
447.302542372881 92.1750557423923
451.963559322034 92.1750557423923
456.624576271186 92.1750557423923
461.285593220339 92.1750557423923
465.946610169491 92.1750557423923
470.607627118644 92.1750557423923
475.268644067797 92.1750557423923
479.929661016949 92.1750557423923
484.590677966102 92.1750557423923
489.251694915254 92.1750557423923
493.912711864407 92.1750557423923
498.573728813559 92.1750557423923
503.234745762712 92.1750557423923
507.895762711864 92.1750557423923
512.556779661017 92.1750557423923
517.217796610169 92.1750557423923
521.878813559322 92.1750557423923
526.539830508475 92.1750557423923
531.200847457627 92.1750557423923
535.86186440678 92.1750557423923
540.522881355932 92.1750557423923
545.183898305085 92.1750557423923
549.844915254237 92.1750557423923
554.50593220339 92.1750557423923
559.166949152542 92.1750557423923
298.15 112.209953711923
302.811016949153 112.209953711923
307.472033898305 112.209953711923
312.133050847458 112.209953711923
316.79406779661 112.209953711923
321.455084745763 112.209953711923
326.116101694915 112.209953711923
330.777118644068 112.209953711923
335.43813559322 112.209953711923
340.099152542373 112.209953711923
344.760169491525 112.209953711923
349.421186440678 112.209953711923
354.082203389831 112.209953711923
358.743220338983 112.209953711923
363.404237288136 112.209953711923
368.065254237288 112.209953711923
372.726271186441 112.209953711923
377.387288135593 112.209953711923
382.048305084746 112.209953711923
386.709322033898 112.209953711923
391.370338983051 112.209953711923
396.031355932203 112.209953711923
400.692372881356 112.209953711923
405.353389830508 112.209953711923
410.014406779661 112.209953711923
414.675423728814 112.209953711923
419.336440677966 112.209953711923
423.997457627119 112.209953711923
428.658474576271 112.209953711923
433.319491525424 112.209953711923
437.980508474576 112.209953711923
442.641525423729 112.209953711923
447.302542372881 112.209953711923
451.963559322034 112.209953711923
456.624576271186 112.209953711923
461.285593220339 112.209953711923
465.946610169491 112.209953711923
470.607627118644 112.209953711923
475.268644067797 112.209953711923
479.929661016949 112.209953711923
484.590677966102 112.209953711923
489.251694915254 112.209953711923
493.912711864407 112.209953711923
498.573728813559 112.209953711923
503.234745762712 112.209953711923
507.895762711864 112.209953711923
512.556779661017 112.209953711923
517.217796610169 112.209953711923
521.878813559322 112.209953711923
526.539830508475 112.209953711923
531.200847457627 112.209953711923
535.86186440678 112.209953711923
540.522881355932 112.209953711923
545.183898305085 112.209953711923
549.844915254237 112.209953711923
554.50593220339 112.209953711923
559.166949152542 112.209953711923
563.827966101695 112.209953711923
568.488983050847 112.209953711923
573.15 112.209953711923
298.15 136.599577951123
302.811016949153 136.599577951123
307.472033898305 136.599577951123
312.133050847458 136.599577951123
316.79406779661 136.599577951123
321.455084745763 136.599577951123
326.116101694915 136.599577951123
330.777118644068 136.599577951123
335.43813559322 136.599577951123
340.099152542373 136.599577951123
344.760169491525 136.599577951123
349.421186440678 136.599577951123
354.082203389831 136.599577951123
358.743220338983 136.599577951123
363.404237288136 136.599577951123
368.065254237288 136.599577951123
372.726271186441 136.599577951123
377.387288135593 136.599577951123
382.048305084746 136.599577951123
386.709322033898 136.599577951123
391.370338983051 136.599577951123
396.031355932203 136.599577951123
400.692372881356 136.599577951123
405.353389830508 136.599577951123
410.014406779661 136.599577951123
414.675423728814 136.599577951123
419.336440677966 136.599577951123
423.997457627119 136.599577951123
428.658474576271 136.599577951123
433.319491525424 136.599577951123
437.980508474576 136.599577951123
442.641525423729 136.599577951123
447.302542372881 136.599577951123
451.963559322034 136.599577951123
456.624576271186 136.599577951123
461.285593220339 136.599577951123
465.946610169491 136.599577951123
470.607627118644 136.599577951123
475.268644067797 136.599577951123
479.929661016949 136.599577951123
484.590677966102 136.599577951123
489.251694915254 136.599577951123
493.912711864407 136.599577951123
498.573728813559 136.599577951123
503.234745762712 136.599577951123
507.895762711864 136.599577951123
512.556779661017 136.599577951123
517.217796610169 136.599577951123
521.878813559322 136.599577951123
526.539830508475 136.599577951123
531.200847457627 136.599577951123
535.86186440678 136.599577951123
540.522881355932 136.599577951123
545.183898305085 136.599577951123
549.844915254237 136.599577951123
554.50593220339 136.599577951123
559.166949152542 136.599577951123
563.827966101695 136.599577951123
568.488983050847 136.599577951123
573.15 136.599577951123
298.15 166.290458904646
302.811016949153 166.290458904646
307.472033898305 166.290458904646
312.133050847458 166.290458904646
316.79406779661 166.290458904646
321.455084745763 166.290458904646
326.116101694915 166.290458904646
330.777118644068 166.290458904646
335.43813559322 166.290458904646
340.099152542373 166.290458904646
344.760169491525 166.290458904646
349.421186440678 166.290458904646
354.082203389831 166.290458904646
358.743220338983 166.290458904646
363.404237288136 166.290458904646
368.065254237288 166.290458904646
372.726271186441 166.290458904646
377.387288135593 166.290458904646
382.048305084746 166.290458904646
386.709322033898 166.290458904646
391.370338983051 166.290458904646
396.031355932203 166.290458904646
400.692372881356 166.290458904646
405.353389830508 166.290458904646
410.014406779661 166.290458904646
414.675423728814 166.290458904646
419.336440677966 166.290458904646
423.997457627119 166.290458904646
428.658474576271 166.290458904646
433.319491525424 166.290458904646
437.980508474576 166.290458904646
442.641525423729 166.290458904646
447.302542372881 166.290458904646
451.963559322034 166.290458904646
456.624576271186 166.290458904646
461.285593220339 166.290458904646
465.946610169491 166.290458904646
470.607627118644 166.290458904646
475.268644067797 166.290458904646
479.929661016949 166.290458904646
484.590677966102 166.290458904646
489.251694915254 166.290458904646
493.912711864407 166.290458904646
498.573728813559 166.290458904646
503.234745762712 166.290458904646
507.895762711864 166.290458904646
512.556779661017 166.290458904646
517.217796610169 166.290458904646
521.878813559322 166.290458904646
526.539830508475 166.290458904646
531.200847457627 166.290458904646
535.86186440678 166.290458904646
540.522881355932 166.290458904646
545.183898305085 166.290458904646
549.844915254237 166.290458904646
554.50593220339 166.290458904646
559.166949152542 166.290458904646
563.827966101695 166.290458904646
568.488983050847 166.290458904646
573.15 166.290458904646
298.15 202.434862079971
302.811016949153 202.434862079971
307.472033898305 202.434862079971
312.133050847458 202.434862079971
316.79406779661 202.434862079971
321.455084745763 202.434862079971
326.116101694915 202.434862079971
330.777118644068 202.434862079971
335.43813559322 202.434862079971
340.099152542373 202.434862079971
344.760169491525 202.434862079971
349.421186440678 202.434862079971
354.082203389831 202.434862079971
358.743220338983 202.434862079971
363.404237288136 202.434862079971
368.065254237288 202.434862079971
372.726271186441 202.434862079971
377.387288135593 202.434862079971
382.048305084746 202.434862079971
386.709322033898 202.434862079971
391.370338983051 202.434862079971
396.031355932203 202.434862079971
400.692372881356 202.434862079971
405.353389830508 202.434862079971
410.014406779661 202.434862079971
414.675423728814 202.434862079971
419.336440677966 202.434862079971
423.997457627119 202.434862079971
428.658474576271 202.434862079971
433.319491525424 202.434862079971
437.980508474576 202.434862079971
442.641525423729 202.434862079971
447.302542372881 202.434862079971
451.963559322034 202.434862079971
456.624576271186 202.434862079971
461.285593220339 202.434862079971
465.946610169491 202.434862079971
470.607627118644 202.434862079971
475.268644067797 202.434862079971
479.929661016949 202.434862079971
484.590677966102 202.434862079971
489.251694915254 202.434862079971
493.912711864407 202.434862079971
498.573728813559 202.434862079971
503.234745762712 202.434862079971
507.895762711864 202.434862079971
512.556779661017 202.434862079971
517.217796610169 202.434862079971
521.878813559322 202.434862079971
526.539830508475 202.434862079971
531.200847457627 202.434862079971
535.86186440678 202.434862079971
540.522881355932 202.434862079971
545.183898305085 202.434862079971
549.844915254237 202.434862079971
554.50593220339 202.434862079971
559.166949152542 202.434862079971
563.827966101695 202.434862079971
568.488983050847 202.434862079971
573.15 202.434862079971
298.15 246.435506013219
302.811016949153 246.435506013219
307.472033898305 246.435506013219
312.133050847458 246.435506013219
316.79406779661 246.435506013219
321.455084745763 246.435506013219
326.116101694915 246.435506013219
330.777118644068 246.435506013219
335.43813559322 246.435506013219
340.099152542373 246.435506013219
344.760169491525 246.435506013219
349.421186440678 246.435506013219
354.082203389831 246.435506013219
358.743220338983 246.435506013219
363.404237288136 246.435506013219
368.065254237288 246.435506013219
372.726271186441 246.435506013219
377.387288135593 246.435506013219
382.048305084746 246.435506013219
386.709322033898 246.435506013219
391.370338983051 246.435506013219
396.031355932203 246.435506013219
400.692372881356 246.435506013219
405.353389830508 246.435506013219
410.014406779661 246.435506013219
414.675423728814 246.435506013219
419.336440677966 246.435506013219
423.997457627119 246.435506013219
428.658474576271 246.435506013219
433.319491525424 246.435506013219
437.980508474576 246.435506013219
442.641525423729 246.435506013219
447.302542372881 246.435506013219
451.963559322034 246.435506013219
456.624576271186 246.435506013219
461.285593220339 246.435506013219
465.946610169491 246.435506013219
470.607627118644 246.435506013219
475.268644067797 246.435506013219
479.929661016949 246.435506013219
484.590677966102 246.435506013219
489.251694915254 246.435506013219
493.912711864407 246.435506013219
498.573728813559 246.435506013219
503.234745762712 246.435506013219
507.895762711864 246.435506013219
512.556779661017 246.435506013219
517.217796610169 246.435506013219
521.878813559322 246.435506013219
526.539830508475 246.435506013219
531.200847457627 246.435506013219
535.86186440678 246.435506013219
540.522881355932 246.435506013219
545.183898305085 246.435506013219
549.844915254237 246.435506013219
554.50593220339 246.435506013219
559.166949152542 246.435506013219
563.827966101695 246.435506013219
568.488983050847 246.435506013219
573.15 246.435506013219
298.15 300
302.811016949153 300
307.472033898305 300
312.133050847458 300
316.79406779661 300
321.455084745763 300
326.116101694915 300
330.777118644068 300
335.43813559322 300
340.099152542373 300
344.760169491525 300
349.421186440678 300
354.082203389831 300
358.743220338983 300
363.404237288136 300
368.065254237288 300
372.726271186441 300
377.387288135593 300
382.048305084746 300
386.709322033898 300
391.370338983051 300
396.031355932203 300
400.692372881356 300
405.353389830508 300
410.014406779661 300
414.675423728814 300
419.336440677966 300
423.997457627119 300
428.658474576271 300
433.319491525424 300
437.980508474576 300
442.641525423729 300
447.302542372881 300
451.963559322034 300
456.624576271186 300
461.285593220339 300
465.946610169491 300
470.607627118644 300
475.268644067797 300
479.929661016949 300
484.590677966102 300
489.251694915254 300
493.912711864407 300
498.573728813559 300
503.234745762712 300
507.895762711864 300
512.556779661017 300
517.217796610169 300
521.878813559322 300
526.539830508475 300
531.200847457627 300
535.86186440678 300
540.522881355932 300
545.183898305085 300
549.844915254237 300
554.50593220339 300
559.166949152542 300
563.827966101695 300
568.488983050847 300
573.15 300
};
\addplot [semithick, black]
table {%
298.15 0.0316992938881239
302.811016949153 0.0416502526748325
307.472033898305 0.0542135031209127
312.133050847458 0.0699379446480481
316.79406779661 0.0894573631288235
321.455084745763 0.11349814624938
326.116101694915 0.142886977832051
330.777118644068 0.178558427224653
335.43813559322 0.221562362278131
340.099152542373 0.273071114093698
344.760169491525 0.334386329262953
349.421186440678 0.406945451585077
354.082203389831 0.492327780973097
358.743220338983 0.592260071535126
363.404237288136 0.708621631256037
368.065254237288 0.843448901418998
372.726271186441 0.998939497096534
377.387288135593 1.17745570205744
382.048305084746 1.38152741613099
386.709322033898 1.6138545617758
391.370338983051 1.87730896217833
396.031355932203 2.17493570898292
400.692372881356 2.50995404318333
405.353389830508 2.88575777644233
410.014406779661 3.30591528414528
414.675423728814 3.77416910593384
419.336440677966 4.29443518921697
423.997457627119 4.87080181690722
428.658474576271 5.50752826050788
433.319491525424 6.20904320199238
437.980508474576 6.97994296809687
442.641525423729 7.82498962397373
447.302542372881 8.74910897141505
451.963559322034 9.75738849941128
456.624576271186 10.8550753353115
461.285593220339 12.0475742454149
465.946610169491 13.3404457346604
470.607627118644 14.7394042977819
475.268644067797 16.2503168748088
479.929661016949 17.8792015668371
484.590677966102 19.6322266713913
489.251694915254 21.5157101004182
493.912711864407 23.53611924974
498.573728813559 25.7000713913868
503.234745762712 28.0143346796471
507.895762711864 30.4858298493258
512.556779661017 33.1216327202799
517.217796610169 35.9289776278006
521.878813559322 38.9152619027171
526.539830508475 42.0880515752612
531.200847457627 45.4550884790218
535.86186440678 49.0242989767954
540.522881355932 52.8038045662037
545.183898305085 56.8019346727322
549.844915254237 61.027242000087
554.50593220339 65.4885208869608
559.166949152542 70.1948292225895
563.827966101695 75.1555146090409
568.488983050847 80.3802456394762
573.15 85.8790494083592
};
\addplot [semithick, black, dashed]
table {%
298.15 64.3424425064156
302.811016949153 71.5805370612018
};

\nextgroupplot[
title={\(z_{CO_2}=\)\qty{1.0}{\mol\percent}},
]
\addplot [semithick, red, mark=*, mark size=1, mark options={solid}, only marks]
table {%
363.404237288136 41.9702457067732
368.065254237288 41.9702457067732
372.726271186441 41.9702457067732
377.387288135593 41.9702457067732
335.43813559322 51.0927743965457
340.099152542373 51.0927743965457
344.760169491525 51.0927743965457
349.421186440678 51.0927743965457
354.082203389831 51.0927743965457
358.743220338983 51.0927743965457
363.404237288136 51.0927743965457
368.065254237288 51.0927743965457
372.726271186441 51.0927743965457
377.387288135593 51.0927743965457
321.455084745763 62.1981489880827
326.116101694915 62.1981489880827
330.777118644068 62.1981489880827
335.43813559322 62.1981489880827
340.099152542373 62.1981489880827
344.760169491525 62.1981489880827
349.421186440678 62.1981489880827
354.082203389831 62.1981489880827
358.743220338983 62.1981489880827
363.404237288136 62.1981489880827
368.065254237288 62.1981489880827
372.726271186441 62.1981489880827
377.387288135593 62.1981489880827
302.811016949153 75.7173550122436
312.133050847458 75.7173550122436
316.79406779661 75.7173550122436
321.455084745763 75.7173550122436
326.116101694915 75.7173550122436
330.777118644068 75.7173550122436
335.43813559322 75.7173550122436
340.099152542373 75.7173550122436
344.760169491525 75.7173550122436
349.421186440678 75.7173550122436
354.082203389831 75.7173550122436
358.743220338983 75.7173550122436
363.404237288136 75.7173550122436
368.065254237288 75.7173550122436
372.726271186441 75.7173550122436
377.387288135593 75.7173550122436
382.048305084746 75.7173550122436
386.709322033898 75.7173550122436
391.370338983051 75.7173550122436
396.031355932203 75.7173550122436
400.692372881356 75.7173550122436
405.353389830508 75.7173550122436
410.014406779661 75.7173550122436
414.675423728814 75.7173550122436
419.336440677966 75.7173550122436
423.997457627119 75.7173550122436
428.658474576271 75.7173550122436
433.319491525424 75.7173550122436
437.980508474576 75.7173550122436
442.641525423729 75.7173550122436
447.302542372881 75.7173550122436
451.963559322034 75.7173550122436
456.624576271186 75.7173550122436
461.285593220339 75.7173550122436
465.946610169491 75.7173550122436
470.607627118644 75.7173550122436
475.268644067797 75.7173550122436
479.929661016949 75.7173550122436
484.590677966102 75.7173550122436
489.251694915254 75.7173550122436
493.912711864407 75.7173550122436
498.573728813559 75.7173550122436
503.234745762712 75.7173550122436
302.811016949153 92.1750557423923
307.472033898305 92.1750557423923
312.133050847458 92.1750557423923
316.79406779661 92.1750557423923
321.455084745763 92.1750557423923
326.116101694915 92.1750557423923
330.777118644068 92.1750557423923
335.43813559322 92.1750557423923
340.099152542373 92.1750557423923
344.760169491525 92.1750557423923
349.421186440678 92.1750557423923
354.082203389831 92.1750557423923
358.743220338983 92.1750557423923
363.404237288136 92.1750557423923
368.065254237288 92.1750557423923
372.726271186441 92.1750557423923
377.387288135593 92.1750557423923
382.048305084746 92.1750557423923
386.709322033898 92.1750557423923
391.370338983051 92.1750557423923
396.031355932203 92.1750557423923
400.692372881356 92.1750557423923
405.353389830508 92.1750557423923
410.014406779661 92.1750557423923
414.675423728814 92.1750557423923
419.336440677966 92.1750557423923
423.997457627119 92.1750557423923
428.658474576271 92.1750557423923
433.319491525424 92.1750557423923
437.980508474576 92.1750557423923
442.641525423729 92.1750557423923
447.302542372881 92.1750557423923
451.963559322034 92.1750557423923
456.624576271186 92.1750557423923
461.285593220339 92.1750557423923
465.946610169491 92.1750557423923
470.607627118644 92.1750557423923
475.268644067797 92.1750557423923
479.929661016949 92.1750557423923
484.590677966102 92.1750557423923
489.251694915254 92.1750557423923
493.912711864407 92.1750557423923
498.573728813559 92.1750557423923
503.234745762712 92.1750557423923
507.895762711864 92.1750557423923
512.556779661017 92.1750557423923
517.217796610169 92.1750557423923
521.878813559322 92.1750557423923
526.539830508475 92.1750557423923
531.200847457627 92.1750557423923
535.86186440678 92.1750557423923
540.522881355932 92.1750557423923
298.15 112.209953711923
307.472033898305 112.209953711923
312.133050847458 112.209953711923
316.79406779661 112.209953711923
321.455084745763 112.209953711923
326.116101694915 112.209953711923
330.777118644068 112.209953711923
335.43813559322 112.209953711923
340.099152542373 112.209953711923
344.760169491525 112.209953711923
349.421186440678 112.209953711923
354.082203389831 112.209953711923
358.743220338983 112.209953711923
363.404237288136 112.209953711923
368.065254237288 112.209953711923
372.726271186441 112.209953711923
377.387288135593 112.209953711923
382.048305084746 112.209953711923
386.709322033898 112.209953711923
391.370338983051 112.209953711923
396.031355932203 112.209953711923
400.692372881356 112.209953711923
405.353389830508 112.209953711923
410.014406779661 112.209953711923
414.675423728814 112.209953711923
419.336440677966 112.209953711923
423.997457627119 112.209953711923
428.658474576271 112.209953711923
433.319491525424 112.209953711923
437.980508474576 112.209953711923
442.641525423729 112.209953711923
447.302542372881 112.209953711923
451.963559322034 112.209953711923
456.624576271186 112.209953711923
461.285593220339 112.209953711923
465.946610169491 112.209953711923
470.607627118644 112.209953711923
475.268644067797 112.209953711923
479.929661016949 112.209953711923
484.590677966102 112.209953711923
489.251694915254 112.209953711923
493.912711864407 112.209953711923
498.573728813559 112.209953711923
503.234745762712 112.209953711923
507.895762711864 112.209953711923
512.556779661017 112.209953711923
517.217796610169 112.209953711923
521.878813559322 112.209953711923
526.539830508475 112.209953711923
531.200847457627 112.209953711923
535.86186440678 112.209953711923
540.522881355932 112.209953711923
545.183898305085 112.209953711923
549.844915254237 112.209953711923
554.50593220339 112.209953711923
559.166949152542 112.209953711923
563.827966101695 112.209953711923
298.15 136.599577951123
302.811016949153 136.599577951123
307.472033898305 136.599577951123
312.133050847458 136.599577951123
316.79406779661 136.599577951123
321.455084745763 136.599577951123
326.116101694915 136.599577951123
330.777118644068 136.599577951123
335.43813559322 136.599577951123
340.099152542373 136.599577951123
344.760169491525 136.599577951123
349.421186440678 136.599577951123
354.082203389831 136.599577951123
358.743220338983 136.599577951123
363.404237288136 136.599577951123
368.065254237288 136.599577951123
372.726271186441 136.599577951123
377.387288135593 136.599577951123
382.048305084746 136.599577951123
386.709322033898 136.599577951123
391.370338983051 136.599577951123
396.031355932203 136.599577951123
400.692372881356 136.599577951123
405.353389830508 136.599577951123
410.014406779661 136.599577951123
414.675423728814 136.599577951123
419.336440677966 136.599577951123
423.997457627119 136.599577951123
428.658474576271 136.599577951123
433.319491525424 136.599577951123
437.980508474576 136.599577951123
442.641525423729 136.599577951123
447.302542372881 136.599577951123
451.963559322034 136.599577951123
456.624576271186 136.599577951123
461.285593220339 136.599577951123
465.946610169491 136.599577951123
470.607627118644 136.599577951123
475.268644067797 136.599577951123
479.929661016949 136.599577951123
484.590677966102 136.599577951123
489.251694915254 136.599577951123
493.912711864407 136.599577951123
498.573728813559 136.599577951123
503.234745762712 136.599577951123
507.895762711864 136.599577951123
512.556779661017 136.599577951123
517.217796610169 136.599577951123
521.878813559322 136.599577951123
526.539830508475 136.599577951123
531.200847457627 136.599577951123
535.86186440678 136.599577951123
540.522881355932 136.599577951123
545.183898305085 136.599577951123
549.844915254237 136.599577951123
554.50593220339 136.599577951123
559.166949152542 136.599577951123
563.827966101695 136.599577951123
568.488983050847 136.599577951123
573.15 136.599577951123
298.15 166.290458904646
302.811016949153 166.290458904646
307.472033898305 166.290458904646
312.133050847458 166.290458904646
316.79406779661 166.290458904646
321.455084745763 166.290458904646
326.116101694915 166.290458904646
330.777118644068 166.290458904646
335.43813559322 166.290458904646
340.099152542373 166.290458904646
344.760169491525 166.290458904646
349.421186440678 166.290458904646
354.082203389831 166.290458904646
358.743220338983 166.290458904646
363.404237288136 166.290458904646
368.065254237288 166.290458904646
372.726271186441 166.290458904646
377.387288135593 166.290458904646
382.048305084746 166.290458904646
386.709322033898 166.290458904646
391.370338983051 166.290458904646
396.031355932203 166.290458904646
400.692372881356 166.290458904646
405.353389830508 166.290458904646
410.014406779661 166.290458904646
414.675423728814 166.290458904646
419.336440677966 166.290458904646
423.997457627119 166.290458904646
428.658474576271 166.290458904646
433.319491525424 166.290458904646
437.980508474576 166.290458904646
442.641525423729 166.290458904646
447.302542372881 166.290458904646
451.963559322034 166.290458904646
456.624576271186 166.290458904646
461.285593220339 166.290458904646
465.946610169491 166.290458904646
470.607627118644 166.290458904646
475.268644067797 166.290458904646
479.929661016949 166.290458904646
484.590677966102 166.290458904646
489.251694915254 166.290458904646
493.912711864407 166.290458904646
498.573728813559 166.290458904646
503.234745762712 166.290458904646
507.895762711864 166.290458904646
512.556779661017 166.290458904646
517.217796610169 166.290458904646
521.878813559322 166.290458904646
526.539830508475 166.290458904646
531.200847457627 166.290458904646
535.86186440678 166.290458904646
540.522881355932 166.290458904646
545.183898305085 166.290458904646
549.844915254237 166.290458904646
554.50593220339 166.290458904646
559.166949152542 166.290458904646
563.827966101695 166.290458904646
568.488983050847 166.290458904646
573.15 166.290458904646
298.15 202.434862079971
302.811016949153 202.434862079971
307.472033898305 202.434862079971
312.133050847458 202.434862079971
316.79406779661 202.434862079971
321.455084745763 202.434862079971
326.116101694915 202.434862079971
330.777118644068 202.434862079971
335.43813559322 202.434862079971
340.099152542373 202.434862079971
344.760169491525 202.434862079971
349.421186440678 202.434862079971
354.082203389831 202.434862079971
358.743220338983 202.434862079971
363.404237288136 202.434862079971
368.065254237288 202.434862079971
372.726271186441 202.434862079971
377.387288135593 202.434862079971
382.048305084746 202.434862079971
386.709322033898 202.434862079971
391.370338983051 202.434862079971
396.031355932203 202.434862079971
400.692372881356 202.434862079971
405.353389830508 202.434862079971
410.014406779661 202.434862079971
414.675423728814 202.434862079971
419.336440677966 202.434862079971
423.997457627119 202.434862079971
428.658474576271 202.434862079971
433.319491525424 202.434862079971
437.980508474576 202.434862079971
442.641525423729 202.434862079971
447.302542372881 202.434862079971
451.963559322034 202.434862079971
456.624576271186 202.434862079971
461.285593220339 202.434862079971
465.946610169491 202.434862079971
470.607627118644 202.434862079971
475.268644067797 202.434862079971
479.929661016949 202.434862079971
484.590677966102 202.434862079971
489.251694915254 202.434862079971
493.912711864407 202.434862079971
498.573728813559 202.434862079971
503.234745762712 202.434862079971
507.895762711864 202.434862079971
512.556779661017 202.434862079971
517.217796610169 202.434862079971
521.878813559322 202.434862079971
526.539830508475 202.434862079971
531.200847457627 202.434862079971
535.86186440678 202.434862079971
540.522881355932 202.434862079971
545.183898305085 202.434862079971
549.844915254237 202.434862079971
554.50593220339 202.434862079971
559.166949152542 202.434862079971
563.827966101695 202.434862079971
568.488983050847 202.434862079971
573.15 202.434862079971
302.811016949153 246.435506013219
307.472033898305 246.435506013219
312.133050847458 246.435506013219
316.79406779661 246.435506013219
321.455084745763 246.435506013219
326.116101694915 246.435506013219
330.777118644068 246.435506013219
335.43813559322 246.435506013219
340.099152542373 246.435506013219
344.760169491525 246.435506013219
349.421186440678 246.435506013219
354.082203389831 246.435506013219
358.743220338983 246.435506013219
363.404237288136 246.435506013219
368.065254237288 246.435506013219
372.726271186441 246.435506013219
377.387288135593 246.435506013219
382.048305084746 246.435506013219
386.709322033898 246.435506013219
391.370338983051 246.435506013219
396.031355932203 246.435506013219
400.692372881356 246.435506013219
405.353389830508 246.435506013219
410.014406779661 246.435506013219
414.675423728814 246.435506013219
419.336440677966 246.435506013219
423.997457627119 246.435506013219
428.658474576271 246.435506013219
433.319491525424 246.435506013219
437.980508474576 246.435506013219
442.641525423729 246.435506013219
447.302542372881 246.435506013219
451.963559322034 246.435506013219
456.624576271186 246.435506013219
461.285593220339 246.435506013219
465.946610169491 246.435506013219
470.607627118644 246.435506013219
475.268644067797 246.435506013219
479.929661016949 246.435506013219
484.590677966102 246.435506013219
489.251694915254 246.435506013219
493.912711864407 246.435506013219
498.573728813559 246.435506013219
503.234745762712 246.435506013219
507.895762711864 246.435506013219
512.556779661017 246.435506013219
517.217796610169 246.435506013219
521.878813559322 246.435506013219
526.539830508475 246.435506013219
531.200847457627 246.435506013219
535.86186440678 246.435506013219
540.522881355932 246.435506013219
545.183898305085 246.435506013219
549.844915254237 246.435506013219
554.50593220339 246.435506013219
559.166949152542 246.435506013219
563.827966101695 246.435506013219
568.488983050847 246.435506013219
573.15 246.435506013219
298.15 300
302.811016949153 300
307.472033898305 300
312.133050847458 300
316.79406779661 300
321.455084745763 300
326.116101694915 300
330.777118644068 300
335.43813559322 300
340.099152542373 300
344.760169491525 300
349.421186440678 300
354.082203389831 300
358.743220338983 300
363.404237288136 300
368.065254237288 300
372.726271186441 300
377.387288135593 300
382.048305084746 300
386.709322033898 300
391.370338983051 300
396.031355932203 300
400.692372881356 300
405.353389830508 300
410.014406779661 300
414.675423728814 300
419.336440677966 300
423.997457627119 300
428.658474576271 300
433.319491525424 300
437.980508474576 300
442.641525423729 300
447.302542372881 300
451.963559322034 300
456.624576271186 300
461.285593220339 300
465.946610169491 300
470.607627118644 300
475.268644067797 300
479.929661016949 300
484.590677966102 300
489.251694915254 300
493.912711864407 300
498.573728813559 300
503.234745762712 300
507.895762711864 300
512.556779661017 300
517.217796610169 300
521.878813559322 300
526.539830508475 300
531.200847457627 300
535.86186440678 300
540.522881355932 300
545.183898305085 300
549.844915254237 300
554.50593220339 300
559.166949152542 300
563.827966101695 300
568.488983050847 300
573.15 300
};
\addplot [semithick, black]
table {%
298.15 0.0316992938881239
302.811016949153 0.0416502526748325
307.472033898305 0.0542135031209127
312.133050847458 0.0699379446480481
316.79406779661 0.0894573631288235
321.455084745763 0.11349814624938
326.116101694915 0.142886977832051
330.777118644068 0.178558427224653
335.43813559322 0.221562362278131
340.099152542373 0.273071114093698
344.760169491525 0.334386329262953
349.421186440678 0.406945451585077
354.082203389831 0.492327780973097
358.743220338983 0.592260071535126
363.404237288136 0.708621631256037
368.065254237288 0.843448901418998
372.726271186441 0.998939497096534
377.387288135593 1.17745570205744
382.048305084746 1.38152741613099
386.709322033898 1.6138545617758
391.370338983051 1.87730896217833
396.031355932203 2.17493570898292
400.692372881356 2.50995404318333
405.353389830508 2.88575777644233
410.014406779661 3.30591528414528
414.675423728814 3.77416910593384
419.336440677966 4.29443518921697
423.997457627119 4.87080181690722
428.658474576271 5.50752826050788
433.319491525424 6.20904320199238
437.980508474576 6.97994296809687
442.641525423729 7.82498962397373
447.302542372881 8.74910897141505
451.963559322034 9.75738849941128
456.624576271186 10.8550753353115
461.285593220339 12.0475742454149
465.946610169491 13.3404457346604
470.607627118644 14.7394042977819
475.268644067797 16.2503168748088
479.929661016949 17.8792015668371
484.590677966102 19.6322266713913
489.251694915254 21.5157101004182
493.912711864407 23.53611924974
498.573728813559 25.7000713913868
503.234745762712 28.0143346796471
507.895762711864 30.4858298493258
512.556779661017 33.1216327202799
517.217796610169 35.9289776278006
521.878813559322 38.9152619027171
526.539830508475 42.0880515752612
531.200847457627 45.4550884790218
535.86186440678 49.0242989767954
540.522881355932 52.8038045662037
545.183898305085 56.8019346727322
549.844915254237 61.027242000087
554.50593220339 65.4885208869608
559.166949152542 70.1948292225895
563.827966101695 75.1555146090409
568.488983050847 80.3802456394762
573.15 85.8790494083592
};
\addplot [semithick, black, dashed]
table {%
298.15 64.3424425064156
302.811016949153 71.5805370612018
};

\nextgroupplot[
title={\(z_{CO_2}=\)\qty{10}{\mol\percent}},
]
\addplot [semithick, red, mark=*, mark size=1, mark options={solid}, only marks]
table {%
302.811016949153 75.7173550122436
302.811016949153 92.1750557423923
307.472033898305 92.1750557423923
298.15 112.209953711923
298.15 136.599577951123
302.811016949153 136.599577951123
307.472033898305 136.599577951123
312.133050847458 136.599577951123
298.15 166.290458904646
302.811016949153 166.290458904646
307.472033898305 166.290458904646
312.133050847458 166.290458904646
316.79406779661 166.290458904646
321.455084745763 166.290458904646
326.116101694915 166.290458904646
330.777118644068 166.290458904646
298.15 202.434862079971
302.811016949153 202.434862079971
307.472033898305 202.434862079971
312.133050847458 202.434862079971
316.79406779661 202.434862079971
321.455084745763 202.434862079971
326.116101694915 202.434862079971
330.777118644068 202.434862079971
335.43813559322 202.434862079971
340.099152542373 202.434862079971
344.760169491525 202.434862079971
349.421186440678 202.434862079971
302.811016949153 246.435506013219
307.472033898305 246.435506013219
312.133050847458 246.435506013219
321.455084745763 246.435506013219
326.116101694915 246.435506013219
330.777118644068 246.435506013219
335.43813559322 246.435506013219
340.099152542373 246.435506013219
344.760169491525 246.435506013219
349.421186440678 246.435506013219
354.082203389831 246.435506013219
358.743220338983 246.435506013219
363.404237288136 246.435506013219
298.15 300
302.811016949153 300
307.472033898305 300
330.777118644068 300
340.099152542373 300
344.760169491525 300
349.421186440678 300
354.082203389831 300
358.743220338983 300
363.404237288136 300
368.065254237288 300
372.726271186441 300
377.387288135593 300
};
\addplot [semithick, black]
table {%
298.15 0.0316992938881239
302.811016949153 0.0416502526748325
307.472033898305 0.0542135031209127
312.133050847458 0.0699379446480481
316.79406779661 0.0894573631288235
321.455084745763 0.11349814624938
326.116101694915 0.142886977832051
330.777118644068 0.178558427224653
335.43813559322 0.221562362278131
340.099152542373 0.273071114093698
344.760169491525 0.334386329262953
349.421186440678 0.406945451585077
354.082203389831 0.492327780973097
358.743220338983 0.592260071535126
363.404237288136 0.708621631256037
368.065254237288 0.843448901418998
372.726271186441 0.998939497096534
377.387288135593 1.17745570205744
382.048305084746 1.38152741613099
386.709322033898 1.6138545617758
391.370338983051 1.87730896217833
396.031355932203 2.17493570898292
400.692372881356 2.50995404318333
405.353389830508 2.88575777644233
410.014406779661 3.30591528414528
414.675423728814 3.77416910593384
419.336440677966 4.29443518921697
423.997457627119 4.87080181690722
428.658474576271 5.50752826050788
433.319491525424 6.20904320199238
437.980508474576 6.97994296809687
442.641525423729 7.82498962397373
447.302542372881 8.74910897141505
451.963559322034 9.75738849941128
456.624576271186 10.8550753353115
461.285593220339 12.0475742454149
465.946610169491 13.3404457346604
470.607627118644 14.7394042977819
475.268644067797 16.2503168748088
479.929661016949 17.8792015668371
484.590677966102 19.6322266713913
489.251694915254 21.5157101004182
493.912711864407 23.53611924974
498.573728813559 25.7000713913868
503.234745762712 28.0143346796471
507.895762711864 30.4858298493258
512.556779661017 33.1216327202799
517.217796610169 35.9289776278006
521.878813559322 38.9152619027171
526.539830508475 42.0880515752612
531.200847457627 45.4550884790218
535.86186440678 49.0242989767954
540.522881355932 52.8038045662037
545.183898305085 56.8019346727322
549.844915254237 61.027242000087
554.50593220339 65.4885208869608
559.166949152542 70.1948292225895
563.827966101695 75.1555146090409
568.488983050847 80.3802456394762
573.15 85.8790494083592
};
\addplot [semithick, black, dashed]
table {%
298.15 64.3424425064156
302.811016949153 71.5805370612018
};
\end{groupplot}

\end{tikzpicture}

        \caption{Convergence maps of the water-carbon dioxide binary \ac{HEOS} mixture implemented in \emph{CoolProp} for a range of compositions.}
        \label{fig:wat_co2_calcmap}
    \end{figure}

    From Figure~\ref{fig:wat_co2_calcmap}, for pure water (\(z_{CO_2}=\)\qty{0.00}{\mol\percent}), the \ac{HEOS} formulation does not converge for a broad range of temperatures and pressures, appearing to shadow the saturation line of pure water. Convergence generally improves for quasi pure water (\qty{0.00}{\mol\percent}\(\leq z_{CO_2}<\)\qty{0.05}{\mol\percent}), however the \ac{HEOS} formulation does not converge at elevated (approaching reservoir-like) pressures. The areas of non-convergence appear to be correlated where the fluid is expected to be liquid (i.e. low temperatures and high pressures). 

    \subsection{Incompressible Binary Mixtures}
    \label{sec:incompressible_fluids}
        Furthermore, \ac{EOS} have also been developed for some industrially relevant mixtures, like seawater \cite{Sharqawy2010}, lithium bromide solution \cite{Patek2006}, and calcium chloride solution \cite{Preisegger2010} or potassium carbonate solution \cite{Melinder2010}. However, the application range of such binary incompressible fluid \ac{EOS} is limited, Table 1, due to the scope in which these fluids are used /handled in industry (e.g. seawater in desalination plants or lithium bromide in adsorption cooling). Moreover, no interaction models exist to allow mixtures of these binary mixtures (e.g. seawater and lithium bromide) or binary mixtures with other pure components (e.g. seawater and carbon dioxide) to be modelled.

        \begin{table}[H]
            \caption[The applicability range of various incompressible fluid \ac{EOS}.]{The applicability range of various incompressible fluid \ac{EOS}. \(x_{min}\) and \(x_{max}\) are the lower and upper limit for the amount of the species other than water.}
            \centering 
            \label{table:IncompressibleEOS}
            \begin{tabular}{|p{10em} c c c c |}
    \hline
    \rowcolor{bluepoli!40} % comment this line to remove the color
    \textbf{Fluid}& \(\mathbf{T_{min}}\)\textbf{, \unit{\degreeCelsius}} & \(\mathbf{T_{max}}\)\textbf{, \unit{\degreeCelsius}} & \(\mathbf{x_{min}}\)\textbf{, \unit{\percent}} & \(\mathbf{x_{max}}\)\textbf{, \unit{\percent}}\T\B \\
    \hline \hline
    Seawater & 0 & 120 & 0 & 12 \T\B\\
    Lithium Bromide & 0 & 227 & 0 & 75 \T\B \\
    Calcium Chloride & -55 & 20 & 15 & 30 \T\B\\
    Potassium Carbonate & -100 & 0 &  & 40 \B\\
    \hline
\end{tabular}        
            \\[10pt]
        \end{table}
    
\subsection{Chemically reactive Systems}
\label{sec:chemically_active_system}
    An alternative approach is to treat the geofluid as a chemically reactive system. In such a system, the constituent species can partition into different phases (e.g. gaseous, aqueous – a water-rich liquid phase, solid, etc.), react with each other to form new species or dissociate into other species, Figure~\ref{fig:chemically_reactive_system}. Determining the amounts and composition of all phases at equilibrium, at a given temperature and pressure, is equivalent to assessing the geofluid’s phase behaviour and also allows the thermophysical properties of the individual phases and overall fluid to be obtained.

    \begin{figure}[H]
        \centering
        \begin{tikzpicture}
    \draw [draw=none, fill=cyan!20](210:4)--(0,0)--(90:4) arc (90:210:4)--cycle;
    \draw [draw=none, fill=gray!20](-30:4)--(0,0)--(-150:4) arc (-150:-30:4)--cycle;
    \draw [draw=none, fill=red!20](90:4)--(0,0)--(-30:4) arc (-30:90:4)--cycle;
    \draw[dashed] (0,0) -- (90:4);
    \draw[dashed] (0,0) -- (210:4);
    \draw[dashed] (0,0) -- (330:4);
    \draw circle(4)[thick];

    \draw [draw=none, postaction={decorate,decoration={text along path,text align=center,text={||Gaseous}}}] (90:3.5) arc (90:-30:3.5);
    \draw [draw=none, postaction={decorate,decoration={text along path,text align=center,text={||Solid}}}] (210:3.5) arc (210:330:3.5);
    \draw [draw=none, postaction={decorate,decoration={text along path,text align=center,text={||Aqueous}}}] (210:3.5) arc (210:90:3.5);

    % draw water
    \node (water) at ($(0,0) + (110:3)$) {\(H_2O\)};
    \node (steam) at ($(0,0) + (70:3)$) {\(H_2O\)};
    \draw[<->]{} (water) to (steam);

    % draw CO2
    \node (CO2) at ($(0,0) + (30:1.5)$) {\(CO_2\)};
    \node (CO3) at ($(0,0) + (150:1.5)$) {\(CO_3^{2-}\)};
    \node (HCO3) at ($(CO3) + (0,1)$) {\(HCO_3^{-}\)};
    \node (Cs) at ($(0,0) + (0,-1)$) {\(C\)};
    \draw[<->]{} (CO2) to (CO3);
    \draw[<->]{} (CO3) to (Cs);
    \draw[<->]{} (CO3) to (HCO3);
    \draw[<->]{} (CO2) to (Cs);

    % draw NaCL
    \node (NaCl) at ($(0,0) + (230:2.5)$) {\(NaCl\)};
    \node (Na) at ($(0,0) + (195:3)$) {\(Na^+\)};
    \node (Cl) at ($(0,0) + (190:2)$) {\(Cl^-\)};
    \draw[<->]{} (NaCl) to (Na);
    \draw[<->]{} (NaCl) to (Cl);
    
    
\end{tikzpicture}
        \caption{Schematic of a possible chemical reactive system describing a geofluid}
        \label{fig:chemically_reactive_system}
    \end{figure}

    A unit amount of geofluid of arbitrary overall composition can be approximated as a closed thermodynamic system (i.e. no mass transfer into or out of the system). From the Second Law of Thermodynamics, such systems reach equilibrium when the system entropy reaches a global maximum. For systems at constant temperature and pressure, it can be shown that this is consistent with the Gibbs free energy,~\ref{eq:Gibbs_energy} of the system reaching a global minimum, also see \nameref{ch:appendix_a}.

    \begin{align}
        G = H - TS \label{eq:Gibbs_energy}
    \end{align}

    Thus, determining the equilibrium state (i.e. finding the phase amounts and compositions) represents a minimisation problem (i.e. \(\min G(P, T, \mathbf{n}) \) ), subject to the constraint that the total mass of each chemical element (i.e. H, O, Na, Cl, etc.) is conserved across all species and phases considered (i.e. \(\sum_{i=0}^N w_{ij} n_i = b_j\), where \(w_{ij} \) is the number of atoms of element \(j\) that make up species \(i\), \(n_i\) is the amount of species \(i\), and \(b_j\) is the total amount of element \(j\) in the system).

    A convenient expression for calculating the Gibbs free energy of the system can be obtained by combining the differential form of the Gibbs free energy (at constant temperature and pressure), Equation~\ref{eq:gibbs_differential}, with the First Law of Thermodynamics (i.e. \(dU= \delta Q + \delta W + \sum _{i=1}^N \mu_i dn_i\)), assuming fully reversible processes (i.e. \(\delta Q = TdS\) ) and mechanical work (i.e. \(\delta W = pdV\)), followed by integration over the molar amounts. This allows the Gibbs free energy of the system to be calculated from the amounts of each species and their respective chemical potential, Equation~\ref{eq:gibbs_energy_from_chemical_potential}. Here, \(G\) is Gibbs free energy, \(P\) is pressure, \(T\) is temperature, \(U\) is internal energy, \(V\) is volume, \(S\) is entropy, \(\mu_i\) is the chemical potential of species \(i\), \(n_i\) is the number of moles of species \(i\) and \(\mathbf{y}\) is the vector of all species’ mole fractions.

    \begin{align}
        dG \big|_{P,T} = dU + PdV - TdS \label{eq:gibbs_differential}
    \end{align}
    \begin{align}
        G(P, T, \mathbf{n}) = \sum _{i=1}^N n_i \mu_i(P, T, \mathbf{y}) \label{eq:gibbs_energy_from_chemical_potential}
    \end{align}

    The partial derivatives of the chemical potential are the partial molar enthalpy, Equation~\ref{eq:partial_molar_enthalpy}, the partial molar entropy, Equation~\ref{eq:partial_molar_entropy}, and the partial molar volume, Equation~\ref{eq:partial_molar_volume}, - the thermodynamic properties of interest.

    \begin{align}
        h_i (P, T, \mathbf{y}) = \frac{ \partial\left( \frac{\mu_i (P, T, \mathbf{y})}{T} \right)}{\partial\left( \frac{1}{T} \right) } \Bigg|_{P, \mathbf{y}} \label{eq:partial_molar_enthalpy}
    \end{align}
    \begin{align}
        s_i (P, T, \mathbf{y}) = - \frac{ \partial\left( \mu_i (P, T, \mathbf{y})\right)}{\partial T} \Bigg|_{P, \mathbf{y}} \label{eq:partial_molar_entropy}
    \end{align}
    \begin{align}
        v_i (P, T, \mathbf{y}) = \frac{ \partial\left( \mu_i (P, T, \mathbf{y})\right)}{\partial P} \Bigg|_{T, \mathbf{y}} = \frac{1}{\rho_i (P, T, \mathbf{y}}) \label{eq:partial_molar_volume}
    \end{align} 

    \begin{notes}{Note}
        For convenience the above equations can also be re-written as shown in Equation~\ref{eq:partial_molar_prop}, where \(\Psi_i (P,T, \mathbf{y})\) is a placeholder for a thermodynamic property and the choice of \(x\) and \(f(x)\) depends on the thermodynamic property of interest, see Table~\ref{table:PartialMolarProperties}. 
        % For instance, when calculating the partial molar enthalpy, \(x=\frac{1}{T}\) and \(f(x)=x\), when calculating the partial molar entropy, \(x=T\) and \(f(x)=-1\), and when calculating the partial molar volume, \(x=P\) and \(f(x)=1\).

        \begin{align}
            \Psi_i (P, T, \mathbf{y}) = \frac{ \partial \left( f(x)*\mu_i (P, T, \mathbf{y}) \right)} {\partial x } \label{eq:partial_molar_prop}
        \end{align}

        \begin{table}[H]
            \caption{The definition of the auxiliary variable and function by partial molar property.}
            \centering 
            \label{table:PartialMolarProperties}
            \begin{tabular}{|c c c |}
    \hline
    \rowcolor{bluepoli!40} % comment this line to remove the color
    \(\mathbf{\Psi_i}\)& \(\mathbf{x}\) & \(\mathbf{f(x)}\) \T\B \\
    \hline \hline
    \rowcolor{white} \(h\) & \(\frac{1}{T}\) & \(x\) \T\B\\
    \rowcolor{white}\(s\) & \(T\) & \(-1\) \T\B \\
    \rowcolor{white}\(v\) & \(P\) & \(1\) \T\B\\
    \hline
\end{tabular}        
            \\[10pt]
        \end{table}
    \end{notes}

    The calculation of the chemical potential is broken down into two components; the standard chemical potential of the species at a reference state, and the species’ activity, Equation~\ref{eq:chemical_potential}. The species’ activity is defined as the difference between the actual chemical potential and the standard chemical potential, Equation~\ref{eq:activity_def}.
    
    The reference state is chosen by convention: For liquid or gaseous species, the reference composition, \(y^o\), is that of the pure component, whereas for aqueous species (e.g. \(Na^+\)), a 1-molal solution of the solute is chosen, with all other species at infinite dilution. Meanwhile, the reference pressure for liquid and aqueous species is taken as the system pressure (i.e. \(P^o=P\)), whereas for gases, the reference pressure is taken to be \qty{1}{\bar} (i.e. \(P^o=1\) \unit{\bar}).

    \begin{align}
        \mu_i (P, T, \mathbf{y}) = \mu_i (P^o, T^o, \mathbf{y}^o) + RT \ln a_i (P, T, \mathbf{y}) \label{eq:chemical_potential}
    \end{align}
        \begin{align}
        RT \ln a_i (P, T, \mathbf{y}) \equiv \mu_i (P, T, \mathbf{y}) - \mu_i (P^o, T^o, \mathbf{y}^o) \label{eq:activity_def}
    \end{align}

    With the above in mind, determining the equilibrium composition and thermophysical properties of a geofluid at a given temperature and pressure requires three inputs: 1) The amounts of all elements across all species, 2) the chemical potential of all species at their respective reference state (also called the standard chemical potential) and 3) the activity of all species.
    
    The elemental amounts can be obtained from the geofluid composition, which is specific to each geothermal site as it is dependent on several factors, such as reservoir rock composition, temperature and pressure. Thus, the geofluid composition can only reliably be obtained from geofluid samples.

    \subsubsection{Standard Chemical Potential}
    The species’ standard chemical potential can be obtained from peer-reviewed open-source databases, such as SUPCRT92 \cite{Johnson1992} or SUPCRTBL \cite{Zimmer2016}. Frameworks, such as ThermoFun \cite{Miron2021} and Reaktoro \cite{Leal2015}, implement several models for computing standard thermodynamic properties from such databases. Alternatively, high-fidelity \ac{EOS} for species, such as water and carbon dioxide, can be used. Computationally cheaper \ac{EOS}, such as \ac{SRK} or \ac{PR}, can be used, provided their input parameters (e.g. critical properties, acentric factor, etc.) have been calibrated to the specific component in question.

    \subsubsection{Activity Models}
    The species’ activity can be calculated from phase and species-specific activity models. The simplest activity models approximate the fluid as an ideal fluid (i.e. Ideal Gas, Ideal Solution or Ideal Solid). However, this approach limits their application to low concentrations (for Ideal Solutions) or low pressures and high temperatures (for Ideal Gases), where the species exhibit ideal behaviour and where interactions among molecules are negligible. For gaseous species, such as \(CH_4\), \(N2\), \(H2S\), etc., the \ac{SRK} \ac{EOS} or the \ac{PR} \ac{EOS} may also be used to approximate the real gas behaviour and interactions among other gaseous species.

    Various activity models have been proposed for the different types of aqueous species. For example, the Setschenow equation \cite{Setschenow1889} for neutral species, the HKF-Debye-Hückel model \cite{Helgeson1981} for water and ionic species, or the Pitzer model for various aqueous species \cite{Pitzer1973}. Moreover, species-specific activity models have been developed for common mixtures of species. For example, for mixtures of \(H_2O\), \(CO_2\), \(CH_4\) and some mineral species, models by \citeauthor{Duan2003}, \citeauthor{Spycher2003} and \citeauthor{Spycher2009}, amongst others, can be used. The selection of an activity model is ultimately dependent on the species present, their relative amounts as well as the system temperature and pressure.

    \subsubsection{Limitations}
    \label{sec:chemically_active_system_limitations}
    In principle, chemically reactive systems allow any number of species and reactions to be modelled. However, the main barrier to this approach, being applied universally to geofluid modelling in a geothermal context, is the availability of appropriate activity models for all species - particularly gaseous water (i.e. steam). While the \ac{WP} \ac{EOS} represents the highest fidelity model for the properties of water and steam \cite{IAPWS2018}, it is computationally expensive and it only works with a single component: water. For this reason, most geochemical modelling codes (e.g., PHREEQC, GEMS, Reaktoro) adopt cubic equations of state for the vapour phase, such as the \ac{PR} \ac{EOS} or the \ac{SRK} \ac{EOS}, to permit other gases such as \(CO_2\), \(H_2S\), \(O_2\), and others to be considered. However, this can result in deviations from the expected phase behaviour when water steam is in higher proportion compared to other gases or simply the only gaseous species.

    For example, for pure water at a pressure of \qty{10}{\bar} the \ac{WP} \ac{EOS} predicts a saturation temperature of around \qty{453}{\K}. To model the same fluid using a chemically reactive system we assume a system consisting of only an aqueous and a gaseous water species (i.e. \(H_2O^{(aq)}\) and \(H_2O^{(g)}\). This system was then simulated in \emph{Reaktoro}, see ~\ref{sec:calc_frameworks}, as part of a \ac{VLE} calculation, for a pressure of \qty{10}{\bar} and a temperature between \qty{445}{\K} and \qty{465}{\K}. The SUPCRTBL database was used for the standard thermodynamic properties and different \ac{EOS} (ideal gas and \ac{SRK}) as activity models for the vapour phase (i.e., \(H_2O^{(g)}\)). The specific volume of the fluid was evaluated for each state and compared against values calculated via the \ac{WP} \ac{EOS} (Figure~\ref{fig:Tsat_at_P}). In the case of \emph{Reaktoro}, the saturation temperature was inferred by the temperature at which the transition from liquid-like to vapour-like densities occurs.

    \begin{figure}[H]
        \centering
        \begin{tikzpicture}
    \begin{axis}[xlabel = {Temperature/\unit{\K}},
                 ylabel = {Specific Volume/\unit{\cubic\m \per \kg}},
                 legend style={at={(0.97,0.03)}, anchor=south east},
                 width=12cm,
                 height=7cm,
                 xmin=445,
                 xmax=465,
                 ymin = 0,
                 y tick label style={/pgf/number format/.cd,fixed,fixed zerofill,precision=2,/tikz/.cd},]
        
        % Plot Liquid Lines
        \addplot[color=blue]
                coordinates {(4.450e+02,1.116e-03) (4.451e+02,1.117e-03) (4.452e+02,1.117e-03) (4.453e+02,1.117e-03) (4.454e+02,1.117e-03) (4.455e+02,1.117e-03) (4.456e+02,1.117e-03) (4.457e+02,1.117e-03) (4.458e+02,1.118e-03) (4.459e+02,1.118e-03) (4.460e+02,1.118e-03) (4.461e+02,1.118e-03) (4.462e+02,1.118e-03) (4.463e+02,1.118e-03) (4.464e+02,1.118e-03) (4.465e+02,1.118e-03) (4.466e+02,1.119e-03) (4.467e+02,1.119e-03) (4.468e+02,1.119e-03) (4.469e+02,1.119e-03) (4.470e+02,1.119e-03) (4.471e+02,1.119e-03) (4.472e+02,1.119e-03) (4.473e+02,1.120e-03) (4.474e+02,1.120e-03) (4.475e+02,1.120e-03) (4.476e+02,1.120e-03) (4.477e+02,1.120e-03) (4.478e+02,1.120e-03) (4.479e+02,1.120e-03) (4.480e+02,1.120e-03) (4.481e+02,1.121e-03) (4.482e+02,1.121e-03) (4.483e+02,1.121e-03) (4.484e+02,1.121e-03) (4.485e+02,1.121e-03) (4.486e+02,1.121e-03) (4.487e+02,1.121e-03) (4.488e+02,1.122e-03) (4.489e+02,1.122e-03) (4.490e+02,1.122e-03) (4.491e+02,1.122e-03) (4.492e+02,1.122e-03) (4.493e+02,1.122e-03) (4.494e+02,1.122e-03) (4.495e+02,1.122e-03) (4.496e+02,1.123e-03) (4.497e+02,1.123e-03) (4.498e+02,1.123e-03) (4.499e+02,1.123e-03) (4.500e+02,1.123e-03) (4.501e+02,1.123e-03) (4.502e+02,1.123e-03) (4.503e+02,1.124e-03) (4.504e+02,1.124e-03) (4.505e+02,1.124e-03) (4.506e+02,1.124e-03) (4.507e+02,1.124e-03) (4.508e+02,1.124e-03) (4.509e+02,1.124e-03) (4.510e+02,1.125e-03) (4.511e+02,1.125e-03) (4.512e+02,1.125e-03) (4.513e+02,1.125e-03) (4.514e+02,1.125e-03) (4.515e+02,1.125e-03) (4.516e+02,1.125e-03) (4.517e+02,1.125e-03) (4.518e+02,1.126e-03) (4.519e+02,1.126e-03) (4.520e+02,1.126e-03) (4.521e+02,1.126e-03) (4.522e+02,1.126e-03) (4.523e+02,1.126e-03) (4.524e+02,1.126e-03) (4.525e+02,1.127e-03) (4.526e+02,1.127e-03) (4.527e+02,1.127e-03) (4.528e+02,1.127e-03) (4.529e+02,1.127e-03)};
        \addlegendentry{\ac{WP} \ac{EOS}};

        \addplot[color=red]
                coordinates {(4.450e+02,1.116e-03) (4.451e+02,1.117e-03) (4.452e+02,1.117e-03) (4.453e+02,1.117e-03) (4.454e+02,1.117e-03) (4.455e+02,1.117e-03) (4.456e+02,1.117e-03) (4.457e+02,1.117e-03) (4.458e+02,1.118e-03) (4.459e+02,1.118e-03) (4.460e+02,1.118e-03) (4.461e+02,1.118e-03) (4.462e+02,1.118e-03) (4.463e+02,1.118e-03) (4.464e+02,1.118e-03) (4.465e+02,1.118e-03) (4.466e+02,1.119e-03) (4.467e+02,1.119e-03) (4.468e+02,1.119e-03) (4.469e+02,1.119e-03) (4.470e+02,1.119e-03) (4.471e+02,1.119e-03) (4.472e+02,1.119e-03) (4.473e+02,1.120e-03) (4.474e+02,1.120e-03) (4.475e+02,1.120e-03) (4.476e+02,1.120e-03) (4.477e+02,1.120e-03) (4.478e+02,1.120e-03) (4.479e+02,1.120e-03) (4.480e+02,1.120e-03) (4.481e+02,1.121e-03) (4.482e+02,1.121e-03) (4.483e+02,1.121e-03) (4.484e+02,1.121e-03) (4.485e+02,1.121e-03) (4.486e+02,1.121e-03) (4.487e+02,1.121e-03) (4.488e+02,1.122e-03) (4.489e+02,1.122e-03) (4.490e+02,1.122e-03) (4.491e+02,1.122e-03) (4.492e+02,1.122e-03) (4.493e+02,1.122e-03) (4.494e+02,1.122e-03) (4.495e+02,1.122e-03) (4.496e+02,1.123e-03) (4.497e+02,1.123e-03) (4.498e+02,1.123e-03) (4.499e+02,1.123e-03) (4.500e+02,1.123e-03) (4.501e+02,1.123e-03) (4.502e+02,1.123e-03) (4.503e+02,1.124e-03) (4.504e+02,1.124e-03) (4.505e+02,1.124e-03) (4.506e+02,1.124e-03) (4.507e+02,1.124e-03) (4.508e+02,1.124e-03) (4.509e+02,1.124e-03) (4.510e+02,1.125e-03) (4.511e+02,1.125e-03) (4.512e+02,1.125e-03) (4.513e+02,1.125e-03) (4.514e+02,1.125e-03) (4.515e+02,1.125e-03) (4.516e+02,1.125e-03) (4.517e+02,1.125e-03) (4.518e+02,1.126e-03) (4.519e+02,1.126e-03) (4.520e+02,1.126e-03) (4.521e+02,1.126e-03) (4.522e+02,1.126e-03) (4.523e+02,1.126e-03) (4.524e+02,1.126e-03) (4.525e+02,1.127e-03) (4.526e+02,1.127e-03) (4.527e+02,1.127e-03) (4.528e+02,1.127e-03) (4.529e+02,1.127e-03) (4.530e+02,1.127e-03) (4.531e+02,1.127e-03) (4.532e+02,1.128e-03) (4.533e+02,1.128e-03) (4.534e+02,1.128e-03) (4.535e+02,1.128e-03) (4.536e+02,1.128e-03) (4.537e+02,1.128e-03) (4.538e+02,1.128e-03) (4.539e+02,1.128e-03) (4.540e+02,1.129e-03) (4.541e+02,1.129e-03) (4.542e+02,1.129e-03) (4.543e+02,1.129e-03) (4.544e+02,1.129e-03) (4.545e+02,1.129e-03) (4.546e+02,1.129e-03) (4.547e+02,1.130e-03) (4.548e+02,1.130e-03) (4.549e+02,1.130e-03) (4.551e+02,1.130e-03) (4.552e+02,1.130e-03) (4.553e+02,1.130e-03) (4.554e+02,1.130e-03) (4.555e+02,1.131e-03) (4.556e+02,1.131e-03) (4.557e+02,1.131e-03) (4.558e+02,1.131e-03) (4.559e+02,1.131e-03) (4.560e+02,1.131e-03) (4.561e+02,1.131e-03) (4.562e+02,1.132e-03) (4.563e+02,1.132e-03) (4.564e+02,1.132e-03) (4.565e+02,1.132e-03) (4.566e+02,1.132e-03) (4.567e+02,1.132e-03) (4.568e+02,1.132e-03) (4.569e+02,1.133e-03)};
        \addlegendentry{Ideal Gas};

        \addplot[color=green]
                coordinates {(4.450e+02,1.116e-03) (4.451e+02,1.117e-03) (4.452e+02,1.117e-03) (4.453e+02,1.117e-03) (4.454e+02,1.117e-03) (4.455e+02,1.117e-03) (4.456e+02,1.117e-03) (4.457e+02,1.117e-03) (4.458e+02,1.118e-03) (4.459e+02,1.118e-03) (4.460e+02,1.118e-03) (4.461e+02,1.118e-03) (4.462e+02,1.118e-03) (4.463e+02,1.118e-03) (4.464e+02,1.118e-03) (4.465e+02,1.118e-03) (4.466e+02,1.119e-03) (4.467e+02,1.119e-03) (4.468e+02,1.119e-03) (4.469e+02,1.119e-03) (4.470e+02,1.119e-03) (4.471e+02,1.119e-03) (4.472e+02,1.119e-03) (4.473e+02,1.120e-03) (4.474e+02,1.120e-03) (4.475e+02,1.120e-03) (4.476e+02,1.120e-03) (4.477e+02,1.120e-03) (4.478e+02,1.120e-03) (4.479e+02,1.120e-03) (4.480e+02,1.120e-03) (4.481e+02,1.121e-03) (4.482e+02,1.121e-03) (4.483e+02,1.121e-03) (4.484e+02,1.121e-03) (4.485e+02,1.121e-03) (4.486e+02,1.121e-03) (4.487e+02,1.121e-03) (4.488e+02,1.122e-03) (4.489e+02,1.122e-03) (4.490e+02,1.122e-03) (4.491e+02,1.122e-03) (4.492e+02,1.122e-03) (4.493e+02,1.122e-03) (4.494e+02,1.122e-03) (4.495e+02,1.122e-03) (4.496e+02,1.123e-03) (4.497e+02,1.123e-03) (4.498e+02,1.123e-03) (4.499e+02,1.123e-03) (4.500e+02,1.123e-03) (4.501e+02,1.123e-03) (4.502e+02,1.123e-03) (4.503e+02,1.124e-03) (4.504e+02,1.124e-03) (4.505e+02,1.124e-03) (4.506e+02,1.124e-03) (4.507e+02,1.124e-03) (4.508e+02,1.124e-03) (4.509e+02,1.124e-03) (4.510e+02,1.125e-03) (4.511e+02,1.125e-03) (4.512e+02,1.125e-03) (4.513e+02,1.125e-03) (4.514e+02,1.125e-03) (4.515e+02,1.125e-03) (4.516e+02,1.125e-03) (4.517e+02,1.125e-03) (4.518e+02,1.126e-03) (4.519e+02,1.126e-03) (4.520e+02,1.126e-03) (4.521e+02,1.126e-03) (4.522e+02,1.126e-03) (4.523e+02,1.126e-03) (4.524e+02,1.126e-03) (4.525e+02,1.127e-03) (4.526e+02,1.127e-03) (4.527e+02,1.127e-03) (4.528e+02,1.127e-03) (4.529e+02,1.127e-03) (4.530e+02,1.127e-03) (4.531e+02,1.127e-03) (4.532e+02,1.128e-03) (4.533e+02,1.128e-03) (4.534e+02,1.128e-03) (4.535e+02,1.128e-03) (4.536e+02,1.128e-03) (4.537e+02,1.128e-03) (4.538e+02,1.128e-03) (4.539e+02,1.128e-03) (4.540e+02,1.129e-03) (4.541e+02,1.129e-03) (4.542e+02,1.129e-03) (4.543e+02,1.129e-03) (4.544e+02,1.129e-03) (4.545e+02,1.129e-03) (4.546e+02,1.129e-03)};
        \addlegendentry{\ac{SRK} \ac{EOS}};

        % Vapour Lines
        \addplot[color=blue]
                coordinates {(4.531e+02,1.944e-01) (4.532e+02,1.944e-01) (4.533e+02,1.945e-01) (4.534e+02,1.946e-01) (4.535e+02,1.946e-01) (4.536e+02,1.947e-01) (4.537e+02,1.947e-01) (4.538e+02,1.948e-01) (4.539e+02,1.949e-01) (4.540e+02,1.949e-01) (4.541e+02,1.950e-01) (4.542e+02,1.950e-01) (4.543e+02,1.951e-01) (4.544e+02,1.952e-01) (4.545e+02,1.952e-01) (4.546e+02,1.953e-01) (4.547e+02,1.953e-01) (4.548e+02,1.954e-01) (4.549e+02,1.955e-01) (4.551e+02,1.955e-01) (4.552e+02,1.956e-01) (4.553e+02,1.956e-01) (4.554e+02,1.957e-01) (4.555e+02,1.958e-01) (4.556e+02,1.958e-01) (4.557e+02,1.959e-01) (4.558e+02,1.959e-01) (4.559e+02,1.960e-01) (4.560e+02,1.961e-01) (4.561e+02,1.961e-01) (4.562e+02,1.962e-01) (4.563e+02,1.962e-01) (4.564e+02,1.963e-01) (4.565e+02,1.964e-01) (4.566e+02,1.964e-01) (4.567e+02,1.965e-01) (4.568e+02,1.965e-01) (4.569e+02,1.966e-01) (4.570e+02,1.967e-01) (4.571e+02,1.967e-01) (4.572e+02,1.968e-01) (4.573e+02,1.968e-01) (4.574e+02,1.969e-01) (4.575e+02,1.970e-01) (4.576e+02,1.970e-01) (4.577e+02,1.971e-01) (4.578e+02,1.971e-01) (4.579e+02,1.972e-01) (4.580e+02,1.973e-01) (4.581e+02,1.973e-01) (4.582e+02,1.974e-01) (4.583e+02,1.974e-01) (4.584e+02,1.975e-01) (4.585e+02,1.975e-01) (4.586e+02,1.976e-01) (4.587e+02,1.977e-01) (4.588e+02,1.977e-01) (4.589e+02,1.978e-01) (4.590e+02,1.978e-01) (4.591e+02,1.979e-01) (4.592e+02,1.980e-01) (4.593e+02,1.980e-01) (4.594e+02,1.981e-01) (4.595e+02,1.981e-01) (4.596e+02,1.982e-01) (4.597e+02,1.983e-01) (4.598e+02,1.983e-01) (4.599e+02,1.984e-01) (4.600e+02,1.984e-01) (4.601e+02,1.985e-01) (4.602e+02,1.985e-01) (4.603e+02,1.986e-01) (4.604e+02,1.987e-01) (4.605e+02,1.987e-01) (4.606e+02,1.988e-01) (4.607e+02,1.988e-01) (4.608e+02,1.989e-01) (4.609e+02,1.990e-01) (4.610e+02,1.990e-01) (4.611e+02,1.991e-01) (4.612e+02,1.991e-01) (4.613e+02,1.992e-01) (4.614e+02,1.993e-01) (4.615e+02,1.993e-01) (4.616e+02,1.994e-01) (4.617e+02,1.994e-01) (4.618e+02,1.995e-01) (4.619e+02,1.995e-01) (4.620e+02,1.996e-01) (4.621e+02,1.997e-01) (4.622e+02,1.997e-01) (4.623e+02,1.998e-01) (4.624e+02,1.998e-01) (4.625e+02,1.999e-01) (4.626e+02,2.000e-01) (4.627e+02,2.000e-01) (4.628e+02,2.001e-01) (4.629e+02,2.001e-01) (4.630e+02,2.002e-01) (4.631e+02,2.002e-01) (4.632e+02,2.003e-01) (4.633e+02,2.004e-01) (4.634e+02,2.004e-01) (4.635e+02,2.005e-01) (4.636e+02,2.005e-01) (4.637e+02,2.006e-01) (4.638e+02,2.007e-01) (4.639e+02,2.007e-01) (4.640e+02,2.008e-01) (4.641e+02,2.008e-01) (4.642e+02,2.009e-01) (4.643e+02,2.009e-01) (4.644e+02,2.010e-01) (4.645e+02,2.011e-01) (4.646e+02,2.011e-01) (4.647e+02,2.012e-01) (4.648e+02,2.012e-01) (4.649e+02,2.013e-01) (4.650e+02,2.013e-01)};
        
        \addplot[color=red]
                coordinates {(4.571e+02,2.109e-01) (4.572e+02,2.109e-01) (4.573e+02,2.110e-01) (4.574e+02,2.110e-01) (4.575e+02,2.111e-01) (4.576e+02,2.111e-01) (4.577e+02,2.112e-01) (4.578e+02,2.112e-01) (4.579e+02,2.113e-01) (4.580e+02,2.113e-01) (4.581e+02,2.114e-01) (4.582e+02,2.114e-01) (4.583e+02,2.115e-01) (4.584e+02,2.115e-01) (4.585e+02,2.115e-01) (4.586e+02,2.116e-01) (4.587e+02,2.116e-01) (4.588e+02,2.117e-01) (4.589e+02,2.117e-01) (4.590e+02,2.118e-01) (4.591e+02,2.118e-01) (4.592e+02,2.119e-01) (4.593e+02,2.119e-01) (4.594e+02,2.120e-01) (4.595e+02,2.120e-01) (4.596e+02,2.121e-01) (4.597e+02,2.121e-01) (4.598e+02,2.121e-01) (4.599e+02,2.122e-01) (4.600e+02,2.122e-01) (4.601e+02,2.123e-01) (4.602e+02,2.123e-01) (4.603e+02,2.124e-01) (4.604e+02,2.124e-01) (4.605e+02,2.125e-01) (4.606e+02,2.125e-01) (4.607e+02,2.126e-01) (4.608e+02,2.126e-01) (4.609e+02,2.127e-01) (4.610e+02,2.127e-01) (4.611e+02,2.128e-01) (4.612e+02,2.128e-01) (4.613e+02,2.128e-01) (4.614e+02,2.129e-01) (4.615e+02,2.129e-01) (4.616e+02,2.130e-01) (4.617e+02,2.130e-01) (4.618e+02,2.131e-01) (4.619e+02,2.131e-01) (4.620e+02,2.132e-01) (4.621e+02,2.132e-01) (4.622e+02,2.133e-01) (4.623e+02,2.133e-01) (4.624e+02,2.134e-01) (4.625e+02,2.134e-01) (4.626e+02,2.134e-01) (4.627e+02,2.135e-01) (4.628e+02,2.135e-01) (4.629e+02,2.136e-01) (4.630e+02,2.136e-01) (4.631e+02,2.137e-01) (4.632e+02,2.137e-01) (4.633e+02,2.138e-01) (4.634e+02,2.138e-01) (4.635e+02,2.139e-01) (4.636e+02,2.139e-01) (4.637e+02,2.140e-01) (4.638e+02,2.140e-01) (4.639e+02,2.141e-01) (4.640e+02,2.141e-01) (4.641e+02,2.141e-01) (4.642e+02,2.142e-01) (4.643e+02,2.142e-01) (4.644e+02,2.143e-01) (4.645e+02,2.143e-01) (4.646e+02,2.144e-01) (4.647e+02,2.144e-01) (4.648e+02,2.145e-01) (4.649e+02,2.145e-01) (4.650e+02,2.146e-01)};

        \addplot[color=green]
                coordinates {(4.548e+02,1.995e-01) (4.549e+02,1.995e-01) (4.551e+02,1.996e-01) (4.552e+02,1.996e-01) (4.553e+02,1.997e-01) (4.554e+02,1.997e-01) (4.555e+02,1.998e-01) (4.556e+02,1.998e-01) (4.557e+02,1.999e-01) (4.558e+02,1.999e-01) (4.559e+02,2.000e-01) (4.560e+02,2.000e-01) (4.561e+02,2.001e-01) (4.562e+02,2.001e-01) (4.563e+02,2.002e-01) (4.564e+02,2.003e-01) (4.565e+02,2.003e-01) (4.566e+02,2.004e-01) (4.567e+02,2.004e-01) (4.568e+02,2.005e-01) (4.569e+02,2.005e-01) (4.570e+02,2.006e-01) (4.571e+02,2.006e-01) (4.572e+02,2.007e-01) (4.573e+02,2.007e-01) (4.574e+02,2.008e-01) (4.575e+02,2.008e-01) (4.576e+02,2.009e-01) (4.577e+02,2.009e-01) (4.578e+02,2.010e-01) (4.579e+02,2.010e-01) (4.580e+02,2.011e-01) (4.581e+02,2.011e-01) (4.582e+02,2.012e-01) (4.583e+02,2.012e-01) (4.584e+02,2.013e-01) (4.585e+02,2.013e-01) (4.586e+02,2.014e-01) (4.587e+02,2.014e-01) (4.588e+02,2.015e-01) (4.589e+02,2.015e-01) (4.590e+02,2.016e-01) (4.591e+02,2.016e-01) (4.592e+02,2.017e-01) (4.593e+02,2.017e-01) (4.594e+02,2.018e-01) (4.595e+02,2.018e-01) (4.596e+02,2.019e-01) (4.597e+02,2.019e-01) (4.598e+02,2.020e-01) (4.599e+02,2.020e-01) (4.600e+02,2.021e-01) (4.601e+02,2.021e-01) (4.602e+02,2.022e-01) (4.603e+02,2.022e-01) (4.604e+02,2.023e-01) (4.605e+02,2.023e-01) (4.606e+02,2.024e-01) (4.607e+02,2.024e-01) (4.608e+02,2.025e-01) (4.609e+02,2.025e-01) (4.610e+02,2.026e-01) (4.611e+02,2.027e-01) (4.612e+02,2.027e-01) (4.613e+02,2.028e-01) (4.614e+02,2.028e-01) (4.615e+02,2.029e-01) (4.616e+02,2.029e-01) (4.617e+02,2.030e-01) (4.618e+02,2.030e-01) (4.619e+02,2.031e-01) (4.620e+02,2.031e-01) (4.621e+02,2.032e-01) (4.622e+02,2.032e-01) (4.623e+02,2.033e-01) (4.624e+02,2.033e-01) (4.625e+02,2.034e-01) (4.626e+02,2.034e-01) (4.627e+02,2.035e-01) (4.628e+02,2.035e-01) (4.629e+02,2.036e-01) (4.630e+02,2.036e-01) (4.631e+02,2.037e-01) (4.632e+02,2.037e-01) (4.633e+02,2.038e-01) (4.634e+02,2.038e-01) (4.635e+02,2.039e-01) (4.636e+02,2.039e-01) (4.637e+02,2.040e-01) (4.638e+02,2.040e-01) (4.639e+02,2.041e-01) (4.640e+02,2.041e-01) (4.641e+02,2.042e-01) (4.642e+02,2.042e-01) (4.643e+02,2.043e-01) (4.644e+02,2.043e-01) (4.645e+02,2.044e-01) (4.646e+02,2.044e-01) (4.647e+02,2.045e-01) (4.648e+02,2.045e-01) (4.649e+02,2.046e-01) (4.650e+02,2.046e-01)};

        % Saturation Lines
        \addplot[color=blue, dashed]
                coordinates {(453.0904,1.127e-03) (453.0904,1.944e-01)};
        \addplot[color=red, dashed]
                coordinates {(457.010,1.133e-03) (457.010,2.109e-01)};
        \addplot[color=green, dashed]
                coordinates {(454.7989,1.129e-03) (454.7989,1.995e-01)};

        \node (WP) at (axis cs:451, 0.175) {\qty{453.1}{\K}};
        \node (SRK) at (axis cs:459, 0.15) {\qty{454.8}{\K}};
        \node (Ideal) at (axis cs:459, 0.1) {\qty{457.0}{\K}};

        \draw (WP) -- (axis cs:453.0904, 0.15) {};
        \draw (SRK) -- (axis cs:454.7989, 0.125) {};
        \draw (Ideal) -- (axis cs:457.010, 0.075) {};
        
    \end{axis}
\end{tikzpicture}
        \caption[The specific volume of water as calculated with \emph{Reaktoro}]{The specific volume of pure water at a pressure of \qty{10}{\bar} over temperatures \num{445} to \qty{465}{\K} calculated with \emph{Reaktoro} using the Ideal Gas and \ac{SRK} \ac{EOS}, compared to the \ac{WP} \ac{EOS}.}
        \label{fig:Tsat_at_P}
    \end{figure}

    Depending on the equation of state selected in \emph{Reaktoro}, the specific volume differs from the values predicted by the \ac{WP} \ac{EOS} by \qty{1.3}{\percent} (\ac{SRK}) and \qty{6.5}{\percent} (Ideal Gas) (Figure~\ref{fig:Tsat_at_P}). These differences indicate that the partial derivatives of the chemical potential (in this case with respect to pressure) of gaseous water equation, see Equation~\ref{eq:partial_molar_volume}, are inconsistent with the \ac{WP} \ac{EOS}. Consequently, in direct steam cycle geothermal power plants, the steam turbine would be designed and optimised for different volumetric rates and velocities, resulting in sub-optimal turbine designs.

    Furthermore, the transition from liquid-like to vapour-like specific volume occurs at higher temperatures compared to \ac{WP} \ac{EOS}, indicating that the selected \ac{EOS} (i.e., \ac{SRK} and Ideal Gas) result in the chemical potential of gaseous water to be overestimated. By definition, at saturation, the chemical potential of the same chemical species in different phases is the same (i.e. \(\mu_i^L=\mu_i^G\)).
    
    Although the differences in saturation temperature are small in relative terms (less than \qty{0.7}{\percent} in the case of the \ac{SRK} activity model), the absolute differences (\qty{2}{\K} in the case of the \ac{SRK} activity model), when compared to key power plant design parameters, such as the minimum approach temperature difference in the heat exchange equipment (typically between \qty{5}{\K} and \qty{10}{\K}), are significant, representing differences of \qty{20}{\percent} to \qty{40}{\percent}. This can affect the required heat transfer area, which is the primary driver for the cost of heat exchange equipment.

    Repeating the above experiment for different system pressures (Figure~\ref{fig:Tsat_vs_P}), for the \ac{SRK} activity model, the deviations range between \qty{0.28}{\percent} and \qty{0.68}{\percent} in relative terms, and \qty{1.16}{\K} to \qty{3.6}{\K} in absolute terms, while for the Ideal Gas activity model the saturation temperature deviation increases from \qty{0.28}{\percent} (corresponding to \qty{1.06}{\K}) at \qty{1}{\bar} to \qty{3.32}{\percent} (corresponding to \qty{18.4}{\K}) at \qty{64}{bar}. The latter can be explained by the deviation from ideal gas behaviour at elevated pressures. Thus, for pressures exceeding \qty{2.5}{\bar} (corresponding to a saturation temperature of \qty{400}{\K} for pure water), the \ac{SRK} activity model provides more accurate saturation temperature estimates than the Ideal Gas activity model.

    \begin{figure}[H]
        \centering
        \begin{tikzpicture}
    \begin{axis}[xlabel = {Pressure/\unit{\bar}},
                 ylabel = {Deviation from \ac{WP} Tsat/\unit{\K}},
                 legend style={at={(0.03,0.97)}, anchor=north west},
                 axis y line*=left,
                 ymin = 0,
                 ymax=20,
                 xmin = 1,
                 xmax = 64,
                 xmode = log,
                 log basis x={4},
                 log ticks with fixed point,
                 height=6.5cm, width=10cm]
        
        \addplot[color=black, smooth]
                coordinates {(1.000e+00,30e+00) (2.000e+00,30e+00)};
        \addlegendentry{Abs Diff.};
        \addplot[color=black, smooth, dashed]
                coordinates {(1.000e+00,2.600e+00) (2.000e+00,1.900e+00)};
        \addlegendentry{\% Diff.};
        
        \addplot[color=blue, smooth]
                coordinates {(1.000e+00,2.600e+00) (2.000e+00,1.900e+00) (4.000e+00,1.300e+00) (8.000e+00,1.600e+00) (1.600e+01,2.100e+00) (3.200e+01,2.700e+00) (6.400e+01,3.700e+00)};
        \addlegendentry{\ac{SRK}};
        \addplot[color=red, smooth]
                coordinates {(1.000e+00,1.100e+00) (2.000e+00,1.500e+00) (4.000e+00,2.200e+00) (8.000e+00,3.400e+00) (1.600e+01,5.600e+00) (3.200e+01,9.900e+00) (6.400e+01,1.840e+01)};
        \addlegendentry{Ideal Gas};
    \end{axis}
    
    \begin{axis}[legend style={at={(0.97,0.03)}, anchor=south east},
                 axis y line*=right,
                 axis x line=none,
                 ylabel = {Deviation from \ac{WP} Tsat/\unit{\percent}},
                 ylabel near ticks,
                 ymin = 0,
                 ymax = 5,
                 xmin = 1,
                 xmax = 64,
                 xmode = log,
                 log basis x={4},
                 log ticks with fixed point,
                 height=6.5cm, width=10cm]
       
        \addplot[color=blue, dashed]
                coordinates {(1.000e+00,6.975e-01) (2.000e+00,4.830e-01) (4.000e+00,3.119e-01) (8.000e+00,3.607e-01) (1.600e+01,4.426e-01) (3.200e+01,5.288e-01) (6.400e+01,6.691e-01)};
        \addplot[color=red, dashed]
                coordinates {(1.000e+00,2.951e-01) (2.000e+00,3.813e-01) (4.000e+00,5.279e-01) (8.000e+00,7.665e-01) (1.600e+01,1.180e+00) (3.200e+01,1.939e+00) (6.400e+01,3.327e+00)};
    \end{axis}
    \label{axis_right}

\end{tikzpicture}
        \caption[The deviation of saturation temperature from the \ac{WP} \ac{EOS}]{The deviation of saturation temperature (from the \ac{WP} \ac{EOS}) when computed as part of a \ac{VLE} calculation in \emph{Reaktoro} assuming either the ideal gas and \ac{SRK} \ac{EOS} for the vapour phase and the SUPCRTLBL thermodynamic database.}
        \label{fig:Tsat_vs_P}
    \end{figure}   
    
\subsection{Empirical Models}
\label{sec:partitioning_models}

    Empirical models for specific geofluid mixtures, most commonly for mixtures comprised of water, and carbon dioxide as well as impurities , such as \(CH_4\), \(N_2\) and \(H_2S\) , have been developed. An example of this is a model originally presented by \citeauthor{Spycher2003} \cite{Spycher2003} for the mutual solubilities of carbon dioxide and water at low temperatures (\qty{12}{\degreeCelsius} to \qty{100}{\degreeCelsius}). This model was later extended to higher temperatures (\qty{12}{\degreeCelsius} to \qty{300}{\degreeCelsius}) by \citeauthor{Spycher2009} \cite{Spycher2009}. Corrections for salinity are applied using an approach similar to that of \citeauthor{Duan2003} \cite{Duan2003}.

    While such empirical models can be used to determine equilibrium phases and compositions, they do not provide methods for estimating the thermophysical properties of the fluid. Moreover, these models make simplifying assumptions, particularly regarding the reactivity of the various aqueous species, meaning that advanced phase behaviours, such as scaling/mineralisation, cannot be captured.

% \subsection{Conclusions}
% \label{sec:tppm_lit_review_conclusions}
    % \begin{itemize}
    %     \item EOS approaches can be used but not for all mixtures and all conditions
    %     \item Fluid specific EOS exist but conditions are limited to non geothermal conditions
    %     \item chemically reactive systems, provide a holistic approach, however availability of underlying models makes application difficult for two-phase systems
    %     \item Empirical models exsist but usually just provide information about the partitioning NOT the properties
    % \end{itemize}

\subsection{Calculation Libraries}
\label{sec:calc_frameworks}
    There are a number of calculation libraries that implement \ac{HEOS} for pure fluids and their mixtures, for example \emph{CoolProp} \cite{Bell2014}, \emph{REFPROP} \cite{Lemmon2018} and \emph{FluidProp} \cite{Colonna2019}. Out of the former, only \emph{CoolProp} is fully open-source, while the others are commercial libraries and represent an black box.

    \emph{Reaktoro} \cite{Leal2015} is a unified open-source framework for modelling chemically reactive systems. \emph{Reaktoro} pairs the aforementioned thermodynamic databases, \ac{EOS} and activity models with scalable optimisation algorithms \cite{Leal2017} and on-demand machine learning acceleration strategies \cite{Kyas2022, Leal2020}. \emph{Reaktoro} was used in \citeauthor{Walsh2017} to produce a computer code to compute both thermodynamic and thermophysical properties, such as viscosity and thermal conductivity. The core \emph{Reaktoro} calculation engine is written in C\textsuperscript{++} for performance reasons, with Python API provided for more convenient usage in Jupyter Notebooks and/or with the rich ecosystem of Python libraries.

    \emph{ThermoFun} \cite{Miron2021} is an open-source framework for calculating the thermodynamic properties of species and reactions from thermodynamic databases such as SUPCRT98 or SUPCRTBL. The core \emph{ThermoFun} calculation engine is written in C\textsuperscript{++} for performance reasons, but also offers a Python API for more convenient usage. \emph{ThermoFun} is integrated into \emph{Reaktoro}, where it can be used to provide the standard thermodynamic properties of species.
