The geofluid's phase behaviour and its thermophysical properties are fundamental inputs to the design of any geothermal power plant. These properties are expensive to obtain experimentally. In the absence of holistic or predictive models to handle the broad range of geofluid compositions we must often rely on mathematical models, equations of state and calibrating model parameters based on what little field data is available.

This chapter aims to provide an overview of the current state-of-the-art modelling approaches for predicting the thermophysical properties of fluids in general and more specifically geofluids. We then present a calculation framework that combines these modelling approaches, and illustrate its use in a number of case studies. Finally, we present a unified calculation framework that combines a number of properties estimation frameworks, which can be used for simulating power plants and other physical processes.   

% \section{Literature Review}
\label{sec:tppm_lit_review}
% The following section aims to provide an overview of the make-up of geothermal geofluids and the various techniques employed for the modelling of thermophysical properties of pure fluids and mixtures.

\section{Geofluids}
\label{sec:pure_fluids}
    Geofluids typically have three main constituents: water, minerals and gases\cite{DiPippo2016}. The compositions of minerals depends not only on the chemical composition of the reservoir rock, but also the composition the waters migrating through the rock and the left over waters from the original deposition of the reservoir rock. \ce{Na}, \ce{Ca}, \ce{K}, \ce{Fe}, \ce{Cl}, \ce{SO4} and \ce{HCO3} are common minerals found in geothermal geofluids. For the gases, \ce{CO2}, \ce{CH4} and \ce{H2S} are common constituents. \ce{CO2} is either produced as part of the dissolution of the reservoir rock (e.g. in case of a carbonate reservoir) or, similarly to \ce{H2S}, migrates into the formation from deeper within the Earth (e.g. magmatic origin). \ce{CH4} is typically created deeper within the Earth in hydrocarbon source rock and then migrates into the reservoir. \citeauthor{kottsova_2022} \cite{kottsova_2022} and \citeauthor{REFLECT_2022} \cite{REFLECT_2022} have collated geofluid compositions from a large number of geothermal wells across Europe. 
    
\section{Equations of State}
\label{sec:equations_of_state}
    In thermodynamics, substances can exist in different \emph{states}, where each \emph{state} is uniquely defined by \emph{state variables}, such as the temperature, pressure, and volume/density of the substance. In this context, an \ac{EOS} is any mathematical expression that correlates the state variables of a substance, allowing the all thermodynamic properties for any given state to be determined. The following sections aim to provide background on the most common \acp{EOS}.
    
    \subsection{Ideal Gas Law}
        The simplest \ac{EOS} is the \emph{Ideal Gas Law}, Equation~\ref{eq:ideal_gas_law}, which links the pressure, molar volume and temperature, and describes the behaviour of a gases, assuming non-interacting, spherical particles of zero volume. 
        
        \begin{align}
            P V_m = R T  \label{eq:ideal_gas_law}
        \end{align}

    \subsection{Real Gas Law}
        Over time, as experimental techniques and equipment improved, temperature and pressure dependent deviations from the ideal gas law were observed. To capture these deviations the \ac{Z} was introduced, leading to the \emph{Real Gas Law}, Equation~\ref{eq:real_gas_law}. Standing and Katz provided graphical methods for evaluating \ac{Z} from the reduced temperature and pressure, based on the law of corresponding states.
    
        \begin{align}
            P V_m = Z R T  \label{eq:real_gas_law}
        \end{align}

    \subsection{Cubic Equations of State}
        \citeauthor{Waals1873} was the first to present a cubic \ac{EOS} based on the the insight that particles did indeed have an irreducible volume and that there are short-ranged attractive interactions between particles. This is now known as the \ac{VDW} \ac{EOS}, Equation~\ref{eq:van_der_Waals_EOS}. One significant advance of the \ac{VDW} \ac{EOS} is that its domain is not restricted to just gases, but also allows the behaviour of liquids to be approximated - though it is considered non-predictive and typically underestimates liquid densities.
    
        \begin{align}
            P = \frac{R T}{V_m - b} + \frac{a}{V_m^2}  \label{eq:van_der_Waals_EOS}
        \end{align}

        To improve on the accuracy of liquid phase predictions modifications to the original \ac{VDW} \ac{EOS} were suggested. These modifications have primarily focused on the definition of the attraction term, for example \citeauthor{Redlich1949} modified the attraction term to include a temperature dependence and the irreducible volume, Equation~\ref{eq:RK_EOS}, \citeauthor{Soave1972} further modified the \ac{EOS} proposed by \citeauthor{Redlich1949} by including a temperature dependent \(\alpha\) term, Equation~\ref{eq:SRK_EOS} and \citeauthor{Peng1976} changed the definition of the denominator in the attraction term, Equation~\ref{eq:PR_EOS}. These \ac{EOS} are commonly referred to as the \ac{RK} \ac{EOS}, \ac{SRK} \ac{EOS} and \ac{PR} \ac{EOS}, respectively.
    
        \begin{align}
            P = \frac{R T}{V_m - b} + \frac{a}{\sqrt{T}V_m(V_m-b)}  \label{eq:RK_EOS}
        \end{align}
        \begin{align}
            P = \frac{R T}{V_m - b} + \frac{a\alpha (T)}{V_m(V_m-b)}  \label{eq:SRK_EOS}
        \end{align}
        \begin{align}
            P = \frac{R T}{V_m - b} + \frac{a\alpha (T)}{V_m^2 + 2V_mb-b^2}  \label{eq:PR_EOS}
        \end{align}

        Cubic \ac{EOS} above can also be used to model mixtures and have been successfully applied in the petroleum sector for many decades. When applied to mixtures, mixing rules are applied to determine the overall values of \(a\), see Equation~\ref{eq:mixing_rule_a}, and \(b\), see Equation~\ref{eq:mixing_rule_b}. Here, \ac{BIC}, \(k_{ij}\), are used to account for differences in the interactions between different species, and are usually obtain by tuning against experimental data of the mixture's phase behaviour.   

        \begin{align}
            a = \sum_{i=0}^N \sum_{j=0}^N x_i x_j \sqrt{a_i a_j} (1-k_{ij})  \label{eq:mixing_rule_a}
        \end{align}
        \begin{align}
            b = \sum_{i=0}^N x_i b_i  \label{eq:mixing_rule_b}
        \end{align}

    \subsection{Excess Property Formulations}
    \label{sec:excess_properties}
        \citeauthor{Peneloux1989} \cite{Peneloux1989} first investigated the use of \emph{Excess Property} formulations, for example in terms of the Helmholtz free energy, as single mathematical formulation to obtain all thermodynamic properties. Here, the \emph{excess} refers to the difference between the actual thermodynamic property or potential at a given state and that of an ideal gas. The equations below shows the \ac{HEOS} for Nitrogen \cite{Span2000}, where \(\delta=\frac{\rho}{\rho_c}\) and \(\tau =\frac{T_c}{T}\) with \(\rho_c\) and \(T_c\) being the critical density and temperature respectively; and \(a_1\) to \(a_7\),the \(N_k\)s, the \(i_k\)s, the \(j_k\)s, the \(l_k\)s, the \(\phi_k\)s, the \(beta_k\)s and the \(\gamma_k\)s as fluid specific correlation parameters - a total of \emph{\num{154}} parameters in the case of Nitrogen.

        \begin{align}
            \frac{a (\delta, \tau)}{R T} = \alpha^0 (\delta, \tau) + \alpha^r (\delta, \tau) \label{eq:helmholtz_EOS}
        \end{align}
        \begin{align}
            \alpha^0 (\delta, \tau) = \ln \delta + a_1 \ln \tau + a_2 + a_3 \tau + \frac{a_4}{\tau} + \frac{a_5}{\tau^2} + \frac{a_6}{\tau^3} + a_7 \ln[1-e^{-a_8\tau}]  \label{eq:HEOS_ig_term}
        \end{align}
        \begin{align}
            \alpha^r (\delta, \tau) = \sum_{k=1}^{6} N_k \delta^{i_k} \tau^{j_k} + \sum_{k=7}^{32} N_k \delta^{i_k} \tau^{j_k} e^{(-\delta^{l_k})} + \sum_{k=33}^{36} N_k \delta^{i_k} \tau^{j_k} e^{-\phi_k(\delta-1)^2-\beta_k(\tau-\gamma_k)^2} \label{eq:HEOS_residual}
        \end{align}

        As a result of their complexity, these \ac{HEOS} are typically only created for fluids where large experimental datasets over a wide range of thermodynamic conditions exist \cite{Lemmon1999}. Given the high-accuracy of \ac{HEOS}, limited only by the experimental uncertainty \cite{Lemmon1999},  they are primarily developed as reference \ac{EOS}s for specific common-use fluids to calculate thermodynamic property charts and look-up tables for scientific and industrial use.

        One such reference \ac{EOS} is the \ac{HEOS} developed by \citeauthor{Wagner2002} for the properties and phase behaviour of water, better known as the IAPWS95 \cite{IAPWS2018} formulation or the \ac{WP} \ac{EOS}. This original model was developed further by \citeauthor{Wagner2000} leading to the IAPWS97 \cite{IAPWS2018} formulation, which primarily improves the speed and is more commonly used in industry than scientific use. Similar reference \ac{EOS} have also been developed for other common geofluid components, such as carbon dioxide \cite{Span1996} (i.e. the \ac{SW} \ac{EOS}), nitrogen \cite{Span2000}, methane \cite{Setzmann1991} and hydrogen sulphide \cite{Lemmon2006}.

        \begin{figure}[H]
            \centering
            \begin{tikzpicture}[baseline]
    \begin{axis}[xlabel = {Temperature/\unit{\K}},
                 ylabel = {Pressure/\unit{\bar}},
                 legend style={at={(0.97, 0.03)}, anchor=south east},
                 ymode = log,
                 ymin = 0.1,
                 xmin = 250,
                 height=7cm,
                 width=8cm]
        \addplot[color=blue]
            coordinates {(273.16,0.006) (282.748,0.012) (292.336,0.022) (301.924,0.04) (311.512,0.068) (321.1,0.111) (330.688,0.178) (340.277,0.275) (349.865,0.414) (359.453,0.609) (369.041,0.874) (378.629,1.229) (388.217,1.696) (397.805,2.298) (407.393,3.064) (416.981,4.025) (426.569,5.214) (436.157,6.67) (445.745,8.431) (455.333,10.542) (464.922,13.047) (474.51,15.996) (484.098,19.441) (493.686,23.434) (503.274,28.034) (512.862,33.3) (522.45,39.294) (532.038,46.081) (541.626,53.73) (551.214,62.313) (560.802,71.906) (570.39,82.59) (579.978,94.452) (589.567,107.586) (599.155,122.099) (608.743,138.108) (618.331,155.754) (627.919,175.208) (637.507,196.692) (647.095,220.637) (647.095,220.637) (637.507,196.692) (627.919,175.208) (618.331,155.754) (608.743,138.108) (599.155,122.099) (589.567,107.586) (579.978,94.452) (570.39,82.59) (560.802,71.906) (551.214,62.313) (541.626,53.73) (532.038,46.081) (522.45,39.294) (512.862,33.3) (503.274,28.034) (493.686,23.434) (484.098,19.441) (474.51,15.996) (464.922,13.047) (455.333,10.542) (445.745,8.431) (436.157,6.67) (426.569,5.214) (416.981,4.025) (407.393,3.064) (397.805,2.298) (388.217,1.696) (378.629,1.229) (369.041,0.874) (359.453,0.609) (349.865,0.414) (340.277,0.275) (330.688,0.178) (321.1,0.111) (311.512,0.068) (301.924,0.04) (292.336,0.022) (282.748,0.012) (273.16,0.006)};
        \addlegendentry{\(Water\)}
        \addplot[color=red]
            coordinates {(216.592,5.18) (218.836,5.704) (221.081,6.268) (223.325,6.872) (225.57,7.519) (227.814,8.211) (230.059,8.949) (232.303,9.735) (234.548,10.572) (236.792,11.461) (239.037,12.403) (241.281,13.402) (243.526,14.459) (245.77,15.576) (248.015,16.754) (250.259,17.997) (252.504,19.307) (254.748,20.685) (256.993,22.133) (259.237,23.654) (261.482,25.25) (263.726,26.924) (265.971,28.678) (268.215,30.513) (270.46,32.434) (272.704,34.442) (274.949,36.54) (277.193,38.732) (279.438,41.019) (281.682,43.406) (283.927,45.895) (286.171,48.491) (288.416,51.198) (290.66,54.02) (292.905,56.962) (295.149,60.03) (297.394,63.232) (299.638,66.577) (301.883,70.08) (304.127,73.771) (301.883,70.08) (299.638,66.577) (297.394,63.232) (295.149,60.03) (292.905,56.962) (290.66,54.02) (288.416,51.198) (286.171,48.491) (283.927,45.895) (281.682,43.406) (279.438,41.019) (277.193,38.732) (274.949,36.54) (272.704,34.442) (270.46,32.434) (268.215,30.513) (265.971,28.678) (263.726,26.924) (261.482,25.25) (259.237,23.654) (256.993,22.133) (254.748,20.685) (252.504,19.307) (250.259,17.997) (248.015,16.754) (245.77,15.576) (243.526,14.459) (241.281,13.402) (239.037,12.403) (236.792,11.461) (234.548,10.572) (232.303,9.735) (230.059,8.949) (227.814,8.211) (225.57,7.519) (223.325,6.872) (221.081,6.268) (218.836,5.704)};
        \addlegendentry{\(CO_2\)}
            
    \end{axis}
\end{tikzpicture}%
~%
%
\begin{tikzpicture}[baseline]
    \begin{axis}[xlabel = {Enthalpy/\unit{\joule \per \mol}},
                 ylabel = {Pressure/\unit{\bar}},
                 legend style={at={(0.5,0.03)}, anchor=south},
                 ymode = log,
                 ymin = 0.1,
                 height=7cm,
                 width=8cm]
        \addplot[color=blue]
            coordinates {(-7549.426,0.006) (-6822.795,0.012) (-6099.057,0.022) (-5376.621,0.04) (-4654.607,0.068) (-3932.499,0.111) (-3209.957,0.178) (-2486.713,0.275) (-1762.516,0.414) (-1037.095,0.609) (-310.149,0.874) (418.663,1.229) (1149.722,1.696) (1883.452,2.298) (2620.319,3.064) (3360.829,4.025) (4105.534,5.214) (4855.034,6.67) (5609.981,8.431) (6371.087,10.542) (7139.135,13.047) (7914.986,15.996) (8699.602,19.441) (9494.058,23.434) (10299.575,28.034) (11117.551,33.3) (11949.607,39.294) (12797.653,46.081) (13663.974,53.73) (14551.362,62.313) (15463.302,71.906) (16404.255,82.59) (17380.11,94.452) (18398.923,107.586) (19472.244,122.099) (20617.874,138.108) (21866.943,155.754) (23286.343,175.208) (25056.85,196.692) (29845.253,220.637) (30157.268,220.637) (36255.986,196.692) (38002.414,175.208) (39179.569,155.754) (40059.906,138.108) (40746.4,122.099) (41292.188,107.586) (41729.218,94.452) (42078.551,82.59) (42354.966,71.906) (42569.309,62.313) (42729.831,53.73) (42842.991,46.081) (42913.975,39.294) (42947.035,33.3) (42945.727,28.034) (42913.07,23.434) (42851.668,19.441) (42763.8,15.996) (42651.488,13.047) (42516.55,10.542) (42360.654,8.431) (42185.353,6.67) (41992.12,5.214) (41782.381,4.025) (41557.526,3.064) (41318.929,2.298) (41067.943,1.696) (40805.889,1.229) (40534.048,0.874) (40253.639,0.609) (39965.808,0.414) (39671.62,0.275) (39372.052,0.178) (39067.993,0.111) (38760.247,0.068) (38449.528,0.04) (38136.453,0.022) (37821.544,0.012) (37505.22,0.006)};
        \addlegendentry{\(Water\)}
        \addplot[color=red]
            coordinates {(-8542.563,5.18) (-8348.728,5.704) (-8154.31,6.268) (-7959.246,6.872) (-7763.469,7.519) (-7566.91,8.211) (-7369.497,8.949) (-7171.153,9.735) (-6971.796,10.572) (-6771.341,11.461) (-6569.696,12.403) (-6366.764,13.402) (-6162.44,14.459) (-5956.611,15.576) (-5749.155,16.754) (-5539.939,17.997) (-5328.82,19.307) (-5115.642,20.685) (-4900.235,22.133) (-4682.411,23.654) (-4461.959,25.25) (-4238.645,26.924) (-4012.199,28.678) (-3782.315,30.513) (-3548.632,32.434) (-3310.734,34.442) (-3068.125,36.54) (-2820.214,38.732) (-2566.284,41.019) (-2305.439,43.406) (-2036.535,45.895) (-1758.065,48.491) (-1468.004,51.198) (-1163.567,54.02) (-840.796,56.962) (-493.647,60.03) (-111.567,63.232) (327.48,66.577) (883.531,70.08) (2688.586,73.771) (4507.633,70.08) (5041.028,66.577) (5404.377,63.232) (5682.78,60.03) (5907.697,56.962) (6094.968,54.02) (6253.895,51.198) (6390.477,48.491) (6508.842,45.895) (6611.968,43.406) (6702.086,41.019) (6780.916,38.732) (6849.818,36.54) (6909.891,34.442) (6962.037,32.434) (7007.011,30.513) (7045.448,28.678) (7077.891,26.924) (7104.809,25.25) (7126.607,23.654) (7143.636,22.133) (7156.207,20.685) (7164.592,19.307) (7169.034,17.997) (7169.752,16.754) (7166.945,15.576) (7160.798,14.459) (7151.48,13.402) (7139.148,12.403) (7123.95,11.461) (7106.02,10.572) (7085.483,9.735) (7062.456,8.949) (7037.046,8.211) (7009.354,7.519) (6979.478,6.872) (6947.506,6.268) (6913.526,5.704)};
        \addlegendentry{\(CO_2\)}
    \end{axis}
\end{tikzpicture}
            \caption[Saturation curves of water and carbon dioxide]{The saturation curves for water and carbon dioxide in Pressure-Temperature and Pressure-Molar Enthalpy domains. Calculated using the Water and Carbon Dioxide \ac{HEOS} implemented in \emph{CoolProp}.}
            \label{fig:SaturationCurves}
        \end{figure}

        \subsubsection{Mixtures}
        Methodologies for modelling mixtures of reference \ac{EOS} have been developed \cite{Lemmon1999b}, but require an additional set of parameters for each component pair to capture their interactions. The equations below outline the general formulation of the mixture model, with \(N_k\), \(d_k\), \(t_k\) common to all mixtures and \(F_{ij}\), \(\xi_{ij}\), \(\beta_{ij}\), \(\phi_{ij}\) and \(\zeta_{ij}\) specific to each component pair.

        \begin{align}
            A = A^{id\;mix} + A^{excess} \label{eq:HEOS_mixture}
        \end{align}
        \begin{align}
            A^{id\;mix} = \sum_{i=1}^n x_i \cdot \left[A^0_i(\rho,T) + A^r_i(\delta, \tau) + R T \ln x_i\right] \label{eq:HEOSmixture_id_term}
        \end{align}
        \begin{align}
            A^{excess} = RT \sum_{i=1}^{n-1} \sum_{j=i+1}^n x_i x_j F_{ij} \sum_{k=1}^{10} N_k \delta^{d_k} \tau^{t_k} \label{eq:HEOS_excess}
        \end{align}
        \begin{align}
            \delta = \frac{\rho}{\rho_{red}} \label{eq:reduced_density}
        \end{align}
        \begin{align}
            \tau = \frac{T_{red}}{T} \label{eq:reduced_temperature}
        \end{align}

        \begin{align}
            \rho_{red} = \left[\sum_{i=1}^n \frac{x_i}{\rho_c} + \sum_{i=1}^{n-1} \sum_{j=i+1}^n x_i x_j \xi_{ij} \right]^{-1} \label{eq:red_density}
        \end{align}
        \begin{align}
            T_{red} = \sum_{i=1}^{n} x_i T_{c_i} + \sum_{i=1}^{n-1} \sum_{j=i+1}{n} x_i^{\beta_{ij}} x_j^{\phi_{ij}}\zeta_{ij} \label{eq:red_temperature}
        \end{align}

    As a result of the complexity of the formulation, the convergence can be challenging and may not be guaranteed for all compositions, temperatures and pressure of interest. The convergence was investigated using binary mixtures of water and carbon dioxide over a range of temperatures (\qtyrange{298}{573}{\K} corresponding to \qtyrange{25}{300}{\degreeCelsius}) and pressures (\qtyrange{1}{300}{\bar}) using \emph{CoolProp} \cite{Bell2014}, see Section~\ref{sec:calc_frameworks}. 

    \begin{figure}[H]
        \centering
        \input{Content/TPPM/LiteratureReview/Plots/CP_H2O_CO2_calcmap}
        \caption{Convergence maps of the water-carbon dioxide binary \ac{HEOS} mixture implemented in \emph{CoolProp} for a range of compositions.}
        \label{fig:wat_co2_calcmap}
    \end{figure}

    From Figure~\ref{fig:wat_co2_calcmap}, for pure water (\(z_{CO_2}=\)\qty{0.00}{\mol\percent}), the \ac{HEOS} formulation does not converge for a broad range of temperatures and pressures, appearing to shadow the saturation line of pure water. Convergence generally improves for quasi pure water (\qty{0.00}{\mol\percent}\(\leq z_{CO_2}<\)\qty{0.05}{\mol\percent}), however the \ac{HEOS} formulation does not converge at elevated (approaching reservoir-like) pressures. The areas of non-convergence appear to be correlated where the fluid is expected to be liquid (i.e. low temperatures and high pressures). 

    \subsection{Incompressible Binary Mixtures}
    \label{sec:incompressible_fluids}
        Furthermore, \ac{EOS} have also been developed for some industrially relevant mixtures, like seawater \cite{Sharqawy2010}, lithium bromide solution \cite{Patek2006}, and calcium chloride solution \cite{Preisegger2010} or potassium carbonate solution \cite{Melinder2010}. However, the application range of such binary incompressible fluid \ac{EOS} is limited, Table 1, due to the scope in which these fluids are used /handled in industry (e.g. seawater in desalination plants or lithium bromide in adsorption cooling). Moreover, no interaction models exist to allow mixtures of these binary mixtures (e.g. seawater and lithium bromide) or binary mixtures with other pure components (e.g. seawater and carbon dioxide) to be modelled.

        \begin{table}[H]
            \caption[The applicability range of various incompressible fluid \ac{EOS}.]{The applicability range of various incompressible fluid \ac{EOS}. \(x_{min}\) and \(x_{max}\) are the lower and upper limit for the amount of the species other than water.}
            \centering 
            \label{table:IncompressibleEOS}
            \begin{tabular}{|p{10em} c c c c |}
    \hline
    \rowcolor{bluepoli!40} % comment this line to remove the color
    \textbf{Fluid}& \(\mathbf{T_{min}}\)\textbf{, \unit{\degreeCelsius}} & \(\mathbf{T_{max}}\)\textbf{, \unit{\degreeCelsius}} & \(\mathbf{x_{min}}\)\textbf{, \unit{\percent}} & \(\mathbf{x_{max}}\)\textbf{, \unit{\percent}}\T\B \\
    \hline \hline
    Seawater & 0 & 120 & 0 & 12 \T\B\\
    Lithium Bromide & 0 & 227 & 0 & 75 \T\B \\
    Calcium Chloride & -55 & 20 & 15 & 30 \T\B\\
    Potassium Carbonate & -100 & 0 &  & 40 \B\\
    \hline
\end{tabular}        
            \\[10pt]
        \end{table}
    
\subsection{Chemically reactive Systems}
\label{sec:chemically_active_system}
    An alternative approach is to treat the geofluid as a chemically reactive system. In such a system, the constituent species can partition into different phases (e.g. gaseous, aqueous – a water-rich liquid phase, solid, etc.), react with each other to form new species or dissociate into other species, Figure~\ref{fig:chemically_reactive_system}. Determining the amounts and composition of all phases at equilibrium, at a given temperature and pressure, is equivalent to assessing the geofluid’s phase behaviour and also allows the thermophysical properties of the individual phases and overall fluid to be obtained.

    \begin{figure}[H]
        \centering
        \begin{tikzpicture}
    \draw [draw=none, fill=cyan!20](210:4)--(0,0)--(90:4) arc (90:210:4)--cycle;
    \draw [draw=none, fill=gray!20](-30:4)--(0,0)--(-150:4) arc (-150:-30:4)--cycle;
    \draw [draw=none, fill=red!20](90:4)--(0,0)--(-30:4) arc (-30:90:4)--cycle;
    \draw[dashed] (0,0) -- (90:4);
    \draw[dashed] (0,0) -- (210:4);
    \draw[dashed] (0,0) -- (330:4);
    \draw circle(4)[thick];

    \draw [draw=none, postaction={decorate,decoration={text along path,text align=center,text={||Gaseous}}}] (90:3.5) arc (90:-30:3.5);
    \draw [draw=none, postaction={decorate,decoration={text along path,text align=center,text={||Solid}}}] (210:3.5) arc (210:330:3.5);
    \draw [draw=none, postaction={decorate,decoration={text along path,text align=center,text={||Aqueous}}}] (210:3.5) arc (210:90:3.5);

    % draw water
    \node (water) at ($(0,0) + (110:3)$) {\(H_2O\)};
    \node (steam) at ($(0,0) + (70:3)$) {\(H_2O\)};
    \draw[<->]{} (water) to (steam);

    % draw CO2
    \node (CO2) at ($(0,0) + (30:1.5)$) {\(CO_2\)};
    \node (CO3) at ($(0,0) + (150:1.5)$) {\(CO_3^{2-}\)};
    \node (HCO3) at ($(CO3) + (0,1)$) {\(HCO_3^{-}\)};
    \node (Cs) at ($(0,0) + (0,-1)$) {\(C\)};
    \draw[<->]{} (CO2) to (CO3);
    \draw[<->]{} (CO3) to (Cs);
    \draw[<->]{} (CO3) to (HCO3);
    \draw[<->]{} (CO2) to (Cs);

    % draw NaCL
    \node (NaCl) at ($(0,0) + (230:2.5)$) {\(NaCl\)};
    \node (Na) at ($(0,0) + (195:3)$) {\(Na^+\)};
    \node (Cl) at ($(0,0) + (190:2)$) {\(Cl^-\)};
    \draw[<->]{} (NaCl) to (Na);
    \draw[<->]{} (NaCl) to (Cl);
    
    
\end{tikzpicture}
        \caption{Schematic of a possible chemical reactive system describing a geofluid}
        \label{fig:chemically_reactive_system}
    \end{figure}

    A unit amount of geofluid of arbitrary overall composition can be approximated as a closed thermodynamic system (i.e. no mass transfer into or out of the system). From the Second Law of Thermodynamics, such systems reach equilibrium when the system entropy reaches a global maximum. For systems at constant temperature and pressure, it can be shown that this is consistent with the Gibbs free energy,~\ref{eq:Gibbs_energy} of the system reaching a global minimum, also see \nameref{ch:appendix_a}.

    \begin{align}
        G = H - TS \label{eq:Gibbs_energy}
    \end{align}

    Thus, determining the equilibrium state (i.e. finding the phase amounts and compositions) represents a minimisation problem (i.e. \(\min G(P, T, \mathbf{n}) \) ), subject to the constraint that the total mass of each chemical element (i.e. H, O, Na, Cl, etc.) is conserved across all species and phases considered (i.e. \(\sum_{i=0}^N w_{ij} n_i = b_j\), where \(w_{ij} \) is the number of atoms of element \(j\) that make up species \(i\), \(n_i\) is the amount of species \(i\), and \(b_j\) is the total amount of element \(j\) in the system).

    A convenient expression for calculating the Gibbs free energy of the system can be obtained by combining the differential form of the Gibbs free energy (at constant temperature and pressure), Equation~\ref{eq:gibbs_differential}, with the First Law of Thermodynamics (i.e. \(dU= \delta Q + \delta W + \sum _{i=1}^N \mu_i dn_i\)), assuming fully reversible processes (i.e. \(\delta Q = TdS\) ) and mechanical work (i.e. \(\delta W = pdV\)), followed by integration over the molar amounts. This allows the Gibbs free energy of the system to be calculated from the amounts of each species and their respective chemical potential, Equation~\ref{eq:gibbs_energy_from_chemical_potential}. Here, \(G\) is Gibbs free energy, \(P\) is pressure, \(T\) is temperature, \(U\) is internal energy, \(V\) is volume, \(S\) is entropy, \(\mu_i\) is the chemical potential of species \(i\), \(n_i\) is the number of moles of species \(i\) and \(\mathbf{y}\) is the vector of all species’ mole fractions.

    \begin{align}
        dG \big|_{P,T} = dU + PdV - TdS \label{eq:gibbs_differential}
    \end{align}
    \begin{align}
        G(P, T, \mathbf{n}) = \sum _{i=1}^N n_i \mu_i(P, T, \mathbf{y}) \label{eq:gibbs_energy_from_chemical_potential}
    \end{align}

    The partial derivatives of the chemical potential are the partial molar enthalpy, Equation~\ref{eq:partial_molar_enthalpy}, the partial molar entropy, Equation~\ref{eq:partial_molar_entropy}, and the partial molar volume, Equation~\ref{eq:partial_molar_volume}, - the thermodynamic properties of interest.

    \begin{align}
        h_i (P, T, \mathbf{y}) = \frac{ \partial\left( \frac{\mu_i (P, T, \mathbf{y})}{T} \right)}{\partial\left( \frac{1}{T} \right) } \Bigg|_{P, \mathbf{y}} \label{eq:partial_molar_enthalpy}
    \end{align}
    \begin{align}
        s_i (P, T, \mathbf{y}) = - \frac{ \partial\left( \mu_i (P, T, \mathbf{y})\right)}{\partial T} \Bigg|_{P, \mathbf{y}} \label{eq:partial_molar_entropy}
    \end{align}
    \begin{align}
        v_i (P, T, \mathbf{y}) = \frac{ \partial\left( \mu_i (P, T, \mathbf{y})\right)}{\partial P} \Bigg|_{T, \mathbf{y}} = \frac{1}{\rho_i (P, T, \mathbf{y}}) \label{eq:partial_molar_volume}
    \end{align} 

    \begin{notes}{Note}
        For convenience the above equations can also be re-written as shown in Equation~\ref{eq:partial_molar_prop}, where \(\Psi_i (P,T, \mathbf{y})\) is a placeholder for a thermodynamic property and the choice of \(x\) and \(f(x)\) depends on the thermodynamic property of interest, see Table~\ref{table:PartialMolarProperties}. 
        % For instance, when calculating the partial molar enthalpy, \(x=\frac{1}{T}\) and \(f(x)=x\), when calculating the partial molar entropy, \(x=T\) and \(f(x)=-1\), and when calculating the partial molar volume, \(x=P\) and \(f(x)=1\).

        \begin{align}
            \Psi_i (P, T, \mathbf{y}) = \frac{ \partial \left( f(x)*\mu_i (P, T, \mathbf{y}) \right)} {\partial x } \label{eq:partial_molar_prop}
        \end{align}

        \begin{table}[H]
            \caption{The definition of the auxiliary variable and function by partial molar property.}
            \centering 
            \label{table:PartialMolarProperties}
            \begin{tabular}{|c c c |}
    \hline
    \rowcolor{bluepoli!40} % comment this line to remove the color
    \(\mathbf{\Psi_i}\)& \(\mathbf{x}\) & \(\mathbf{f(x)}\) \T\B \\
    \hline \hline
    \rowcolor{white} \(h\) & \(\frac{1}{T}\) & \(x\) \T\B\\
    \rowcolor{white}\(s\) & \(T\) & \(-1\) \T\B \\
    \rowcolor{white}\(v\) & \(P\) & \(1\) \T\B\\
    \hline
\end{tabular}        
            \\[10pt]
        \end{table}
    \end{notes}

    The calculation of the chemical potential is broken down into two components; the standard chemical potential of the species at a reference state, and the species’ activity, Equation~\ref{eq:chemical_potential}. The species’ activity is defined as the difference between the actual chemical potential and the standard chemical potential, Equation~\ref{eq:activity_def}.
    
    The reference state is chosen by convention: For liquid or gaseous species, the reference composition, \(y^o\), is that of the pure component, whereas for aqueous species (e.g. \(Na^+\)), a 1-molal solution of the solute is chosen, with all other species at infinite dilution. Meanwhile, the reference pressure for liquid and aqueous species is taken as the system pressure (i.e. \(P^o=P\)), whereas for gases, the reference pressure is taken to be \qty{1}{\bar} (i.e. \(P^o=1\) \unit{\bar}).

    \begin{align}
        \mu_i (P, T, \mathbf{y}) = \mu_i (P^o, T^o, \mathbf{y}^o) + RT \ln a_i (P, T, \mathbf{y}) \label{eq:chemical_potential}
    \end{align}
        \begin{align}
        RT \ln a_i (P, T, \mathbf{y}) \equiv \mu_i (P, T, \mathbf{y}) - \mu_i (P^o, T^o, \mathbf{y}^o) \label{eq:activity_def}
    \end{align}

    With the above in mind, determining the equilibrium composition and thermophysical properties of a geofluid at a given temperature and pressure requires three inputs: 1) The amounts of all elements across all species, 2) the chemical potential of all species at their respective reference state (also called the standard chemical potential) and 3) the activity of all species.
    
    The elemental amounts can be obtained from the geofluid composition, which is specific to each geothermal site as it is dependent on several factors, such as reservoir rock composition, temperature and pressure. Thus, the geofluid composition can only reliably be obtained from geofluid samples.

    \subsubsection{Standard Chemical Potential}
    The species’ standard chemical potential can be obtained from peer-reviewed open-source databases, such as SUPCRT92 \cite{Johnson1992} or SUPCRTBL \cite{Zimmer2016}. Frameworks, such as ThermoFun \cite{Miron2021} and Reaktoro \cite{Leal2015}, implement several models for computing standard thermodynamic properties from such databases. Alternatively, high-fidelity \ac{EOS} for species, such as water and carbon dioxide, can be used. Computationally cheaper \ac{EOS}, such as \ac{SRK} or \ac{PR}, can be used, provided their input parameters (e.g. critical properties, acentric factor, etc.) have been calibrated to the specific component in question.

    \subsubsection{Activity Models}
    The species’ activity can be calculated from phase and species-specific activity models. The simplest activity models approximate the fluid as an ideal fluid (i.e. Ideal Gas, Ideal Solution or Ideal Solid). However, this approach limits their application to low concentrations (for Ideal Solutions) or low pressures and high temperatures (for Ideal Gases), where the species exhibit ideal behaviour and where interactions among molecules are negligible. For gaseous species, such as \(CH_4\), \(N2\), \(H2S\), etc., the \ac{SRK} \ac{EOS} or the \ac{PR} \ac{EOS} may also be used to approximate the real gas behaviour and interactions among other gaseous species.

    Various activity models have been proposed for the different types of aqueous species. For example, the Setschenow equation \cite{Setschenow1889} for neutral species, the HKF-Debye-Hückel model \cite{Helgeson1981} for water and ionic species, or the Pitzer model for various aqueous species \cite{Pitzer1973}. Moreover, species-specific activity models have been developed for common mixtures of species. For example, for mixtures of \(H_2O\), \(CO_2\), \(CH_4\) and some mineral species, models by \citeauthor{Duan2003}, \citeauthor{Spycher2003} and \citeauthor{Spycher2009}, amongst others, can be used. The selection of an activity model is ultimately dependent on the species present, their relative amounts as well as the system temperature and pressure.

    \subsubsection{Limitations}
    \label{sec:chemically_active_system_limitations}
    In principle, chemically reactive systems allow any number of species and reactions to be modelled. However, the main barrier to this approach, being applied universally to geofluid modelling in a geothermal context, is the availability of appropriate activity models for all species - particularly gaseous water (i.e. steam). While the \ac{WP} \ac{EOS} represents the highest fidelity model for the properties of water and steam \cite{IAPWS2018}, it is computationally expensive and it only works with a single component: water. For this reason, most geochemical modelling codes (e.g., PHREEQC, GEMS, Reaktoro) adopt cubic equations of state for the vapour phase, such as the \ac{PR} \ac{EOS} or the \ac{SRK} \ac{EOS}, to permit other gases such as \(CO_2\), \(H_2S\), \(O_2\), and others to be considered. However, this can result in deviations from the expected phase behaviour when water steam is in higher proportion compared to other gases or simply the only gaseous species.

    For example, for pure water at a pressure of \qty{10}{\bar} the \ac{WP} \ac{EOS} predicts a saturation temperature of around \qty{453}{\K}. To model the same fluid using a chemically reactive system we assume a system consisting of only an aqueous and a gaseous water species (i.e. \(H_2O^{(aq)}\) and \(H_2O^{(g)}\). This system was then simulated in \emph{Reaktoro}, see ~\ref{sec:calc_frameworks}, as part of a \ac{VLE} calculation, for a pressure of \qty{10}{\bar} and a temperature between \qty{445}{\K} and \qty{465}{\K}. The SUPCRTBL database was used for the standard thermodynamic properties and different \ac{EOS} (ideal gas and \ac{SRK}) as activity models for the vapour phase (i.e., \(H_2O^{(g)}\)). The specific volume of the fluid was evaluated for each state and compared against values calculated via the \ac{WP} \ac{EOS} (Figure~\ref{fig:Tsat_at_P}). In the case of \emph{Reaktoro}, the saturation temperature was inferred by the temperature at which the transition from liquid-like to vapour-like densities occurs.

    \begin{figure}[H]
        \centering
        \begin{tikzpicture}
    \begin{axis}[xlabel = {Temperature/\unit{\K}},
                 ylabel = {Specific Volume/\unit{\cubic\m \per \kg}},
                 legend style={at={(0.97,0.03)}, anchor=south east},
                 width=12cm,
                 height=7cm,
                 xmin=445,
                 xmax=465,
                 ymin = 0,
                 y tick label style={/pgf/number format/.cd,fixed,fixed zerofill,precision=2,/tikz/.cd},]
        
        % Plot Liquid Lines
        \addplot[color=blue]
                coordinates {(4.450e+02,1.116e-03) (4.451e+02,1.117e-03) (4.452e+02,1.117e-03) (4.453e+02,1.117e-03) (4.454e+02,1.117e-03) (4.455e+02,1.117e-03) (4.456e+02,1.117e-03) (4.457e+02,1.117e-03) (4.458e+02,1.118e-03) (4.459e+02,1.118e-03) (4.460e+02,1.118e-03) (4.461e+02,1.118e-03) (4.462e+02,1.118e-03) (4.463e+02,1.118e-03) (4.464e+02,1.118e-03) (4.465e+02,1.118e-03) (4.466e+02,1.119e-03) (4.467e+02,1.119e-03) (4.468e+02,1.119e-03) (4.469e+02,1.119e-03) (4.470e+02,1.119e-03) (4.471e+02,1.119e-03) (4.472e+02,1.119e-03) (4.473e+02,1.120e-03) (4.474e+02,1.120e-03) (4.475e+02,1.120e-03) (4.476e+02,1.120e-03) (4.477e+02,1.120e-03) (4.478e+02,1.120e-03) (4.479e+02,1.120e-03) (4.480e+02,1.120e-03) (4.481e+02,1.121e-03) (4.482e+02,1.121e-03) (4.483e+02,1.121e-03) (4.484e+02,1.121e-03) (4.485e+02,1.121e-03) (4.486e+02,1.121e-03) (4.487e+02,1.121e-03) (4.488e+02,1.122e-03) (4.489e+02,1.122e-03) (4.490e+02,1.122e-03) (4.491e+02,1.122e-03) (4.492e+02,1.122e-03) (4.493e+02,1.122e-03) (4.494e+02,1.122e-03) (4.495e+02,1.122e-03) (4.496e+02,1.123e-03) (4.497e+02,1.123e-03) (4.498e+02,1.123e-03) (4.499e+02,1.123e-03) (4.500e+02,1.123e-03) (4.501e+02,1.123e-03) (4.502e+02,1.123e-03) (4.503e+02,1.124e-03) (4.504e+02,1.124e-03) (4.505e+02,1.124e-03) (4.506e+02,1.124e-03) (4.507e+02,1.124e-03) (4.508e+02,1.124e-03) (4.509e+02,1.124e-03) (4.510e+02,1.125e-03) (4.511e+02,1.125e-03) (4.512e+02,1.125e-03) (4.513e+02,1.125e-03) (4.514e+02,1.125e-03) (4.515e+02,1.125e-03) (4.516e+02,1.125e-03) (4.517e+02,1.125e-03) (4.518e+02,1.126e-03) (4.519e+02,1.126e-03) (4.520e+02,1.126e-03) (4.521e+02,1.126e-03) (4.522e+02,1.126e-03) (4.523e+02,1.126e-03) (4.524e+02,1.126e-03) (4.525e+02,1.127e-03) (4.526e+02,1.127e-03) (4.527e+02,1.127e-03) (4.528e+02,1.127e-03) (4.529e+02,1.127e-03)};
        \addlegendentry{\ac{WP} \ac{EOS}};

        \addplot[color=red]
                coordinates {(4.450e+02,1.116e-03) (4.451e+02,1.117e-03) (4.452e+02,1.117e-03) (4.453e+02,1.117e-03) (4.454e+02,1.117e-03) (4.455e+02,1.117e-03) (4.456e+02,1.117e-03) (4.457e+02,1.117e-03) (4.458e+02,1.118e-03) (4.459e+02,1.118e-03) (4.460e+02,1.118e-03) (4.461e+02,1.118e-03) (4.462e+02,1.118e-03) (4.463e+02,1.118e-03) (4.464e+02,1.118e-03) (4.465e+02,1.118e-03) (4.466e+02,1.119e-03) (4.467e+02,1.119e-03) (4.468e+02,1.119e-03) (4.469e+02,1.119e-03) (4.470e+02,1.119e-03) (4.471e+02,1.119e-03) (4.472e+02,1.119e-03) (4.473e+02,1.120e-03) (4.474e+02,1.120e-03) (4.475e+02,1.120e-03) (4.476e+02,1.120e-03) (4.477e+02,1.120e-03) (4.478e+02,1.120e-03) (4.479e+02,1.120e-03) (4.480e+02,1.120e-03) (4.481e+02,1.121e-03) (4.482e+02,1.121e-03) (4.483e+02,1.121e-03) (4.484e+02,1.121e-03) (4.485e+02,1.121e-03) (4.486e+02,1.121e-03) (4.487e+02,1.121e-03) (4.488e+02,1.122e-03) (4.489e+02,1.122e-03) (4.490e+02,1.122e-03) (4.491e+02,1.122e-03) (4.492e+02,1.122e-03) (4.493e+02,1.122e-03) (4.494e+02,1.122e-03) (4.495e+02,1.122e-03) (4.496e+02,1.123e-03) (4.497e+02,1.123e-03) (4.498e+02,1.123e-03) (4.499e+02,1.123e-03) (4.500e+02,1.123e-03) (4.501e+02,1.123e-03) (4.502e+02,1.123e-03) (4.503e+02,1.124e-03) (4.504e+02,1.124e-03) (4.505e+02,1.124e-03) (4.506e+02,1.124e-03) (4.507e+02,1.124e-03) (4.508e+02,1.124e-03) (4.509e+02,1.124e-03) (4.510e+02,1.125e-03) (4.511e+02,1.125e-03) (4.512e+02,1.125e-03) (4.513e+02,1.125e-03) (4.514e+02,1.125e-03) (4.515e+02,1.125e-03) (4.516e+02,1.125e-03) (4.517e+02,1.125e-03) (4.518e+02,1.126e-03) (4.519e+02,1.126e-03) (4.520e+02,1.126e-03) (4.521e+02,1.126e-03) (4.522e+02,1.126e-03) (4.523e+02,1.126e-03) (4.524e+02,1.126e-03) (4.525e+02,1.127e-03) (4.526e+02,1.127e-03) (4.527e+02,1.127e-03) (4.528e+02,1.127e-03) (4.529e+02,1.127e-03) (4.530e+02,1.127e-03) (4.531e+02,1.127e-03) (4.532e+02,1.128e-03) (4.533e+02,1.128e-03) (4.534e+02,1.128e-03) (4.535e+02,1.128e-03) (4.536e+02,1.128e-03) (4.537e+02,1.128e-03) (4.538e+02,1.128e-03) (4.539e+02,1.128e-03) (4.540e+02,1.129e-03) (4.541e+02,1.129e-03) (4.542e+02,1.129e-03) (4.543e+02,1.129e-03) (4.544e+02,1.129e-03) (4.545e+02,1.129e-03) (4.546e+02,1.129e-03) (4.547e+02,1.130e-03) (4.548e+02,1.130e-03) (4.549e+02,1.130e-03) (4.551e+02,1.130e-03) (4.552e+02,1.130e-03) (4.553e+02,1.130e-03) (4.554e+02,1.130e-03) (4.555e+02,1.131e-03) (4.556e+02,1.131e-03) (4.557e+02,1.131e-03) (4.558e+02,1.131e-03) (4.559e+02,1.131e-03) (4.560e+02,1.131e-03) (4.561e+02,1.131e-03) (4.562e+02,1.132e-03) (4.563e+02,1.132e-03) (4.564e+02,1.132e-03) (4.565e+02,1.132e-03) (4.566e+02,1.132e-03) (4.567e+02,1.132e-03) (4.568e+02,1.132e-03) (4.569e+02,1.133e-03)};
        \addlegendentry{Ideal Gas};

        \addplot[color=green]
                coordinates {(4.450e+02,1.116e-03) (4.451e+02,1.117e-03) (4.452e+02,1.117e-03) (4.453e+02,1.117e-03) (4.454e+02,1.117e-03) (4.455e+02,1.117e-03) (4.456e+02,1.117e-03) (4.457e+02,1.117e-03) (4.458e+02,1.118e-03) (4.459e+02,1.118e-03) (4.460e+02,1.118e-03) (4.461e+02,1.118e-03) (4.462e+02,1.118e-03) (4.463e+02,1.118e-03) (4.464e+02,1.118e-03) (4.465e+02,1.118e-03) (4.466e+02,1.119e-03) (4.467e+02,1.119e-03) (4.468e+02,1.119e-03) (4.469e+02,1.119e-03) (4.470e+02,1.119e-03) (4.471e+02,1.119e-03) (4.472e+02,1.119e-03) (4.473e+02,1.120e-03) (4.474e+02,1.120e-03) (4.475e+02,1.120e-03) (4.476e+02,1.120e-03) (4.477e+02,1.120e-03) (4.478e+02,1.120e-03) (4.479e+02,1.120e-03) (4.480e+02,1.120e-03) (4.481e+02,1.121e-03) (4.482e+02,1.121e-03) (4.483e+02,1.121e-03) (4.484e+02,1.121e-03) (4.485e+02,1.121e-03) (4.486e+02,1.121e-03) (4.487e+02,1.121e-03) (4.488e+02,1.122e-03) (4.489e+02,1.122e-03) (4.490e+02,1.122e-03) (4.491e+02,1.122e-03) (4.492e+02,1.122e-03) (4.493e+02,1.122e-03) (4.494e+02,1.122e-03) (4.495e+02,1.122e-03) (4.496e+02,1.123e-03) (4.497e+02,1.123e-03) (4.498e+02,1.123e-03) (4.499e+02,1.123e-03) (4.500e+02,1.123e-03) (4.501e+02,1.123e-03) (4.502e+02,1.123e-03) (4.503e+02,1.124e-03) (4.504e+02,1.124e-03) (4.505e+02,1.124e-03) (4.506e+02,1.124e-03) (4.507e+02,1.124e-03) (4.508e+02,1.124e-03) (4.509e+02,1.124e-03) (4.510e+02,1.125e-03) (4.511e+02,1.125e-03) (4.512e+02,1.125e-03) (4.513e+02,1.125e-03) (4.514e+02,1.125e-03) (4.515e+02,1.125e-03) (4.516e+02,1.125e-03) (4.517e+02,1.125e-03) (4.518e+02,1.126e-03) (4.519e+02,1.126e-03) (4.520e+02,1.126e-03) (4.521e+02,1.126e-03) (4.522e+02,1.126e-03) (4.523e+02,1.126e-03) (4.524e+02,1.126e-03) (4.525e+02,1.127e-03) (4.526e+02,1.127e-03) (4.527e+02,1.127e-03) (4.528e+02,1.127e-03) (4.529e+02,1.127e-03) (4.530e+02,1.127e-03) (4.531e+02,1.127e-03) (4.532e+02,1.128e-03) (4.533e+02,1.128e-03) (4.534e+02,1.128e-03) (4.535e+02,1.128e-03) (4.536e+02,1.128e-03) (4.537e+02,1.128e-03) (4.538e+02,1.128e-03) (4.539e+02,1.128e-03) (4.540e+02,1.129e-03) (4.541e+02,1.129e-03) (4.542e+02,1.129e-03) (4.543e+02,1.129e-03) (4.544e+02,1.129e-03) (4.545e+02,1.129e-03) (4.546e+02,1.129e-03)};
        \addlegendentry{\ac{SRK} \ac{EOS}};

        % Vapour Lines
        \addplot[color=blue]
                coordinates {(4.531e+02,1.944e-01) (4.532e+02,1.944e-01) (4.533e+02,1.945e-01) (4.534e+02,1.946e-01) (4.535e+02,1.946e-01) (4.536e+02,1.947e-01) (4.537e+02,1.947e-01) (4.538e+02,1.948e-01) (4.539e+02,1.949e-01) (4.540e+02,1.949e-01) (4.541e+02,1.950e-01) (4.542e+02,1.950e-01) (4.543e+02,1.951e-01) (4.544e+02,1.952e-01) (4.545e+02,1.952e-01) (4.546e+02,1.953e-01) (4.547e+02,1.953e-01) (4.548e+02,1.954e-01) (4.549e+02,1.955e-01) (4.551e+02,1.955e-01) (4.552e+02,1.956e-01) (4.553e+02,1.956e-01) (4.554e+02,1.957e-01) (4.555e+02,1.958e-01) (4.556e+02,1.958e-01) (4.557e+02,1.959e-01) (4.558e+02,1.959e-01) (4.559e+02,1.960e-01) (4.560e+02,1.961e-01) (4.561e+02,1.961e-01) (4.562e+02,1.962e-01) (4.563e+02,1.962e-01) (4.564e+02,1.963e-01) (4.565e+02,1.964e-01) (4.566e+02,1.964e-01) (4.567e+02,1.965e-01) (4.568e+02,1.965e-01) (4.569e+02,1.966e-01) (4.570e+02,1.967e-01) (4.571e+02,1.967e-01) (4.572e+02,1.968e-01) (4.573e+02,1.968e-01) (4.574e+02,1.969e-01) (4.575e+02,1.970e-01) (4.576e+02,1.970e-01) (4.577e+02,1.971e-01) (4.578e+02,1.971e-01) (4.579e+02,1.972e-01) (4.580e+02,1.973e-01) (4.581e+02,1.973e-01) (4.582e+02,1.974e-01) (4.583e+02,1.974e-01) (4.584e+02,1.975e-01) (4.585e+02,1.975e-01) (4.586e+02,1.976e-01) (4.587e+02,1.977e-01) (4.588e+02,1.977e-01) (4.589e+02,1.978e-01) (4.590e+02,1.978e-01) (4.591e+02,1.979e-01) (4.592e+02,1.980e-01) (4.593e+02,1.980e-01) (4.594e+02,1.981e-01) (4.595e+02,1.981e-01) (4.596e+02,1.982e-01) (4.597e+02,1.983e-01) (4.598e+02,1.983e-01) (4.599e+02,1.984e-01) (4.600e+02,1.984e-01) (4.601e+02,1.985e-01) (4.602e+02,1.985e-01) (4.603e+02,1.986e-01) (4.604e+02,1.987e-01) (4.605e+02,1.987e-01) (4.606e+02,1.988e-01) (4.607e+02,1.988e-01) (4.608e+02,1.989e-01) (4.609e+02,1.990e-01) (4.610e+02,1.990e-01) (4.611e+02,1.991e-01) (4.612e+02,1.991e-01) (4.613e+02,1.992e-01) (4.614e+02,1.993e-01) (4.615e+02,1.993e-01) (4.616e+02,1.994e-01) (4.617e+02,1.994e-01) (4.618e+02,1.995e-01) (4.619e+02,1.995e-01) (4.620e+02,1.996e-01) (4.621e+02,1.997e-01) (4.622e+02,1.997e-01) (4.623e+02,1.998e-01) (4.624e+02,1.998e-01) (4.625e+02,1.999e-01) (4.626e+02,2.000e-01) (4.627e+02,2.000e-01) (4.628e+02,2.001e-01) (4.629e+02,2.001e-01) (4.630e+02,2.002e-01) (4.631e+02,2.002e-01) (4.632e+02,2.003e-01) (4.633e+02,2.004e-01) (4.634e+02,2.004e-01) (4.635e+02,2.005e-01) (4.636e+02,2.005e-01) (4.637e+02,2.006e-01) (4.638e+02,2.007e-01) (4.639e+02,2.007e-01) (4.640e+02,2.008e-01) (4.641e+02,2.008e-01) (4.642e+02,2.009e-01) (4.643e+02,2.009e-01) (4.644e+02,2.010e-01) (4.645e+02,2.011e-01) (4.646e+02,2.011e-01) (4.647e+02,2.012e-01) (4.648e+02,2.012e-01) (4.649e+02,2.013e-01) (4.650e+02,2.013e-01)};
        
        \addplot[color=red]
                coordinates {(4.571e+02,2.109e-01) (4.572e+02,2.109e-01) (4.573e+02,2.110e-01) (4.574e+02,2.110e-01) (4.575e+02,2.111e-01) (4.576e+02,2.111e-01) (4.577e+02,2.112e-01) (4.578e+02,2.112e-01) (4.579e+02,2.113e-01) (4.580e+02,2.113e-01) (4.581e+02,2.114e-01) (4.582e+02,2.114e-01) (4.583e+02,2.115e-01) (4.584e+02,2.115e-01) (4.585e+02,2.115e-01) (4.586e+02,2.116e-01) (4.587e+02,2.116e-01) (4.588e+02,2.117e-01) (4.589e+02,2.117e-01) (4.590e+02,2.118e-01) (4.591e+02,2.118e-01) (4.592e+02,2.119e-01) (4.593e+02,2.119e-01) (4.594e+02,2.120e-01) (4.595e+02,2.120e-01) (4.596e+02,2.121e-01) (4.597e+02,2.121e-01) (4.598e+02,2.121e-01) (4.599e+02,2.122e-01) (4.600e+02,2.122e-01) (4.601e+02,2.123e-01) (4.602e+02,2.123e-01) (4.603e+02,2.124e-01) (4.604e+02,2.124e-01) (4.605e+02,2.125e-01) (4.606e+02,2.125e-01) (4.607e+02,2.126e-01) (4.608e+02,2.126e-01) (4.609e+02,2.127e-01) (4.610e+02,2.127e-01) (4.611e+02,2.128e-01) (4.612e+02,2.128e-01) (4.613e+02,2.128e-01) (4.614e+02,2.129e-01) (4.615e+02,2.129e-01) (4.616e+02,2.130e-01) (4.617e+02,2.130e-01) (4.618e+02,2.131e-01) (4.619e+02,2.131e-01) (4.620e+02,2.132e-01) (4.621e+02,2.132e-01) (4.622e+02,2.133e-01) (4.623e+02,2.133e-01) (4.624e+02,2.134e-01) (4.625e+02,2.134e-01) (4.626e+02,2.134e-01) (4.627e+02,2.135e-01) (4.628e+02,2.135e-01) (4.629e+02,2.136e-01) (4.630e+02,2.136e-01) (4.631e+02,2.137e-01) (4.632e+02,2.137e-01) (4.633e+02,2.138e-01) (4.634e+02,2.138e-01) (4.635e+02,2.139e-01) (4.636e+02,2.139e-01) (4.637e+02,2.140e-01) (4.638e+02,2.140e-01) (4.639e+02,2.141e-01) (4.640e+02,2.141e-01) (4.641e+02,2.141e-01) (4.642e+02,2.142e-01) (4.643e+02,2.142e-01) (4.644e+02,2.143e-01) (4.645e+02,2.143e-01) (4.646e+02,2.144e-01) (4.647e+02,2.144e-01) (4.648e+02,2.145e-01) (4.649e+02,2.145e-01) (4.650e+02,2.146e-01)};

        \addplot[color=green]
                coordinates {(4.548e+02,1.995e-01) (4.549e+02,1.995e-01) (4.551e+02,1.996e-01) (4.552e+02,1.996e-01) (4.553e+02,1.997e-01) (4.554e+02,1.997e-01) (4.555e+02,1.998e-01) (4.556e+02,1.998e-01) (4.557e+02,1.999e-01) (4.558e+02,1.999e-01) (4.559e+02,2.000e-01) (4.560e+02,2.000e-01) (4.561e+02,2.001e-01) (4.562e+02,2.001e-01) (4.563e+02,2.002e-01) (4.564e+02,2.003e-01) (4.565e+02,2.003e-01) (4.566e+02,2.004e-01) (4.567e+02,2.004e-01) (4.568e+02,2.005e-01) (4.569e+02,2.005e-01) (4.570e+02,2.006e-01) (4.571e+02,2.006e-01) (4.572e+02,2.007e-01) (4.573e+02,2.007e-01) (4.574e+02,2.008e-01) (4.575e+02,2.008e-01) (4.576e+02,2.009e-01) (4.577e+02,2.009e-01) (4.578e+02,2.010e-01) (4.579e+02,2.010e-01) (4.580e+02,2.011e-01) (4.581e+02,2.011e-01) (4.582e+02,2.012e-01) (4.583e+02,2.012e-01) (4.584e+02,2.013e-01) (4.585e+02,2.013e-01) (4.586e+02,2.014e-01) (4.587e+02,2.014e-01) (4.588e+02,2.015e-01) (4.589e+02,2.015e-01) (4.590e+02,2.016e-01) (4.591e+02,2.016e-01) (4.592e+02,2.017e-01) (4.593e+02,2.017e-01) (4.594e+02,2.018e-01) (4.595e+02,2.018e-01) (4.596e+02,2.019e-01) (4.597e+02,2.019e-01) (4.598e+02,2.020e-01) (4.599e+02,2.020e-01) (4.600e+02,2.021e-01) (4.601e+02,2.021e-01) (4.602e+02,2.022e-01) (4.603e+02,2.022e-01) (4.604e+02,2.023e-01) (4.605e+02,2.023e-01) (4.606e+02,2.024e-01) (4.607e+02,2.024e-01) (4.608e+02,2.025e-01) (4.609e+02,2.025e-01) (4.610e+02,2.026e-01) (4.611e+02,2.027e-01) (4.612e+02,2.027e-01) (4.613e+02,2.028e-01) (4.614e+02,2.028e-01) (4.615e+02,2.029e-01) (4.616e+02,2.029e-01) (4.617e+02,2.030e-01) (4.618e+02,2.030e-01) (4.619e+02,2.031e-01) (4.620e+02,2.031e-01) (4.621e+02,2.032e-01) (4.622e+02,2.032e-01) (4.623e+02,2.033e-01) (4.624e+02,2.033e-01) (4.625e+02,2.034e-01) (4.626e+02,2.034e-01) (4.627e+02,2.035e-01) (4.628e+02,2.035e-01) (4.629e+02,2.036e-01) (4.630e+02,2.036e-01) (4.631e+02,2.037e-01) (4.632e+02,2.037e-01) (4.633e+02,2.038e-01) (4.634e+02,2.038e-01) (4.635e+02,2.039e-01) (4.636e+02,2.039e-01) (4.637e+02,2.040e-01) (4.638e+02,2.040e-01) (4.639e+02,2.041e-01) (4.640e+02,2.041e-01) (4.641e+02,2.042e-01) (4.642e+02,2.042e-01) (4.643e+02,2.043e-01) (4.644e+02,2.043e-01) (4.645e+02,2.044e-01) (4.646e+02,2.044e-01) (4.647e+02,2.045e-01) (4.648e+02,2.045e-01) (4.649e+02,2.046e-01) (4.650e+02,2.046e-01)};

        % Saturation Lines
        \addplot[color=blue, dashed]
                coordinates {(453.0904,1.127e-03) (453.0904,1.944e-01)};
        \addplot[color=red, dashed]
                coordinates {(457.010,1.133e-03) (457.010,2.109e-01)};
        \addplot[color=green, dashed]
                coordinates {(454.7989,1.129e-03) (454.7989,1.995e-01)};

        \node (WP) at (axis cs:451, 0.175) {\qty{453.1}{\K}};
        \node (SRK) at (axis cs:459, 0.15) {\qty{454.8}{\K}};
        \node (Ideal) at (axis cs:459, 0.1) {\qty{457.0}{\K}};

        \draw (WP) -- (axis cs:453.0904, 0.15) {};
        \draw (SRK) -- (axis cs:454.7989, 0.125) {};
        \draw (Ideal) -- (axis cs:457.010, 0.075) {};
        
    \end{axis}
\end{tikzpicture}
        \caption[The specific volume of water as calculated with \emph{Reaktoro}]{The specific volume of pure water at a pressure of \qty{10}{\bar} over temperatures \num{445} to \qty{465}{\K} calculated with \emph{Reaktoro} using the Ideal Gas and \ac{SRK} \ac{EOS}, compared to the \ac{WP} \ac{EOS}.}
        \label{fig:Tsat_at_P}
    \end{figure}

    Depending on the equation of state selected in \emph{Reaktoro}, the specific volume differs from the values predicted by the \ac{WP} \ac{EOS} by \qty{1.3}{\percent} (\ac{SRK}) and \qty{6.5}{\percent} (Ideal Gas) (Figure~\ref{fig:Tsat_at_P}). These differences indicate that the partial derivatives of the chemical potential (in this case with respect to pressure) of gaseous water equation, see Equation~\ref{eq:partial_molar_volume}, are inconsistent with the \ac{WP} \ac{EOS}. Consequently, in direct steam cycle geothermal power plants, the steam turbine would be designed and optimised for different volumetric rates and velocities, resulting in sub-optimal turbine designs.

    Furthermore, the transition from liquid-like to vapour-like specific volume occurs at higher temperatures compared to \ac{WP} \ac{EOS}, indicating that the selected \ac{EOS} (i.e., \ac{SRK} and Ideal Gas) result in the chemical potential of gaseous water to be overestimated. By definition, at saturation, the chemical potential of the same chemical species in different phases is the same (i.e. \(\mu_i^L=\mu_i^G\)).
    
    Although the differences in saturation temperature are small in relative terms (less than \qty{0.7}{\percent} in the case of the \ac{SRK} activity model), the absolute differences (\qty{2}{\K} in the case of the \ac{SRK} activity model), when compared to key power plant design parameters, such as the minimum approach temperature difference in the heat exchange equipment (typically between \qty{5}{\K} and \qty{10}{\K}), are significant, representing differences of \qty{20}{\percent} to \qty{40}{\percent}. This can affect the required heat transfer area, which is the primary driver for the cost of heat exchange equipment.

    Repeating the above experiment for different system pressures (Figure~\ref{fig:Tsat_vs_P}), for the \ac{SRK} activity model, the deviations range between \qty{0.28}{\percent} and \qty{0.68}{\percent} in relative terms, and \qty{1.16}{\K} to \qty{3.6}{\K} in absolute terms, while for the Ideal Gas activity model the saturation temperature deviation increases from \qty{0.28}{\percent} (corresponding to \qty{1.06}{\K}) at \qty{1}{\bar} to \qty{3.32}{\percent} (corresponding to \qty{18.4}{\K}) at \qty{64}{bar}. The latter can be explained by the deviation from ideal gas behaviour at elevated pressures. Thus, for pressures exceeding \qty{2.5}{\bar} (corresponding to a saturation temperature of \qty{400}{\K} for pure water), the \ac{SRK} activity model provides more accurate saturation temperature estimates than the Ideal Gas activity model.

    \begin{figure}[H]
        \centering
        \begin{tikzpicture}
    \begin{axis}[xlabel = {Pressure/\unit{\bar}},
                 ylabel = {Deviation from \ac{WP} Tsat/\unit{\K}},
                 legend style={at={(0.03,0.97)}, anchor=north west},
                 axis y line*=left,
                 ymin = 0,
                 ymax=20,
                 xmin = 1,
                 xmax = 64,
                 xmode = log,
                 log basis x={4},
                 log ticks with fixed point,
                 height=6.5cm, width=10cm]
        
        \addplot[color=black, smooth]
                coordinates {(1.000e+00,30e+00) (2.000e+00,30e+00)};
        \addlegendentry{Abs Diff.};
        \addplot[color=black, smooth, dashed]
                coordinates {(1.000e+00,2.600e+00) (2.000e+00,1.900e+00)};
        \addlegendentry{\% Diff.};
        
        \addplot[color=blue, smooth]
                coordinates {(1.000e+00,2.600e+00) (2.000e+00,1.900e+00) (4.000e+00,1.300e+00) (8.000e+00,1.600e+00) (1.600e+01,2.100e+00) (3.200e+01,2.700e+00) (6.400e+01,3.700e+00)};
        \addlegendentry{\ac{SRK}};
        \addplot[color=red, smooth]
                coordinates {(1.000e+00,1.100e+00) (2.000e+00,1.500e+00) (4.000e+00,2.200e+00) (8.000e+00,3.400e+00) (1.600e+01,5.600e+00) (3.200e+01,9.900e+00) (6.400e+01,1.840e+01)};
        \addlegendentry{Ideal Gas};
    \end{axis}
    
    \begin{axis}[legend style={at={(0.97,0.03)}, anchor=south east},
                 axis y line*=right,
                 axis x line=none,
                 ylabel = {Deviation from \ac{WP} Tsat/\unit{\percent}},
                 ylabel near ticks,
                 ymin = 0,
                 ymax = 5,
                 xmin = 1,
                 xmax = 64,
                 xmode = log,
                 log basis x={4},
                 log ticks with fixed point,
                 height=6.5cm, width=10cm]
       
        \addplot[color=blue, dashed]
                coordinates {(1.000e+00,6.975e-01) (2.000e+00,4.830e-01) (4.000e+00,3.119e-01) (8.000e+00,3.607e-01) (1.600e+01,4.426e-01) (3.200e+01,5.288e-01) (6.400e+01,6.691e-01)};
        \addplot[color=red, dashed]
                coordinates {(1.000e+00,2.951e-01) (2.000e+00,3.813e-01) (4.000e+00,5.279e-01) (8.000e+00,7.665e-01) (1.600e+01,1.180e+00) (3.200e+01,1.939e+00) (6.400e+01,3.327e+00)};
    \end{axis}
    \label{axis_right}

\end{tikzpicture}
        \caption[The deviation of saturation temperature from the \ac{WP} \ac{EOS}]{The deviation of saturation temperature (from the \ac{WP} \ac{EOS}) when computed as part of a \ac{VLE} calculation in \emph{Reaktoro} assuming either the ideal gas and \ac{SRK} \ac{EOS} for the vapour phase and the SUPCRTLBL thermodynamic database.}
        \label{fig:Tsat_vs_P}
    \end{figure}   
    
\subsection{Empirical Models}
\label{sec:partitioning_models}

    Empirical models for specific geofluid mixtures, most commonly for mixtures comprised of water, and carbon dioxide as well as impurities , such as \(CH_4\), \(N_2\) and \(H_2S\) , have been developed. An example of this is a model originally presented by \citeauthor{Spycher2003} \cite{Spycher2003} for the mutual solubilities of carbon dioxide and water at low temperatures (\qty{12}{\degreeCelsius} to \qty{100}{\degreeCelsius}). This model was later extended to higher temperatures (\qty{12}{\degreeCelsius} to \qty{300}{\degreeCelsius}) by \citeauthor{Spycher2009} \cite{Spycher2009}. Corrections for salinity are applied using an approach similar to that of \citeauthor{Duan2003} \cite{Duan2003}.

    While such empirical models can be used to determine equilibrium phases and compositions, they do not provide methods for estimating the thermophysical properties of the fluid. Moreover, these models make simplifying assumptions, particularly regarding the reactivity of the various aqueous species, meaning that advanced phase behaviours, such as scaling/mineralisation, cannot be captured.

% \subsection{Conclusions}
% \label{sec:tppm_lit_review_conclusions}
    % \begin{itemize}
    %     \item EOS approaches can be used but not for all mixtures and all conditions
    %     \item Fluid specific EOS exist but conditions are limited to non geothermal conditions
    %     \item chemically reactive systems, provide a holistic approach, however availability of underlying models makes application difficult for two-phase systems
    %     \item Empirical models exsist but usually just provide information about the partitioning NOT the properties
    % \end{itemize}

\subsection{Calculation Libraries}
\label{sec:calc_frameworks}
    There are a number of calculation libraries that implement \ac{HEOS} for pure fluids and their mixtures, for example \emph{CoolProp} \cite{Bell2014}, \emph{REFPROP} \cite{Lemmon2018} and \emph{FluidProp} \cite{Colonna2019}. Out of the former, only \emph{CoolProp} is fully open-source, while the others are commercial libraries and represent an black box.

    \emph{Reaktoro} \cite{Leal2015} is a unified open-source framework for modelling chemically reactive systems. \emph{Reaktoro} pairs the aforementioned thermodynamic databases, \ac{EOS} and activity models with scalable optimisation algorithms \cite{Leal2017} and on-demand machine learning acceleration strategies \cite{Kyas2022, Leal2020}. \emph{Reaktoro} was used in \citeauthor{Walsh2017} to produce a computer code to compute both thermodynamic and thermophysical properties, such as viscosity and thermal conductivity. The core \emph{Reaktoro} calculation engine is written in C\textsuperscript{++} for performance reasons, with Python API provided for more convenient usage in Jupyter Notebooks and/or with the rich ecosystem of Python libraries.

    \emph{ThermoFun} \cite{Miron2021} is an open-source framework for calculating the thermodynamic properties of species and reactions from thermodynamic databases such as SUPCRT98 or SUPCRTBL. The core \emph{ThermoFun} calculation engine is written in C\textsuperscript{++} for performance reasons, but also offers a Python API for more convenient usage. \emph{ThermoFun} is integrated into \emph{Reaktoro}, where it can be used to provide the standard thermodynamic properties of species.


\section{GeoProp}
\label{sec:tppm_geoprop}
    The aforementioned frameworks/models fall into two categories: partition models for determining the number, amounts and composition of equilibrium phases; and property models for estimating the thermophysical properties of fluids of known composition. GeoProp was developed in recognition of these synergies and allows different partitioning and property frameworks/models to be coupled with another (Figure~\ref{fig:geoprop_v1}), all while maintaining the flexibility of customising the underlying calculation engines, \cite{Merbecks2024}.

\subsection{Structure}
\label{sec:geopropv1_structure}

    The main underlying data structure, a \emph{Fluid}, is a container for the compositional data of the geofluid. The individual species are stored in \emph{Phases}, both in their native phase (i.e. aqueous, gaseous or mineral) and in a total phase, capturing all species. The \emph{Fluid} can be passed from one calculation engine (e.g. partition model or property model) to another, with the required input data and parameters being automatically passed to the underlying models. The user defines the initial species and their total amounts. The phase compositions are populated after performing a partitioning calculation, while the thermophysical properties are updated following the property calculation.

    The \emph{Partition} module equilibrates an input \emph{Fluid} and partitions the \emph{Fluid} into the equilibrium phases. This determines the number, amounts and composition of the equilibrium phases at the given temperature and pressure. To date, two partition models are available: a) \emph{Reaktoro} and b) \emph{\ac{SP2009}} (details of adaptation of the original model by \citeauthor{Spycher2009} can be found in \nameref{ch:appendix_b}). However, the open architecture of GeoProp allows other partition models to be included. Additionally, the user retains the ability to fully customise the underlying equilibrium and partitioning calculations, such as selecting non-default activity models in \emph{Reaktoro}.

    The \emph{Property} module evaluates the properties of a given \emph{Fluid} at the specified pressure and temperature. This module currently uses two calculation engines: a) \emph{CoolProp} and b) \emph{ThermoFun}.

    \begin{figure}[H]
        \centering
        \resizebox{.75\linewidth}{!}{\begin{tikzpicture} [node distance=2cm]

    % the flowchart elements
    \node (PTin) [io, minimum width=1.5cm, minimum height= 1.2cm] at (0,0) {\(P, T\)};
    
    \node (Zspecies) [io, minimum height= 1.2cm, minimum width= 3cm] at ($(PTin) + (-2,-1.5)$) {Species};
    \node (Zelems) [io, minimum height= 1.2cm, minimum width= 3cm] at ($(PTin)+ (4,-1.5)$) {Elements};

    \node (SP2009) [process, minimum height= 1.2cm, fill=red!30] at ($(PTin) + (0, -3.7)$){Spycher Pruess 2009};
    \node (User) [process, minimum height= 1.2cm, fill=red!30] at ($(PTin) + (-4,-3.7)$) {User Entered};
    \node (Reaktoro) [process, minimum height= 1.2cm, fill=red!30] at ($(PTin) + (4,-3.7)$) {Reaktoro};

    \node (Phases) [process, minimum height= 1.2cm, text width = 11cm] at ($(SP2009) + (0,-2.2)$) {Phases \& Phase Compositions};
    \node (Gaseous) [process, minimum height= 2.4cm] at ($(User) + (0, -4.3)$){Gaseous Phase};
    \node (Aqueous) [process, minimum height= 2.4cm] at ($(SP2009) + (0,-4.3)$) {Aqueous Phase};
    \node (Mineral) [process, minimum height= 2.4cm] at ($(Reaktoro) + (0,-4.3)$) {Mineral Phase};
    \node (Water) [process, minimum height= 0.6cm, text width =1.2cm, minimum width = 1.2cm,] at ($(Aqueous) + (-0.75,-0.75)$) {Water};

    \node (CP) [process, minimum height= 1.2cm, text width = 5cm, fill=green!30] at ($(Aqueous) + (-3,-2.8)$) {CoolProp};
    \node (TF) [process, minimum height= 1.2cm, text width = 5cm, fill=green!30] at ($(Aqueous) + (3,-2.8)$) {ThermoFun};

    \node (Gprops) [io, minimum height= 1.2cm] at ($(Gaseous) + (0, -5)$){\(n, \mathbf{x}, \rho, h, s\)};
    \node (Aprops) [io, minimum height= 1.2cm] at ($(Aqueous) + (0,-5)$) {\(n, \mathbf{x}, \rho, h, s\)};
    \node (Mprops) [io, minimum height= 1.2cm] at ($(Mineral) + (0,-5)$) {\(n, \mathbf{x}, \rho, h, s\)};
    \node (Tprops) [io, minimum height= 1.2cm] at ($(Aprops) + (0,-2.7)$) {\(n, \mathbf{x}, \rho, h, s\)};

    % the connectors
    \draw [arrow] (PTin) to [out=180, in = 90] (Zspecies);
    \draw [arrow] (PTin) to [out=0, in = 90] (Zelems);

    \draw [arrow] (Zspecies) to [out=180, in = 90] (User);
    \draw [arrow] (Zspecies) to [out=0, in = 90] (SP2009);
    \draw [arrow] (Zelems) -- (Reaktoro);

    \draw [arrow] (User) to [out=-90, in = 170] (Phases);
    \draw [arrow] (SP2009) -- (Phases);
    \draw [arrow] (Reaktoro) to [out=-90, in = 10] (Phases);

    \draw [arrow] (Phases) to [out=190, in = 90] (Gaseous);
    \draw [arrow] (Phases) -- (Aqueous);
    \draw [arrow] (Phases) to [out=-10, in = 90] (Mineral);

    \draw [arrow] (Gaseous) to [bend right=20] (Gprops);
    \draw [arrow] (Water) to [bend right=30] (Aprops) ;
    \draw [arrow] (Aqueous) to [bend left=30] (Aprops);
    \draw [arrow] (Mineral) to [bend left=20] (Mprops);

    \draw [arrow] (Gprops) to [out=-90, in = 180] (Tprops) ;
    \draw [arrow] (Aprops) to (Tprops);
    \draw [arrow] (Mprops) to [out=-90, in = 0] (Tprops);

    % the spatial separators
    \draw [dashed] ($(SP2009) + (-6.5, 1.1)$) to ($(SP2009) + (6, 1.1)$);
    \draw [dashed] ($(SP2009) + (-6.5, -1.1)$) to ($(SP2009) + (6, -1.1)$);

    \draw [dashed] ($(Aqueous) + (-6.5, -1.7)$) to ($(Aqueous) + (6, -1.7)$);

    \draw [dashed] ($(Aprops) + (-6.5, 1.1)$) to ($(Aprops) + (6, 1.1)$);

    % annotations
    \node [text centered, text width=2cm] at ($(Aprops) + (1.5, -1.2)$) {Aqueous Properties};
    \node [text centered, text width=2cm] at ($(Gprops) + (1.5, -1.2)$) {Gaseous Properties};
    \node [text centered, text width=2cm] at ($(Mprops) + (1.5, -1.2)$) {Mineral Properties};
    \node [text centered] at ($(Tprops) + (0, -1.2)$) {Total Properties};

    
    \node [text centered, text width=2cm] at ($(PTin) + (-7, -0.75)$) {Inputs};
    \node [text centered, text width=2cm] at ($(SP2009) + (-7, 0)$) {Partition Model};
    \node [text centered, text width=2cm] at ($0.5*(Phases)+0.5*(Aqueous) + (-7, 0)$) {Phase Splitting};
    \node [text centered, text width=2cm] at ($0.5*(CP)+0.5*(TF) + (-7, 0)$) {Property Model};
    \node [text centered, text width=2cm] at ($0.5*(Aprops)+0.5*(Tprops) + (-7, 0)$) {Outputs};
    
    
\end{tikzpicture}}
        \caption{Flow diagram of GeoProp for partitioning geofluids and calculating their thermophysical properties}
        \label{fig:geoprop_v1}
    \end{figure}

    \subsubsection{Property Estimation}
        The properties of gaseous phase species are evaluated in \emph{CoolProp} using a mixture of pure components and the default \ac{BIC} data, which effectively takes the role of the activity model correcting the partial molar properties from ideal to real conditions. In case of non-convergence, an ideal mixture is assumed (i.e. mixture effects are negligible), allowing the pure component properties to be calculated and then aggregated to the phase properties.

        \emph{ThermoFun} is used to calculate the properties of all aqueous species besides water, which is calculated using \emph{CoolProp}’s implementation of the \ac{WP} \ac{EOS}. Aqueous species are assumed to be dilute and hence mixture effects are insignificant (i.e. unit activity for all aqueous species). As such, the species chemical potential, \(\mu_i^{(aq)} (P,T, \mathbf{y})\), approaches that at the reference conditions, \(\mu_i^{(aq)} (P,T, \mathbf{y}^o)\), since the activity contribution approaches zero (i.e. \(RT\ln a_i (P, T, \mathbf{y})\rightarrow 0\) as \(a_i\approx1\)). In turn this allows the pure component properties, \(\Psi_i^{(aq)}(P, T, \mathbf{y}^o)\), to be used, see Equation~\ref{eq:props_aqueous_i}, where \(\Psi_i^{(aq)}\) is a placeholder for properties such as the specific enthalpy, entropy or density of component \(i\), and \(x\) as well as \(f(x)\) are the supplementary variable and function to calculate the property, see Equation~\ref{eq:partial_molar_prop} and Table~\ref{table:PartialMolarProperties} in  Section~\ref{sec:chemically_active_system}. To obtain the aqueous phase properties, the species properties, \(\Psi_i^{(aq)}\), are aggregated using Equation~\ref{eq:props_aqueous}.

        \begin{align}
            \Psi_i^{(aq)} (P, T, \mathbf{y}) \approx \frac{1}{M_{r_i}} \frac{\partial \mu_i (P, T, \mathbf{y}^o)*f(x)}{\partial x} \label{eq:props_aqueous_i}
        \end{align}
        \begin{align}
            \Psi^{(aq)} (P, T, \mathbf{y}) = \frac{\sum_{i=0}^N m_i^{(aq)} \Psi_i^{(aq)} (P, T, \mathbf{y})}{\sum_{i=0}^N m_i^{(aq)}} \label{eq:props_aqueous}
        \end{align}
    
    Mineral phase properties are computed with \emph{ThermoFun}, assuming that each mineral constitutes a separate pure phase. Thus, unit activities are assumed, allowing the pure component properties, \(\Psi_i^{(s)} (P, T, \mathbf{y}^o)\), to be used, Equation~\ref{eq:props_solid_i}, which are then aggregated to the overall mineral phase properties, Equation~\ref{eq:props_solid}.

    \begin{align}
        \Psi_i^{(s)} (P, T, \mathbf{y}) \approx \frac{1}{M_{r_i}} \frac{\partial \mu_i (P, T, \mathbf{y}^o)*f(x)}{\partial x} \label{eq:props_solid_i}
    \end{align}
    \begin{align}
        \Psi^{(s)} (P, T, \mathbf{y}) = \frac{\sum_{i=0}^N m_i^{(s)} \Psi_i^{(s)} (P, T, \mathbf{y})}{\sum_{i=0}^N m_i^{(i)}} \label{eq:props_solid}
    \end{align}

    The properties of all phases are aggregated to the overall fluid properties employing a mass-fraction-based mixing rule, Equation~\ref{eq:geo_props_total}. 
    
    \begin{align}
        \Psi^{(t)} (P, T, \mathbf{y}) = \frac{\sum_{j=0}^K m^j \Psi^j (P, T, \mathbf{z})}{\sum_{j=0}^K m^j} \label{eq:geo_props_total}
    \end{align}

\subsection{Validation}
\label{sec:tppm_geoprop_validation}
    We use geofluid samples, collected from geothermal fields near Makhachkala, Dagestan in Russia, by \citeauthor{Abdulagatov2016} as the primary validation dataset. This dataset includes the fluid density, speed of sound and specific enthalpy (inferred from density and speed of sound measurements) for various temperatures. The composition of the fluid samples is summarised in Table~\ref{table:SampleComposition}. The salinity of these fluids ranges between about \qty{1.7}{\g \per \L} to \qty{15}{\g \per \L}. For reference, seawater has a salinity of \qty{35}{\g\per\kg} \cite{Millero2008}.

    \begin{table}[H]
        \caption{The composition of the geothermal fluid samples near Makhachkala \cite{Abdulagatov2016} in \unit{\milli \g \per \L}. Species exclusively below the detection threshold of \qty{0.1}{\milli \g \per \L} have been omitted.}
        \centering
        \begin{tabular}{|p{6em} c | c | c | c |}
    \hline
    \rowcolor{bluepoli!40}
    \multicolumn{2}{|c|}{} & \multicolumn{3}{c|}{\textbf{Sample}} \T\B \\
    \rowcolor{bluepoli!40}
    \multicolumn{2}{|c|}{\textbf{Species}} & \textbf{No. 68} & \textbf{No. 129} & \textbf{No. 27T}\T\B \\
    \hline \hline
    \emph{Cations} & B & 1.2 & 2.4 & 59.3 \T\B\\
     & Ba & <0.1 & <0.1 & 1.7 \T\B \\
     & Ca & 49.2 & 2.8 & 73.6 \T\B \\
     & K & 10.2 & 4.7 & 145 \T\B \\
     & Li & 0.2 & 0.1 & 2.2 \T\B \\
     & Mg & 32.9 & 1.3 & 28.5 \T\B \\ 
     & Na & 396 & 590 & 7540 \T\B \\
     & P & <0.1 & 0.2 & <0.1 \T\B \\
     & S & 240 & 211 & 39.8 \T\B \\
     & Se & 2.4 & 0.2 & <0.1 \T\B \\
     & Si & 13.8 & 12.3 & 29.4 \T\B \\
     & Sr & 1.1 & 0.1 & 6.7 \T\B \\
     \hline
     Anions & Cl & 152 & 276 & 7387 \T\B \\
     & SO\textsubscript{4} & 749 & 616 & 30.7 \T\B \\
    \hline
    \multicolumn{2}{|c|}{\textbf{Total}} & \textbf{1667.7} & \textbf{1830.0} & \textbf{15345.9} \T\B \\
     \hline
\end{tabular}
        \label{table:SampleComposition}
        \\[10pt]
    \end{table}

    We also consider several “synthetic” datasets. For example, seawater is a good analogue for simple geothermal brines as it is primarily comprised of water and NaCl. Although lithium bromide is not typically present in large quantities in geothermal fluids, it is also considered to test the applicability of GeoProp to unconventional brines. The thermophysical properties of these fluids were obtained from the MITSW and LiBr incompressible binary mixture \ac{EOS}, implemented in \emph{CoolProp}. Moreover, with these \ac{EOS}, it is possible to explore a wider range of salinities and temperatures.
    
    A final benchmark is performed against the ELECNRTL electrolyte model in \emph{ASPEN Plus v11}, a common process simulation. 
    
    The fluids were recreated in GeoProp and \emph{Aspen Plus v11}, and then equilibrated over a range of temperatures, determining their thermophysical properties (Figure~\ref{fig:geoprop_density_validation}, Figure~\ref{fig:geoprop_enthalpy_validation} and Figure~\ref{fig:geoprop_entropy_validation}). We find that GeoProp reproduces the “measured” densities of all fluids at all temperatures to within \qty{3}{\percent}, narrowly outperforming the ELECNRTL model in \emph{Aspen Plus v11}. For the specific enthalpy, both GeoProp and \emph{Aspen Plus v11} reproduced the measurements to within \qty{1}{\percent}.

    \begin{figure}[H]
        \centering
        \begin{tikzpicture}
    \definecolor{lightgray204}{RGB}{204,204,204}
    \begin{axis}[
        xlabel = {Temperature/\unit{\K}}, 
        xmin=273.15, 
        xmax=373.15,
        ymin=950,
        ymax=1200,
        ylabel = {Density/\unit{\kg \per \cubic \m}},
        ylabel near ticks,
        legend style={
          fill opacity=0.8,
          draw opacity=1,
          text opacity=1,
          at={(1.03,0.5)},
          anchor=west,
          draw=lightgray204
        },
        width=10cm, 
        height=6.5cm,
        ytick distance=25]
        
        % adding the dummy lines
        \addplot[color=black, only marks] coordinates {(0,0)};
        \addlegendentry{Measured}
        \addplot[color=black] coordinates {(0,0)};
        \addlegendentry{GeoProp}
        \addplot[color=black, dashed] coordinates {(0,0)};
        \addlegendentry{\emph{Aspen Plus v11}}

        %define colors
        \definecolor{libr}{HTML}{4472C4}
        \definecolor{nacl}{HTML}{ED7D31}
        \definecolor{sample27T}{HTML}{70AD47}
        \definecolor{sample68}{HTML}{FFC000}
        \definecolor{sample129}{HTML}{5B9BD5}


        % adding the geoprop lines for the legend
        \addplot[color=libr,] coordinates {(2.980e+02,1.134e+03) (3.160e+02,1.127e+03) (3.340e+02,1.119e+03) (3.520e+02,1.109e+03) (3.700e+02,1.098e+03)};
        \addlegendentry{LiBr: \qty{0.2}{\kg\per\kg}}
        
        \addplot[color=nacl,] coordinates {(2.980e+02,1.064e+03) (3.160e+02,1.056e+03) (3.340e+02,1.047e+03) (3.520e+02,1.036e+03) (3.700e+02,1.025e+03)};
        \addlegendentry{NaCl: \qty{0.1}{\kg\per\kg}}
        
        \addplot[color=sample27T,] coordinates {(2.732e+02,1.015e+03) (2.782e+02,1.015e+03) (2.832e+02,1.014e+03) (2.882e+02,1.013e+03) (2.932e+02,1.012e+03) (2.982e+02,1.011e+03) (3.032e+02,1.009e+03) (3.082e+02,1.007e+03) (3.132e+02,1.006e+03) (3.182e+02,1.003e+03) (3.232e+02,1.001e+03) (3.282e+02,9.988e+02) (3.331e+02,9.963e+02)};
        \addlegendentry{No.27T: \qty{15.3}{\g\per\kg}}
        
        \addplot[color=sample68,] coordinates {(2.732e+02,1.001e+03) (2.782e+02,1.001e+03) (2.832e+02,1.000e+03) (2.882e+02,9.997e+02) (2.932e+02,9.988e+02) (2.982e+02,9.976e+02) (3.032e+02,9.962e+02) (3.082e+02,9.946e+02) (3.132e+02,9.928e+02) (3.182e+02,9.908e+02) (3.232e+02,9.886e+02) (3.282e+02,9.862e+02) (3.331e+02,9.837e+02)};
        \addlegendentry{No.68: \qty{1.7}{\g\per\kg}}
        
        \addplot[color=sample129,] coordinates {(2.732e+02,1.001e+03) (2.782e+02,1.001e+03) (2.832e+02,1.000e+03) (2.882e+02,9.998e+02) (2.932e+02,9.988e+02) (2.982e+02,9.977e+02) (3.032e+02,9.963e+02) (3.082e+02,9.946e+02) (3.132e+02,9.928e+02) (3.182e+02,9.908e+02) (3.232e+02,9.886e+02) (3.282e+02,9.863e+02) (3.331e+02,9.838e+02)};
        \addlegendentry{No.129: \qty{1.8}{\g\per\kg}}

        \addplot[color=blue,] coordinates {(2.980e+02,9.971e+02) (3.160e+02,9.911e+02) (3.340e+02,9.828e+02) (3.520e+02,9.725e+02) (3.700e+02,9.606e+02)};
        \addlegendentry{Water}
        
        % adding the remaining lines
        % LIBR:
        %CoolProp
        \addplot[color=libr, only marks] coordinates {(2.980e+02,1.159e+03) (3.160e+02,1.153e+03) (3.340e+02,1.144e+03) (3.520e+02,1.134e+03) (3.700e+02,1.121e+03)};
        % Aspen
        \addplot[color=libr, dashed] coordinates {(2.981e+02,1.126e+03) (3.031e+02,1.124e+03) (3.081e+02,1.123e+03) (3.131e+02,1.120e+03) (3.181e+02,1.118e+03) (3.231e+02,1.116e+03) (3.281e+02,1.113e+03) (3.331e+02,1.110e+03) (3.381e+02,1.107e+03) (3.431e+02,1.104e+03) (3.481e+02,1.101e+03) (3.531e+02,1.097e+03) (3.581e+02,1.094e+03) (3.631e+02,1.090e+03) (3.731e+02,1.082e+03)};

        % MITSW:
        %CoolProp
        \addplot[color=nacl, only marks] coordinates {(2.980e+02,1.073e+03) (3.160e+02,1.066e+03) (3.340e+02,1.056e+03) (3.520e+02,1.046e+03) (3.700e+02,1.034e+03)};
        % Aspen
        \addplot[color=nacl, dashed] coordinates {(2.981e+02,1.063e+03) (3.031e+02,1.061e+03) (3.081e+02,1.059e+03) (3.131e+02,1.057e+03) (3.181e+02,1.055e+03) (3.231e+02,1.053e+03) (3.281e+02,1.050e+03) (3.331e+02,1.048e+03) (3.381e+02,1.045e+03) (3.431e+02,1.042e+03) (3.481e+02,1.039e+03) (3.531e+02,1.035e+03) (3.581e+02,1.032e+03) (3.631e+02,1.029e+03) (3.731e+02,1.021e+03)};

        % Sample No. 27T:
        % RawData
        \addplot[color=sample27T, only marks] coordinates {(2.771e+02,1.016e+03) (2.831e+02,1.016e+03) (2.931e+02,1.014e+03) (3.031e+02,1.011e+03) (3.131e+02,1.007e+03) (3.231e+02,1.003e+03) (3.331e+02,nan)};

        % Sample No. 68:
        % RawData
        \addplot[color=sample68, only marks] coordinates {(2.771e+02,1.001e+03) (2.831e+02,1.001e+03) (2.931e+02,9.990e+02) (3.031e+02,9.964e+02) (3.131e+02,9.929e+02) (3.231e+02,9.887e+02) (3.331e+02,9.837e+02)};

        % Sample No. 129:
        % RawData
        \addplot[color=sample129, only marks] coordinates {(2.771e+02,1.002e+03) (2.831e+02,1.002e+03) (2.931e+02,1.000e+03) (3.031e+02,9.975e+02) (3.131e+02,9.940e+02) (3.231e+02,9.896e+02) (3.331e+02,9.846e+02)};

    \end{axis}
\end{tikzpicture}        
        \caption{The density of various brines as a function of temperature at \qty{1}{\bar} pressure. Solid circles represent “measured” data and lines represent property models}
        \label{fig:geoprop_density_validation}
    \end{figure}

    \begin{figure}[H]
        \centering
        \begin{tikzpicture}
    \definecolor{lightgray204}{RGB}{204,204,204}
    \begin{axis}[xlabel = {Temperature/\unit{\K}}, 
                 xmin=298, xmax=373.15,
                 ymin=0, ymax=350,
                 ylabel = {Enthalpy/\unit{\kilo\joule \per \kg}},
        legend style={
          fill opacity=0.8,
          draw opacity=1,
          text opacity=1,
          at={(1.03,0.5)},
          anchor=west,
          draw=lightgray204
        },
                 width=10cm,
                         height=6.5cm,
        ytick distance=50]
        
        % adding the dummy lines
        \addplot[color=black, only marks] coordinates {(0,0)};
        \addlegendentry{Measured}
        \addplot[color=black] coordinates {(0,0)};
        \addlegendentry{GeoProp}
        \addplot[color=black, dashed] coordinates {(0,0)};
        \addlegendentry{\emph{Aspen Plus v11}}

        %define colors
        \definecolor{libr}{HTML}{4472C4}
        \definecolor{nacl}{HTML}{ED7D31}
        \definecolor{sample27T}{HTML}{70AD47}
        \definecolor{sample68}{HTML}{FFC000}
        \definecolor{sample129}{HTML}{5B9BD5}

        % adding the geoprop lines for the legend
        \addplot[color=libr,] coordinates {(2.980e+02,0.000e+00) (3.160e+02,6.073e+01) (3.340e+02,1.218e+02) (3.520e+02,1.830e+02) (3.700e+02,2.442e+02)};
        \addlegendentry{LiBr: \qty{0.2}{\kg\per\kg}}
        
        \addplot[color=nacl,] coordinates {(2.980e+02,0.000e+00) (3.160e+02,6.697e+01) (3.340e+02,1.343e+02) (3.520e+02,2.018e+02) (3.700e+02,2.695e+02)};
        \addlegendentry{NaCl: \qty{0.1}{\kg\per\kg}}
        
        \addplot[color=sample27T,] coordinates {(2.732e+02,-1.037e+02) (2.782e+02,-8.274e+01) (2.832e+02,-6.182e+01) (2.882e+02,-4.095e+01) (2.932e+02,-2.009e+01) (2.982e+02,7.472e-01) (3.032e+02,2.158e+01) (3.082e+02,4.240e+01) (3.132e+02,6.323e+01) (3.182e+02,8.406e+01) (3.232e+02,1.049e+02) (3.282e+02,1.257e+02) (3.331e+02,1.466e+02)};
        \addlegendentry{No.27T: \qty{15.3}{\g\per\kg}}
        
        \addplot[color=sample68,] coordinates {(2.732e+02,-1.037e+02) (2.782e+02,-8.274e+01) (2.832e+02,-6.182e+01) (2.882e+02,-4.095e+01) (2.932e+02,-2.009e+01) (2.982e+02,7.472e-01) (3.032e+02,2.158e+01) (3.082e+02,4.240e+01) (3.132e+02,6.323e+01) (3.182e+02,8.406e+01) (3.232e+02,1.049e+02) (3.282e+02,1.257e+02) (3.331e+02,1.466e+02)};
        \addlegendentry{No.68: \qty{1.7}{\g\per\kg}}
        
        \addplot[color=sample129,] coordinates {(2.732e+02,-1.037e+02) (2.782e+02,-8.273e+01) (2.832e+02,-6.182e+01) (2.882e+02,-4.094e+01) (2.932e+02,-2.009e+01) (2.982e+02,7.471e-01) (3.032e+02,2.158e+01) (3.082e+02,4.240e+01) (3.132e+02,6.323e+01) (3.182e+02,8.406e+01) (3.232e+02,1.049e+02) (3.282e+02,1.257e+02) (3.331e+02,1.466e+02)};
        \addlegendentry{No.129: \qty{1.8}{\g\per\kg}}

        \addplot[color=blue,] coordinates {(2.980e+02,0.000e+00) (3.160e+02,7.524e+01) (3.340e+02,1.505e+02) (3.520e+02,2.259e+02) (3.700e+02,3.016e+02)};
        \addlegendentry{Water}
        
        % adding the remaining lines
        % LIBR:
        %CoolProp
        \addplot[color=libr, only marks] coordinates {(2.980e+02,0.000e+00) (3.160e+02,5.861e+01) (3.340e+02,1.177e+02) (3.520e+02,1.772e+02) (3.700e+02,2.370e+02)};
        % Aspen
        \addplot[color=libr, dashed] coordinates {(2.981e+02,0.000e+00) (3.031e+02,1.687e+01) (3.081e+02,3.375e+01) (3.131e+02,5.063e+01) (3.181e+02,6.753e+01) (3.231e+02,8.444e+01) (3.281e+02,1.014e+02) (3.331e+02,1.183e+02) (3.381e+02,1.353e+02) (3.431e+02,1.523e+02) (3.481e+02,1.693e+02) (3.531e+02,1.863e+02) (3.581e+02,2.034e+02) (3.631e+02,2.205e+02) (3.731e+02,2.548e+02)};

        % MITSW:
        %CoolProp
        \addplot[color=nacl, only marks] coordinates {(2.980e+02,0.000e+00) (3.160e+02,6.688e+01) (3.340e+02,1.340e+02) (3.520e+02,2.013e+02) (3.700e+02,2.687e+02)};
        % Aspen
        \addplot[color=nacl, dashed] coordinates {(2.981e+02,0.000e+00) (3.031e+02,1.876e+01) (3.081e+02,3.752e+01) (3.131e+02,5.627e+01) (3.181e+02,7.503e+01) (3.231e+02,9.379e+01) (3.281e+02,1.126e+02) (3.331e+02,1.313e+02) (3.381e+02,1.501e+02) (3.431e+02,1.689e+02) (3.481e+02,1.878e+02) (3.531e+02,2.066e+02) (3.581e+02,2.255e+02) (3.631e+02,2.444e+02) (3.731e+02,2.823e+02)};

        % Sample No. 27T:
        % RawData
        \addplot[color=sample27T, only marks] coordinates {(2.732e+02,-1.051e+02) (2.781e+02,-8.381e+01) (2.831e+02,-6.270e+01) (2.931e+02,-2.081e+01) (3.031e+02,2.081e+01) (3.131e+02,6.234e+01) (3.231e+02,1.040e+02) (3.331e+02,1.458e+02)};

        % Sample No. 68:
        % RawData
        \addplot[color=sample68, only marks] coordinates {(2.732e+02,-1.051e+02) (2.781e+02,-8.381e+01) (2.831e+02,-6.270e+01) (2.931e+02,-2.081e+01) (3.031e+02,2.081e+01) (3.131e+02,6.234e+01) (3.231e+02,1.040e+02) (3.331e+02,1.458e+02)};

        % Sample No. 129:
        % RawData
        \addplot[color=sample129, only marks] coordinates {(2.732e+02,-1.067e+02) (2.781e+02,-8.523e+01) (2.831e+02,-6.373e+01) (2.931e+02,-2.116e+01) (3.031e+02,2.116e+01) (3.131e+02,6.337e+01) (3.231e+02,1.056e+02) (3.331e+02,1.479e+02)};

    \end{axis}
\end{tikzpicture}      
        \caption[The specific enthalpy of various brines as a function of temperature at \qty{1}{\bar} pressure.]{The specific enthalpy of various brines as a function of temperature at \qty{1}{\bar} pressure. The reference temperature is \qty{298}{\K} or \qty{25}{\degreeCelsius}. Solid circles represent “measured” data and lines represent property models.}
        \label{fig:geoprop_enthalpy_validation}
    \end{figure}

    \begin{figure}[H]
        \centering
        \begin{tikzpicture}
    \definecolor{lightgray204}{RGB}{204,204,204}
    \begin{axis}[xlabel = {Temperature/\unit{\K}}, 
                 xmin=298, xmax=373.15,
                 ymin=0, ymax=1,
                 ylabel = {Entropy/\unit{\kilo\joule \per \kg \K}},
                         legend style={
          fill opacity=0.8,
          draw opacity=1,
          text opacity=1,
          at={(1.03,0.5)},
          anchor=west,
          draw=lightgray204
        },
        width=10cm, 
        height=6.5cm,
        ytick distance=0.1
]
        
        % adding the dummy lines
        \addplot[color=black, only marks] coordinates {(0,0)};
        \addlegendentry{Measured}
        \addplot[color=black] coordinates {(0,0)};
        \addlegendentry{GeoProp}
        \addplot[color=black, dashed] coordinates {(0,0)};
        \addlegendentry{\emph{Aspen Plus v11}}

        %define colors
        \definecolor{libr}{HTML}{4472C4}
        \definecolor{nacl}{HTML}{ED7D31}
        \definecolor{sample27T}{HTML}{70AD47}
        \definecolor{sample68}{HTML}{FFC000}
        \definecolor{sample129}{HTML}{5B9BD5}

        % adding the geoprop lines for the legend
        \addplot[color=libr,] coordinates {(2.980e+02,0.000e+00) (3.160e+02,1.979e-01) (3.340e+02,3.859e-01) (3.520e+02,5.644e-01) (3.700e+02,7.340e-01)};
        \addlegendentry{LiBr: \qty{0.2}{\kg\per\kg}}
        
        \addplot[color=nacl,] coordinates {(2.980e+02,0.000e+00) (3.160e+02,2.182e-01) (3.340e+02,4.255e-01) (3.520e+02,6.224e-01) (3.700e+02,8.098e-01)};
        \addlegendentry{NaCl: \qty{0.1}{\kg\per\kg}}

        \addplot[color=blue,] coordinates {(2.980e+02,0.000e+00) (3.160e+02,2.451e-01) (3.340e+02,4.768e-01) (3.520e+02,6.968e-01) (3.700e+02,9.064e-01)};
        \addlegendentry{Water}
        
        % adding the remaining lines
        % LIBR:
        %CoolProp
        \addplot[color=libr, only marks] coordinates {(2.980e+02,0.000e+00) (3.160e+02,1.910e-01) (3.340e+02,3.729e-01) (3.520e+02,5.463e-01) (3.700e+02,7.119e-01)};
        % Aspen
        \addplot[color=libr, dashed] coordinates {(2.981e+02,0.000e+00) (3.031e+02,5.612e-02) (3.081e+02,1.113e-01) (3.131e+02,1.657e-01) (3.181e+02,2.192e-01) (3.231e+02,2.720e-01) (3.281e+02,3.240e-01) (3.331e+02,3.752e-01) (3.381e+02,4.258e-01) (3.431e+02,4.756e-01) (3.481e+02,5.249e-01) (3.531e+02,5.735e-01) (3.581e+02,6.215e-01) (3.631e+02,6.689e-01) (3.731e+02,7.622e-01)};

        % MITSW:
        %CoolProp
        \addplot[color=nacl, only marks] coordinates {(2.980e+02,0.000e+00) (3.160e+02,2.179e-01) (3.340e+02,4.245e-01) (3.520e+02,6.206e-01) (3.700e+02,8.074e-01)};
        % Aspen
        \addplot[color=nacl, dashed] coordinates {(2.981e+02,0.000e+00) (3.031e+02,6.241e-02) (3.081e+02,1.238e-01) (3.131e+02,1.842e-01) (3.181e+02,2.436e-01) (3.231e+02,3.021e-01) (3.281e+02,3.597e-01) (3.331e+02,4.165e-01) (3.381e+02,4.725e-01) (3.431e+02,5.277e-01) (3.481e+02,5.822e-01) (3.531e+02,6.360e-01) (3.581e+02,6.890e-01) (3.631e+02,7.414e-01) (3.731e+02,8.445e-01)};

    \end{axis}
\end{tikzpicture}
        \caption[The specific entropy of various brines as a function of temperature at \qty{1}{\bar} pressure.]{The specific entropy of various brines as a function of temperature at \qty{1}{\bar} pressure. The reference temperature is \qty{298}{\K} or \qty{25}{\degreeCelsius}. Solid circles represent “measured” data and lines represent property models.}
        \label{fig:geoprop_entropy_validation}
    \end{figure}

\subsection{Examples}
\label{sec:tppm_geoprop_examples}
    For code examples on using \emph{GeoProp}, please refer to \nameref{ch:appendix_d} or the \emph{GeoProp} Github repository \url{https://github.com/EASYGO-ITN/GeoProp}.

\subsection{GeoProp v2}
\label{sec:tppm_geoprop_v2}
    Given the open-source architecture of GeoProp, it is in principle possible to extend the functionality of GeoProp v1 to include further partition and property models. However, the first implementation of the properties back-end is somewhat static, making it difficult for users to extend the functionality to other property estimation engines.

    Moreover, the initial approach divides the models into two classes without recognising that in fact they are all \emph{models} and the only difference between them are their capabilities. For example, in GeoProp v1, \emph{CoolProp} is treated purely as a property model, when in reality it also has some partitioning-like capabilities. A summary of the \emph{models} currently used and their capabilities can be found in Table~\ref{table:ModelCapabiities}.

    With the above in mind, a revised plug-in based architecture is proposed. This streamlines the process of adding \emph{models} to GeoProp by simply requiring an \emph{interface} or \emph{driver} script, without having to adjust the overall GeoProp architecture. Besides the default configuration, this enables the user to define custom thermophysical property models by specifying arbitrary combinations of the \emph{models}. While the general workflow is unchanged, the new architecture can be seen in Figure~\ref{fig:geoprop_v2}.

    \begin{table}[H]
        \caption{\emph{Model} capabilities.}
        \centering 
        \label{table:ModelCapabiities}
        \begin{tabular}{|p{6em} | c | c | c | c | c|}
    \hline
    \rowcolor{bluepoli!40}
     &  & \multicolumn{3}{c|}{\textbf{Properties}} \T\B \\
    \rowcolor{bluepoli!40}
    \textbf{Model} & \textbf{Partition} & \textbf{Vapour} & \textbf{Aqueous} & \textbf{Mineral} \T\B \\
    \hline \hline
    \emph{CoolProp} & \checkmark & \checkmark & \checkmark & \T\B\\
    \emph{Reaktoro} & \checkmark & \checkmark & \checkmark & \checkmark \T\B\\
    \emph{\ac{SP2009}} & \checkmark &  &  &  \T\B\\
    \emph{ThermoFun} &  &  & \checkmark & \checkmark \T\B\\
    \emph{...} &  &  &  &  \T\B\\
    \hline
\end{tabular}        
        \\[10pt]
    \end{table}

    \begin{figure}[H]
        \centering
        \resizebox{.75\linewidth}{!}{\begin{tikzpicture} [node distance=2cm]

    % the flowchart elements
    \node (PTin) [io, minimum width=1.5cm, minimum height= 1.2cm] at (0,0) {\(P, T\)};
    \node (Z) [io, minimum height= 1.2cm, minimum width= 3cm] at ($(PTin) + (0,-1.5)$) {Total Composition};

    \node (PartitionModel) [process, minimum height= 1.2cm, fill=red!30] at ($(Z) + (0, -2.2)$){Partition Model};

    \node (Phases) [process, minimum height= 1.2cm, text width = 11cm] at ($(PartitionModel) + (0,-2.2)$) {Phases \& Phase Compositions};
    \node (Gaseous) [process, minimum height= 1.2cm] at ($(Phases) + (-4, -1.5)$){Gaseous Phase};
    \node (Aqueous) [process, minimum height= 1.2cm] at ($(Phases) + (0,-1.5)$) {Aqueous Phase};
    \node (Mineral) [process, minimum height= 1.2cm] at ($(Phases) + (4,-1.5)$) {Mineral Phase};

    \node (PropertyModelG) [process, minimum height= 1.2cm, fill=green!30] at ($(Gaseous) + (0, -2.2)$){Gaseous Property Model};
    \node (PropertyModelA) [process, minimum height= 1.2cm, fill=green!30] at ($(Aqueous) + (0,-2.2)$) {Aqueous Property Model};
    \node (PropertyModelM) [process, minimum height= 1.2cm, fill=green!30] at ($(Mineral) + (0,-2.2)$) {Mineral Property Model};

    \node (Gprops) [io, minimum height= 1.2cm] at ($(PropertyModelG) + (0, -2.2)$){\(n, \mathbf{x}, \rho, h, s\)};
    \node (Aprops) [io, minimum height= 1.2cm] at ($(PropertyModelA) + (0,-2.2)$) {\(n, \mathbf{x}, \rho, h, s\)};
    \node (Mprops) [io, minimum height= 1.2cm] at ($(PropertyModelM) + (0,-2.2)$) {\(n, \mathbf{x}, \rho, h, s\)};
    \node (Tprops) [io, minimum height= 1.2cm] at ($(Aprops) + (0,-2.7)$) {\(n, \mathbf{x}, \rho, h, s\)};

    % the connectors
    \draw [arrow] (PTin) to [out=-90, in = 90] (Z);
    \draw [arrow] (Z) to [out=-90, in = 90] (PartitionModel);

    \draw [arrow] (PartitionModel) -- (Phases);

    \draw [arrow] (Phases) to [out=190, in = 90] (Gaseous);
    \draw [arrow] (Phases) -- (Aqueous);
    \draw [arrow] (Phases) to [out=-10, in = 90] (Mineral);
    \draw [arrow] (Gaseous) -- (PropertyModelG);
    \draw [arrow] (Aqueous) -- (PropertyModelA);
    \draw [arrow] (Mineral) -- (PropertyModelM);

    \draw [arrow] (PropertyModelG) -- (Gprops);
    \draw [arrow] (PropertyModelA) -- (Aprops);
    \draw [arrow] (PropertyModelM) -- (Mprops);

    \draw [arrow] (Gprops) to [out=-90, in = 180] (Tprops) ;
    \draw [arrow] (Aprops) to (Tprops);
    \draw [arrow] (Mprops) to [out=-90, in = 0] (Tprops);

    % the spatial separators
    \draw [dashed] ($(Z) + (-6.5, -1.1)$) to ($(Z) + (6, -1.1)$);
    \draw [dashed] ($(PartitionModel) + (-6.5, -1.1)$) to ($(PartitionModel) + (6, -1.1)$);
    \draw [dashed] ($(Aqueous) + (-6.5, -1.1)$) to ($(Aqueous) + (6, -1.1)$);
    \draw [dashed] ($(Aprops) + (-6.5, 1.1)$) to ($(Aprops) + (6, 1.1)$);

    % annotations
    \node [text centered, text width=2cm] at ($(Aprops) + (1.5, -1.2)$) {Aqueous Properties};
    \node [text centered, text width=2cm] at ($(Gprops) + (1.5, -1.2)$) {Gaseous Properties};
    \node [text centered, text width=2cm] at ($(Mprops) + (1.5, -1.2)$) {Mineral Properties};
    \node [text centered] at ($(Tprops) + (0, -1.2)$) {Total Properties};

    \node [text centered, text width=2cm] at ($(PTin) + (-7, -0.75)$) {Inputs};
    \node [text centered, text width=2cm] at ($(SP2009) + (-7, 0)$) {Partition Model};
    \node [text centered, text width=2cm] at ($0.5*(Phases)+0.5*(Aqueous) + (-7, 0)$) {Phase Splitting};
    \node [text centered, text width=2cm] at ($(PropertyModelA) + (-7, 0)$) {Property Models};
    \node [text centered, text width=2cm] at ($0.5*(Aprops)+0.5*(Tprops) + (-7, 0)$) {Outputs};
    
\end{tikzpicture}}
        \caption{Flow diagram of GeoProp v2 for partitioning geofluids and calculating their thermophysical properties}
        \label{fig:geoprop_v2}
    \end{figure}


% \section{Semi-Empirical Modelling Approaches}
% \label{sec:semiempirical_models}
%     \todo{perhaps this could be framed as future work with potential}

In specific cases it is also possible to tightly couple different models to extend their respective capabilities or to improve speed of calculations. For example, mixtures of water and carbon dioxide can be modelled using their respective high-fidelity pure fluid formulation; however, convergence is not guaranteed for all conditions, particularly for quasi-pure water, see Section~\ref{sec:excess_properties}. On the other hand, the \acf{SP2009} model can be used to partition mixtures of water and carbon dioxide, providing stable convergence for a broad range of compositions, temperatures and pressure.

The following section aims to investigate the coupling of the \acf{SP2009} model with the \acf{WP} \ac{EOS} for water and the \acf{SW} \ac{EOS} for pure carbon dioxide.

\subsection{Formulation}
\label{sec_semiempirical_formulation}
    Assuming a system comprised of two phases, one water-rich phase, hereinafter \emph{wp}, and one carbon dioxide-rich, hereinafter \emph{cp}, where the only reactions taking place are the migrations of water and carbon dioxide molecules from one phase to another, see Reaction~\ref{eq:reaction_1} and ~\ref{eq:reaction_2}.

    \begin{align}
        H_2O^{(wp)} \rightleftharpoons H_2O^{(cp)} \label{eq:reaction_1}
    \end{align}
    \begin{align}
        CO_2^{(cp)} \rightleftharpoons CO_2^{(wp)} \label{eq:reaction_2}
    \end{align}

    At equilibrium the chemical potential of a species \(i\) is equal across all phases, see Equation~\ref{eq:chempot_at_equib}, where the chemical potential of species \(i\) at a given temperature, pressure and composition can be calculated from its chemical potential at some reference conditions and the activity of species \(i\), as previously shown in Equations~\ref{eq:chemical_potential} and ~\ref{eq:activity_def}. By convention, the reference conditions depend on the phase that the species occupies. 

    \begin{align}
        \mu_i^{(x)}= \mu_i^{(y)}= \mu_i^{(z)}= ... \label{eq:chempot_at_equib}
    \end{align}
    \begin{align}
        \ln a_i^{j}(P, T, \mathbf{z}_j)= \frac{\mu_i^{j} (P, T, \mathbf{z}_j) - \mu_i^{j} (P^o, T^o, \mathbf{z}_j^o)}{RT} \tag{\ref{eq:activity_def}}
    \end{align}
    \begin{align}
        \mu_i^{j} (P, T, \mathbf{z}_j) = \mu_i^{j} (P^o, T^o, \mathbf{z}_j^o) +RT*\ln a_i (P, T, \mathbf{z}_j) \tag{\ref{eq:chemical_potential}}
    \end{align}

    It is then possible to calculate the thermodynamic properties of species \(i\) for the partial derivaties of chemical potential. The property \(\Psi\), a placeholder for the molar enthalpy, molar entropy or molar volume, can be calculated via Equation~\ref{eq:props_from_mu}. The choice of \(x\) and \(f(x)\) depends on the property to be calculated: for the partial molar enthalpy \(x=1/T\) and \(f(x)=x\), for the partial molar entropy \(x=T\) and \(f(x)=-1\), and for the partial molar volume \(x=P\) and \(f(x)=1\), with the unused properties being constant.

    \begin{align}
        \Psi_i^{j} (P, T, \mathbf{z}_j) = \frac{\partial \left( \mu_i^{j} (P, T, \mathbf{z}_j)*f(x) \right)}{\partial x} \label{eq:props_from_mu}
    \end{align}

    Substituting Equation~\ref{eq:chemical_potential} into Equation~\ref{eq:props_from_mu}, yields Equation~\ref{eq:props_from_mu_expandend}, which simplifies to Equation~\ref{eq:props_from_mu_expandend_simp} by recognising that \(\Psi_i^{j} (P^o, T^o, \mathbf{z}_j^o)=\frac{\partial \left( \mu_i (P^o, T^o, \mathbf{z}_j^o)*f(x) \right)}{\partial x}\).

    \begin{align}
        \Psi_i^{j} (P, T, \mathbf{z}_j) = \frac{\partial \left( \mu_i^{j} (P^o, T^o, \mathbf{z}_j^o)*f(x) \right)}{\partial x} + \frac{\partial \left(RT * f(x)* \ln a_i^{j} (P, T, \mathbf{z}_j) \right)}{\partial x} \label{eq:props_from_mu_expandend}
    \end{align}
    \begin{align}
        \Psi_i^{j} (P, T, \mathbf{z}_j) = \Psi_i^{j} (P^o, T^o, \mathbf{z}_j^o) + \frac{\partial \left(RT * f(x)* \ln a_i^{j} (P, T, \mathbf{y}_j) \right)}{\partial x} \label{eq:props_from_mu_expandend_simp}
    \end{align}

    Equation~\ref{eq:props_from_mu_expandend_simp} provides the basis for coupling the \ac{WP} and \ac{SW} \ac{EOS} with the \ac{SP2009} model. The \ac{WP} and \ac{SW} \ac{EOS} can be used to evaluate the properties of water and carbon dioxide at their respective reference conditions, i.e. \(\Psi_i^{j} (P^o, T^o, \mathbf{z}_j^o)\). Meanwhile, the \ac{SP2009} model is used to 1) partition the fluid into water-rich and carbon dioxide-rich phase and 2) to provide the species activity (i.e. \(\ln a_i^{j} (P, T, \mathbf{z}_j)\), which accounts for the deviation from the reference conditions and composition.

\subsection{Change of Reference Conditions}
\label{sec:change_ref_cond}

    However, Equation~\ref{eq:props_from_mu_expandend_simp} gives undue importance to the the activity model as it not only corrects for the presence of other species but also the temperature and pressure being different to the reference conditions. For example, considering the boundary case where the fluid is pure (either water or carbon dioxide) then it would be best to simply use the corresponding \ac{HEOS} to calculate the fluid's properties. However, with the current formulation it would be the activity model's responsibility to correct the properties at the reference conditions to the temperature and pressure of interest, which can introduce inconsistencies in the pure component properties, as the activity model does not have the same level of accuracy as the \ac{HEOS}. This has previously been illustrated in Section~\ref{sec:chemically_active_system_limitations}.

    An alternative approach is to change the reference conditions to be the current temperature and pressure and composition corresponding to a pure substance, i.e. \(P, T, \mathbf{z}_j^o\), by adding and subtracting \(\mu_i (P, T, \mathbf{z}_j^o)\) to Equation~\ref{eq:chemical_potential}, yielding Equation~\ref{eq:chem_pot_change_ref}. The definition of the species activity allows this expression to be simplified to Equation~\ref{eq:chem_pot_change_ref_simp}, where \(i\) is the species index, and \(j\) is the phase index

    \begin{align}
        \mu_i^{j} (P, T, \mathbf{z}_j) = \mu_i^{j} (P^o, T^o, \mathbf{z}_j^o) +RT*\ln a_i^{j} (P, T, \mathbf{z}_j) + \mu_i^{j} (P, T, \mathbf{z}_j^o) - \mu_i^{j} (P, T, \mathbf{z}_j^o) \label{eq:chem_pot_change_ref}
    \end{align}
    \begin{align}
        \mu_i^{j} (P, T, \mathbf{z}_j) = \mu_i^{j} (P, T, \mathbf{z}_j^o) + RT*\ln a_i^{j} (P, T, \mathbf{z}_j) -RT*\ln a_i^{j} (P, T, \mathbf{z}_j^o)
    \end{align}
    \begin{align}
        \mu_i^{j} (P, T, \mathbf{z}_j) = \mu_i^{j} (P, T, \mathbf{z}_j^o) + RT*\ln \frac{a_i^{j} (P, T, \mathbf{z}_j)}{a_i^{j} (P, T, \mathbf{z}_j^0)} \label{eq:chem_pot_change_ref_simp}
    \end{align}

    From Equation~\ref{eq:chem_pot_change_ref_simp}, the thermodynamic properties can then be obtained as outlined above, yielding Equation~\ref{eq:props_chempot_changed_ref}, where \(A_i^{j} (P, T, \mathbf{z}_j) = \frac{a_i^{j} (P, T, \mathbf{z}_j)}{a_i^{j} (P, T, \mathbf{z}_j^o)}\)

    \begin{align}
        \Psi_i^{j} (P, T, \mathbf{z}_j) = \Psi_i^{j} (P, T, \mathbf{z}_{j}^o) + \frac{\partial \left(RT * f(x)* \ln A_i^{j} (P, T, \mathbf{z}_j) \right)}{\partial x} \label{eq:props_chempot_changed_ref}
    \end{align}

    This formulation allows the properties to be evaluated at the temperature and pressure of interest directly from the corresponding \ac{HEOS} and the activity model only corrects for the difference in composition. The additional benefit being that if the fluid is pure, the properties are entirely consistent with those predicted by the \ac{HEOS} as \(A_i^{j} (P, T, \mathbf{z}_j) = 1\) and hence the activity contribution is zero.

    The properties of species \(i\) in phase \(j\) can then be aggregated by phase or by all phases to obtain the phase or total properties, Equations~\ref{eq:props_phase} and \ref{eq:props_total}.

    \begin{align}
        \Psi_i^j (P, T, \mathbf{z}_j) = \Psi_i^j (P, T, \mathbf{z}_j^o) + \frac{\partial \left(RT * f(x)* \ln A_i^j (P, T, \mathbf{z}_j) \right)}{\partial x} \label{eq:props_comp_phase}
    \end{align}
    \begin{align}
        \Psi^j (P, T, \mathbf{z}_j) = \sum_{i=0}^N n_i^j\Psi_i^j (P, T, \mathbf{z}_j) \label{eq:props_phase}
    \end{align}
    \begin{align}
        \Psi (P, T, \mathbf{z}) = \sum_{i=0}^N n^j\Psi^j (P, T, \mathbf{z}_j) \label{eq:props_total}
    \end{align}

\subsection{Extrapolation}
\label{sec:extrapolation}
    \todo{After speaking with Allan, this section needs some more work... essentially the extrapolation should be treated as tuning to reduce the differences to the CoolProp model}
    One drawback of the above formulation is that the models used to obtain \(\Psi_i^j (P, T, \mathbf{z}_j)\) are not necessarily continuous over the fully temperature and pressure domain of interest. For example, considering pure water, if the temperature is below the saturation temperature at a given pressure (but above the melting point), water can only exist in its liquid state - a vapour or solid state is not feasible. However, the presence of impurities changes the saturation point. Specifically, for a mixture of water and carbon dioxide, some water may exist in its vapour state despite the temperature being below the saturation temperature of pure water. This poses an obstacle, because the \ac{WP} \ac{EOS} is designed to model physical states of pure water, but such equilibrium states are non-physical for pure water. 

    \todo{insert the chemical potential diagram}

    As an alternative, the desired properties of water vapour could be obtained by extrapolating from the saturation point. The following methods were considered, also see Table~\ref{table:SemiEmpirical_ExtrapolationFuncs}:

    \begin{itemize}
        \item \emph{Gibbs Energy}: the Gibbs energy is linearly extrapolated from the saturation point, with the remaining properties being determined from the partial derivatives. This has the unfortunate consequence that the molar enthalpy and entropy are constant with temperature.
        \item \emph{Enthalpy \& Entropy}: both the molar enthalpy and molar entropy are linearly extrapolated from the saturation point, with the Gibbs energy being back-calculated from the resulting values of the molar enthalpy and entropy; the volume is not computed.
        \item \emph{Density}: the ideal gas law is used to compute the density of the water vapour at the current temperature. In turn, the temperature and density are then used to compute the properties from the \ac{WP} \ac{EOS}.
        \item \emph{Power Law}: it is assumed that \(v \propto T^a\), with the value of \(a\)  being back-calculated from the saturation point.
    \end{itemize}

    \begin{table}[H]
        \caption{The extrapolation schemes investigated for obtaining water vapour properties for sub-saturation conditions}
        \centering 
        \label{table:SemiEmpirical_ExtrapolationFuncs}
        \begin{tabular}{|p{7em} | c | c | c | c|}
    \hline
    \rowcolor{bluepoli!40}
    \textbf{Extrapolation} & \textbf{Gibbs Energy} & \textbf{Enthalpy} & \textbf{Entropy} & \textbf{Volume} \T\B \\
    \hline \hline 
    Gibbs Energy & \(G_{sat} - \Delta T*\frac{dG}{dT}\) & \(\frac{d(G/T)}{d(1/T)}\) & \(\frac{dG}{dT}\) & \(\frac{dG}{dP}\) \T\B\\
    \hline
    Enthalpy \& Entropy  & H - TS & \(H_{sat} - \Delta T*\frac{dH}{dT}\) & \(S_{sat} - \Delta T*\frac{dS}{dT}\) & - \T\B\\
    \hline
    Density & \multicolumn{4}{c|}{\(\rho=\rho _{sat}*\frac{T_{sat}}{T}\)} \T\B\\
     & \(G(T,\rho)\) & \(H(T,\rho)\) & \(S(T,\rho)\) & \(1/\rho\)\T\B\\
    \hline
    Ideal Gas & - & - & - & \(v_{sat}*\frac{T}{T_{sat}}\)\T\B\\
    \hline
    Power Law & - & - & - & \(a=\frac{T_{sat}}{v_{sat}}*\frac{dv}{dT}\)\T\B\\
     & - & - & - & \(v_{sat} * \left(\frac{T}{T_{sat}} \right)^a\)\T\B\\
    \hline
\end{tabular}        
        \\[10pt]
    \end{table}

    While the selection of extrapolation method is arbitrary, provided it allows existing property data to be reproduced, care was taken to ensure that the selected methods display smoothly around the saturation point, and that when extrapolating backwards (i.e. to conditions where \(T>T_{sat}\) and water vapour can exist) the extrapolated properties are close to those predicted by the \ac{WP} \ac{EOS}, see Figures~\ref{fig:SemiEmpirical_extrapolation1} and \ref{fig:SemiEmpirical_extrapolation2}. Based on these criteria the \emph{Enthalpy \& Entropy} method was selected for the molar enthalpy and molar entropy, and the \emph{Power Law} method was chosen for the molar volume.

    \begin{notes}{Note:}
        While it was attempted to also extrapolate the derivatives of the species activity from the saturation point, it was ultimately decided to simply use the derivatives evaluated at the saturation point to ensure model stability.
    \end{notes}

    \begin{figure}[H]
        \centering
        \input{Content/TPPM/SemiEmpirical/Plots/ExtrapolationFuncs_part1}
        \caption{The molar Gibbs energy, molar enthalpy and molar entropy of water vapour at sub-saturation conditions, as extrapolated by the various methods considered, see Table~\ref{table:SemiEmpirical_ExtrapolationFuncs}}
        \label{fig:SemiEmpirical_extrapolation1}
    \end{figure}

    \begin{figure}[H]
        \centering
        % This file was created with tikzplotlib v0.10.1.
\begin{tikzpicture}

\definecolor{darkgray176}{RGB}{176,176,176}
\definecolor{darkorange25512714}{RGB}{255,127,14}
\definecolor{forestgreen4416044}{RGB}{44,160,44}
\definecolor{lightgray204}{RGB}{204,204,204}
\definecolor{steelblue31119180}{RGB}{31,119,180}

\begin{axis}[
legend cell align={left},
legend style={
  % fill opacity=0.8,
  % draw opacity=1,
  % text opacity=1,
  at={(1.03,0.5)},
  anchor=west,
  % draw=lightgray204
},
% log basis y={10},
% tick align=outside,
% tick pos=left,
% x grid style={darkgray176},
xlabel={Temperature/\unit{\K}},
xmin=298, xmax=700,
% xtick style={color=black},
% y grid style={darkgray176},
ylabel={Volume/\unit{\cubic\m \per\mole}},
ymin=1e-05, ymax=0.1,
ymode=log,
% ytick style={color=black},
% ytick={1e-07,1e-05,0.001,0.1,10,1000},
% yticklabels={
%   \(\displaystyle {10^{-7}}\),
%   \(\displaystyle {10^{-5}}\),
%   \(\displaystyle {10^{-3}}\),
%   \(\displaystyle {10^{-1}}\),
%   \(\displaystyle {10^{1}}\),
%   \(\displaystyle {10^{3}}\)
% }
ylabel near ticks,
xlabel near ticks,
width = 7.5cm, height=7cm
]
    \addplot [semithick, black]
    table {%
    0 1
    };
    \addlegendentry{WP EOS}
    \addplot [semithick, black, mark=*, mark size=3, mark options={solid}, only marks]
    table {%
    0 1
    };
    \addlegendentry{Saturation}
    \addplot [semithick, black, dashed]
    table {%
    0 1
    };
    \addlegendentry{G Extrap.}
    \addplot [semithick, black, dotted]
    table {%
    0 1
    };
    \addlegendentry{IdealGas Extrap.}
    \addplot [semithick, black, dash pattern=on 1pt off 3pt on 3pt off 3pt]
    table {%
    0 1
    };
    \addlegendentry{Power Extrap.}
    \addplot [semithick, steelblue31119180]
    table {%
    373.755928897123 0.0306051041977436
    378.607980593138 0.0310332038999433
    383.460032289153 0.0314591088609619
    388.312083985168 0.0318831453330401
    393.164135681182 0.0323055539817563
    398.016187377197 0.0327265241658882
    402.868239073212 0.033146211586911
    407.720290769227 0.0335647479603254
    412.572342465241 0.0339822467328254
    417.424394161256 0.03439880675304
    422.276445857271 0.0348145148218167
    427.128497553286 0.0352294475853638
    431.980549249301 0.0356436730125589
    436.832600945315 0.036057251588772
    441.68465264133 0.036470237303519
    446.536704337345 0.0368826784803917
    451.38875603336 0.0372946184818288
    456.240807729375 0.0377060963120227
    461.092859425389 0.0381171471354881
    465.944911121404 0.0385278027249808
    };
    \addlegendentry{P=1 bar}
    \addplot [semithick, steelblue31119180, mark=*, mark size=3, mark options={solid}, only marks, forget plot]
    table {%
    372.755928897123 0.0305165608645238
    };
    \addplot [semithick, steelblue31119180, forget plot]
    table {%
    279.566946672842 1.80165658036436e-05
    284.418998368857 1.80228443189366e-05
    289.271050064872 1.80346424918743e-05
    294.123101760887 1.80513579732168e-05
    298.975153456902 1.80725192775695e-05
    303.827205152916 1.80977517447863e-05
    308.679256848931 1.81267541561107e-05
    313.531308544946 1.81592822172187e-05
    318.383360240961 1.81951366145866e-05
    323.235411936975 1.82341542071955e-05
    328.08746363299 1.82762014231897e-05
    332.939515329005 1.8321169242286e-05
    337.79156702502 1.83689693414718e-05
    342.643618721035 1.84195311094006e-05
    347.495670417049 1.84727993201621e-05
    352.347722113064 1.85287323152513e-05
    357.199773809079 1.85873005830908e-05
    362.051825505094 1.86484856542136e-05
    366.903877201108 1.87122792510062e-05
    371.755928897123 1.87786826461874e-05
    };
    \addplot [semithick, steelblue31119180, dashed, forget plot]
    table {%
    279.566946672842 0.0222598827740029
    284.418998368857 0.0226897815450415
    289.271050064872 0.0231196803160802
    294.123101760887 0.0235495790871188
    298.975153456902 0.0239794778581575
    303.827205152916 0.0244093766291961
    308.679256848931 0.0248392754002347
    313.531308544946 0.0252691741712734
    318.383360240961 0.025699072942312
    323.235411936975 0.0261289717133507
    328.08746363299 0.0265588704843893
    332.939515329005 0.0269887692554279
    337.79156702502 0.0274186680264666
    342.643618721035 0.0278485667975052
    347.495670417049 0.0282784655685438
    352.347722113064 0.0287083643395825
    357.199773809079 0.0291382631106211
    362.051825505094 0.0295681618816598
    366.903877201108 0.0299980606526984
    371.755928897123 0.030427959423737
    373.755928897123 0.0306051623053105
    378.607980593138 0.0310350610763492
    383.460032289153 0.0314649598473878
    388.312083985168 0.0318948586184264
    393.164135681182 0.0323247573894651
    398.016187377197 0.0327546561605037
    402.868239073212 0.0331845549315423
    407.720290769227 0.033614453702581
    412.572342465241 0.0340443524736196
    417.424394161256 0.0344742512446583
    422.276445857271 0.0349041500156969
    427.128497553286 0.0353340487867355
    431.980549249301 0.0357639475577742
    436.832600945315 0.0361938463288128
    441.68465264133 0.0366237450998515
    446.536704337345 0.0370536438708901
    451.38875603336 0.0374835426419287
    456.240807729375 0.0379134414129674
    461.092859425389 0.038343340184006
    465.944911121404 0.0387732389550447
    };
    \addplot [semithick, steelblue31119180, dotted, forget plot]
    table {%
    279.566946672842 0.0228874206483928
    284.418998368857 0.0232846455331514
    289.271050064872 0.0236818704179101
    294.123101760887 0.0240790953026687
    298.975153456902 0.0244763201874273
    303.827205152916 0.0248735450721859
    308.679256848931 0.0252707699569445
    313.531308544946 0.0256679948417031
    318.383360240961 0.0260652197264618
    323.235411936975 0.0264624446112204
    328.08746363299 0.026859669495979
    332.939515329005 0.0272568943807376
    337.79156702502 0.0276541192654962
    342.643618721035 0.0280513441502548
    347.495670417049 0.0284485690350135
    352.347722113064 0.0288457939197721
    357.199773809079 0.0292430188045307
    362.051825505094 0.0296402436892893
    366.903877201108 0.0300374685740479
    371.755928897123 0.0304346934588065
    373.755928897123 0.030598428270241
    378.607980593138 0.0309956531549996
    383.460032289153 0.0313928780397582
    388.312083985168 0.0317901029245169
    393.164135681182 0.0321873278092755
    398.016187377197 0.0325845526940341
    402.868239073212 0.0329817775787927
    407.720290769227 0.0333790024635513
    412.572342465241 0.0337762273483099
    417.424394161256 0.0341734522330686
    422.276445857271 0.0345706771178272
    427.128497553286 0.0349679020025858
    431.980549249301 0.0353651268873444
    436.832600945315 0.035762351772103
    441.68465264133 0.0361595766568616
    446.536704337345 0.0365568015416203
    451.38875603336 0.0369540264263789
    456.240807729375 0.0373512513111375
    461.092859425389 0.0377484761958961
    465.944911121404 0.0381457010806547
    };
    \addplot [semithick, steelblue31119180, dash pattern=on 1pt off 3pt on 3pt off 3pt, forget plot]
    table {%
    279.566946672842 0.0223521841923462
    284.418998368857 0.0227723276211109
    289.271050064872 0.0231930610521656
    294.123101760887 0.0236143753960784
    298.975153456902 0.0240362618512102
    303.827205152916 0.024458711890075
    308.679256848931 0.0248817172465554
    313.531308544946 0.0253052699039049
    318.383360240961 0.0257293620834796
    323.235411936975 0.0261539862341419
    328.08746363299 0.0265791350222888
    332.939515329005 0.0270048013224581
    337.79156702502 0.0274309782084711
    342.643618721035 0.0278576589450746
    347.495670417049 0.0282848369800468
    352.347722113064 0.0287125059367348
    357.199773809079 0.0291406596069961
    362.051825505094 0.0295692919445148
    366.903877201108 0.0299983970584698
    371.755928897123 0.0304279692075312
    373.755928897123 0.030605172073059
    378.607980593138 0.0310353942661578
    383.460032289153 0.0314660702264989
    388.312083985168 0.0318971946816662
    393.164135681182 0.0323287624856601
    398.016187377197 0.0327607686143423
    402.868239073212 0.0331932081610989
    407.720290769227 0.0336260763327083
    412.572342465241 0.0340593684454009
    417.424394161256 0.034493079921102
    422.276445857271 0.0349272062838446
    427.128497553286 0.0353617431563442
    431.980549249301 0.0357966862567266
    436.832600945315 0.036232031395399
    441.68465264133 0.0366677744720578
    446.536704337345 0.037103911472825
    451.38875603336 0.037540438467507
    456.240807729375 0.0379773516069682
    461.092859425389 0.0384146471206151
    465.944911121404 0.0388523213139828
    };
    \addplot [semithick, darkorange25512714]
    table {%
    454.028007881676 0.00351235661124847
    459.936271143277 0.00357546402804142
    465.844534404878 0.00363702644052686
    471.752797666479 0.00369740442685451
    477.66106092808 0.00375681626266118
    483.569324189681 0.00381540884795562
    489.477587451282 0.00387329010002339
    495.385850712883 0.00393054451663538
    501.294113974484 0.00398724113522943
    507.202377236085 0.00404343792506486
    513.110640497686 0.0040991844270412
    519.018903759287 0.00415452348112089
    524.927167020888 0.00420949244407219
    530.835430282489 0.00426412409938716
    536.74369354409 0.00431844736633356
    542.651956805691 0.00437248786868834
    548.560220067292 0.00442626839999029
    554.468483328893 0.00447980930936753
    560.376746590494 0.0045331288246579
    566.285009852095 0.00458624332503592
    };
    \addlegendentry{P=10 bar}
    \addplot [semithick, darkorange25512714, mark=*, mark size=3, mark options={solid}, only marks, forget plot]
    table {%
    453.028007881676 0.00350148206214618
    };
    \addplot [semithick, darkorange25512714, forget plot]
    table {%
    339.771005911257 1.83818456197703e-05
    345.679269172858 1.84450180255277e-05
    351.587532434459 1.8512143164282e-05
    357.49579569606 1.85831621943783e-05
    363.404058957661 1.86580406680686e-05
    369.312322219262 1.87367662331484e-05
    375.220585480863 1.88193470027022e-05
    381.128848742464 1.89058104627799e-05
    387.037112004065 1.89962028268197e-05
    392.945375265666 1.90905887744168e-05
    398.853638527267 1.91890515340157e-05
    404.761901788868 1.92916932866154e-05
    410.670165050469 1.93986358822562e-05
    416.57842831207 1.95100218741922e-05
    422.486691573671 1.96260158882238e-05
    428.394954835272 1.97468063575961e-05
    434.303218096873 1.98726076680406e-05
    440.211481358474 2.00036627739409e-05
    446.119744620075 2.01402463663932e-05
    452.028007881676 2.02826686986277e-05
    };
    \addplot [semithick, darkorange25512714, dashed, forget plot]
    table {%
    339.771005911257 0.00226619512968994
    345.679269172858 0.00233063618210693
    351.587532434459 0.00239507723452392
    357.49579569606 0.00245951828694091
    363.404058957661 0.0025239593393579
    369.312322219262 0.00258840039177489
    375.220585480863 0.00265284144419189
    381.128848742464 0.00271728249660888
    387.037112004065 0.00278172354902587
    392.945375265666 0.00284616460144286
    398.853638527267 0.00291060565385985
    404.761901788868 0.00297504670627684
    410.670165050469 0.00303948775869383
    416.57842831207 0.00310392881111082
    422.486691573671 0.00316836986352781
    428.394954835272 0.0032328109159448
    434.303218096873 0.0032972519683618
    440.211481358474 0.00336169302077879
    446.119744620075 0.00342613407319578
    452.028007881676 0.00349057512561277
    454.028007881676 0.0035123889986796
    459.936271143277 0.00357683005109659
    465.844534404878 0.00364127110351358
    471.752797666479 0.00370571215593057
    477.66106092808 0.00377015320834756
    483.569324189681 0.00383459426076455
    489.477587451282 0.00389903531318154
    495.385850712883 0.00396347636559853
    501.294113974484 0.00402791741801553
    507.202377236085 0.00409235847043252
    513.110640497686 0.00415679952284951
    519.018903759287 0.0042212405752665
    524.927167020888 0.00428568162768349
    530.835430282489 0.00435012268010048
    536.74369354409 0.00441456373251747
    542.651956805691 0.00447900478493446
    548.560220067292 0.00454344583735145
    554.468483328893 0.00460788688976844
    560.376746590494 0.00467232794218544
    566.285009852095 0.00473676899460243
    };
    \addplot [semithick, darkorange25512714, dotted, forget plot]
    table {%
    339.771005911257 0.00262611154660964
    345.679269172858 0.00267177688621119
    351.587532434459 0.00271744222581275
    357.49579569606 0.00276310756541431
    363.404058957661 0.00280877290501586
    369.312322219262 0.00285443824461742
    375.220585480863 0.00290010358421897
    381.128848742464 0.00294576892382053
    387.037112004065 0.00299143426342209
    392.945375265666 0.00303709960302364
    398.853638527267 0.0030827649426252
    404.761901788868 0.00312843028222676
    410.670165050469 0.00317409562182831
    416.57842831207 0.00321976096142987
    422.486691573671 0.00326542630103142
    428.394954835272 0.00331109164063298
    434.303218096873 0.00335675698023454
    440.211481358474 0.00340242231983609
    446.119744620075 0.00344808765943765
    452.028007881676 0.0034937529990392
    454.028007881676 0.00350921112525316
    459.936271143277 0.00355487646485472
    465.844534404878 0.00360054180445627
    471.752797666479 0.00364620714405783
    477.66106092808 0.00369187248365939
    483.569324189681 0.00373753782326094
    489.477587451282 0.0037832031628625
    495.385850712883 0.00382886850246405
    501.294113974484 0.00387453384206561
    507.202377236085 0.00392019918166717
    513.110640497686 0.00396586452126872
    519.018903759287 0.00401152986087028
    524.927167020888 0.00405719520047184
    530.835430282489 0.00410286054007339
    536.74369354409 0.00414852587967495
    542.651956805691 0.0041941912192765
    548.560220067292 0.00423985655887806
    554.468483328893 0.00428552189847962
    560.376746590494 0.00433118723808117
    566.285009852095 0.00437685257768273
    };
    \addplot [semithick, darkorange25512714, dash pattern=on 1pt off 3pt on 3pt off 3pt, forget plot]
    table {%
    339.771005911257 0.00233315445563773
    345.679269172858 0.00239061072905434
    351.587532434459 0.00244847220993624
    357.49579569606 0.0025067348744086
    363.404058957661 0.00256539480463296
    369.312322219262 0.00262444818432837
    375.220585480863 0.00268389129455054
    381.128848742464 0.00274372050971037
    387.037112004065 0.0028039322938148
    392.945375265666 0.00286452319691418
    398.853638527267 0.00292548985174211
    404.761901788868 0.00298682897053441
    410.670165050469 0.00304853734201527
    416.57842831207 0.00311061182853956
    422.486691573671 0.00317304936338101
    428.394954835272 0.0032358469481569
    434.303218096873 0.00329900165038063
    440.211481358474 0.0033625106011341
    446.119744620075 0.0034263709928526
    452.028007881676 0.00349058007721521
    454.028007881676 0.0035123939459932
    459.936271143277 0.00357706555669724
    465.844534404878 0.00364207965316137
    471.752797666479 0.00370743367080838
    477.66106092808 0.00377312509622894
    483.569324189681 0.00383915146553871
    489.477587451282 0.00390551036280773
    495.385850712883 0.00397219941855803
    501.294113974484 0.00403921630832569
    507.202377236085 0.0041065587512839
    513.110640497686 0.00417422450892369
    519.018903759287 0.00424221138378937
    524.927167020888 0.00431051721826573
    530.835430282489 0.00437913989341432
    536.74369354409 0.00444807732785641
    542.651956805691 0.00451732747670005
    548.560220067292 0.00458688833050932
    554.468483328893 0.00465675791431336
    560.376746590494 0.00472693428665349
    566.285009852095 0.00479741553866643
    };
    \addplot [semithick, forestgreen4416044]
    table {%
    585.147146966519 0.000327519412666254
    592.780662058184 0.000346311310697216
    600.414177149848 0.000362802123852569
    608.047692241513 0.00037775357288756
    615.681207333178 0.000391576987791443
    623.314722424843 0.000404528199999452
    630.948237516507 0.000416780743285698
    638.581752608172 0.000428459290607768
    646.215267699837 0.000439657127347321
    653.848782791501 0.000450446205264266
    661.482297883166 0.000460883351805354
    669.115812974831 0.000471014315760819
    676.749328066496 0.000480876515295257
    684.38284315816 0.00049050096783296
    692.016358249825 0.000499913682965948
    699.64987334149 0.000509136691111628
    707.283388433154 0.000518188818169375
    714.916903524819 0.000527086278839013
    722.550418616484 0.000535843137817035
    730.183933708149 0.000544471673001094
    };
    \addlegendentry{P=100 bar}
    \addplot [semithick, forestgreen4416044, mark=*, mark size=3, mark options={solid}, only marks, forget plot]
    table {%
    584.147146966519 0.000324815468175749
    };
    \addplot [semithick, forestgreen4416044, forget plot]
    table {%
    438.110360224889 1.98376085719536e-05
    445.743875316554 2.00053256920947e-05
    453.377390408219 2.01818721839686e-05
    461.010905499883 2.03678739811543e-05
    468.644420591548 2.05640575954332e-05
    476.277935683213 2.07712677731207e-05
    483.911450774877 2.09904899316858e-05
    491.544965866542 2.12228789350273e-05
    499.178480958207 2.14697964240414e-05
    506.811996049872 2.17328599118282e-05
    514.445511141536 2.20140083863679e-05
    522.079026233201 2.23155915958625e-05
    529.712541324866 2.2640494166552e-05
    537.346056416531 2.29923124216735e-05
    544.979571508195 2.3375613583208e-05
    552.61308659986 2.37963287889712e-05
    560.246601691525 2.42623736815989e-05
    567.880116783189 2.4784678434187e-05
    575.513631874854 2.5379008743034e-05
    583.147146966519 2.60694648284117e-05
    };
    \addplot [semithick, forestgreen4416044, dashed, forget plot]
    table {%
    438.110360224889 -7.51804604765381e-05
    445.743875316554 -5.42722014953695e-05
    453.377390408219 -3.3363942514201e-05
    461.010905499883 -1.24556835330324e-05
    468.644420591548 8.45257544813613e-06
    476.277935683213 2.93608344293047e-05
    483.911450774877 5.02690934104732e-05
    491.544965866542 7.11773523916418e-05
    499.178480958207 9.20856113728103e-05
    506.811996049872 0.000112993870353979
    514.445511141536 0.000133902129335147
    522.079026233201 0.000154810388316316
    529.712541324866 0.000175718647297485
    537.346056416531 0.000196626906278653
    544.979571508195 0.000217535165259822
    552.61308659986 0.00023844342424099
    560.246601691525 0.000259351683222159
    567.880116783189 0.000280259942203327
    575.513631874854 0.000301168201184496
    583.147146966519 0.000322076460165665
    585.147146966519 0.000327554476185833
    592.780662058184 0.000348462735167002
    600.414177149848 0.00036937099414817
    608.047692241513 0.000390279253129339
    615.681207333178 0.000411187512110507
    623.314722424843 0.000432095771091676
    630.948237516507 0.000453004030072844
    638.581752608172 0.000473912289054013
    646.215267699837 0.000494820548035181
    653.848782791501 0.00051572880701635
    661.482297883166 0.000536637065997519
    669.115812974831 0.000557545324978687
    676.749328066496 0.000578453583959856
    684.38284315816 0.000599361842941024
    692.016358249825 0.000620270101922193
    699.64987334149 0.000641178360903361
    707.283388433154 0.00066208661988453
    714.916903524819 0.000682994878865698
    722.550418616484 0.000703903137846867
    730.183933708149 0.000724811396828036
    };
    \addplot [semithick, forestgreen4416044, dotted, forget plot]
    table {%
    438.110360224889 0.000243611601131812
    445.743875316554 0.000247856223041208
    453.377390408219 0.000252100844950605
    461.010905499883 0.000256345466860002
    468.644420591548 0.000260590088769399
    476.277935683213 0.000264834710678796
    483.911450774877 0.000269079332588192
    491.544965866542 0.000273323954497589
    499.178480958207 0.000277568576406986
    506.811996049872 0.000281813198316383
    514.445511141536 0.00028605782022578
    522.079026233201 0.000290302442135176
    529.712541324866 0.000294547064044573
    537.346056416531 0.00029879168595397
    544.979571508195 0.000303036307863367
    552.61308659986 0.000307280929772764
    560.246601691525 0.00031152555168216
    567.880116783189 0.000315770173591557
    575.513631874854 0.000320014795500954
    583.147146966519 0.000324259417410351
    585.147146966519 0.000325371518941147
    592.780662058184 0.000329616140850544
    600.414177149848 0.00033386076275994
    608.047692241513 0.000338105384669337
    615.681207333178 0.000342350006578734
    623.314722424843 0.000346594628488131
    630.948237516507 0.000350839250397528
    638.581752608172 0.000355083872306924
    646.215267699837 0.000359328494216321
    653.848782791501 0.000363573116125718
    661.482297883166 0.000367817738035115
    669.115812974831 0.000372062359944512
    676.749328066496 0.000376306981853908
    684.38284315816 0.000380551603763305
    692.016358249825 0.000384796225672702
    699.64987334149 0.000389040847582099
    707.283388433154 0.000393285469491495
    714.916903524819 0.000397530091400892
    722.550418616484 0.000401774713310289
    730.183933708149 0.000406019335219686
    };
    \addplot [semithick, forestgreen4416044, dash pattern=on 1pt off 3pt on 3pt off 3pt, forget plot]
    table {%
    438.110360224889 7.87427393757862e-05
    445.743875316554 8.57360290468106e-05
    453.377390408219 9.32156264571468e-05
    461.010905499883 0.000101206299993923
    468.644420591548 0.000109733634289132
    476.277935683213 0.00011882404293552
    483.911450774877 0.000128504781186872
    491.544965866542 0.000138803958642948
    499.178480958207 0.000149750551919356
    506.811996049872 0.000161374417302603
    514.445511141536 0.000173706303390572
    522.079026233201 0.000186777863718673
    529.712541324866 0.000200621669371887
    537.346056416531 0.00021527122158294
    544.979571508195 0.000230760964316817
    552.61308659986 0.000247126296841828
    560.246601691525 0.000264403586287438
    567.880116783189 0.000282630180189051
    575.513631874854 0.000301844419019944
    583.147146966519 0.000322085648710547
    585.147146966519 0.00032756369546383
    592.780662058184 0.000349158730797641
    600.414177149848 0.000371873532577751
    608.047692241513 0.000395750813413037
    615.681207333178 0.000420834344588929
    623.314722424843 0.000447168968522958
    630.948237516507 0.000474800611208723
    638.581752608172 0.0005037762946484
    646.215267699837 0.000534144149273979
    653.848782791501 0.000565953426357345
    661.482297883166 0.000599254510409364
    669.115812974831 0.000634098931568115
    676.749328066496 0.000670539377976381
    684.38284315816 0.00070862970814857
    692.016358249825 0.000748424963327153
    699.64987334149 0.000789981379828788
    707.283388433154 0.000833356401380216
    714.916903524819 0.000878608691444078
    722.550418616484 0.000925798145534747
    730.183933708149 0.000974985903524325
    };
        
    \end{axis}
    

\end{tikzpicture}

        \caption{The molar volume of water vapour at sub-saturation conditions, as extrapolated by the various methods considered, see Table~\ref{table:SemiEmpirical_ExtrapolationFuncs}}
        \label{fig:SemiEmpirical_extrapolation2}
    \end{figure}


\subsection{Algorithm}
\label{sec:algorithm}
    Similarly to \emph{GeoProp}, the fluid is first equilibrated to determine the composition of the water and carbon dioxide-rich phases. Using these compositions, the composition contribution to the component properties are then evaluated (i.e. the partial derivatives of the component activity, \(A_i\)). For the carbon dioxide-rich phase the properties of pure carbon dioxide are then determined from the \ac{SW} \ac{HEOS}, while the properties of pure water are determined from the \ac{WP} \ac{HEOS} or extrapolated from the saturated properties if the temperature is below the saturation temperature of pure water. For the water-rich phase, the properties of pure water are obtained from the \ac{WP} \ac{HEOS}, while the properties of aqueous carbon dioxide is calculated using \emph{ThermoFun} using the "CO2@" component in the \emph{slop98-inorganic} database \cite{Johnson1992}. The overall fluid properties are then obtained by aggregating the component and phase properties, see Figure~\ref{fig:coupled_model}.

    \todo{maybe I should just write a wee routine to calculate the CO2@ properties directly instead of using ThermoFUN- for speed}

    \begin{figure}[H]
        \centering
        \begin{tikzpicture} [node distance=1.5cm]
    \node (start) [startstop] {Start};
    \node (input) [io, right of=start, xshift=2cm] {\(P, T, \mathbf{z}\)};
    \node (SP2009) [process, right of=input, text width=4.5cm, xshift=3cm] {\(\mathbf{x}, \mathbf{y}=SP2009(P,T,\mathbf{z})\)};
    \node (cp_derivs) [process, below of=SP2009, xshift=-2.5cm, yshift=-0.2cm, text width=3.5cm] {\(\frac{\partial RT*f(x)*A_i(P,T,\mathbf{y})}{\partial x}\)};
    \node (wp_derivs) [process, right of=cp_derivs, xshift=3cm, yshift=-0.2cm, text width=3.5cm] {\(\frac{\partial RT*f(x)*A_i(P,T,\mathbf{x})}{\partial x}\)};
    \node (cp_CO2) [process, below of=cp_derivs, text width=3.5cm] {\(\Psi_{CO_2}^{cp}=SW(P,T)\)};
    \node (Tsat) [decision, left of=cp_CO2, xshift=-3.7cm] {\(T<T_{sat}^{wat}\)};        

    \node (extrap) [process, below of=Tsat, text width=4cm, xshift=2.5cm] {\(\Psi_{H_2O}^{cp}=Extrap(P,T)\)};
    \node (cp_H2O) [process, below of=Tsat, text width=3.5cm, yshift=-1.5cm] {\(\Psi_{H_2O}^{cp}=WP(P,T)\)};
    \node (cp_props) [process, right of=cp_H2O, xshift=3.5cm] {\(\Psi^{cp}(P,T, \mathbf{y})\)};

    \node (wp_CO2) [process, below of=wp_derivs, text width=3.5cm] {\(\Psi_{CO_2}^{wp}=SW(P,T)\)};
    \node (wp_H2O) [process, below of=wp_CO2, text width=3.5cm] {\(\Psi_{H_2O}^{wp}=WP(P,T)\)};
    \node (wp_props) [process, below of=wp_H2O, text width=3.5cm] {\(\Psi^{wp}(P,T, \mathbf{x})\)};
    
    \node (out1) [io, below of=wp_props] {\(\Psi(P,T\mathbf{z})\)};
    \node (stop) [startstop, right of=out1, xshift=2cm] {Stop};
    
    % \node (stop) [startstop, below of=out1] {Stop};
    
    \draw [arrow] (start) -- (input);
    \draw [arrow] (input) -- (SP2009);
    \draw [arrow] (SP2009) |- ($0.5*(SP2009)+0.5*(cp_derivs)$) -| (cp_derivs);
    \draw [arrow] (cp_derivs) -- (cp_CO2);
    \draw [arrow] (cp_CO2) -- (Tsat);
    \draw [arrow] (Tsat) -- node[anchor=west] {yes} (extrap);
    \draw [arrow] (Tsat) -- node[anchor=east] {no} (cp_H2O);
    \draw [arrow] (cp_H2O) -- (cp_props);
    \draw [arrow] (extrap) -| (cp_props);
    \draw [arrow] (cp_props) -| ($0.5*(cp_props)+0.5*(wp_derivs)$) |- (wp_derivs);
    \draw [arrow] (wp_derivs) -- (wp_CO2);
    \draw [arrow] (wp_CO2) -- (wp_H2O);
    \draw [arrow] (wp_H2O) -- (wp_props);
    \draw [arrow] (wp_props) -- (out1);
    \draw [arrow] (out1) -- (stop);

\end{tikzpicture}
        \caption{Calculation steps for equilibrating a mixture of water and carbon dioxide and determining the thermophysical properties}
        \label{fig:coupled_model}
    \end{figure}

\subsection{Validation}
\label{sec:SemiEmpirical_validation}
    The proposed model was validated against the \ac{HEOS} mixture model for water and carbon dioxide implemented in \emph{CoolProp}. An extract of the validation plots is shown in Figures~\ref{fig:SemiEmpirical_properties_maintext} and \ref{fig:SemiEmpirical_ratios_maintext} for a 50:50 mixture of carbon dioxide and water, further validation plots are provided in \nameref{ch:appendix_e}.

    For temperatures up to \qty{466}{\K} (\qty{193}{\degreeCelsius}) the differences between the coupled model and the \ac{HEOS} mixture are below \qty{10}{\percent}, see Figure~\ref{fig:SemiEmpirical_ratios_maintext}, particularly for the density and molar volume, as well as the vapour quality. Comparing the relative differences in molar enthalpy and molar enthalpy is difficult at low temperatures, as the values are small in magnitude and transition from positive to negative, which can lead to \emph{infinite} relative differences. In absolute terms, the differences are small, see Figure~\ref{fig:SemiEmpirical_properties_maintext}.

    \begin{figure}[H]
        \centering
        \input{Content/TPPM/SemiEmpirical/Plots/SemiEmpirical_properties}
        \caption{The molar density/volume/enthalpy/entropy/vapour quality for a carbon dioxide mole fraction of \num{0.5} as a function of temperature and pressure, calculated using the \ac{HEOS} mixture in \emph{CoolProp} and the coupled model}
        \label{fig:SemiEmpirical_properties_maintext}
    \end{figure}

    \begin{figure}[H]
        \centering
        \input{Content/TPPM/SemiEmpirical/Plots/SemiEmpirical_ratios}
        \caption{The ratio of molar density/volume/enthalpy/entropy/vapour quality, as a function of temperature and pressure for a mole fraction of carbon dioxide of \num{0.5}, as calculated using the \ac{HEOS} mixture in \emph{CoolProp} and the coupled model}
        \label{fig:SemiEmpirical_ratios_maintext}
    \end{figure}

\subsection{Performance}
\label{sec:SemiEmpirical_performance}
    The performance of the coupled model, the \ac{HEOS} mixture as well as the pure component \ac{HEOS} was compared for randomised values of pressure, temperature and composition. The temperatures were obtained by linear sampling of temperatures between \qty{298}{\K} and \qty{573}{\K}.The pressures were obtained by linear sampling \(\log P\) for pressures between \qty{1}{\bar} to \qty{100}{\bar}. The compositions were obtained by linear sampling the standard deviation for values between \num{-3.5} and \num{+3.5} of the Normal distribution, to ensure more representative sampling of the tails (i.e. quasi pure compositions). The results are provided in Table~\ref{table:SemiEmpirical_Performance}

    \begin{table}[H]
        \caption{The computational performance over 10000 randomised calculations}
        \centering 
        \label{table:SemiEmpirical_Performance}
        \begin{tabular}{|p{7em} | c | c | c |}
    \hline
    \rowcolor{bluepoli!40}
    \textbf{Model} & \textbf{Runtime}/\unit{\s} & \textbf{Speed-up} & \textbf{Failure Rate}/\unit{\percent} \T\B \\
    \hline \hline 
    \ac{HEOS} mixture & 252.97 & 1 & 5.69 \T\B\\
    \ac{WP} \ac{EOS}  & 0.34 & 744 & 0 \T\B\\
    \ac{SW} \ac{EOS} & 0.27 & 937 & 0 \T\B\\
    Coupled Model & 14.28 & 17 & 0 \T\B\\
    \hline
\end{tabular}    
        \\[10pt]
    \end{table}



\subsection{Conclusions}
\label{sec:tppm_semi_conclusions}
    The approach of tightly coupling the \ac{WP} \ac{EOS} and \ac{SW} \ac{EOS} with the \ac{SP2009} model has yielded a useful model for evaluating the properties of water and carbon dioxide for a wide range of temperatures, pressures and compositions. The calculated properties are similar to the values obtained from a \ac{HEOS} mixture of water and carbon dioxide in \emph{CoolProp}, while being computationally cheaper, speed-ups of \num{17} times have been observed, and stable, converging for all conditions tested. The performance could be further improved by implementing the model in C++ to streamline the evaluations and reduce computational overheads. Moreover, extending the \ac{SP2009} to sub-atmospheric conditions (i.e. below pressures of \qty{1}{bar} and temperatures below \qty{298}{\K}) would allow this model to be used geothermal power plant simulations, particularly direct steam cycles. 

\section{Case Studies}
\label{sec_tppm_case_study}
    \subsection{Cooling Curves of Geothermal Geofluids}
\label{sec:tppm_geoprop_casestudies}

    A common approach for investigating the comparative performance of different geothermal power plant technologies is to consider the geofluid to be pure water. However, this approach neglects the impact that impurities, such as dissolved salts and non-condensable gases, can have on the phase behaviour and thermophysical properties.

    \begin{table}[H]
        \caption{Used compositions and models of the considered geofluids.}
        \centering 
        \label{table:CasestudyComposition}
        \begin{tabular}{|p{4.5em} c | c | c | c | c| c |}
    \hline
    \rowcolor{bluepoli!40}
    \multicolumn{2}{|c|}{} &  & \multicolumn{4}{c|}{\textbf{Geofluid}} \T\B \\
    \rowcolor{bluepoli!40}
    \multicolumn{2}{|c|}{} & \textbf{Units} & \textbf{Water} & \textbf{Brine} & \textbf{Water \& NCG} & \textbf{Brine \& NCG} \T\B \\
    \hline \hline
    \emph{Component} &   &   &  &  &  &  \T\B\\
     & \(H_2O\) & \unit{\kg \per \kg} & 1.00 & 0.95 & 0.95 & 0.90 \T\B\\
    & \(NaCl\) & \unit{\kg \per \kg} & - & 0.05 & - & 0.05 \T\B\\
    & \(CO_2\) & \unit{\kg \per \kg} & - & - & 0.05 & 0.05 \T\B\\
    \hline
    \multicolumn{2}{|c|}{Partition Model} & - & \ac{WP} \ac{EOS} & \emph{Reaktoro} & \ac{SP2009} & \ac{SP2009} \T\B\\
    \hline
    \multicolumn{2}{|c|}{Property Model} & - & \ac{WP} \ac{EOS} & \multicolumn{3}{|c|}{Default - i.e. \emph{CoolProp} \& \emph{ThermoFun}} \T\B\\

     \hline
\end{tabular}   
        \\[10pt]
    \end{table}

    These differences can be illustrated by considering the \ac{PHE} in a simple binary ORC geothermal power plant and calculating the heat released by different geofluids as a function of reinjection temperature. Four geofluids, Table~\ref{table:CasestudyComposition} are considered and their temperature-heat content (TQ) curves are generated in GeoProp, Figure~\ref{fig:geoprop_casestudy}.
    
    The inlet conditions are defined in terms of a common temperature of \qty{473}{\K} (\qty{200}{\degreeCelsius}) and a heat content of \qty{1135}{\kilo\joule \per \kg} (relative to \qty{298}{\K} and \qty{1.01325}{\bar}). The heat content corresponds to that of pure water at \qty{473}{\K} and a vapour quality of \qty{0.2}{\kg\per\kg}. The same inlet temperature has been considered in order to investigate similar geothermal heat sources, while the heat content has been fixed to have Primary Heat Exchangers units of similar capacity. For the other three fluids, the vapour quality and inlet pressure are calculated in GeoProp assuming an inlet temperature of \qty{473}{\K} and a heat content of \qty{1135}{\kilo\joule \per \kg}. The resulting inlet conditions are summarised in Table~\ref{table:CasestudyBoundaryConditions}.

    \begin{table}[H]
        \caption{Used compositions and models of the considered geofluids.}
        \centering 
        \label{table:CasestudyBoundaryConditions}
        \begin{tabular}{|p{2em} c | c | c | c | c| c |}
    \hline
    \rowcolor{bluepoli!40}
    \multicolumn{2}{|c|}{} &  & \multicolumn{4}{c|}{\textbf{Geofluid}} \T\B \\
    \rowcolor{bluepoli!40}
    \multicolumn{2}{|c|}{Conditions} & \textbf{Units} & \textbf{Water} & \textbf{Brine} & \textbf{Water \& NCG} & \textbf{Brine \& NCG} \T\B \\
    \hline \hline
    \emph{Inlet} &   & & \multicolumn{4}{c|}{} \T\B\\
     & Mass Rate & \unit{\kg \per \s} & \multicolumn{4}{c|}{1.00} \T\B\\
    & Temperature & \unit{\K} & \multicolumn{4}{c|}{473} \T\B\\
    & Heat Content & \unit{\kilo \joule \per \kg} & \multicolumn{4}{c|}{1135} \T\B\\
    & Pressure & \unit{\bar} & 15.55 & 14.4 & 16.53 & 16.26 \T\B\\
    & Vapour Quality & \unit{\kg \per \kg} & 0.200 & 0.223 & 0.265 & 0.318 \T\B\\
    \hline
\end{tabular}         
        \\[10pt]
    \end{table}

    Unlike \emph{Water}, \emph{Brine} experiences a small temperature glide in the two-phase region as condensing water reduces the effective salinity of the aqueous phase, thereby reducing the saturation temperature. Moreover, liquid “Brine” has a lower specific heat capacity than liquid \emph{Water}, as indicated by the steeper slope. 
    
    Thus, a binary ORC operating on a liquid-dominated \emph{Brine}-like geofluid has a higher cycle working fluid mass rate to geofluid mass rate ratio compared to a \emph{Water}-like geofluid. Consequently, for the same net power, a higher mass rate of the \emph{Brine}-like geofluid is required. In turn, the higher geofluid mass rate also affects the heat exchanger design, in particular the required heat transfer and, hence, the cost.
    
    Above \qty{440}{\K}, the specific heat capacity of \emph{Water} \& \emph{\ac{NCG}} deviates from \emph{Water} significantly, which can be attributed to the presence of \ac{NCG}, reducing the boiling point of the geofluid, allowing the water species to remain in the vapour phase at lower temperatures. For example, at the inlet, the vapour quality of \emph{Water} \& \emph{\ac{NCG}} is \qty{0.265}{\kg\per\kg}, compared to just \qty{0.200}{\kg\per\kg} for \emph{Water}. Discounting the initial \ac{NCG} content of \qty{0.050}{\kg\per\kg}, this means that an additional \qty{0.015}{\kg\per\kg} of water is in the vapour phase. Similarly, when the vapour quality of \emph{Water} reaches zero, \emph{Water} \& \emph{\ac{NCG}} still has a vapour quality of \qty{0.09}{\kg\per\kg}, implying that about \qty{0.04}{\kg\per\kg} of water still remains in the vapour phase.
    
    The curvature of the TQ curve for \emph{Water} \& \emph{\ac{NCG}} (Figure 9) also has practical implications, as it reduces the average temperature difference between the hot geofluid and the cold working fluid, compared to the \emph{Water} case. This increases the heat transfer area required and, in turn, the cost of the heat exchanger.
    
    The \emph{Brine} \& \emph{\ac{NCG}} case has a slightly higher vapour quality, compared to the \emph{Water} \& \emph{\ac{NCG}} case. This can be attributed to the presence of \(Na^+\) and \(Cl^-\) ions.

    \begin{figure}[H]
        \centering
        \begin{tikzpicture}
    \begin{axis}[xlabel = {Heat Released/\unit{\kilo\joule\per\kg}},
                 ylabel = {Temperature/\unit{\K}},
                 ymin=298, ymax=500,
                 xmin=0,
                 xmax=1200,
                 legend style={at={(1.25,0.5)}, anchor= west},
                 width=10cm, height=8cm]
        
        % adding the dummy lines
        \addplot[color=black] coordinates {(0,0)};
        \addlegendentry{Heat Released}
        \addplot[color=black, dashed] coordinates {(0,0)};
        \addlegendentry{Vapour Quality}

        %define colors
        \definecolor{water}{HTML}{4472C4}
        \definecolor{brine}{HTML}{ED7D31}
        \definecolor{watncg}{HTML}{70AD47}
        \definecolor{brinencg}{HTML}{FFC000}

        % adding the geoprop lines for the legend
        \addplot[color=water,] coordinates {(1.136e+03,2.977e+02) (1.113e+03,3.032e+02) (1.090e+03,3.088e+02) (1.066e+03,3.143e+02) (1.043e+03,3.199e+02) (1.020e+03,3.254e+02) (9.968e+02,3.310e+02) (9.737e+02,3.365e+02) (9.505e+02,3.421e+02) (9.273e+02,3.476e+02) (9.041e+02,3.531e+02) (8.809e+02,3.587e+02) (8.577e+02,3.642e+02) (8.346e+02,3.697e+02) (8.114e+02,3.752e+02) (7.882e+02,3.807e+02) (7.650e+02,3.862e+02) (7.418e+02,3.916e+02) (7.186e+02,3.971e+02) (6.955e+02,4.026e+02) (6.723e+02,4.080e+02) (6.491e+02,4.134e+02) (6.259e+02,4.188e+02) (6.027e+02,4.242e+02) (5.796e+02,4.296e+02) (5.564e+02,4.350e+02) (5.332e+02,4.403e+02) (5.100e+02,4.456e+02) (4.868e+02,4.509e+02) (4.636e+02,4.562e+02) (4.405e+02,4.614e+02) (4.173e+02,4.666e+02) (3.941e+02,4.718e+02) (3.879e+02,4.732e+02) (3.709e+02,4.732e+02) (3.477e+02,4.732e+02) (3.246e+02,4.732e+02) (3.014e+02,4.732e+02) (2.782e+02,4.732e+02) (2.550e+02,4.732e+02) (2.318e+02,4.732e+02) (2.086e+02,4.732e+02) (1.855e+02,4.732e+02) (1.623e+02,4.732e+02) (1.391e+02,4.732e+02) (1.159e+02,4.732e+02) (9.273e+01,4.732e+02) (6.955e+01,4.732e+02) (4.636e+01,4.732e+02) (2.318e+01,4.732e+02) (0.000e+00,4.732e+02)};
        \addlegendentry{Water}

        \addplot[color=brine,] coordinates {(1.135e+03,2.980e+02) (1.099e+03,3.071e+02) (1.064e+03,3.161e+02) (1.028e+03,3.252e+02) (9.925e+02,3.342e+02) (9.568e+02,3.433e+02) (9.210e+02,3.523e+02) (8.852e+02,3.614e+02) (8.493e+02,3.704e+02) (8.133e+02,3.795e+02) (7.772e+02,3.885e+02) (7.410e+02,3.976e+02) (7.047e+02,4.066e+02) (6.682e+02,4.157e+02) (6.315e+02,4.247e+02) (5.946e+02,4.338e+02) (5.575e+02,4.428e+02) (5.202e+02,4.519e+02) (4.825e+02,4.609e+02) (4.444e+02,4.700e+02) (4.444e+02,4.700e+02) (4.401e+02,4.710e+02) (4.358e+02,4.721e+02) (7.158e+01,4.731e+02) (-5.576e+02,4.741e+02) (-8.323e+02,4.751e+02) (-9.791e+02,4.762e+02) (-1.072e+03,4.772e+02) (-1.135e+03,4.782e+02) (-1.182e+03,4.792e+02) (-1.218e+03,4.803e+02) (-1.247e+03,4.813e+02) (-1.270e+03,4.823e+02) (-1.290e+03,4.833e+02) (-1.307e+03,4.844e+02) (-1.321e+03,4.854e+02) (-1.334e+03,4.864e+02) (-1.346e+03,4.874e+02) (-1.356e+03,4.885e+02) (-1.365e+03,4.895e+02) (-1.374e+03,4.905e+02) (-1.382e+03,4.915e+02) (-1.389e+03,4.926e+02) (-1.396e+03,4.936e+02) (-1.403e+03,4.946e+02) (-1.409e+03,4.956e+02) (-1.415e+03,4.967e+02) (-1.420e+03,4.977e+02) (-1.425e+03,4.987e+02) (-1.430e+03,4.997e+02) (-1.435e+03,5.008e+02) (-1.440e+03,5.018e+02) (nan,5.028e+02) (nan,5.038e+02) (nan,5.049e+02) (nan,5.059e+02) (nan,5.069e+02) (nan,5.079e+02) (nan,5.090e+02) (nan,5.100e+02)};
        \addlegendentry{Brine}

        \addplot[color=watncg,] coordinates {(1.136e+03,2.978e+02) (1.113e+03,3.035e+02) (1.090e+03,3.091e+02) (1.066e+03,3.149e+02) (1.043e+03,3.206e+02) (1.020e+03,3.263e+02) (9.968e+02,3.321e+02) (9.737e+02,3.378e+02) (9.505e+02,3.436e+02) (9.273e+02,3.493e+02) (9.041e+02,3.551e+02) (8.809e+02,3.607e+02) (8.577e+02,3.663e+02) (8.346e+02,3.719e+02) (8.114e+02,3.775e+02) (7.882e+02,3.830e+02) (7.650e+02,3.885e+02) (7.418e+02,3.939e+02) (7.186e+02,3.993e+02) (6.955e+02,4.046e+02) (6.723e+02,4.099e+02) (6.491e+02,4.150e+02) (6.259e+02,4.201e+02) (6.027e+02,4.250e+02) (5.796e+02,4.297e+02) (5.564e+02,4.343e+02) (5.332e+02,4.387e+02) (5.100e+02,4.429e+02) (4.868e+02,4.467e+02) (4.636e+02,4.503e+02) (4.405e+02,4.535e+02) (4.173e+02,4.563e+02) (3.941e+02,4.589e+02) (3.879e+02,4.595e+02) (3.709e+02,4.610e+02) (3.477e+02,4.629e+02) (3.246e+02,4.645e+02) (3.014e+02,4.658e+02) (2.782e+02,4.670e+02) (2.550e+02,4.680e+02) (2.318e+02,4.688e+02) (2.086e+02,4.695e+02) (1.855e+02,4.702e+02) (1.623e+02,4.707e+02) (1.391e+02,4.712e+02) (1.159e+02,4.716e+02) (9.273e+01,4.720e+02) (6.955e+01,4.723e+02) (4.636e+01,4.726e+02) (2.318e+01,4.729e+02) (0.000e+00,4.731e+02)};
        \addlegendentry{Water \& NCG}

        \addplot[color=brinencg,] coordinates {(1.136e+03,2.978e+02) (1.113e+03,3.039e+02) (1.090e+03,3.100e+02) (1.066e+03,3.162e+02) (1.043e+03,3.223e+02) (1.020e+03,3.284e+02) (9.968e+02,3.346e+02) (9.737e+02,3.407e+02) (9.505e+02,3.469e+02) (9.273e+02,3.530e+02) (9.041e+02,3.591e+02) (8.809e+02,3.651e+02) (8.577e+02,3.711e+02) (8.346e+02,3.770e+02) (8.114e+02,3.829e+02) (7.882e+02,3.888e+02) (7.650e+02,3.946e+02) (7.418e+02,4.004e+02) (7.186e+02,4.060e+02) (6.955e+02,4.116e+02) (6.723e+02,4.171e+02) (6.491e+02,4.224e+02) (6.259e+02,4.276e+02) (6.027e+02,4.326e+02) (5.796e+02,4.374e+02) (5.564e+02,4.419e+02) (5.332e+02,4.461e+02) (5.100e+02,4.499e+02) (4.868e+02,4.533e+02) (4.636e+02,4.563e+02) (4.405e+02,4.589e+02) (4.173e+02,4.611e+02) (3.941e+02,4.629e+02) (3.879e+02,4.634e+02) (3.709e+02,4.645e+02) (3.477e+02,4.658e+02) (3.246e+02,4.669e+02) (3.014e+02,4.679e+02) (2.782e+02,4.687e+02) (2.550e+02,4.694e+02) (2.318e+02,4.700e+02) (2.086e+02,4.705e+02) (1.855e+02,4.709e+02) (1.623e+02,4.713e+02) (1.391e+02,4.717e+02) (1.159e+02,4.720e+02) (9.273e+01,4.723e+02) (6.955e+01,4.725e+02) (4.636e+01,4.728e+02) (2.318e+01,4.730e+02) (0.000e+00,4.732e+02)};
        \addlegendentry{Brine \& NCG}

    \end{axis}

    \begin{axis}[width=10cm, height=8cm, axis y line*=right, axis x line=none,
                ylabel = {Vapour Quality/\unit{\kg\per\kg}},
                ylabel near ticks,
                ymin=0, ymax=0.4,
                xmin=0,
                xmax=1200,
                y tick label style={/pgf/number format/.cd,fixed,fixed zerofill,precision=2,/tikz/.cd}]

        \definecolor{water}{HTML}{4472C4}
        \definecolor{brine}{HTML}{ED7D31}
        \definecolor{watncg}{HTML}{70AD47}
        \definecolor{brinencg}{HTML}{FFC000}
   
        \addplot[color=water, dashed] coordinates {(1.136e+03,-1.000e+00) (1.113e+03,-1.000e+00) (1.090e+03,-1.000e+00) (1.066e+03,-1.000e+00) (1.043e+03,-1.000e+00) (1.020e+03,-1.000e+00) (9.968e+02,-1.000e+00) (9.737e+02,-1.000e+00) (9.505e+02,-1.000e+00) (9.273e+02,-1.000e+00) (9.041e+02,-1.000e+00) (8.809e+02,-1.000e+00) (8.577e+02,-1.000e+00) (8.346e+02,-1.000e+00) (8.114e+02,-1.000e+00) (7.882e+02,-1.000e+00) (7.650e+02,-1.000e+00) (7.418e+02,-1.000e+00) (7.186e+02,-1.000e+00) (6.955e+02,-1.000e+00) (6.723e+02,-1.000e+00) (6.491e+02,-1.000e+00) (6.259e+02,-1.000e+00) (6.027e+02,-1.000e+00) (5.796e+02,-1.000e+00) (5.564e+02,-1.000e+00) (5.332e+02,-1.000e+00) (5.100e+02,-1.000e+00) (4.868e+02,-1.000e+00) (4.636e+02,-1.000e+00) (4.405e+02,-1.000e+00) (4.173e+02,-1.000e+00) (3.941e+02,-1.000e+00) (3.879e+02,-3.363e-14) (3.709e+02,8.781e-03) (3.477e+02,2.073e-02) (3.246e+02,3.268e-02) (3.014e+02,4.463e-02) (2.782e+02,5.659e-02) (2.550e+02,6.854e-02) (2.318e+02,8.049e-02) (2.086e+02,9.244e-02) (1.855e+02,1.044e-01) (1.623e+02,1.163e-01) (1.391e+02,1.283e-01) (1.159e+02,1.402e-01) (9.273e+01,1.522e-01) (6.955e+01,1.641e-01) (4.636e+01,1.761e-01) (2.318e+01,1.880e-01) (0.000e+00,2.000e-01)};
        
        \addplot[color=brine, dashed] coordinates {(1.135e+03,5.203e-18) (1.099e+03,5.203e-18) (1.064e+03,5.203e-18) (1.028e+03,5.203e-18) (9.925e+02,5.203e-18) (9.568e+02,5.203e-18) (9.210e+02,5.203e-18) (8.852e+02,5.203e-18) (8.493e+02,5.203e-18) (8.133e+02,5.203e-18) (7.772e+02,5.203e-18) (7.410e+02,5.203e-18) (7.047e+02,5.203e-18) (6.682e+02,5.203e-18) (6.315e+02,5.221e-18) (5.946e+02,6.040e-18) (5.575e+02,7.402e-18) (5.202e+02,1.002e-17) (4.825e+02,1.676e-17) (4.444e+02,6.778e-17) (4.444e+02,6.778e-17) (4.401e+02,1.066e-16) (4.358e+02,2.534e-16) (7.158e+01,1.853e-01) (-5.576e+02,5.080e-01) (-8.323e+02,6.434e-01) (-9.791e+02,7.168e-01) (-1.072e+03,7.625e-01) (-1.135e+03,7.936e-01) (-1.182e+03,8.160e-01) (-1.218e+03,8.329e-01) (-1.247e+03,8.461e-01) (-1.270e+03,8.566e-01) (-1.290e+03,8.653e-01) (-1.307e+03,8.724e-01) (-1.321e+03,8.785e-01) (-1.334e+03,8.837e-01) (-1.346e+03,8.882e-01) (-1.356e+03,8.921e-01) (-1.365e+03,8.955e-01) (-1.374e+03,8.986e-01) (-1.382e+03,9.013e-01) (-1.389e+03,9.037e-01) (-1.396e+03,9.060e-01) (-1.403e+03,9.080e-01) (-1.409e+03,9.098e-01) (-1.415e+03,9.115e-01) (-1.420e+03,9.130e-01) (-1.425e+03,9.144e-01) (-1.430e+03,9.157e-01) (-1.435e+03,9.169e-01) (-1.440e+03,9.180e-01) (nan,nan) (nan,nan) (nan,nan) (nan,nan) (nan,nan) (nan,nan) (nan,nan) (nan,nan)};

        \addplot[color=watncg, dashed] coordinates {(1.136e+03,2.910e-02) (1.113e+03,3.162e-02) (1.090e+03,3.373e-02) (1.066e+03,3.551e-02) (1.043e+03,3.701e-02) (1.020e+03,3.828e-02) (9.968e+02,3.937e-02) (9.737e+02,4.032e-02) (9.505e+02,4.114e-02) (9.273e+02,4.187e-02) (9.041e+02,4.252e-02) (8.809e+02,4.311e-02) (8.577e+02,4.367e-02) (8.346e+02,4.421e-02) (8.114e+02,4.476e-02) (7.882e+02,4.538e-02) (7.650e+02,4.601e-02) (7.418e+02,4.672e-02) (7.186e+02,4.754e-02) (6.955e+02,4.847e-02) (6.723e+02,4.955e-02) (6.491e+02,5.080e-02) (6.259e+02,5.226e-02) (6.027e+02,5.397e-02) (5.796e+02,5.598e-02) (5.564e+02,5.834e-02) (5.332e+02,6.114e-02) (5.100e+02,6.445e-02) (4.868e+02,6.832e-02) (4.636e+02,7.285e-02) (4.405e+02,7.809e-02) (4.173e+02,8.408e-02) (3.941e+02,9.080e-02) (3.879e+02,9.271e-02) (3.709e+02,9.824e-02) (3.477e+02,1.063e-01) (3.246e+02,1.150e-01) (3.014e+02,1.242e-01) (2.782e+02,1.338e-01) (2.550e+02,1.437e-01) (2.318e+02,1.540e-01) (2.086e+02,1.645e-01) (1.855e+02,1.752e-01) (1.623e+02,1.861e-01) (1.391e+02,1.971e-01) (1.159e+02,2.083e-01) (9.273e+01,2.195e-01) (6.955e+01,2.309e-01) (4.636e+01,2.423e-01) (2.318e+01,2.537e-01) (0.000e+00,2.653e-01)};

        \addplot[color=brinencg, dashed] coordinates {(1.136e+03,3.329e-02) (1.113e+03,3.544e-02) (1.090e+03,3.722e-02) (1.066e+03,3.870e-02) (1.043e+03,3.995e-02) (1.020e+03,4.100e-02) (9.968e+02,4.190e-02) (9.737e+02,4.268e-02) (9.505e+02,4.337e-02) (9.273e+02,4.400e-02) (9.041e+02,4.457e-02) (8.809e+02,4.511e-02) (8.577e+02,4.564e-02) (8.346e+02,4.620e-02) (8.114e+02,4.702e-02) (7.882e+02,4.768e-02) (7.650e+02,4.845e-02) (7.418e+02,4.934e-02) (7.186e+02,5.039e-02) (6.955e+02,5.162e-02) (6.723e+02,5.308e-02) (6.491e+02,5.482e-02) (6.259e+02,5.689e-02) (6.027e+02,5.936e-02) (5.796e+02,6.232e-02) (5.564e+02,6.592e-02) (5.332e+02,7.018e-02) (5.100e+02,7.522e-02) (4.868e+02,8.113e-02) (4.636e+02,8.792e-02) (4.405e+02,9.560e-02) (4.173e+02,1.041e-01) (3.941e+02,1.134e-01) (3.879e+02,1.159e-01) (3.709e+02,1.233e-01) (3.477e+02,1.337e-01) (3.246e+02,1.446e-01) (3.014e+02,1.559e-01) (2.782e+02,1.675e-01) (2.550e+02,1.793e-01) (2.318e+02,1.913e-01) (2.086e+02,2.035e-01) (1.855e+02,2.159e-01) (1.623e+02,2.283e-01) (1.391e+02,2.409e-01) (1.159e+02,2.535e-01) (9.273e+01,2.662e-01) (6.955e+01,2.790e-01) (4.636e+01,2.918e-01) (2.318e+01,3.047e-01) (0.000e+00,3.176e-01)};
    \end{axis}

\end{tikzpicture}        
        \caption[The temperature and heat released by different geofluids as well as the corresponding vapour quality.]{The temperature and heat released by different geofluids as well as the corresponding vapour quality. The heat content is relative to \qty{298}{\K} (\qty{25}{\degreeCelsius}) and \qty{1.01325}{\bar}.}
        \label{fig:geoprop_casestudy}
    \end{figure}

\subsection{Model Comparison}
\label{sec:tppm_model_comparison}
    A particular focus of this work is geothermal power generation from two-phase and \ac{NCG}-rich geofluids. In this respect, it is important to have the appropriate tools for describing these fluids over the typical range of temperatures and pressures encountered in the geothermal energy system. This case study aims to compare the \ac{SP2009} model against the \ac{HEOS} mixture model implemented in \emph{CoolProp}. The primary objective of this study is to compare the equilibrium compositions of the water-rich and carbon dioxide-rich phases.

    A mixture of water and carbon dioxide was modelled using an \ac{HEOS} mixture in \emph{CoolProp} using the default binary interaction data from \citeauthor{Gernert2013}. Similarly to the comparison above, this approach relies on the fluid to be two-phase at the temperature and pressure of interest in order for the mole fraction to be representative. In this respect, the same 1:10 ratio of carbon dioxide to water was used to ensure two-phase behaviour over a wide range of conditions.

    \inputminted[bgcolor=bg,linenos, fontsize=\footnotesize]{python}{Content/TPPM/CaseStudies/Code/SP2009vsCoolProp.py}

    Regarding the definition of phases, the same caveats as with the ELECNRTL model apply, the \ac{HEOS} mixture considers vapour and liquid phases, whereas the \ac{SP2009} model considers a carbon-dioxiderich and a water-rich phase. These definitions are congruent for most conditions, however below the critical temperautre of carbon dioxide, it is possible for both water and carbon dioxide to be in their liquid state.

    Comparing the equilibrium mole fraction of water in the vapour/carbon dioxide-rich phase, see Figure~\ref{fig:SP2009vsCoolProp_yH2O}, it can be seen that the \ac{HEOS} mixture and \ac{SP2009} models are in close agreement at the lower temperatures and pressures. Unlike the ELECNRTL model, the \ac{HEOS} mixture model appears to capture the phase transition at temperatures below \qty{31}{\degreeCelsius}. At higher pressures and temperatures more significant deviations are observed, for example at \qty{250}{\degreeCelsius} and \qty{600}{\bar}, the ratio of the equilibrium mole fractions of water is around \num{0.75}, representing a difference of \qty{25}{\percent}. Perhaps due to the complexity of the formulation, it is worth noting that the calculations did not converge for all conditions considered, see the gaps in the lines in Figure~\ref{fig:SP2009vsCoolProp_yH2O}.

    \begin{figure}[H]
        \centering
        % This file was created with tikzplotlib v0.10.1.
\begin{tikzpicture}

\definecolor{crimson2143940}{RGB}{214,39,40}
\definecolor{darkgray176}{RGB}{176,176,176}
\definecolor{darkorange25512714}{RGB}{255,127,14}
\definecolor{forestgreen4416044}{RGB}{44,160,44}
\definecolor{lightgray204}{RGB}{204,204,204}
\definecolor{mediumpurple148103189}{RGB}{148,103,189}
\definecolor{orchid227119194}{RGB}{227,119,194}
\definecolor{sienna1408675}{RGB}{140,86,75}
\definecolor{steelblue31119180}{RGB}{31,119,180}

\begin{axis}[
legend cell align={left},
legend style={
  % fill opacity=0.8,
  % draw opacity=1,
  % text opacity=1,
  at={(1.03,0.5)},
  anchor=west,
  % draw=lightgray204
},
log basis y={10},
% tick align=outside,
% tick pos=left,
unbounded coords=jump,
% x grid style={darkgray176},
xlabel={Pressure/\unit{\bar}},
xmin=0, xmax=600,
% xtick style={color=black},
% y grid style={darkgray176},
ylabel={\(y_{H_2O}\)/\unit{\percent}},
ymin=0.0478066537154532, ymax=100,
ymode=log,
% ytick style={color=black},
% ytick={0.001,0.01,0.1,1,10,100,1000,10000,100000},
% yticklabels={
%   \(\displaystyle {10^{-3}}\),
%   \(\displaystyle {10^{-2}}\),
%   \(\displaystyle {10^{-1}}\),
%   \(\displaystyle {10^{0}}\),
%   \(\displaystyle {10^{1}}\),
%   \(\displaystyle {10^{2}}\),
%   \(\displaystyle {10^{3}}\),
%   \(\displaystyle {10^{4}}\),
%   \(\displaystyle {10^{5}}\)
% }
ylabel near ticks,
xlabel near ticks
]
\addplot [semithick, black, dashed]
table {%
-1 1000
-1 1000
};
\addlegendentry{CoolProp}
\addplot [semithick, black]
table {%
-1 1000
-1 1000
};
\addlegendentry{SP2009}
\addplot [semithick, steelblue31119180]
table {%
1 3373.19628441501
2 1638.35814509514
3 929.438254932963
4 714.066105009353
5 601.538317778862
6 520.234131916313
7 457.802146052199
8 408.337966914968
9 368.251267183253
10 335.180855846863
11 307.49411434631
13 263.898057882313
15 231.264869966687
17 206.000394594738
21 169.534057082698
23 155.96385183873
25 144.530537718346
27 134.767787719483
29 126.335317167584
31 118.979222751117
33 112.50630948205
35 106.767013676241
37 101.643761915768
39 97.0428584203852
41 92.8887176073862
43 89.1196893914688
45 85.684987653441
47 82.5423965971268
49 79.6565346753605
51 76.9975241994237
54 73.3798022129648
57 70.1452390721801
60 67.2369628135922
63 64.6088513008231
66 62.2231048688591
69 60.0484487078598
72 58.0587821872011
75 56.2321505675229
80 53.5007968028779
85 51.1015597449896
90 48.9797284948602
95 47.0920370156331
100 45.4038185724287
125 39.1277015673981
150 35.1558048327492
200 30.7255372465483
250 28.6777010621692
300 27.8119322448473
350 27.5664846647821
400 27.635432569002
450 27.846152943384
500 28.104082810654
550 28.3601988366968
600 28.5909404160945
};
\addlegendentry{\qty{250}{\degreeCelsius}}
\addplot [semithick, steelblue31119180, dashed, forget plot]
table {%
1 25.6099666108255
2 25.6099666108255
3 25.6099666108255
4 25.6099666108255
5 25.6099666108255
6 25.6099666108255
7 25.6099666108255
8 25.6099666108255
9 25.6099666108255
10 25.6099666108255
11 25.6099666108255
13 25.6099666108255
15 25.6099666108255
17 25.6099666108255
21 25.6099666108255
23 25.6099666108255
25 25.6099666108255
27 25.6099666108255
29 25.6099666108255
31 25.6099666108255
33 25.6099666108255
35 25.6099666108255
37 25.6099666108255
39 25.6099666108255
41 89.4853019568454
43 89.3754134901206
45 90.4350551018516
47 87.3075432953675
49 84.4144489922775
51 81.7280017631996
54 78.0365632753564
57 74.696593189901
60 71.658930005306
63 68.8841079214371
66 66.3397786056885
69 63.9989865671242
72 61.8389845774672
75 59.8403751263099
80 56.8237004038673
85 54.1455957111165
90 51.7546115010738
95 49.609047477251
100 47.6747498307029
125 40.331314497689
150 35.4978792001588
200 29.6746845801153
250 26.4319148396335
300 24.460574185156
350 23.1841703462142
400 22.3128283317205
450 21.6898707297886
500 21.2262792859492
550 20.8691603641936
600 20.5858273116429
};
\addplot [semithick, darkorange25512714]
table {%
1 1156.51829327808
2 652.623858395477
3 458.830571630523
4 351.871315593291
5 284.987706905881
6 239.529580785241
7 206.716806503633
8 181.944082162249
9 162.586585006466
10 147.04581982364
11 134.294854299503
13 114.615288190719
15 100.135573867999
17 89.036063355412
21 73.1423030454507
23 67.2589954907123
25 62.3140616499141
27 58.100288332983
29 54.4673120114626
31 51.3033700415453
33 48.5236585798501
35 46.062665039542
37 43.8689770363057
39 41.9016821836461
41 40.127817517342
43 38.5205280321057
45 37.057714485989
47 35.7210252270183
49 34.4950940740325
51 33.3669569240847
54 31.8344670713569
57 30.4668770116671
60 29.2395854662059
63 28.1326640017945
66 27.1297978585099
69 26.2175023718406
72 25.3845348024306
75 24.6214469873916
80 23.4837413378863
85 22.4882570175331
90 21.6115235242842
95 20.8350008587108
100 20.143848505633
125 17.6150226082096
150 16.0739860889773
200 14.4980817743486
250 13.9225059261508
300 13.799580045365
350 13.8719266377027
400 14.0146184372955
450 14.1707891918517
500 14.3168943960795
550 14.4446397400657
600 14.5523487339373
};
\addlegendentry{\qty{200}{\degreeCelsius}}
\addplot [semithick, darkorange25512714, dashed, forget plot]
table {%
1 16.7081193559301
2 16.7081193559301
3 16.7081193559301
4 16.7081193559301
5 16.7081193559301
6 16.7081193559301
7 16.7081193559301
8 16.7081193559301
9 16.7081193559301
10 16.7081193559301
11 16.7081193559301
13 16.7081193559301
15 16.7081193559301
17 89.9442923955346
21 76.2517380723002
23 70.2197234056989
25 65.1018196382996
27 60.7069651334064
29 56.8946297304421
31 53.5584433881791
33 50.6162048112905
35 48.0033736019204
37 45.6685887535886
39 43.5704947229684
41 41.675444384445
43 39.9558043533722
45 38.3886848434187
47 36.9549751463151
49 35.6386027823824
51 34.4259585656987
54 32.7765768288244
57 31.3024326310187
60 29.9774760667637
63 28.7805599253779
66 27.6943433770443
69 26.7044757611655
72 25.7989808395029
75 24.9677866016302
80 23.7249816682987
85 22.6332030524984
90 21.6673914394492
95 20.8077460345922
100 20.0384146871302
125 17.1687575093459
150 15.3327925556081
200 13.2132664143456
250 12.1143543783255
300 11.4959453738656
350 11.1225658032969
400 10.8814975771172
450 10.7160718224561
500 10.5964029409003
550 10.5059397936664
600 10.4350326132276
};
\addplot [semithick, forestgreen4416044]
table {%
1 450.35786229226
2 229.449350106173
3 154.215813867376
4 116.359756241962
5 93.5693149779796
6 78.343708239149
7 67.452957992816
8 59.2769033650954
9 52.9133622712664
10 47.8200665006706
11 43.6514543709304
13 37.2365578361712
15 32.5320068253014
17 28.935173041047
21 23.8004984433452
23 21.9050649979064
25 20.3144250783058
27 18.9609431992183
29 17.7956244137679
31 16.782106642357
33 15.8928357662703
35 15.106551073667
37 14.4065856218961
39 13.7796891909833
41 13.2151955038036
43 12.7044216987639
45 12.2402278434159
47 11.8166888454963
49 11.4288466660984
51 11.0725208032811
54 10.5895041390204
57 10.1596358952072
60 9.77499820083833
63 9.42918155842476
66 9.11694214469342
69 8.83394846871651
72 8.57659133444167
75 8.34183938844646
80 7.9940284723662
85 7.69237766564122
90 7.42937188548925
95 7.19907889372545
100 6.99675267027732
125 6.29152858973381
150 5.91800549819175
200 5.68206601314327
250 5.75163272098748
300 5.90449093842759
350 6.05642570334671
400 6.1850943393286
450 6.28859782216541
500 6.37021269826036
550 6.43387939057336
600 6.48306419504824
};
\addlegendentry{\qty{150}{\degreeCelsius}}
\addplot [semithick, forestgreen4416044, dashed, forget plot]
table {%
1 9.32811898710353
2 9.32811898710354
3 9.32811898710354
4 9.32811898710354
5 90.3610288199216
6 80.0988052533426
7 69.0346069668385
8 60.6817956379252
9 54.1593759769876
10 48.9295816669098
11 44.6453553301284
13 38.0494141993531
15 33.2120000038693
17 29.5144965928866
21 24.2381488069291
23 22.2909973635659
25 20.6571042452397
27 19.2668380074967
29 18.0697764843625
31 17.0285125046025
33 16.1147104134701
35 15.306514075355
37 14.5867953541535
39 13.9419414966151
41 13.3609974499088
43 12.8350475666127
45 12.3567622218664
47 11.9200602166168
49 11.519853881032
51 11.1518541721703
54 10.652424574516
57 10.2072400881798
60 9.80818134241856
63 9.44867664315062
66 9.12335002400888
69 8.82776114243478
72 8.55821022411903
75 8.31158983620389
80 7.94455041965839
85 7.62415587147289
90 7.34270016574561
95 7.09409894362951
100 6.87348328491032
125 6.07414399103947
150 5.59773279679242
200 5.12772509843167
250 4.95443729164524
300 4.89375606486414
350 4.87464846586847
400 4.87100710173647
450 4.87311024064398
500 4.87708448526441
550 4.88131031074558
600 4.88514083944386
};
\addplot [semithick, crimson2143940]
table {%
1 95.8606316813074
2 48.1757796031571
3 32.2821241340086
4 24.3362788064538
5 19.5695664614579
6 16.3924281601276
7 14.1236244568558
8 12.4225357176395
9 11.0999288928803
10 10.0422642265305
11 9.17728921481984
13 7.84755613362324
15 6.87359414564234
17 6.12986168652296
21 5.07004910767003
23 4.67963419448738
25 4.35248060035318
27 4.07454848949184
29 3.8356716410119
31 3.62830791508808
33 3.44674409564568
35 3.28657335893589
37 3.14434221224398
39 3.0173060672773
41 2.9032563534378
43 2.80039587889069
45 2.70724742896406
47 2.62258570135033
49 2.54538591039606
51 2.47478448458696
54 2.37968117578582
57 2.29575921968184
60 2.22138780492659
63 2.15524893809483
66 2.0962664220923
69 2.04355335369373
72 1.99637273562772
75 1.95410752742201
80 1.89321210656189
85 1.84260872883118
90 1.80079359753984
95 1.76658999103182
100 1.73906428613391
125 1.67935765329994
150 1.71044117131609
200 1.87552652280567
250 2.01822869760922
300 2.11276919497652
350 2.1746996649175
400 2.21554171070386
450 2.24218342684361
500 2.25884950997315
550 2.2682625210935
600 2.27225878878612
};
\addlegendentry{\qty{99}{\degreeCelsius}}
\addplot [semithick, crimson2143940, dashed, forget plot]
table {%
1 4.14800449074032
2 49.1438746612167
3 32.9080855549648
4 24.8060273941503
5 19.9503458105443
6 16.7159365446246
7 14.4072848219058
8 12.676928624331
9 11.3319482752574
10 10.2566445727937
11 9.3774178242023
13 8.02609168372826
15 7.03656243319465
17 6.28108223218173
21 5.20471435120641
23 4.80823923175611
25 4.47601075930467
27 4.19375948624821
29 3.9511542883851
31 3.74053150363104
33 3.55608634476892
35 3.39334162571033
37 3.24878884343437
39 3.11963972788286
41 3.00365052919184
43 2.89899535426838
45 2.80417328904654
47 2.71793924010151
49 2.63925171697073
51 2.56723290365052
54 2.47010902295248
57 2.38426258968383
60 2.30803387375571
63 2.24008046357486
66 2.17930513030836
69 2.124802503869
72 2.07581907089923
75 2.03172276176783
80 1.96767530808539
85 1.91374369661983
90 1.86836917760713
95 1.83032315729892
100 1.79862270134898
125 1.7115941701309
150 1.7031656803934
200 1.77686616833938
250 1.85253479126495
300 nan
350 nan
400 nan
450 nan
500 2.01271160092641
550 2.02647916542156
600 2.03771633603268
};
\addplot [semithick, mediumpurple148103189]
table {%
1 19.9873187323954
2 10.0598646152343
3 6.75119703716956
4 5.09723223239361
5 4.10515357134599
6 3.44402228404596
7 2.97200756355852
8 2.61819409198314
9 2.34318454923113
10 2.12334061664519
11 1.94361979693063
13 1.6675201157532
15 1.46551622196881
17 1.3114722637043
21 1.09250698759603
23 1.01211265345535
25 0.944922551656159
27 0.888015530016043
29 0.839277047625009
31 0.797139041341842
33 0.760414393006203
35 0.728188157944251
37 0.699744078469779
39 0.674513717018478
41 0.652040486298789
43 0.63195372756195
45 0.613949712359501
47 0.597777507104801
49 0.583228312911115
51 0.570127328803914
54 0.55287586158314
57 0.538153149718716
60 0.525643846917749
63 0.515101728629109
66 0.50633574557022
69 0.499200143076213
72 0.493587578088459
75 0.489424538578709
80 0.48560568212928
85 0.48569681015941
90 0.489927368208858
95 0.498787705258638
100 0.513012799866181
125 0.651979826849887
150 0.756648383869949
200 0.860520897313372
250 0.915725684192483
300 0.950381880919357
350 0.97359545237227
400 0.989533245484759
450 1.00047120041752
500 1.00779444492035
550 1.01240991135276
600 1.01494358195364
};
\addlegendentry{\qty{60}{\degreeCelsius}}
\addplot [semithick, mediumpurple148103189, dashed, forget plot]
table {%
1 19.9484125467747
2 10.0532848039603
3 6.75864547843177
4 5.11250793859896
5 4.12547148519392
6 3.46789550993877
7 2.99854949540246
8 2.64683388182151
9 2.37353329915007
10 2.15512166422781
11 1.97662947346386
13 1.70255940683532
15 1.50220179787564
17 1.34955503900935
21 1.13292344098665
23 1.05354771483003
25 0.987310603702407
27 0.931306750504794
29 0.883433081374634
31 0.842130037561752
33 0.806216916103827
35 0.774783673833811
37 0.747117829424788
39 0.722653864002265
41 0.700937438074067
43 0.681599601261127
45 0.664337886523108
47 0.648902238854507
49 0.63508439795699
51 0.62270978761356
54 0.606538294805588
57 0.592882572272854
60 0.581424101887183
63 0.571911628580714
66 0.564146847960511
69 0.557973988887119
72 0.553272174746051
75 0.549949798424816
80 0.54730736249206
85 0.548144179404196
90 0.552423964507478
95 0.560219011596841
100 0.57164992534887
125 0.669872164373478
150 0.748257927719281
200 nan
250 nan
300 nan
350 nan
400 nan
450 0.9245481072892
500 0.932991912864759
550 0.939935955926555
600 0.94573405654296
};
\addplot [semithick, sienna1408675]
table {%
1 4.51683471711622
2 2.27670675425374
3 1.53017271297495
4 1.15704018560931
5 0.933270777901633
6 0.784185101544104
7 0.67777777332292
8 0.598046165107499
9 0.536099987375552
10 0.486605122814147
11 0.4461671872364
13 0.384106940303503
15 0.338779488157144
17 0.304288013838946
21 0.255461044327703
23 0.237636518417898
25 0.222810265218576
27 0.210324850981041
29 0.199705100025449
31 0.190599436999602
33 0.182742587090295
35 0.175931098462732
37 0.170006844832548
39 0.164845655985083
41 0.16034934184477
43 0.156440027799543
45 0.153056113644619
47 0.150149417372258
49 0.147683231185634
51 0.145631139351113
54 0.143295892203336
57 0.141845726774367
60 0.141335921738369
63 0.141931447920552
66 0.144034909992659
69 0.148808095651535
72 0.16821009142237
75 0.320010303606056
80 0.333501876413481
85 0.344080753070529
90 0.352894816263377
95 0.360496155659726
100 0.367200071047013
125 0.392474485018069
150 0.409982140846556
200 0.433644145666457
250 0.449092452262799
300 0.459818598357877
350 0.467458663503529
400 0.472927960804077
450 0.476795608341932
500 0.479441370018738
550 0.481131829307522
600 0.482061393794867
};
\addlegendentry{\qty{31}{\degreeCelsius}}
\addplot [semithick, sienna1408675, dashed, forget plot]
table {%
1 4.50895511497688
2 2.28177716339494
3 1.53990331986344
4 1.1692483057848
5 0.947065675440635
6 0.799117425232975
7 0.693590105751473
8 0.61457801440808
9 0.553245118034577
10 0.504290059934553
11 0.46433942517757
13 0.403148242275114
15 0.358601612996218
17 0.324840305741456
21 0.277402932530355
23 0.260262007171191
25 0.246122393238929
27 0.234332281155601
29 0.224421141732197
31 0.216041460882767
33 0.208931705046689
35 0.202892036211297
37 0.197767966717069
39 0.193439122363745
41 0.189811394290159
43 0.186811409790016
45 0.184382646864986
47 0.182482768346003
49 0.181081922293374
51 0.180161886025306
54 0.179672286068012
57 0.180285807676709
60 0.182123516662266
63 0.185452826532261
66 0.190845902768782
69 0.199724293007145
72 0.217802756837637
75 0.341687515642143
80 nan
85 0.374117597421795
90 nan
95 0.388923244373214
100 0.394471260233982
125 nan
150 nan
200 nan
250 0.456428203082959
300 nan
350 nan
400 nan
450 0.481880219346684
500 0.485590996488721
550 0.488723092138453
600 0.491396464192714
};
\addplot [semithick, orchid227119194]
table {%
1 2.35264971957208
2 1.18661188500407
3 0.798042909436359
4 0.603843305887396
5 0.487393248140123
6 0.409819523614363
7 0.354462241156084
8 0.312991508211641
9 0.280779656774538
10 0.255050133984396
11 0.234036088984049
13 0.201805473247432
15 0.17828960125294
17 0.160419461398019
21 0.135189088372316
23 0.126014099160879
25 0.118407897546522
27 0.112029314173569
29 0.10663225186952
31 0.102035204580275
33 0.0981019116914533
35 0.0947287122752502
37 0.091836090587067
39 0.0893629467939429
41 0.0872627205157718
43 0.0855008552146039
45 0.0840533362794944
47 0.0829062432368411
49 0.0820565085528615
51 0.081514521397142
54 0.0813526744282246
57 0.0822426012889671
60 0.0853166534215115
63 0.260240859772041
66 0.263243171610668
69 0.266038659674326
72 0.268656045336382
75 0.271118010367697
80 0.2749240461933
85 0.278413679086521
90 0.281635983014775
95 0.28462867004342
100 0.287421460584148
125 0.299091809975646
150 0.30807355014898
200 0.321147373390524
250 0.33016837792347
300 0.336627312675522
350 0.341316439505947
400 0.344713586944738
450 0.347131366124806
500 0.348786274810539
550 0.349834691526042
600 0.350393285369793
};
\addlegendentry{\qty{20}{\degreeCelsius}}
\addplot [semithick, orchid227119194, dashed, forget plot]
table {%
1 2.34905084740619
2 1.19099326061286
3 0.805259154668618
4 0.612577117640366
5 0.497114005534821
6 0.420261972871692
7 0.365475695453841
8 0.32448288523219
9 0.292688040093931
10 0.267334011199964
11 0.246666374632667
13 0.215072214382151
15 0.192149069712051
17 0.174850791401096
21 0.150753022644825
23 0.142155250035009
25 0.135141552691522
27 0.129375752617698
29 0.124616485549956
31 0.120687032758427
33 0.117456243075645
35 0.114826154970117
37 0.112723849750632
39 0.111096115823572
41 0.109906117146257
43 0.109131660962077
45 0.108764989139685
47 0.108814403921105
49 0.109308699771926
51 0.110306856295461
54 0.113016711019413
57 nan
60 nan
63 nan
66 nan
69 nan
72 nan
75 nan
80 nan
85 nan
90 nan
95 0.314964386966846
100 0.317093579937162
125 nan
150 nan
200 0.342838462988412
250 0.350201935238151
300 0.355867685715983
350 0.36038101944356
400 0.36406353344316
450 nan
500 nan
550 0.371893264455306
600 0.37378097125288
};
\end{axis}

\end{tikzpicture}
         
        \caption[Comparison of the equilibrium mole fractions of water in a carbon dioxide-rich phase]{The equilibrium mole fractions of water in a carbon dioxide-rich phase, as a function of temperature and pressure, as calculated using the \ac{HEOS} mixture model in \emph{CoolProp} and the \ac{SP2009} model}
        \label{fig:SP2009vsCoolProp_yH2O}
    \end{figure}
    
    % \begin{figure}[H]
    %     \centering
    %     % This file was created with tikzplotlib v0.10.1.
\begin{tikzpicture}

\definecolor{crimson2143940}{RGB}{214,39,40}
\definecolor{darkgray176}{RGB}{176,176,176}
\definecolor{darkorange25512714}{RGB}{255,127,14}
\definecolor{forestgreen4416044}{RGB}{44,160,44}
\definecolor{lightgray204}{RGB}{204,204,204}
\definecolor{mediumpurple148103189}{RGB}{148,103,189}
\definecolor{orchid227119194}{RGB}{227,119,194}
\definecolor{sienna1408675}{RGB}{140,86,75}
\definecolor{steelblue31119180}{RGB}{31,119,180}

\begin{axis}[
legend cell align={left},
legend style={
  % fill opacity=0.8,
  % draw opacity=1,
  % text opacity=1,
  at={(1.03,0.5)},
  anchor=west,
  % draw=lightgray204
},
% tick align=outside,
% tick pos=left,
unbounded coords=jump,
% x grid style={darkgray176},
xlabel={Pressure/\unit{\bar}},
xmin=0, xmax=600,
% xtick style={color=black},
% y grid style={darkgray176},
ylabel={\(\frac{y_{H_2O}^{Aspen}}{y_{H_2O}^{SP2009}}\)},
ymin=0, ymax=1.5,
% ytick style={color=black}
ylabel near ticks,
xlabel near ticks
]
\addplot [semithick, steelblue31119180]
table {%
1 0.00759219578449964
2 0.0156314824615703
3 0.0275542420111303
4 0.0358649800503975
5 0.0425741234662963
6 0.049227770804829
7 0.0559411239804573
8 0.0627175738869231
9 0.0695448159804462
10 0.0764064121326967
11 0.0832860383857061
13 0.0970449226354157
15 0.110738680779811
17 0.124319988130157
21 0.151060896267781
23 0.164204501933607
25 0.177194155748129
27 0.19003032582335
29 0.202714230549275
31 0.21524738537246
33 0.227631381108554
35 0.23986778059078
37 0.251958075224021
39 0.26390367130247
41 0.963360290267694
43 1.00286944557816
45 1.05543640232082
47 1.05772968673903
49 1.0597303703495
51 1.06143674894694
54 1.06346107405518
57 1.06488471887647
60 1.06576690865668
63 1.06617137643739
66 1.06615988940934
69 1.06578917431296
72 1.06510991529373
75 1.06416657592447
80 1.06210942265463
85 1.05956835723445
90 1.05665370330717
95 1.05344874889957
100 1.0500163054491
125 1.03076114573757
150 1.00973023854914
200 0.965798721174484
250 0.921688763765716
300 0.879499272823368
350 0.841027451564523
400 0.807399279023706
450 0.778918034885746
500 0.755273866397182
550 0.735860862061018
600 0.720012249056863
};
\addlegendentry{\qty{250}{\degreeCelsius}}
\addplot [semithick, darkorange25512714]
table {%
1 0.0144469131643148
2 0.0256014534880906
3 0.03641457302323
4 0.0474836072606785
5 0.058627509015496
6 0.069753887186528
7 0.0808261294208626
8 0.091831067860897
9 0.102764439976801
10 0.113625258956487
11 0.124413697331007
13 0.145775660644223
15 0.166854981806516
17 1.01020068729339
21 1.04251212906049
23 1.04401980572837
25 1.04473722165709
27 1.04486512675263
29 1.04456466877728
31 1.04395565719772
33 1.04312424686603
35 1.04213192095404
37 1.04102242265174
39 1.03982686260681
41 1.03856743184287
43 1.03726003755894
45 1.03591614798406
47 1.03454407905301
49 1.03314989389204
51 1.03173803484757
54 1.02959401693001
57 1.02742504980184
60 1.02523601442334
63 1.02303002387339
66 1.02080905731324
69 1.01857436236367
72 1.01632671389482
75 1.01406658245618
80 1.0102726531919
85 1.00644540992449
90 1.00258509841299
95 0.998691873146377
100 0.994765954555637
125 0.974665652790267
150 0.95388862916352
200 0.911380320514107
250 0.870127435576876
300 0.833064871255036
350 0.801803966657866
400 0.776439089355409
450 0.756208541202342
500 0.740132786325642
550 0.72732445964198
600 0.717068619232047
};
\addlegendentry{\qty{200}{\degreeCelsius}}
\addplot [semithick, forestgreen4416044]
table {%
1 0.0207126815542304
2 0.0406543709223284
3 0.0604874348043557
4 0.0801661956708331
5 0.965712197863017
6 1.02240252668199
7 1.02344817812424
8 1.02370050041543
9 1.02354818617145
10 1.02320187418024
11 1.02276902278564
13 1.02182952481156
15 1.02090228193482
17 1.0200214303546
21 1.018388285633
23 1.0176184076923
25 1.01686876028305
27 1.0161328898602
29 1.01540558871216
31 1.01468265382271
33 1.01396067073635
35 1.01323684014391
37 1.01250884400975
39 1.01177474349261
41 1.01103290118282
43 1.01028192159754
45 1.00952060533033
47 1.008747913436
49 1.00796293953296
51 1.00716488777024
54 1.00594177354007
57 1.0046856199832
60 1.00339469541564
63 1.00206752671004
66 1.00070285400672
69 0.999299596742766
72 0.997856828009418
75 0.996373755135532
80 0.993810623407354
85 0.991131247433021
90 0.988333910177127
95 0.985417585826508
100 0.982381914700135
125 0.965448047228292
150 0.945881648555888
200 0.902440254402299
250 0.861396673255351
300 0.8288193031197
350 0.804872164645691
400 0.787539661402356
450 0.774912051692627
500 0.765607793064161
550 0.758688501046114
600 0.753523440840694
};
\addlegendentry{\qty{150}{\degreeCelsius}}
\addplot [semithick, crimson2143940]
table {%
1 0.0432711992189926
2 1.02009505743413
3 1.019390341799
4 1.01930239998615
5 1.01945773044264
6 1.01973523271457
7 1.02008410560027
8 1.02047833972658
9 1.02090278096519
10 1.02134780975956
11 1.02180694153774
13 1.02275046486639
15 1.02370932645996
17 1.02466948740316
21 1.02656093475163
23 1.02748185689818
25 1.02838155302552
27 1.02925747406463
29 1.0301075426109
31 1.03093000681565
33 1.03172334414422
35 1.0324861961423
37 1.0332173230966
39 1.03391557181267
41 1.03457985225286
43 1.0352091202965
45 1.03580236480992
47 1.03635859781516
49 1.03687684692184
51 1.03735614945032
54 1.03799998423604
57 1.03855080674107
60 1.03900537701565
63 1.03936043024107
66 1.03961266914402
69 1.03975876138903
72 1.03979534174841
75 1.03971901917199
80 1.03933167407149
85 1.03860557408396
90 1.03752544442606
95 1.03607694291865
100 1.0342473913644
125 1.01919574235281
150 0.995746424346713
200 0.947395916151218
250 0.917901322808188
300 nan
350 nan
400 nan
450 nan
500 0.891033949822681
550 0.893405920424335
600 0.896780043756048
};
\addlegendentry{\qty{99}{\degreeCelsius}}
\addplot [semithick, mediumpurple148103189]
table {%
1 0.998053456487006
2 0.999345934411077
3 1.0011032771257
4 1.00299686290695
5 1.0049493675437
6 1.00693178612792
7 1.0089306407458
8 1.01093875733891
9 1.01295192473376
10 1.01496747499364
11 1.01698360789767
13 1.02101281462881
15 1.02503252803135
17 1.02903818583055
21 1.03699422873217
23 1.04093917928327
25 1.04485875797117
27 1.04875052183825
29 1.05261198775134
31 1.05644058801106
33 1.06023363513222
35 1.06398829118714
37 1.06770153890921
39 1.07137015270287
41 1.0749906682219
43 1.07855934941741
45 1.08207215208222
47 1.08552468291642
49 1.08891215309326
51 1.09222932519295
54 1.09706054641052
57 1.10169860119326
60 1.1061179642766
63 1.1102886998706
66 1.11417543180797
69 1.1177360355883
72 1.12091997308509
75 1.12366617338369
80 1.12706128168071
85 1.12857273908035
90 1.12756298250311
95 1.12316122809472
100 1.11429953696669
125 1.02744308456604
150 0.988911023495809
200 nan
250 nan
300 nan
350 nan
400 nan
450 0.924112665015606
500 0.925776002802339
550 0.928414415333643
600 0.931809485136645
};
\addlegendentry{\qty{60}{\degreeCelsius}}
\addplot [semithick, sienna1408675]
table {%
1 0.998255503547766
2 1.00222708046688
3 1.00635915593448
4 1.01055116350092
5 1.01478123805614
6 1.01904183547924
7 1.02332967094957
8 1.02764309891965
9 1.03198121817342
10 1.0363435078902
11 1.04072966022833
13 1.04957291830386
15 1.05851040435447
17 1.06754223291026
21 1.08589132742488
23 1.09521048744497
25 1.10462771092474
27 1.11414452482709
29 1.12376269661415
31 1.13348425516713
33 1.14331152017375
35 1.15324714040978
37 1.16329414213801
39 1.17345598953029
41 1.18373665963601
43 1.19414073506424
45 1.20467351793019
47 1.21534116841481
49 1.22615086925988
51 1.23711101092837
54 1.2538551057211
57 1.27099921708244
60 1.28858618829684
63 1.3066366140087
66 1.32499754940319
69 1.34216013001632
72 1.29482574437667
75 1.06773910649693
80 nan
85 1.08729591551757
90 nan
95 1.07885545592425
100 1.07426793003937
125 nan
150 nan
200 nan
250 1.01633461168897
300 nan
350 nan
400 nan
450 1.01066413137158
500 1.01282664962713
550 1.01577792689762
600 1.01936489940495
};
\addlegendentry{\qty{31}{\degreeCelsius}}
\addplot [semithick, orchid227119194]
table {%
1 0.998470289845552
2 1.00369234091126
3 1.00904242760249
4 1.01446370551402
5 1.01994438255309
6 1.02548060464575
7 1.03107088151854
8 1.03671466068267
9 1.0424118451322
10 1.04816259855923
11 1.05396725651777
13 1.06574024441078
15 1.07773570843007
17 1.08995997042573
21 1.11512714864713
23 1.12809003898463
25 1.14132212032922
27 1.15483838825662
29 1.16865660590608
31 1.18279796914091
33 1.19728801458084
35 1.21215787919158
37 1.22744608388749
39 1.24320112316509
41 1.25948533917634
43 1.2763809284499
45 1.29399966680703
47 1.31249951358007
49 1.33211492542981
51 1.35321724773481
54 1.38921936880053
57 nan
60 nan
63 nan
66 nan
69 nan
72 nan
75 nan
80 nan
85 nan
90 nan
95 1.10657997635585
100 1.10323557361621
125 nan
150 nan
200 1.06754247861
250 1.06067678994784
300 1.05715630406677
350 1.05585602605374
400 1.05613340242816
450 nan
500 nan
550 1.06305427524366
600 1.06674695794585
};
\addlegendentry{\qty{20}{\degreeCelsius}}
\end{axis}

\end{tikzpicture}
         
    %     \caption{The ratio of equilibrium mole fractions of water in a carbon dioxide-rich phase, as a function of temperature and pressure, as calculated using the \ac{HEOS} mixture model in \emph{CoolProp} and the \ac{SP2009} model}
    %     \label{fig:SP2009vsCoolProp_yH2O_ratio}
    % \end{figure}

    Comparing the equilibrium mole fraction of carbon dioxide in the liquid/water-rich phase, see Figures~\ref{fig:SP2009vsCoolProp_xCO2_part2}, \ref{fig:SP2009vsCoolProp_xCO2_part1}, the \ac{HEOS} mixture model and the \ac{SP2009} appear in close agreement for temperatures up to \qty{150}{\degreeCelsius}.
    
    \begin{figure}[H]
        \centering
        % This file was created with tikzplotlib v0.10.1.
\begin{tikzpicture}

\definecolor{crimson2143940}{RGB}{214,39,40}
\definecolor{darkgray176}{RGB}{176,176,176}
\definecolor{darkorange25512714}{RGB}{255,127,14}
\definecolor{forestgreen4416044}{RGB}{44,160,44}
\definecolor{lightgray204}{RGB}{204,204,204}
\definecolor{steelblue31119180}{RGB}{31,119,180}

\begin{axis}[
legend cell align={left},
legend style={
  % fill opacity=0.8,
  % draw opacity=1,
  % text opacity=1,
  at={(1.03,0.5)},
  anchor=west,
  % draw=lightgray204
},
% tick align=outside,
% tick pos=left,
unbounded coords=jump,
% x grid style={darkgray176},
xlabel={Pressure/\unit{\bar}},
xmin=0, xmax=600,
% xtick style={color=black},
% y grid style={darkgray176},
ylabel={\(x_{CO_2}\)/\unit{\percent}},
ymin=0, ymax=10,
% ytick style={color=black}
ylabel near ticks,
xlabel near ticks
]
\addplot [semithick, black, dashed]
table {%
-1 1000
-1 1000
};
\addlegendentry{CoolProp}
\addplot [semithick, black]
table {%
-1 1000
-1 1000
};
\addlegendentry{SP2009}
\addplot [semithick, steelblue31119180]
table {%
1 -0.653841086378044
2 -0.587318523834475
3 -0.527281299132599
4 -0.506015076503974
5 -0.493764302302543
6 -0.482140363523801
7 -0.470356117000907
8 -0.458236300156135
9 -0.445714655475876
10 -0.432776119439212
11 -0.419434502395587
13 -0.391664375281455
15 -0.362685201804279
17 -0.332765308382717
21 -0.270921271293673
23 -0.239288020715552
25 -0.207316001819702
27 -0.175076269618617
29 -0.142623276698775
31 -0.109999283136248
33 -0.0772375075615523
35 -0.0443643864785528
37 -0.0114012077192016
39 0.0216346960746392
41 0.0547290654310063
43 0.0878700546399711
45 0.12104773317503
47 0.154253704608061
49 0.187480812314075
51 0.220722909979323
54 0.270602723554086
57 0.320489293744996
60 0.370370119834738
63 0.420234761156227
66 0.470074388206378
69 0.5198814466336
72 0.569649402483058
75 0.619372546783056
80 0.702131450227207
85 0.784732039760643
90 0.867157434785795
95 0.949393166279459
100 1.03142662525023
125 1.43820715596948
150 1.83864431033483
200 2.61804066756336
250 3.36722783281498
300 4.08756180335536
350 4.78365190525344
400 5.46224457543851
450 6.13083340315622
500 6.79657752970924
550 7.4657308112937
600 8.14344582145374
};
\addlegendentry{\qty{250}{\degreeCelsius}}
\addplot [semithick, steelblue31119180, dashed, forget plot]
table {%
1 0.34308497389756
2 0.34308497389756
3 0.34308497389756
4 0.34308497389756
5 0.34308497389756
6 0.34308497389756
7 0.34308497389756
8 0.34308497389756
9 0.34308497389756
10 0.34308497389756
11 0.34308497389756
13 0.34308497389756
15 0.34308497389756
17 0.34308497389756
21 0.34308497389756
23 0.34308497389756
25 0.34308497389756
27 0.34308497389756
29 0.34308497389756
31 0.34308497389756
33 0.34308497389756
35 0.34308497389756
37 0.34308497389756
39 0.34308497389756
41 0.11485361370615
43 0.11637547915228
45 0.122083370801884
47 0.168945144424781
49 0.215906499094609
51 0.262938325700224
54 0.333564318128798
57 0.40422497571414
60 0.474868681937113
63 0.545456461952964
66 0.615958582627008
69 0.686351936397434
72 0.756618159231959
75 0.826742301383765
80 0.943267060264352
85 1.05931548551882
90 1.17484828654251
95 1.2898313044867
100 1.40423364940025
125 1.96658462131998
150 2.51069103420849
200 3.53547944805797
250 4.4682600611982
300 5.30843804858109
350 6.06170799881477
400 6.73656832259001
450 7.34200548793215
500 7.88641337296359
550 8.37725245309129
600 8.82100941090856
};
\addplot [semithick, darkorange25512714]
table {%
1 -0.209830680589779
2 -0.196192370565988
3 -0.184165282072564
4 -0.170960331838576
5 -0.156819285309495
6 -0.14204804380314
7 -0.126863830642195
8 -0.111406385982936
9 -0.0957653235864359
10 -0.0799994737128217
11 -0.0641486180210685
13 -0.032294794281159
15 -0.000345378909574973
17 0.0316212542844356
21 0.0954361845356526
23 0.127235132346014
25 0.158941527098648
27 0.190545607840569
29 0.222040184910984
31 0.253419828970211
33 0.284680349656216
35 0.315818450158069
37 0.346831492241519
39 0.37771733266618
41 0.408474206944814
43 0.439100645219942
45 0.469595410377783
47 0.499957451843782
49 0.530185870621655
51 0.560279892516352
54 0.605167489678274
57 0.649749301971555
60 0.694023779734938
63 0.737989661648074
66 0.781645926345186
69 0.824991756734955
72 0.868026513327649
75 0.910749714047461
80 0.981261837575989
85 1.05090684576567
90 1.11968484313151
95 1.18759675272537
100 1.2546442624723
125 1.57702429580435
150 1.87843899555021
200 2.42243363504952
250 2.89758542429717
300 3.31734951101392
350 3.69465896122736
400 4.04003161191148
450 4.36122490138826
500 4.66372269607187
550 4.95137032345634
600 5.22688825336974
};
\addlegendentry{\qty{200}{\degreeCelsius}}
\addplot [semithick, darkorange25512714, dashed, forget plot]
table {%
1 0.200195628941994
2 0.200195628941994
3 0.200195628941994
4 0.200195628941994
5 0.200195628941994
6 0.200195628941994
7 0.200195628941994
8 0.200195628941994
9 0.200195628941994
10 0.200195628941994
11 0.200195628941994
13 0.200195628941994
15 0.200195628941994
17 0.0334856676733024
21 0.103678457847423
23 0.14164333974277
25 0.179515121018203
27 0.217281333143626
29 0.254936360296454
31 0.292477882007125
33 0.32990500207063
35 0.36721733852596
37 0.404414599914549
39 0.44149639670632
41 0.478462169281435
43 0.515311172937227
45 0.552042489093507
47 0.588655047028952
49 0.625147648453772
51 0.661518991338983
54 0.715845600255316
57 0.769891326841032
60 0.823651045210949
63 0.877119517565705
66 0.93029146574658
69 0.983161626511786
72 1.03572479437231
75 1.08797585496088
80 1.17435387377312
85 1.25982885369779
90 1.3443798000867
95 1.42798710211671
100 1.51063263607163
125 1.90890946484127
150 2.28133718650005
200 2.94755829614577
250 3.51479404130568
300 3.9963927788033
350 4.40762564694651
400 4.76214704777789
450 5.07094931821178
500 5.34252062998216
550 5.58331889419891
600 5.79826416475641
};
\addplot [semithick, forestgreen4416044]
table {%
1 -0.0609944522738537
2 -0.0450239450828673
3 -0.0282821716908178
4 -0.0113677177321072
5 0.00557777432155286
6 0.0225049629355507
7 0.039392321806293
8 0.0562290127699571
9 0.0730090138005047
10 0.0897287322702979
11 0.106385908950199
13 0.139507209209479
15 0.172365928872223
17 0.204958741227428
21 0.269341261773402
23 0.301130204362337
25 0.33265110939945
27 0.363904404284206
29 0.394890663945583
31 0.425610569714571
33 0.456064883115899
35 0.486254428618943
37 0.516180081962328
39 0.54584276205432
41 0.575243425230086
43 0.604383061100091
45 0.633262689495958
47 0.661883358188064
49 0.690246141155386
51 0.718352137256959
54 0.760032118275156
57 0.801140745980103
60 0.841682018139158
63 0.881660041275716
66 0.92107902821152
69 0.959943296244722
72 0.998257265781078
75 1.03602545929171
80 1.09777221670985
85 1.1580378409528
90 1.2168451285062
95 1.2742176409104
100 1.33017969974697
125 1.58973676139832
150 1.81810031810904
200 2.19666609082599
250 2.49735923208106
300 2.74686771662808
350 2.96299418345139
400 3.15633601632836
450 3.33318082407821
500 3.49742663083469
550 3.6516210888643
600 3.79751201982417
};
\addlegendentry{\qty{150}{\degreeCelsius}}
\addplot [semithick, forestgreen4416044, dashed, forget plot]
table {%
1 0.10277201913951
2 0.10277201913951
3 0.10277201913951
4 0.10277201913951
5 0.00867516161284781
6 0.0213513925453435
7 0.0384866761840241
8 0.0555513478404546
9 0.0725550439917622
10 0.0895040828515437
11 0.106402634639791
13 0.140058272547738
15 0.173534230904782
17 0.206835410125816
21 0.272917605869594
23 0.305697251222159
25 0.338300590127316
27 0.370725722769216
29 0.402970599995833
31 0.435033097234727
33 0.466911060467012
35 0.498602335894771
37 0.530104789716535
39 0.561416321691488
41 0.592534874679529
43 0.623458441505287
45 0.654185069994928
47 0.68471286673647
49 0.71503999993705
51 0.745164701601775
54 0.789968489018955
57 0.834307527707168
60 0.878176700472788
63 0.921571241071495
66 0.964486737863002
69 1.00691913623241
72 1.04886474005721
75 1.09032021240781
80 1.15831556017205
85 1.22492973483955
90 1.29015391751008
95 1.35398201379499
100 1.41641062946947
125 1.70762558812968
150 1.96479984793502
200 2.38687737483804
250 2.7096343164443
300 2.96187760571022
350 3.16558704852847
400 3.33515557473609
450 3.47961363563255
500 3.60473260610812
550 3.71437558415835
600 3.81125413821783
};
\addplot [semithick, crimson2143940]
table {%
1 0.000842091372508321
2 0.0210139570098535
3 0.0410476956525778
4 0.0609438987978718
5 0.0807031559452749
6 0.100326054609207
7 0.119813180331676
8 0.139165116695176
9 0.158382445335765
10 0.177465745956357
11 0.196415596340209
13 0.233917248014918
15 0.270891984759432
17 0.307344361160281
21 0.378700105070782
23 0.413612437039206
25 0.448020338621845
27 0.48192822281512
29 0.515340475936113
31 0.548261458224602
33 0.580695504472868
35 0.612646924685472
37 0.644120004771363
39 0.675119007270853
41 0.705648172120167
43 0.735711717456504
45 0.765313840466724
47 0.794458718283002
49 0.823150508929045
51 0.851393352320671
54 0.892924855690442
57 0.933469349192333
60 0.973040610177722
63 1.01165235463741
66 1.0493182481707
69 1.08605191799506
72 1.12186696609924
75 1.15677698364687
80 1.21298610437518
85 1.26678214802429
90 1.31822865910022
95 1.36738989165389
100 1.41433113270418
125 1.61814201534475
150 1.77747593487639
200 2.00598203845093
250 2.17160118515993
300 2.30761918369489
350 2.42658564914289
400 2.53413423695859
450 2.63327998505786
500 2.72584501247257
550 2.81302759281723
600 2.89566559681683
};
\addlegendentry{\qty{99}{\degreeCelsius}}
\addplot [semithick, crimson2143940, dashed, forget plot]
table {%
1 0.0432563786719489
2 0.020204593133466
3 0.0397853045960739
4 0.0592684469303273
5 0.0786628468913642
6 0.0979703404675458
7 0.117191244817438
8 0.136325383995806
9 0.15537238396125
10 0.17433178172348
11 0.193203072937065
13 0.230679240327636
15 0.26779667765171
17 0.30455125102633
21 0.376956054687516
23 0.412598876318593
25 0.447864034805467
27 0.482748323715709
29 0.517248736731823
31 0.551362465772065
33 0.58508689850868
35 0.618419615549892
37 0.651358387444656
39 0.683901171607093
41 0.716046109227531
43 0.747791522204844
45 0.779135910136053
47 0.810077947378096
49 0.840616480204404
51 0.870750524059405
54 0.91519142853359
57 0.958718359413635
60 1.0013300580251
63 1.0430260619549
66 1.08380669021195
69 1.12367302998
72 1.16262692497503
75 1.20067096543379
80 1.26206517180086
85 1.32096146401031
90 1.37738534193975
95 1.43136786465032
100 1.48294565277556
125 1.70643932457429
150 1.87855617537021
200 2.11188760303553
250 2.26323741469897
300 nan
350 nan
400 nan
450 nan
500 2.66269045589053
550 2.71228559777669
600 2.75603772913738
};
\end{axis}

\end{tikzpicture}

        \caption[Comparison of the equilibrium mole fractions of carbon dioxide in a water-rich phase. Part 1.]{The equilibrium mole fractions of carbon dioxide in a water-rich phase, as a function of temperature and pressure, as calculated using the \ac{HEOS} mixture model in \emph{CoolProp} and the \ac{SP2009} model}
        \label{fig:SP2009vsCoolProp_xCO2_part2}
    \end{figure}
    
    \begin{figure}[H]
        \centering
        % This file was created with tikzplotlib v0.10.1.
\begin{tikzpicture}

\definecolor{crimson2143940}{RGB}{214,39,40}
\definecolor{darkgray176}{RGB}{176,176,176}
\definecolor{lightgray204}{RGB}{204,204,204}
\definecolor{mediumpurple148103189}{RGB}{148,103,189}
\definecolor{orchid227119194}{RGB}{227,119,194}
\definecolor{sienna1408675}{RGB}{140,86,75}

\begin{axis}[
legend cell align={left},
legend style={
    % fill opacity=0.8,
    % draw opacity=1,
    % text opacity=1,
    % draw=lightgray204,
    at={(1.03, 0.5)},
    anchor=west},
% tick align=outside,
% tick pos=left,
unbounded coords=jump,
% x grid style={darkgray176},
xlabel={Pressure/\unit{\bar}},
xmin=0, xmax=600,
% xtick style={color=black},
% y grid style={darkgray176},
ylabel={\(x_{CO_2}\)/\unit{\percent}},
ymin=0, ymax=10,
% ytick style={color=black}
ylabel near ticks,
xlabel near ticks
]
\addplot [semithick, black, dashed]
table {%
-1 1000
-1 1000
};
\addlegendentry{CoolProp}
\addplot [semithick, black]
table {%
-1 1000
-1 1000
};
\addlegendentry{SP2009}
\addplot [semithick, crimson2143940]
table {%
1 0.000842091372508321
2 0.0210139570098535
3 0.0410476956525778
4 0.0609438987978718
5 0.0807031559452749
6 0.100326054609207
7 0.119813180331676
8 0.139165116695176
9 0.158382445335765
10 0.177465745956357
11 0.196415596340209
13 0.233917248014918
15 0.270891984759432
17 0.307344361160281
21 0.378700105070782
23 0.413612437039206
25 0.448020338621845
27 0.48192822281512
29 0.515340475936113
31 0.548261458224602
33 0.580695504472868
35 0.612646924685472
37 0.644120004771363
39 0.675119007270853
41 0.705648172120167
43 0.735711717456504
45 0.765313840466724
47 0.794458718283002
49 0.823150508929045
51 0.851393352320671
54 0.892924855690442
57 0.933469349192333
60 0.973040610177722
63 1.01165235463741
66 1.0493182481707
69 1.08605191799506
72 1.12186696609924
75 1.15677698364687
80 1.21298610437518
85 1.26678214802429
90 1.31822865910022
95 1.36738989165389
100 1.41433113270418
125 1.61814201534475
150 1.77747593487639
200 2.00598203845093
250 2.17160118515993
300 2.30761918369489
350 2.42658564914289
400 2.53413423695859
450 2.63327998505786
500 2.72584501247257
550 2.81302759281723
600 2.89566559681683
};
\addlegendentry{\qty{99}{\degreeCelsius}}
\addplot [semithick, crimson2143940, dashed, forget plot]
table {%
1 0.0432563786719489
2 0.020204593133466
3 0.0397853045960739
4 0.0592684469303273
5 0.0786628468913642
6 0.0979703404675458
7 0.117191244817438
8 0.136325383995806
9 0.15537238396125
10 0.17433178172348
11 0.193203072937065
13 0.230679240327636
15 0.26779667765171
17 0.30455125102633
21 0.376956054687516
23 0.412598876318593
25 0.447864034805467
27 0.482748323715709
29 0.517248736731823
31 0.551362465772065
33 0.58508689850868
35 0.618419615549892
37 0.651358387444656
39 0.683901171607093
41 0.716046109227531
43 0.747791522204844
45 0.779135910136053
47 0.810077947378096
49 0.840616480204404
51 0.870750524059405
54 0.91519142853359
57 0.958718359413635
60 1.0013300580251
63 1.0430260619549
66 1.08380669021195
69 1.12367302998
72 1.16262692497503
75 1.20067096543379
80 1.26206517180086
85 1.32096146401031
90 1.37738534193975
95 1.43136786465032
100 1.48294565277556
125 1.70643932457429
150 1.87855617537021
200 2.11188760303553
250 2.26323741469897
300 nan
350 nan
400 nan
450 nan
500 2.66269045589053
550 2.71228559777669
600 2.75603772913738
};
\addplot [semithick, mediumpurple148103189]
table {%
1 0.0240980682246103
2 0.0539282236908063
3 0.0834840626515818
4 0.112766906803538
5 0.141778069866682
6 0.170518857564195
7 0.198990567600705
8 0.227194489638992
9 0.255131905275075
10 0.282804088011631
11 0.310212303229661
13 0.364241851843101
15 0.417230510537358
17 0.469188106657489
21 0.570048752496079
23 0.61897078448649
25 0.666899715160711
27 0.713844689560489
29 0.759814710683541
31 0.804818637153197
33 0.84886518065199
35 0.891962903098731
37 0.934120213546735
39 0.975345364778864
41 1.01564644957297
43 1.0550313966092
45 1.09350796598841
47 1.13108374432907
49 1.16776613940797
51 1.20356237430902
54 1.25561050490897
57 1.30570344043661
60 1.35386300902356
63 1.400109911549
66 1.4444636487019
69 1.48694244141512
72 1.52756314660185
75 1.56634117249993
80 1.62691336651363
85 1.68245947457602
90 1.73302272385372
95 1.77863992774758
100 1.81935769866457
125 1.95945622148193
150 2.04612502273426
200 2.17676564827454
250 2.28371754263978
300 2.37786910403317
350 2.463384999714
400 2.54242614590541
450 2.61629882353388
500 2.68587051039568
550 2.75175556138895
600 2.81441003966434
};
\addlegendentry{\qty{60}{\degreeCelsius}}
\addplot [semithick, mediumpurple148103189, dashed, forget plot]
table {%
1 0.0249551288774019
2 0.0558234326269891
3 0.0864249631574499
4 0.116760896529688
5 0.146831378483037
6 0.176636431313659
7 0.206176086392662
8 0.235450416183203
9 0.264459542960427
10 0.293203640405904
11 0.321682932638812
13 0.37784823790763
15 0.432958177702114
17 0.487016129009343
21 0.59199240840101
23 0.642920116618695
25 0.692814472258296
27 0.741681104902669
29 0.78952592794711
31 0.836355095273933
33 0.882174962287763
35 0.926992050891018
37 0.970813018014649
39 1.01364462735699
41 1.05549372399695
43 1.09636721160783
45 1.13627203199409
47 1.17521514673441
49 1.21320352071859
51 1.25024410739789
54 1.30404303340607
57 1.35574823985662
60 1.40538268829608
63 1.45296901957532
66 1.49852950711673
69 1.54208602676033
72 1.58366004672992
75 1.62327264405366
80 1.68499114150708
85 1.7414137365059
90 1.79263685372716
95 1.83876497475414
100 1.87992403531131
125 2.02077516987152
150 2.09836493642813
200 nan
250 nan
300 nan
350 nan
400 nan
450 2.48562781846157
500 2.52388343098841
550 2.55824405245131
600 2.58914074351589
};
\addplot [semithick, sienna1408675]
table {%
1 0.0492827466363969
2 0.100290118453331
3 0.150699066946316
4 0.200512743348626
5 0.249734266316697
6 0.298366721434451
7 0.346413160684667
8 0.393876601885051
9 0.440760028086468
10 0.487066386930558
11 0.532798589963713
13 0.62255198985819
15 0.710042769087295
17 0.795292838911824
21 0.959155148732422
23 1.03780774870784
25 1.11430026766752
27 1.18865087723116
29 1.26087683456559
31 1.33099440365012
33 1.39901876210206
35 1.46496389016083
37 1.52884243740475
39 1.59066556136614
41 1.65044273024263
43 1.70818147910462
45 1.76388710493679
47 1.8175622798233
49 1.86920655240391
51 1.91881569332855
54 1.98939253443561
57 2.05531242943015
60 2.11647359474034
63 2.17270057074217
66 2.22368164135647
69 2.26878917620921
72 2.30584807264241
75 2.29600128843308
80 2.31145177865509
85 2.32606166293033
90 2.34002568841602
95 2.35346465853231
100 2.366461262498
125 2.42650252604883
150 2.48073142450865
200 2.57799601199996
250 2.66503754558897
300 2.74474923496738
350 2.81873684422587
400 2.8880181030512
450 2.95329362235444
500 3.01507407528993
550 3.07374773147589
600 3.12961957499625
};
\addlegendentry{\qty{31}{\degreeCelsius}}
\addplot [semithick, sienna1408675, dashed, forget plot]
table {%
1 0.05934515377225
2 0.1201305299251
3 0.179603740663281
4 0.237807764747933
5 0.294782989492148
6 0.350567487984488
7 0.405197197008911
8 0.458706070449733
9 0.511126217901281
10 0.562488031251673
11 0.612820300710193
13 0.710503990885603
15 0.804379414601741
17 0.894629055634266
21 1.06489649170316
23 1.14520053376643
25 1.22245445143232
27 1.29677158457794
29 1.36825546236163
31 1.43700071075569
33 1.50309382326224
35 1.56661381029802
37 1.62763273882283
39 1.68621617011306
41 1.7424235000319
43 1.79630820209912
45 1.84791796912659
47 1.89729474320043
49 1.94447461576835
51 1.98948756781784
54 2.05299323705936
57 2.11172110797136
60 2.16567012670558
63 2.21477115350038
66 2.25884402615711
69 2.29748861380831
72 2.32967419825004
75 2.34615425239195
80 nan
85 2.37032094087297
90 nan
95 2.39090779367398
100 2.40048485021841
125 nan
150 nan
200 nan
250 2.61054406310093
300 nan
350 nan
400 nan
450 2.80090559193469
500 2.83960805997039
550 2.87549896574758
600 2.90880481351756
};
\addplot [semithick, orchid227119194]
table {%
1 0.0653247926336365
2 0.131360247318469
3 0.196540672844325
4 0.260870725067352
5 0.324354999479936
6 0.3869980298946
7 0.448804287030193
8 0.509778176992118
9 0.569924039637462
10 0.629246146814943
11 0.687748700468461
13 0.802311593020832
15 0.913644852836863
17 1.02177934772966
21 1.22856859671631
23 1.32727781705191
25 1.42289691708249
27 1.5154486203101
29 1.60495347262642
31 1.6914295612775
33 1.77489216647402
35 1.85535332411346
37 1.93282126909016
39 2.00729971489879
41 2.07878690349778
43 2.14727432386141
45 2.21274493712173
47 2.2751706379597
49 2.33450847652741
51 2.39069474575702
54 2.46884604789077
57 2.53911916860705
60 2.60019599323272
63 2.51441493832889
66 2.52219759386243
69 2.52984060875024
72 2.53735511134764
75 2.54475056853287
80 2.55683345629869
85 2.56863954506079
90 2.58019457609512
95 2.59151990430326
100 2.60263353829511
125 2.6554999674246
150 2.70472701965934
200 2.79510892268245
250 2.87738460949771
300 2.95343535127034
350 3.02441618976896
400 3.09110719549519
450 3.15406988156671
500 3.21372750000453
550 3.27041047261781
600 3.32438399595852
};
\addlegendentry{\qty{20}{\degreeCelsius}}
\addplot [semithick, orchid227119194, dashed, forget plot]
table {%
1 0.08782935270093
2 0.174597078177963
3 0.258395988247329
4 0.339416296476569
5 0.417827518104152
6 0.49378148221505
7 0.567414781853326
8 0.638850802337187
9 0.708201414878173
10 0.775568400734084
11 0.841044656371781
13 0.966658145828181
15 1.08564283696209
17 1.19850981010862
21 1.40758437369958
23 1.50449988503722
25 1.59673157493007
27 1.68453173842293
29 1.7681223059054
31 1.84769878886132
33 1.9234334027539
35 1.99547750325708
37 2.06396342418774
39 2.12900576204199
41 2.19070210634302
43 2.24913315736105
45 2.30436208705141
47 2.35643285318102
49 2.40536689761037
51 2.45115706616716
54 2.51382805304928
57 nan
60 nan
63 nan
66 nan
69 nan
72 nan
75 nan
80 nan
85 nan
90 nan
95 2.63780353411166
100 2.64589337675245
125 nan
150 nan
200 2.78602591249258
250 2.84580910855533
300 2.90082961334277
350 2.9518592651116
400 2.99942144173266
450 nan
500 nan
550 3.12467242826016
600 3.16138587552253
};
\end{axis}

\end{tikzpicture}

        \caption[Comparison of the equilibrium mole fractions of carbon dioxide in a water-rich phase. Part 2.]{The equilibrium mole fractions of carbon dioxide in a water-rich phase, as a function of temperature and pressure, as calculated using the \ac{HEOS} mixture model in \emph{CoolProp} and the \ac{SP2009} model}
        \label{fig:SP2009vsCoolProp_xCO2_part1}
    \end{figure}

    % \begin{figure}[H]
    %     \centering
    %     % This file was created with tikzplotlib v0.10.1.
\begin{tikzpicture}

\definecolor{crimson2143940}{RGB}{214,39,40}
\definecolor{darkgray176}{RGB}{176,176,176}
\definecolor{darkorange25512714}{RGB}{255,127,14}
\definecolor{forestgreen4416044}{RGB}{44,160,44}
\definecolor{lightgray204}{RGB}{204,204,204}
\definecolor{mediumpurple148103189}{RGB}{148,103,189}
\definecolor{orchid227119194}{RGB}{227,119,194}
\definecolor{sienna1408675}{RGB}{140,86,75}
\definecolor{steelblue31119180}{RGB}{31,119,180}

\begin{axis}[
legend cell align={left},
legend style={
  fill opacity=0.8,
  draw opacity=1,
  text opacity=1,
  at={(0.97,0.03)},
  anchor=south east,
  draw=lightgray204
},
tick align=outside,
tick pos=left,
unbounded coords=jump,
x grid style={darkgray176},
xlabel={Pressure/\unit{\bar}},
xmin=0, xmax=600,
xtick style={color=black},
y grid style={darkgray176},
ylabel={\(\frac{x_{CO_2}\^{Aspen}}{x_{CO_2}\^{SP2009}}\)},
ymin=0, ymax=2,
ytick style={color=black}
]
\addplot [semithick, steelblue31119180]
table {%
1 -0.524722262099008
2 -0.584154866523931
3 -0.650667820880334
4 -0.678013343531011
5 -0.694835516252736
6 -0.711587329859851
7 -0.72941535465754
8 -0.748707541896311
9 -0.769741290044094
10 -0.792753940171483
11 -0.817970319413497
13 -0.875966760191081
15 -0.945958015906874
17 -1.03101184304637
21 -1.26636410740027
23 -1.43377413073843
25 -1.65488901428811
27 -1.95963150599981
29 -2.40553282632937
31 -3.11897463434018
33 -4.44194776254461
35 -7.73334201439748
37 -30.0919851955462
39 15.8580907591179
41 2.09858532758867
43 1.32440431076438
45 1.00855561355582
47 1.09524205499019
49 1.15161917867581
51 1.19125978234364
54 1.23267169578997
57 1.26127450621103
60 1.28214630853321
63 1.29798034901302
66 1.31034278420755
69 1.32020856070511
72 1.32821724368343
75 1.3348061771187
80 1.34343371167766
85 1.34990727005607
90 1.35482697767876
95 1.35858498912666
100 1.36144793533866
125 1.3673862024378
150 1.36551208958478
200 1.35042953757799
250 1.32698477294979
300 1.29868080385318
350 1.26717163348733
400 1.23329672070738
450 1.19755423204819
500 1.16035068216179
550 1.12209409431407
600 1.08320354851134
};
\addlegendentry{\qty{250}{\degreeCelsius}}
\addplot [semithick, darkorange25512714]
table {%
1 -0.954081778600233
2 -1.02040476071754
3 -1.08704326184082
4 -1.17100631935496
5 -1.27660082461729
6 -1.40935153756456
7 -1.57803550411955
8 -1.79698521925536
9 -2.09048141273497
10 -2.50246182444458
11 -3.12080969345658
13 -6.1990061679628
15 -579.640572692633
17 1.05896076645462
21 1.08636423754651
23 1.11324079388367
25 1.12944127500918
27 1.14031142258303
29 1.14815415236056
31 1.15412390259921
33 1.15886116645925
35 1.16274821291209
37 1.16602617974763
39 1.16885395115428
41 1.17133997972625
43 1.17356050041583
45 1.17557045255062
47 1.17741028733158
49 1.17911035184844
51 1.18069379282548
54 1.18288839447717
57 1.18490520036717
60 1.18677640343321
63 1.18852548097615
66 1.19016991503612
69 1.1917229699395
72 1.19319488341637
75 1.19459368274277
80 1.19677931903897
85 1.19880164333681
90 1.2006769657851
95 1.2024174862719
100 1.20403263399531
125 1.21045025743731
150 1.21448564041966
200 1.21677566456243
250 1.21300791059791
300 1.20469452059088
350 1.19297225892868
400 1.17874004592869
450 1.16273511063316
500 1.14554851952969
550 1.12763104543985
600 1.10931473635739
};
\addlegendentry{\qty{200}{\degreeCelsius}}
\addplot [semithick, forestgreen4416044]
table {%
1 -1.68494043815793
2 -2.28260804223966
3 -3.63380932210648
4 -9.04069062598541
5 1.55530882261164
6 0.948741511216403
7 0.97700959017541
8 0.987948126845567
9 0.993782003274514
10 0.99749634912842
11 1.00015721715175
13 1.00395007069084
15 1.00677803345593
17 1.00915632525429
21 1.0132781144361
23 1.01516635260648
25 1.01698320122264
27 1.01874480881436
29 1.02046119796685
31 1.0221388475537
33 1.02378209275183
35 1.0253939225004
37 1.02697645306512
39 1.02853121946429
41 1.03005936042211
43 1.0315617389582
45 1.03303902289842
47 1.03449173976953
49 1.0359203149476
51 1.03732509858911
54 1.03938829692058
57 1.04139944434669
60 1.04335922776907
63 1.04526824164338
66 1.04712701985601
69 1.04893605713114
72 1.05069582362272
75 1.05240677497754
80 1.05515109832498
85 1.05776313305247
90 1.06024496239211
95 1.06259870396049
100 1.06482652662561
125 1.07415619339875
150 1.08068835826318
200 1.08659089554231
250 1.08499981966405
300 1.07827456989668
350 1.06837437150859
400 1.05665415769508
450 1.04393185347058
500 1.0306814085326
550 1.01718537979897
600 1.00361871623366
};
\addlegendentry{\qty{150}{\degreeCelsius}}
\addplot [semithick, crimson2143940]
table {%
1 51.3677969922693
2 0.961484461207947
3 0.969245750914044
4 0.972508292042468
5 0.974718348619547
6 0.976519418102935
7 0.978116468430437
8 0.979594507827775
9 0.980994981052766
10 0.98234045552854
11 0.983644255023517
13 0.986157465023378
15 0.988573648236694
17 0.990912115246214
21 0.995394639822082
23 0.997549491674215
25 0.999651123391276
27 1.00170170756093
29 1.00370291270493
31 1.00565607430715
33 1.0075623007273
35 1.00942254116004
37 1.01123763059628
39 1.01300832037264
41 1.0147352994285
43 1.01641920940189
45 1.01806065556178
47 1.01966021485528
49 1.02121844193243
51 1.02273587371333
54 1.02493667042781
57 1.02704856912886
60 1.02907324478699
63 1.03101234052753
66 1.0328674757171
69 1.03464025187156
72 1.03633225694978
75 1.03794506841633
80 1.0404613599848
85 1.04276924494912
90 1.04487588889162
95 1.04678839107041
100 1.04851375924971
125 1.05456709509562
150 1.05686729058351
200 1.0527948718157
250 1.04219754076635
300 nan
350 nan
400 nan
450 nan
500 0.976831200492666
550 0.964187342030423
600 0.951780389340215
};
\addlegendentry{\qty{99}{\degreeCelsius}}
\addplot [semithick, mediumpurple148103189]
table {%
1 1.03556553350265
2 1.03514317376832
3 1.03522708900909
4 1.03541810127956
5 1.03564238546276
6 1.03587622997744
7 1.03610984620324
8 1.03633858619252
9 1.03656005968871
10 1.0367729917463
11 1.03697670688663
13 1.03735536154256
15 1.03769539083923
17 1.03799760074666
21 1.03849434949001
23 1.03869218504727
25 1.03885855175593
27 1.03899505837792
29 1.03910324036348
31 1.03918455247543
33 1.03924036748708
35 1.03927197831949
37 1.03928060214926
39 1.03926738564735
41 1.0392334108397
43 1.03917970131646
45 1.03910722860352
47 1.0390169186204
49 1.03890965817322
51 1.03878630147081
54 1.03857289207739
57 1.03832784525963
60 1.03805383478914
63 1.03775354176862
66 1.03742971203424
69 1.03708521850565
72 1.03672313007345
75 1.03634678865197
80 1.03569813623076
85 1.03504052419731
90 1.03439893144671
95 1.03380394540153
100 1.03328995540085
125 1.03129385985629
150 1.02553114453586
200 nan
250 nan
300 nan
350 nan
400 nan
450 0.950055015162292
500 0.939689170129273
550 0.929677071738172
600 0.919958608385538
};
\addlegendentry{\qty{60}{\degreeCelsius}}
\addplot [semithick, sienna1408675]
table {%
1 1.20417707661654
2 1.1978301728799
3 1.19180393284891
4 1.18599826014281
5 1.18038663191828
6 1.17495505631148
7 1.16969342679724
8 1.1645933479024
9 1.15964739388983
10 1.15484879750462
11 1.15019129602413
13 1.14127655595069
15 1.13286051153751
17 1.12490520706607
21 1.11024425309136
23 1.10348042322126
25 1.09706017929188
27 1.09096086110552
29 1.08516186898861
31 1.07964444239198
33 1.07439146920646
35 1.06938732129844
37 1.06461771272243
39 1.06006957783436
41 1.05573096727552
43 1.05159096037073
45 1.04763959323395
47 1.04386780263996
49 1.04026738685869
51 1.03683098628753
54 1.03196991117783
57 1.02744530599509
60 1.02324457630253
63 1.01936326768849
66 1.01581268835731
69 1.01264967141947
72 1.01033291216812
75 1.02184361315977
80 nan
85 1.01902756003763
90 nan
95 1.01590979282647
100 1.01437741164818
125 nan
150 nan
200 nan
250 0.979552452242844
300 nan
350 nan
400 nan
450 0.948400650288791
500 0.941803746462624
550 0.935502590632863
600 0.929443577346315
};
\addlegendentry{\qty{31}{\degreeCelsius}}
\addplot [semithick, orchid227119194]
table {%
1 1.34450258714952
2 1.32914699646287
3 1.31472017729378
4 1.30109001839489
5 1.28817967589243
6 1.27592763805421
7 1.2642811092737
8 1.25319370496918
9 1.24262421941119
10 1.23253579646023
11 1.22289530434431
13 1.20484130384874
15 1.188254750838
17 1.1729634316564
21 1.14571085201245
23 1.13352296385014
25 1.12216953720302
27 1.11157297967531
29 1.10166577166375
31 1.09238884737582
33 1.08369028783026
35 1.07552425585061
37 1.06785012002652
39 1.06063172641328
41 1.05383678464441
43 1.04743633934786
45 1.04140429761817
47 1.03571697606567
49 1.03035260818087
51 1.02529069029722
54 1.01821985019963
57 nan
60 nan
63 nan
66 nan
69 nan
72 nan
75 nan
80 nan
85 nan
90 nan
95 1.017859646662
100 1.01662156343596
125 nan
150 nan
200 0.996750391329598
250 0.9890263189571
300 0.982188288663591
350 0.976009609754503
400 0.970338863079175
450 nan
500 nan
550 0.955437384518587
600 0.950968925180078
};
\addlegendentry{\qty{20}{\degreeCelsius}}
\end{axis}

\end{tikzpicture}

    %     \caption{The ratio of equilibrium mole fractions of carbon dioxide in a water-rich phase, as a function of temperature and pressure, as calculated using the \ac{HEOS} mixture model in \emph{CoolProp} and the \ac{SP2009} model}
    %     \label{fig:SP2009vsCoolProp_xCO2_ratio}
    % \end{figure}
    


